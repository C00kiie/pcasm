% To create PDF version, type
%   pdflatex pcasm.tex
% This will produce errors the first time, type R at the error prompt
% Then rerun again (twice to get all the references.

\documentclass[10pt, a4paper]{book} % <<<<<<<<<<<<<<<<<<<<<<<<<<<<<<<<<<<<<<<<<
\typeout{-----------------------------------------}
\typeout{Enter files to be included. (*=all)}
\typeout{(pcasm1,pcasm2,...)}
\typeout{-----------------------------------------}
\typein[\infiles]{ }
\if*\infiles\else\includeonly{\infiles}\fi


\newif\ifmypdf
\ifx\pdfoutput\undefined
    \pdffalse          % we are not running PDFLaTeX
\else
%    \pdfoutput=1       % we are running PDFLaTeX
%    \pdftrue
\fi
%----------------------------------------------------------------------
% The following is the construct that interests us in the end:
%\ifpdf
%   % Put PDF-specific stuff here
%\else
%   % Put LaTeX-specific stuff here
%\fi


\usepackage{ngerman} % this seems to work nicely, more or less <<<<<<<<<<<<<<<<
%\usepackage{indentfirst}  % indent first paragraph of sections <<<bad style<<<
%\usepackage{graphicx}
\usepackage{listings}
\usepackage{epsfig} % what for?
\usepackage{longtable}
\usepackage{color} % needed by vtable.latex
\usepackage{makeidx}
\ifmypdf
\usepackage[pdftex,
            bookmarks=true,
            bookmarksnumbered=true,
            pdftitle={Die PC Assemblersprache},
            pdfauthor={Paul A. Carter},
            pdfsubject={Programmieren in 80x86 Assemblersprache},
            pdfkeywords={80x86 Assembler Programmierung}]{hyperref}
\fi
\author{Paul~A.~Carter}
\title{Die PC Assemblersprache}
\usepackage{lecnote}
\makeindex

\hyphenation{UNIX Dis-place-ment Bib-li-o-thek As-semb-ler
             Mak-ro Zyk-len da-rum Byte Bytes Code IEEE
             Va-ri-ab-len} % <<<<<<<<<<<<<<<<<<<<<<<<<<<<<<<<<<<<<<<<<<<<<<<<<<
\begin{document}
\maketitle
\newlength{\AsmMargin}
\setlength{\AsmMargin}{-1cm}
\DefineVerbatimEnvironment{AsmCodeListing}{Verbatim}
 {numbers=left, frame=lines, framesep=0.75em, xleftmargin=\AsmMargin, labelposition=all, commentchar=^ }

\newcommand{\MarginNote}[1]{\marginpar{\sloppy \em \small #1}}
\thispagestyle{empty}
\vspace*{\fill}
\noindent Copyright \copyright\  2001, 2002, 2003, 2004, 2006 by Paul Carter\\

\noindent Dieses Dokument kann in seiner Gesamtheit reproduziert und
verteilt werden (zusammen mit dieser Autorenschaft-, Copyright- und
Erlaubnis-Notiz), vorausgesetzt, dass f\"{u}r das Dokument selbst, ohne
Einwilligung des Autors, keine Kosten erhoben werden. Dies schlie{\ss}t
"`fair use"' Ausz\"{u}ge wie Reviews und Werbung sowie abgeleitete
Erzeugnisse wie \"{U}bersetzungen mit ein.
\\

\noindent Beachte, dass diese Einschr\"{a}nkung nicht darauf hinzielt,
zu verhindern, dass Forderungen f\"{u}r die Leistung, das Dokument zu
drucken oder zu kopieren, erhoben werden.\\

\noindent Dozenten werden angeregt, dieses Dokument als
Kurs-Hilfsmittel zu verwenden; jedoch w\"{u}rde es der Autor begr\"{u}{\ss}en,
in diesem Fall verst\"{a}ndigt zu werden.
\\
\\
\\
\\

\noindent This may be reproduced and distributed in its entirety
(including this authorship, copyright and permission notice), provided
that no charge is made for the document itself, without the author's
consent. This includes ``fair use'' excerpts like reviews and advertising,
and derivative works like translations.\\

\noindent Note that this restriction is not intended to prohibit charging for
the service of printing or copying the document.\\

\noindent Instructors are encouraged to use this document as a class
resource; however, the author would appreciate being notified in this
case.

\index{Subroutine|see{Unterprogramm}}
\index{C++!Mitgliedsfunktionen|see{Methoden}}
\index{text Segment|see{Codesegment}} % <<<<<<<<<<<<<<<<<<<<<<<<<<<<<<<<<<<<<<<
\index{AND|see{Bitoperationen}} % <<<<<<<<<<<<<<<<<<<<<<<<<<<<<<<<<<<<<<<<<<<<<
\index{NOT|see{Bitoperationen}} % <<<<<<<<<<<<<<<<<<<<<<<<<<<<<<<<<<<<<<<<<<<<<
\index{OR|see{Bitoperationen}} % <<<<<<<<<<<<<<<<<<<<<<<<<<<<<<<<<<<<<<<<<<<<<<
\index{XOR|see{Bitoperationen}} % <<<<<<<<<<<<<<<<<<<<<<<<<<<<<<<<<<<<<<<<<<<<<
\index{Pointer|see{Zeiger}} % <<<<<<<<<<<<<<<<<<<<<<<<<<<<<<<<<<<<<<<<<<<<<<<<<
\index{Ganzzahl|see{Integer}} % <<<<<<<<<<<<<<<<<<<<<<<<<<<<<<<<<<<<<<<<<<<<<<<
\index{Signatur|see{Unterprogramm}} % <<<<<<<<<<<<<<<<<<<<<<<<<<<<<<<<<<<<<<<<<
\index{Vtable|see{C++}} % <<<<<<<<<<<<<<<<<<<<<<<<<<<<<<<<<<<<<<<<<<<<<<<<<<<<<
\index{Asm|see{Assembler}} % <<<<<<<<<<<<<<<<<<<<<<<<<<<<<<<<<<<<<<<<<<<<<<<<<<
\index{Masm|see{Assembler}} % <<<<<<<<<<<<<<<<<<<<<<<<<<<<<<<<<<<<<<<<<<<<<<<<<
\index{Tasm|see{Assembler}} % <<<<<<<<<<<<<<<<<<<<<<<<<<<<<<<<<<<<<<<<<<<<<<<<<
\index{Ganzzahl|see{Integer}} % <<<<<<<<<<<<<<<<<<<<<<<<<<<<<<<<<<<<<<<<<<<<<<<
\index{Epilog|see{Unterprogramm}} % <<<<<<<<<<<<<<<<<<<<<<<<<<<<<<<<<<<<<<<<<<<
\index{Prolog|see{Unterprogramm}} % <<<<<<<<<<<<<<<<<<<<<<<<<<<<<<<<<<<<<<<<<<<


\vfill
\frontmatter
%\tableofcontents
 % <<< where is it ???? <<<<<<<<<<<<<<<<<<<<<<<<<<<<<<<<<<<<<<<<<
\tableofcontents % <<<<<<<<<<<<<<<<<<<<<<<<<<<<<<<<<<<<<<<<<<<<<<<<<<<<<<<<<<<<
\listoffigures % <<<<<<<<<<<<<<<<<<<<<<<<<<<<<<<<<<<<<<<<<<<<<<<<<<<<<<<<<<<<<<
\listoftables % <<<<<<<<<<<<<<<<<<<<<<<<<<<<<<<<<<<<<<<<<<<<<<<<<<<<<<<<<<<<<<<
\newpage % <<<<<<<<<<<<<<<<<<<<<<<<<<<<<<<<<<<<<<<<<<<<<<<<<<<<<<<<<<<<<<<<<<<<
% -*-LaTex-*-
%front matter of pcasm book

\chapter{Prefacio}

\section*{Prop�sito}

El prop�sito de este libro es dar al lector un mejor entendimiento de
c�mo trabajan realmente los computadores a un nivel m�s bajo que los
lenguajes de alto nivel como Pascal. Teniendo un conocimiento profundo
de c�mo trabajan los computadores, el lector puede ser m�s productivo
desarrollando software en lenguajes de alto nivel tales como C y C++.
Aprender a programar en lenguaje ensamblador es una manera excelente
de lograr este objetivo. Otros libros de lenguaje ensamblador a�n
ense�an a programar el procesador 8086 que us� el PC original en 1981.
El procesador 8086 s�lo soporta el modo \emph{real}. En este modo, 
cualquier programa puede acceder a cualquier direcci�n de memoria
o dispositivo en el computador. Este modo no es apropiado para un sistema
operativo multitarea seguro. Este libro, en su lugar discute c�mo
programar los procesadores 80386 y posteriores en modo \emph{protegido}
(el modo en que corren Windows y Linux). Este modo soporta las
caracter�sticas que los sistemas operativos modernos esperan, como
memoria virtual y protecci�n de memoria.
Hay varias razones para usar el modo protegido
\begin{enumerate}
\item Es m�s f�cil de programar en modo protegido que en el modo real del
8086 que usan los otros libros.
\item Todos los sistemas operativos de PC se ejecutan en modo protegido.
\item Hay disponible software libre que se ejecuta en este modos.
\end{enumerate}
La carencia de libros de texto para la programaci�n en ensamblador de PC
para modo protegido es la principal raz�n por la cual el autor escribi�
este libro.

C�mo lo dicho antes, este libro hace uso de Software Libre: es decir el
ensamblador NASM y el compilador de C/C++ DJGPP. Ambos se pueden
descargar de Internet. El texto tambi�n discute c�mo usar el c�digo del 
ensamblador NASM bajo el sistema operativo Linux y con los compiladores
de C/C++ de Borland y Microsoft bajo Windows. Todos los ejemplos de estas
plataformas se pueden encontrar en mi sitio web:
{\url{http://www.drpaulcarter.com/pcasm}}.
Debe descargar el c�digo de los ejemplos, si desea ensamblar y correr los
muchos ejemplos de este tutorial.

Tenga en cuenta que este libro no intenta cubrir cada aspecto de la
programaci�n en ensamblador. El autor ha intentado cubrir los t�picos m�s
importantes que \emph{todos} los programadores deber�an tener

\section*{Reconocimientos}

El autor quiere agradecer a los muchos programadores alrededor del mundo
que han contribuido al movimiento de Software Libre. Todos los programe y
a�n este libro en s� mismo fueron producidos usando software libre. 
El autor desear�a agradecerle especialmente a John~S.~Fine,
Simon~Tatham, Julian~Hall 
y otros por desarrollar el ensamblador NASM ya que todos los ejemplos de
este libro est�n basados en �l; a DJ Delorie por desarrollar el
compilador usado de C/C++ DJGPP; la numerosa gente que ha contribuido al
compilador GNU gcc en el cual est� basado DJGPP; a Donald Knuth y otros
por desarrollar los lenguajes de composici�n de textos \TeX\ y \LaTeXe\
que fueron usados para producir este libro; a Richar Stallman (fundador
de la Free Software Fundation), Linus Torvalds 
(creador del n�cleo de Linux) y a otros que han desarrollado el software
que el autor ha usado para producir este trabajo.

Gracias a las siguientes personas por correcciones:
\begin{itemize}
\item John S. Fine
\item Marcelo Henrique Pinto de Almeida
\item Sam Hopkins
\item Nick D'Imperio
\item Jeremiah Lawrence
\item Ed Beroset
\item Jerry Gembarowski
\item Ziqiang Peng
\item Eno Compton
\item Josh I Cates
\item Mik Mifflin
\item Luke Wallis
\item Gaku Ueda
\item Brian Heward
\item Chad Gorshing
\item F. Gotti
\item Bob Wilkinson
\item Markus Koegel
\item Louis Taber
\item Dave Kiddell
\item Eduardo Horowitz
\item S\'{e}bastien Le Ray
\item Nehal Mistry
\item Jianyue Wang
\item Jeremias Kleer
\item Marc Janicki
\end{itemize}


\section*{Recursos en Internet}
\begin{center}
\begin{tabular}{|ll|}
\hline
P�gina del autor & \url{http://www.drpaulcarter.com/} \\
%NASM   & {\code http://nasm.2y.net/} \\
P�gina de NASM en SourceForge & \url{http://nasm.sourceforge.net/} \\
DJGPP  & \url{http://www.delorie.com/djgpp} \\
Ensamblador con Linux & \url{http://www.linuxassembly.org/} \\
The Art of Assembly & \url{http://webster.cs.ucr.edu/} \\
USENET & {\code comp.lang.asm.x86} \\
Documentaci�n de Intel & \url{http://www.intel.com/design/Pentium4/documentation.htm} \\
\hline
\end{tabular}
\end{center}


\section*{Comentarios}

El autor agradece cualquier comentario sobre este trabajo.
\begin{center}
\begin{tabular}{ll}
\textbf{E-mail:} & {\code pacman128@gmail.com} \\
\textbf{WWW:}    & \url{http://www.drpaulcarter.com/pcasm} \\
\end{tabular}
\end{center}




\mainmatter
% -*-latex-*-

\chapter{시작하기 앞서서}
\section{기수법}
% 번역이 약간 odd 함
컴퓨터 내부의 메모리는 수들로 구성되어 있다. 컴퓨터 메모리는 그 수들을 
10 개의 숫자를 이용하는 십진법(decimal)을 이용하여 저장하지 않는다. 그 대신 
하드웨어를 매우 단순화 시키는 이진법(binary)을 이용하여 컴퓨터는 모든 정보를
저장한다. 먼저, 십진 체계에 대해 알아 보도록 하자.

\subsection{십진법\index{십진법}}

십진체계에서는 10 개의 숫자(0-9)들을 이용하여 수를 나타낸다. 십진법에서 수의 각 자리
숫자는 그 수에서의 위치에 대한 10 의 멱수로 나타나게 된다. 예를 들면

\begin{displaymath}
234 = 2 \times 10^2 + 3 \times 10^1 + 4 \times 10^0
\end{displaymath}

\subsection{이진법\index{이진법|(}}

이진체계에서는 오직 2 개의 숫자(0,1) 만을 이용하며 십진법과는 달리 2 를 기준으로 나타난다.
이진수의 각 자리 숫자는 그 수에서의 위치에 대한 2 의 멱수로 나타나게 된다.(참고로, 
이진수에서 각 자리 숫자를 비트(bit)라고 부른다) 예를 들자면 

\begin{eqnarray*}
11001_2 & = & 1 \times 2^4 + 1 \times 2^3 + 0 \times 2^2 + 0 \times 2^1 
             + 1 \times 2^0 \\
& = & 16 + 8 + 1 \\
& = & 25 
\end{eqnarray*}

위 예를 통해 어떻게 이진수가 십진수로 바뀌어지는지 알 수 있다. 아래 표 ~\ref{tab:dec-bin}에서 일부 십진수들이
이진수로 어떻게 나타나는지 보여주고 있다. 

\begin{table}[t]
\begin{center}
\begin{tabular}{||c|c||cc||c|c||}
\hline
십진수 & 이진수 & & & 십진수 & 이진수 \\
\hline
0       & 0000   & & & 8       & 1000 \\
\hline
1       & 0001   & & & 9       & 1001 \\
\hline
2       & 0010   & & & 10      & 1010 \\
\hline
3       & 0011   & & & 11      & 1011 \\
\hline
4       & 0100   & & & 12      & 1100 \\
\hline
5       & 0101   & & & 13      & 1101 \\
\hline
6       & 0110   & & & 14      & 1110 \\
\hline
7       & 0111   & & & 15      & 1111 \\
\hline
\end{tabular}
\caption{0 부터 15 까지의 십진수들을 이진수로 나타냄\label{tab:dec-bin}}
\end{center}
\end{table}


\begin{figure}[h]
\begin{center}
\begin{tabular}{|rrrrrrrrp{.1cm}|p{.1cm}rrrrrrrr|}
\hline
& \multicolumn{7}{c}{이전에 올림(carry) 없음} & & & \multicolumn{7}{c}{이전에 올림 있음} & \\
\hline
&  0 & &  0 & &  1 & &  1 & & &  0 & &  0 & &  1 & & 1  & \\
& +0 & & +1 & & +0 & & +1 & & & +0 & & +1 & & +0 & & +1 &  \\
\cline{2-2} \cline{4-4} \cline{6-6} \cline{8-8} \cline{11-11} \cline{13-13} \cline{15-15} \cline{17-17}
& 0  & & 1  & & 1  & & 0  & & & 1  & & 0  & & 0  & & 1 & \\
&    & &    & &    & & c  & & &    & & c  & & c  & & c & \\
\hline
\end{tabular}

\caption{이진수의 덧셈 (c 는 \emph{올림(carry)}을 나타낸다 )\label{fig:bin-add}}
%TODO: Change this so that it is clear that single bits are being added,
%not 4-bit numbers. Ideas: Table or do sums horizontally.
\index{이진수의 덧셈}
\end{center}
\end{figure}

그림 ~\ref{fig:bin-add} 는 어떻게 두 이진숫자가 더해지는지 ({\em i.e.}, 비트(bits)) 보여주고 있다.
여기, 그 예가 있다.
\newline
\begin{center}
\begin{tabular}{r}
 $11011_2$ \\
+$10001_2$ \\
\hline
$101100_2$ \\
\end{tabular}
\end{center}

누군가가 아래와 같은 십진 나눗셈을 하였다고 하자. 
\[ 1234 \div 10 = 123\; r\; 4 \]
그는 이 나눗셈이 피제수의 가장 오른쪽 숫자를 띄어버리고 나머지 숫자를 오른쪽으로 한 자리 쉬프트 했다는 사실을 
알 수 있다. 이진체계에서는 2 로 나누는 연산을 통해 위와 동일한 작업을 발생시킬 수 있다. 예를 들어 아래와 같은
이진 나눗셈을 생각하자. \footnote{참고로 수 옆에 아래 첨자로 적힌 2 는 이 숫자가 십진법으로 나타낸 수가 아니라
이진법으로 나타낸 수 임을 알려준다}:
\[ 1101_2 \div 10_2 = 110_2\; r\; 1 \]
이 사실을 통해서 어떤 십진수를 값이 같은 이진수로 바꿀 수 있다는 것을 알 수 있다. 이는 그림 ~\ref{fig:dec-convert}
에서 나타난다. 이 방법은 가장 오른쪽의 비트 부터 찾아 나가는데 그 비트를 흔히 \emph{최하위 비트(least significant bit, lsb)} 
라 부른다. 또한 가장 왼쪽의 비트는 \emph{최상위 비트(most significant bit, msb)} 라고 부른다. 메모리의 가장 기본적인
단위는 \emph{바이트(byte)} 라 부르는 8 개의 비트들로 이루어져 있다. 

\index{이진법|)}

\begin{figure}[t]
\centering
\fbox{\parbox{\textwidth}{
\begin{eqnarray*}
\mathrm{십진수} & \mathrm{이진수} \\
25 \div 2 = 12\;r\;1 & 11001 \div 10 = 1100\;r\;1 \\
12 \div 2 = 6\;r\;0  & 1100 \div 10 = 110\;r\;0 \\
6 \div 2 = 3\;r\;0   & 110 \div 10 = 11\;r\;0 \\
3 \div 2 = 1\;r\;1   & 11 \div 10 = 1\;r\;1 \\
1 \div 2 = 0\;r\;1   & 1 \div 10 = 0\;r\;1 \\
\end{eqnarray*}

\centering
따라서 $25_{10} = 11001_{2}$
}}
\caption{십진수의 변환 \label{fig:dec-convert}}
\end{figure}

\subsection{16 진법\index{16진법|(}}

  16진수들은 16 을 밑으로 하여 나타내어 진다. 16 진수들은 흔히 자리수가 큰 이진수들의 크기를 줄이기 위해
  사용된다. 16 진수가 16 개의 숫자를 사용하기 때문에 6 개의 숫자가 모자르게 된다. 이 때문에 9 다음으로
  나타나는 숫자들은 영어 대문자 A, B, C, D, E, F 를 사용하게 된다. 따라서 16 진수 A 는 10 진수 10 과 동일한
  값을 가지며 B 는 11, 그리고 F 는 15 의 값을 가진다. 16 진수는 16 을 밑으로 하여 수를 나타내는데 아래 예를 보면
  
\begin{eqnarray*}
%\rm
2BD_{16} & = & 2 \times 16^2 + 11 \times 16^1 + 13 \times 16^0 \\
         & = & 512 + 176 + 13 \\
         & = & 701 \\
\end{eqnarray*}
십진수를 16 진수로 변환하기 위해서는 우리가 앞서 이진수를 십진수로 변환하는데 사용했던 방법을 이용하면
된다. 다만, 이진수로 변환하는 것은 2 로 나누는 것이였지만 16 진수로 변환하는 것은 16 으로 나누어 주면 된다.
예시로 아래 그림 ~\ref{fig:hex-conv} 을 보면 알 수 있다. 

\begin{figure}[t]
\centering
\fbox{\parbox{\textwidth}{
\begin{eqnarray*}
589 \div 16 & = & 36\;r\;13 \\
36 \div 16 & = & 2\;r\;4 \\
2 \div 16 & = & 0\;r\;2 \\
\end{eqnarray*}

\centering
결과적으로 $589 = 24\mathrm{D}_{16}$
}}
\caption{\label{fig:hex-conv}}
\end{figure}

16 진수가 유용한 이유는 바로 2진수를 손쉽게 16 진수로 바꿀 수 있기 때문이다. 이진수들은 그 자리수가
너무 커서 다루기 불편하지만 ~16 진수들은 그 길이가 훨씬 짧아지므로 표현하기가 매우 쉬워진다. 

16 진수를 2 진수로 바꾸는 방법은 단지 16 진수의 각 자리수를 4 비트의 이진수로 바꾸어 주면 된다. 
예를 들어 $\mathrm{24D}_{16}$ 은 \mbox{$0010\;0100\; 1101_2$} 으로 변환된다. 주의 해야 할 점은
16 진수의 모든 자리수는 반드시 4 비트로 바꾸어 주어야 하는데 (다만 16 진수의 첫 번째 자리수는 꼭
그렇게 할 필요는 없다) 왜냐하면 4 를 이진수로 바꾸었을 때 $\mathrm{4}_{16} = {100}_{2}$ 라고 해서
그냥 100 으로 한다면 그 결과가 완전히 달라지게 된다. 
마찬가지로 이진수를 16 진수로 바꾸는 것도 매우 쉬운데 오른쪽 끝에서 부터 4 비트 씩 끊어 가면서 
각각을 16 진수로 바꾸면 된다. 물론 이 때 반드시 오른쪽 끝에서 부터 변환해야 한다.\footnote{왜 그래야
하는지 이해가 잘 가지 않는다면 직접 어떤 이진수를 골라서 왼쪽 부터 16 진수로 변환해 보아라} 
예를 들면 \newline

\begin{tabular}{cccccc}
$110$ & $0000$ & $0101$ & $1010$ & $0111$ & $1110_2$ \\
  $6$ & $0$    &   $5$  &   A  &  $7$   &    $\mathrm{E}_{16}$ \\
\end{tabular}\newline

4 비트의 이진수를 \emph{니블(nibble)} \index{니블} 이라 하고, 따라서 16 진수의 각 자리수는 1 개의 니블에 대응된다.
또한 2 개의 니블은 1 바이트를 나타내며, 1 개의 바이트는 2 자리 16 진수에 대응된다. 1 바이트가 가질 수 있
는 값의 범위는 이진수로 나타내면 0 부터 11111111 , 16 진수로 나타내면 0 부터 FF, 십진수로 나타내면 0 부터 255 까지 임을
알 수 있다. 
\index{16진법|)}

\section{컴퓨터의 구조}

\subsection{메모리\index{메모리|(}}
메모리의 기본적인 단위는 바이트 이다. \index{바이트} \MarginNote{메모리의 크기를 
나타내는 단위들은 킬로바이트(~$2^{10} = 1,024$ 바이트), 메가바이트(~$2^{20} = 1,048,576$ 바이트) 
그리고 기가바이트(~$2^{30} = 1,073,741,824$ 바이트).} 만약 어떤 컴퓨터가 32 메가바이트의 메모리를 가진
다면 그 컴퓨터는 대략 3200 만 바이트의 정보를 보관할 수 있다. 메모리의 각 바이트는 그림 ~\ref{fig:memory} 에
잘 나와 있듯이 주소(adress)라 불리는 고유의 값을 가지고 있다. 

\begin{figure}[ht]
\begin{center}
\begin{tabular}{rcccccccc}
주소 & 0 & 1 & 2 & 3 & 4 & 5 & 6 & 7 \\
\cline{2-9}
메모리 & \multicolumn{1}{|c}{2A}  & \multicolumn{1}{|c}{45}  
       & \multicolumn{1}{|c}{B8} & \multicolumn{1}{|c}{20} 
       & \multicolumn{1}{|c}{8F} & \multicolumn{1}{|c}{CD} 
       & \multicolumn{1}{|c}{12} & \multicolumn{1}{|c|}{2E} \\
\cline{2-9}
\end{tabular}
\caption{ 메모리 주소 \label{fig:memory} }
\end{center}
\end{figure}

\begin{table}
\begin{center}
\begin{tabular}{|l|l|}
\hline
워드(word) & 2 바이트 \\ \hline
더블워드(double word) & 4 바이트 \\ \hline
쿼드워드(quad word) & 8 바이트 \\ \hline
패러그래프(paragraph) & 16 바이트 \\ \hline
\end{tabular}
\caption{메모리의 단위 \label{tab:mem_units} }
\end{center}
\end{table}


보통 메모리는 단일 바이트 보다는 여러개의 바이트들의 묶음으로 많이 사용된다. PC 아키텍쳐에서는
그 묶음들을 표 ~\ref{tab:mem_units} 에서 나타나는 것 처럼 이름 붙인다. 

메모리에 있는 모든 데이터들은 수 이다. 문자들은 \emph{문자 코드(character code)} 에 나타난 숫자
들에 대응되어 기록된다. 문자 코드 중에서 가장 유명한 것으로는 \emph{아스키(ASCII, American
Standard Code for Information Interchange)} 코드를 들 수 있다. 현재 새롭게 아스키 코드를 대체해
나가고 있는 코드는 유니코드(Unicode) 이다. 유니코드와 아스키코드의 가장 큰 차이점은 아스키 코드에서는
한 개의 문자를 나타내기 위해 1 바이트를 사용하지만 유니코드는 2 바이트(혹은 \emph{워드}) 를 사용한다. 
예를 들어서 아스키코드에서는 대문자 \emph{A} 를 나타내기 위해 바이트 $41_{16}$($65_{10}$) 를 사용하지만
유니코드에서는 워드 $0041_{16}$ 을 사용한다.아스키코드가 1 바이트 만을 사용하기 때문에 오직 256 개의 서로
다른 문자들만 사용할 수 있으나 \footnote{사실 아스키코드는 오직 하위 7 비트 만을 사용하므로 오직 128 개의 
서로 다른 값들만 나타낼 수 있다.} 반면에 유니코드는 아스키코드를 확장하여 훨씬 많은 문자들을 사용할 수 있게
하였다. 이를 통해 전 세계의 많은 언어들의 문자를 나타낼 수 있게 되었다. 

\index{메모리|)}

\subsection{CPU\index{CPU|(}}

중앙처리장치(CPU) 는 명령을 수행하는 장치이다. 보통 CPU 들이 수행하는 명령들은 매우 단순하다. 명령들은 대부분
CPU 자체에 있는 특별한 데이타 저장소인 \emph{레지스터} \index{레지스터}에 들어 있는 값들을 필요로 한다. 
CPU 는 레지스터에 있는 데이터에 메모리에 있는 데이터 보다 훨씬 빠른 속도로 접근 할 수 있다. 그러나 CPU 한 개에
있는 레지스터의 수는 제한적이기 때문에 프로그래머는 반드시 현재 사용되고 있는 데이터들만 레지스터에 보관
하도록 해야 한다. 
% 번역이 미숙함. 특히 아래 Clock 설명도 좀더 확실한 이해가 필요하다.
CPU 가 실행하는 명령들은 CPU 의 \emph{기계어(Machine language)} \index{기계어}에 의해 작성되어 있다. 기계 프로그램은 
고급 언어들 보다 훨씬 간단한 구조를 가지고 있다. 기계어의 명령들은 숫자들에 대응이 된다. CPU 는 이러한 명령들을 
 빠른 속도로 해석하여 효율적으로 실행해야만 한다. 따라서 기계어는 이러한 점을 항상 
염두에 두어 디자인 되기 때문에 인간이 해석하기에는 매우 난해한 면이 있다. 다른 언어로 쓰여진 프로그램들은
실행되기 위해서는 반드시 CPU 가 해석할 수 있는 기계어로 변환되어야만 한다. 따라서 특정한 컴퓨터 아키텍쳐에 대한 기계어 코드로 
소스를 바꾸어 주는 프로그램을 \emph{컴파일러(compiler)} \index{컴파일러} 라고 한다. 보통 각 CPU 마다 특정한
기계어를 가지고 있다. 이 때문에  Mac 에서 쓰여진 코드가 IBM PC 에서 잘 작동이 되지 않는다. 

컴퓨터는 \emph{클록(clock)} \index{clock} 을 이용하여 \MarginNote{\emph{GHz} 는 기가헤르츠의 약자로 초당 10억 번의 
펄스가 발생했다는 뜻이다.  예로 1.5 GHz CPU 는 초당 15억번의 펄스가 발생한다.} 명령들의 실행을 동기화 한다.
클록은 특정한 진동수에 따라 펄스를 방출하는데 이 진동수를 \emph{클록 속도} 라고 한다. 당신이 만약 1.5GHz 의 컴퓨터를 
구매 한다면 1.5Ghz 가 바로 이 컴퓨터의 클록의 진동수이다. \footnote{사실 클록 펄스는 컴퓨터의 여러 다른 부분들에서도
많이 사용되는데 그 부분들의 클록 속도는 CPU 와 다르다.} 우리가 메트로놈을 이용하여 음악을 정확한 리듬으로 연주할 
수 있는 것 처럼 CPU 도 언제나 일정하게 클록에서 방출되는 펄스를 이용하여 명령을 정확하게 수행 할 수 있다. 각 명령이
필요로 하는 클록의 진동 횟수(보통 \emph{사이클} 이라 부른다) 는 CPU 의 세대와 모델에 따라 차이가 난다. 


\subsection{CPU 의 80x86 계열 \index{80x86!CPU}}

IBM 형의 PC 는 인텔의 80x86 계열의 CPU 를 장착하고 있다. 이 계열의 CPU 들은 모두 동일한 기반의 기계어를 사용한다. 
최근에 출시된 것들의 경우 그 기능이 대폭 강화되었다. 

\begin{description}

\item[8088,8086:]   
이 CPU 들은 초기의 PC 에서 사용되었던 것들이다. 이들은 AX, BX, CX, DX, SI, DI, BP, SP, CS, DS, SS,
ES, IP, FLAGS 와 같은 16비트 레지스터들을 사용한다. 뿐만 아니라 최대 1 MB 의 메모리만 지원하며 오직 
실제 모드에서만 실행된다. 이 모드에서는 하나의 프로그램이 메모리의 모든 부분에 접근할 수 있다. 심지어
다른 프로그램이 그 부분을 이미 사용하고 있다고 해도 말이다. 이 때문이 디버깅과 보안을 유지하기가 매우
힘들어 지게 된다. 뿐만 아니라 프로그램 메모리는 반드시 \emph{세그먼트(segments)} 들로 나누어 져야 하는
데 각각의 세그먼트들의 크기는 64KB 를 넘을 수 없다. 

\item[80286:] 
이 CPU 는 AT 계열의 PC 들에서 사용되었다. 이 CPU 는 몇 개의 새로운 명령들이 추가된 8088/86 기계어를 
사용하였다. 그러나 이 CPU 의 가장 중요한 새기능은 바로 \emph{16 비트 보호 모드(16-bits protected mode)} 이다. 
이 모드에서는 최대 16 MB 의 메모리에 접근 할 수 있고 특정한 프로그램이 다른 프로그램의 메모리에 함부로
접근하는 것을 방지해 준다. 하지만 아직도 각각의 프로그램들은 64 KB 보다 작아야 하는 세그먼트들에서만 실행 될 수 있었다.

\item[80386:] 
이 CPU 는 80286 CPU 를 크게 발전시킨 것이다. 첫번째로, 기존의 16 비트 레지스터들을 32 비트 레지스터
(EAX, EBX, ECX, EDX, ESI, EDI, EBP, ESP, EIP) 로 확장하였고 새로운 16 비트 레지스터인 FS 와 GS 를 추가하였다.
또한 이 CPU 는 \emph{32 비트 보호 모드(32-bit protected mode)} 를 지원한다. 이 모드에서는 최대 4 GB 의 메모리를
사용할 수 있다. 아직도 프로그램들은 세그먼트들에 나뉘어 들어가야만 하지만 각 세그먼트의 크기는 최대
4 GB 까지 될 수 있다.  

\item[80486/펜티엄/펜티엄 프로:]
이 CPU 들은 기존의 CPU 들과는 크게 달라진 것이 없다. 다만 명령들의 연산속도가 매우 빨라졌다.

\item[펜티엄 MMX:] 
이 프로세서에는 MMX(멀티미디아 확장, MultiMedia eXtensions) 명령이 추가되었다. 이 명령은 자주 실행되는
그래픽 관련 명령들의 실행 속도를 향상 시켰다. 

\item[펜티엄 II:] 
이는 펜티엄 프로 프로세서에 MMX 명령들이 추가된 프로세서 이다. (펜티엄 III 의 경우 단지 II 보다 빨라진 것이다.) 

\end{description}
\index{CPU|)}

\subsection{8086 16비트 레지스터\index{레지스터|(}}

기존의 8086 CPU 는 4 개의 16 비트 범용 프로세서들을 지원했는데 이는 AX, BX, CX, 그리고 DX 이다. 각각의 레지스터들은
2 개의 8 비트 레지스터들로 나뉘어 질 수 있다. 예를 들어 AX 레지스터는 AH, AL 레지스터로 나뉠 수 있는데 이는 그림 ~\ref{fig:AX_reg}
에 잘 나타나 있다. AH 레지스터는 AX 의 상위 8 비트를 나타내고, AL 은 AX 의 하위 8 비트를 나타낸다. AH 와 AL 을 종종 독립적인
1 바이트 레지스터 들로 사용된다. 그러나 두 레지스터들이 AX 와 다른 것이라고 생각하면 안된다. AX 레지스터의 값을 바꾸게 되면
AH 와 AL 의 값도 바뀌고 AH 와 AL 의 값을 바꾸면 AX 의 값이 바뀐다. 범용 레지스터들은 산술 연산과 데이터 이동 명령 들에서 
잘 쓰인다. 

\begin{figure}
\begin{center}
\begin{tabular}{cc}
\multicolumn{2}{c}{AX} \\
\hline
\multicolumn{1}{||c|}{AH} & \multicolumn{1}{c||}{AL} \\
\hline
\end{tabular} 
\caption{AX 레지스터 \label{fig:AX_reg} }
\end{center}
\end{figure}

CPU 에는 2 개의 SI 와 DI 라 불리는 16 비트 색인 레지스터가 있다. \index{레지스터!색인} 이 들은 포인터로 자주 쓰이지만 범용 레지스터
들과 같은 용도로 종종 쓰인다. 그러나 색인 레지스터들은 범용 레지스터와는 달리 2 개의 8 비트 레지스터들로 나누어 질 수 없다.

16 비트 BP 와 SP 레지스터들은 기계어 스택에 들어 있는 데이터의 위치를 가리키는데 쓰인다. 이 2 개의 레지스터를 각각 기준 포인터
\index{레지스터!위치 레지스터} 와 스택 포인터\index{레지스터!스택 포인터} 라고 부른다. 이 들에 대해선 나중에 다루기로 하자. 

16 비트 CS, DS, SS, ES 레지스터들은 \emph{세그먼트 레지스터(segment register)} 라고 부른다. \index{레지스터!세그먼트} 이 프로그램의 각 부분에서 
메모리의 어떤 부분이 사용되는지 가리킨다. CS 는 코드 세그먼트(Code Segment), DS 는 데이타 세그먼트(Data Segment), SS 는 스택 세그
먼트(Stack Segment), 그리고 ES 는 보조 세그먼트(Extra Segment) 의 약자 이다. ES 는 임시적인 세그먼트 레지스터로 사용된다. 이 레지스터
들에 대한 자세한 내용은 %~\ref{실제 모드} 와 \ref{16비트 보호 모드} 현재 이 부분이 정의되 있지 않다고 오류뜸
에서 다룰 것이다. 

명령 포인터(IP) \index{레지스터!IP} 레지스터는 CS 레지스터와 함께 사용되는데 이는 CPU 에서 다음으로 실행될 명령의 주소를 저장한다. 보통 CPU
에서 명령이 하나 실행되면 IP 는 메모리 상에서 그 다음으로 실행될 명령을 가리키게 된다. 

플래그(FLAG) \index{레지스터!플래그} 레지스터들은 이전에 실행된 명령들의 중요한 결과값들을 저장하고 있다. 이 값들은 플래그 레지스터의 각각
의 비트에 저장되는데 예를 들며 이전 연산의 결과가 0 이 였다면 플래그 레지스터의 Z 비트의 값은 1 이되고 0 이 아니라면 0 이 된다. 
모든 명령이 플래그 레지스터의 비트 값들을 바꾸는 것은 아니며 이 책의 부록에 보면 어떠한 명령들이 플래그 레지스터의 값에 영향을 주는지 알 수 있다.

\subsection{80386 32비트 레지스터\index{레지스터!32비트}}

80386 과 그 이후에 출시된 프로세서들은 모두 확장 레지스터를 가지고 있다. 예를 들어 16 비트 AX 레지스터는 32 비트로 확장되었다. 이 때
하위 호환성을 위해 AX 는 계속 16 비트 레지스터로 남아 있고 그 대신 EAX 가 확장된 32 비트 레지스터를 가리키게 된다. AL 이 단지 AX 의 하위
8 비트인 것 처럼 AX 도 단지 EAX 의 하위 16 비트를 나타낸다. 이 때, EAX 의 상위 16 비트에 직접적으로 접근 할 수 있는 방법은 없다. 다른
확장된 레지스터들은 EBX, ECX, EDX, ESI, 그리고 EDI 가 있다. 

다른 많은 수의 레지스터들도 확장되었다. BP 는 EBP\index{레지스터!베이스 포인터}로, SP 는 ESP\index{레지스터!스택 포인터}로, FLAGS 는 
EFLAGS\index{레지스터!EFLAGS}로, 그리고 IP 는 EIP\index{레지스터!EIP}가 되었다. 그러나 32 비트 보호 모드에서는 범용 레지스터를 뺀 나머지 레지스터들은
모두 확장된 것으로만 사용해야 한다. 

80386 에선 아직 세그먼트 레지스터들이 16 비트로 남아 있다. 하지만 새로운 2 개의 세그먼트 레지스터들이 추가되었는데 바로 FS 와 GS 이다.
\index{레지스터!세그먼트} 이 레지스터들은 단지 ES 와 같이 임시적인 세그먼트 레지스터들로 사용된다. 

\emph{워드(word)} 라는 단어는 CPU 데이터 레지스터의 크기로 정의된다. \index{워드} 80x86 계열 에서는 위 정의가 약간 혼란스럽다. 
아래 표 ~\ref{tab:mem_units} 에 나온 것 처럼 \emph{워드} 는 2 바이트 (혹은 16 비트) 로 정의됨을 알 수 있다. 이는 처음 8086 이 발표되었을 때
정해진 것이다. 하지만 80386 이 개발 되었을 때 레지스터의 크기가 바뀌었는데도 불구하고 기존의 \emph{워드} 의 정의를 그대로 남기기로 했다.
\index{레지스터|)}

\subsection{실제 모드}\index{실제 모드|}

실제 모드 에선 \MarginNote{그렇다면 DOS 의 640KB 제한은 어디서 나타난 것일까? BIOS 는 비디오 스크린과 같은 하드웨어를 위해 1MB 의 일부분을 필요로 한다}
메모리 사용이 오직 1 MB ($2^{20}$ 바이트) 로 제한된다. 이는 00000 부터 FFFFF 에 해당하는 값이다. 따라서 메모리의 주소를 나타내기 위해선 20 비트의
숫자를 저장할 수 있는 레지스터가 필요한데 안타깝게도 8086 의 레지스터는 모두 16 비트 밖에 되지 않으므로 그럴 수 없다. 하지만 인텔은 하나의 메모리 
주소를 정하기 위해 2 개의 16 비트 값을 사용하여 문제를 해결하였다. 처음의 16 비트 값을 \emph{실렉터(selector)} 라고 부른다. 실렉터의 값은 반드시
세그먼트 레지스터에 저장되어야만 한다. 두 번째 16 비트 값을 \emph{오프셋(offset)} 이라 부른다. 메모리의 물리 주소는 32 비트의 \emph{실렉터:오프셋} 쌍
으로 나타나며 그 값은
\[ 16 * \text{selector} + \text{offset} \]
이다. 16 을 곱하는 것은 16 진수에서는 단순히 오른쪽 끝에 0 을 하나 추가하는 것이므로 매우 쉬운 작업이다. 예를 들어 특정한 메모리 공간의 주소가 047C:0048 
로 주어진다면 이는
\begin{center}
\begin{tabular}{r}
047C0 \\
+0048 \\
\hline
04808 \\
\end{tabular}
\end{center}
에 데이터가 저장되어 있다는 것이다. 사실상 실릭터의 값은 언제나 16 의 배수(패러그래프 수)가 된다. (표 ~\ref{tab:mem_units} 참조)

그러나 위와 같은 방법도 몇 가지 단점이 있는데 : 
\begin{itemize}
\item
하나의 실렉터를 사용하면 최대 64 KB 의 메모리 까지 밖에 사용 할 수 없다. 따라서 만일 프로그램이 64KB 보다 더 큰 메모리를
필요로 한다면 하나의 실렉터 만으로는 충분하지 못할 것이다. 이 때문에 프로그램이 64KB 이 상의 메모리를 사용시 
프로그램은 반드시 64KB 이하의 조각들로 쪼개져야 한다 (이 조각들을 \emph{세그먼트}\index{메모리!세그먼트} 라
부른다) 만약 프로그램의 명령이 다른 세그먼트로 넘어갈 경우 CS 의 값도 반드시 달라져야 한다. 큰 데이터를 다룰 때 DS 레지스터
에서도 비슷한 일이 발생하게 된다. 상당히 짜증나는 일이다!

\item 
위와 같이 주소 지정을 해준다면 메모리의 공간들은 유일한 주소값을 가지지 않는다. 예를 들어 물리 주소 04808 은 
047C:0048, 047D:0038, 047E:0028, 047B:0058 등과 같이 나타날 수 있다. 이 때문에 두 개의 주소값이 같은지 비교시 
복잡해진다.  

\end{itemize}

\subsection{16비트 보호 모드}\index{보호 모드!16비트}

80286 의 16비트 보호 모드에서 실렉터의 값들은 실제 모드에서의 값과 다른 방식으로 해석된다. 실제 모드에서는 실릭터의 값은
언제나 패러그래프 수 이였다. 보호 모드에서는 실릭터의 값은 \emph{디스크립터 테이블(descriptor table)} 의 \emph{인덱스(index)} 이다.  
두 모드 모두 프로그램들이 세그먼트 \index{메모리:세그먼트} 들로 나뉘어 진다. 실제 모드에서는 이러한 세그먼트가 물리 메모리에 
고정된 위치에 저장되며 실릭터는 그 세그먼트의 시작 부분의 패러그래프 값을 가리키게 된다. 그러나 보호 모드에서는 세그먼트 들이 물리 메모리에
고정된 위치에 저장되는 것이 아니다. 사실 메모리 상에 존재할 필요도 없다!

보호 모드는 \emph{가상 메모리(virtual memory)} 라는 기술을 이용한다. \index{메모리!가상} 가상 메모리 시스템의 기본적인 개념은
프로그램의 현재 사용하고 있는 데이터와 코드 만 메모리 상에 저장하는 것이다. 다른 데이터나 코드들은 나중에 필요할 때 까지 임시적으로
디스크에 보관하게 된다. 16비트 보호 모드에선 세그먼트들이 메모리와 디스크 사이를 왔다 갔다 하게 되므로 세그먼트들이 언제나 
메모리 상의 고정된 부분을 점유 하고 있을 수 가 없게 된다. 이러한 작업은 운영체에 의해 제어되므로 프로그램이 가상 메모리에서의 작업을
위해 특별히 해 주어야 될 것은 없다. 

보호 모드에서는 개개의 세그먼트가 디스크립터 테이블의 엔트리에 할당되는데 이 디스크립터 테이블에는 시스템이 그 세그먼트에 대해 알아야
할 모든 정보가 수록되어 있다. 예를 들면 현재 메모리 상에 있는지의 유무, 메모리 상에 있다면 어디에 있는지, 접근 가능한지({\em e.g.\/}, 읽기 전용).
세그먼트 엔트리의 인덱스 값이 바로 세그먼트 레지스터에 저장되어 있는 실렉터 값이다. 

16비트 보호 모드의 가장 큰 단점은 오프셋이 아직도 16비트 이라는 점이다. 이 때문에 세그먼트의 크기는 아직도 64KB 로 제한된다. 이는 큰 
배열을 만들기 매우 골치아프게 한다. \MarginNote{한 유명한 PC 칼럼니스트는 286 CPU 를 가리켜 '뇌사 상태' 라고 말했다}

\subsection{32비트 보호 모드}\index{보호 모드!32비트|(}

80386 이 처음으로 32비트 보호 모드의 장을 열었다. 386 32비트와 286 16비트 보호 모드에는 2 개의 큰 차이점이 있는데

\begin{enumerate}
\item

오프셋이 32 비트로 확장되었다. 이는 오프셋의 값이 최대 40억 까지 가능하다는 뜻이다. 따라서, 하나의 세그먼트는 최대 4GB 까지 가질 수 있다. 

\item

세그먼트들은 \index{메모리!세그먼트} 작은 4KB 크기의 \emph{페이지(page)} \index{메모리!페이지} 라 부르는 단위들로 나뉘어 질 수 있다. 
가상 메모리 \index{메모리!가상} 시스템도 세그먼트가 아닌 페이지를 기준으로 작업하게 된다. 이 말은 세그먼트의 일부분도 
메모리에 존재할 수 있게 되었다는 뜻이다. 이전의 286 16 비트 모드에서는 전체 세그먼트가 메모리 상에 한꺼번에 존재하던지, 아니면 메모리 상에 
아무 것도 존재 못하는 2 가지 경우 밖에 없었었다. 이를 거대한 32비트 모드의 세그먼트에 그대로 적용하는 것은 이치에 맞지 않다.  

\end{enumerate}

\index{보호 모드!32비트|)}

윈도우즈 3.x 버전에서 \emph{표준 모드(standard mode)} 는 286 16비트 보호 모드를 가리키고 \emph{향상 모드(enhanced mode)} 는 32 비트 모드
를 가리킨다. 윈도우즈 9X, 윈도우즈 NT/2000/XP, OS/2, 그리고 리눅스에선 모든 프로그램이 페이지 가능한 32비트 보호 모드에서 실행된다. 
% 인터럽트에 대한 정확한 번역이 필요하다. 번역이 약간 stub 하다. 
\subsection{인터럽트\index{인터럽트}}
종종 보통의 프로그램의 흐름이 즉시 응답이 필요한 프로세스 이벤트(event)로 부터 중지되는 경우가 있다. 컴퓨터 하드웨어는 \emph{인터럽트(interrupt)} 라는
메커니즘을 통해 이러한 이벤트를 처리한다. 예를 들어 마우스가 움직였을 때, 마우스 하드웨어는 마우스 움직임을 처리(마우스 커서를 움직
이기 위해) 하기 위해 현재 실행되는 일시적으로 프로그램을 중지시킨다. 인터럽트는 제어 흐름이 \emph{인터럽트 핸들러(interrupt handler)} 을 통과하게
한다. 인터럽트 핸들러는 인터럽트를 처리하는 루틴이다. 각각의 인터럽트는 하나의 정수에 대응이 되어 있다.
물리 메모리 상단에 있는 \emph{인터럽트 벡터 테이블(interrupt vector table)}에 인터럽트 핸들러들의 주소값을 보관한다. 인터럽트의 번호는 본질적으로 테이블의
인덱스가 되는 것이다. 

CPU 외부에서 오는 인터럽트를 외부 인터럽트(External inturrpt) 라 한다. 앞에서 예로 나온 마우스가 외부 인터럽트의 한 예 이다. 대다수의 입출력(I/O) 장치들, 예를 
들어 키보드, 타이머, 디스크 드라이브, CD-ROM, 그리고 사운드 카드 들은 모두 외부 인터럽트를 발생시킨다. 내부 인터럽트는 CPU 내부에서 발생되는 인터럽
트로 오류나 인터럽트 명령에 의해 발생된다. 오류 인터럽트는 \emph{트랩(trap)} 이라고도 불린다. 인터럽트 명령을 통해 발생되는 인터럽트는 
\emph{소프트웨어 인터럽트(software interrupt)} 라 부른다. DOS 는 자체 API(Application Programming Interface) 를 구현하기 위해 이러한 인터럽트를 
사용하였다. 많은 수의 현대 운영체제 (윈도우즈, 유닉스 등등) 에서는 C 기반의 인터페이스(interface) 를 사용한다. 
\footnote{그러나 이들도 커널 수준에서는 저급 수준의 인터페이스를 사용할 것이다.}

많은 인터럽트 핸들러 들은 자신의 역할이 끝나면 중단되었던 프로그램으로 제어 흐름이 돌려진다. 그들은 인터럽트 되기 전에 레지스터에 저장되어 있던
값들을 모두 복원한다. 따라서 중단되었던 프로그램도 마치 아무 일이 없었던 것 처럼 다시 원상태로 실행 될 수 있다.
(다만, 일부 CPU 사이클들을 잃어버렸다) 트랩의 경우 대부분 리턴되지 않으며 종종 프로그램을 종료 시킨다. 

\section{어셈블리어}

\subsection{기계어\index{기계어}}
CPU 들은 자기들 만이 이해할 수 있는 기계어를 가지고 있다. 기계어의 명령들은 메모리에 바이트로 저장되어 있는 숫자들이다. 각 명령들은 자기만의 
유일한 수 코드를 가지고 있고 이것을 \emph{연산 부호(operation code)} 또는 줄여서 \emph{opcode} 라 부른다. 80x86 프로세서 명령의 크기는 
각각 다르며 연산 부호는 언제나 명령 앞 부분에 위치한다. 많은 명령들은 명령에 의해 사용되는 데이터를 포함한다({\em e.g.\/}, 상수나 주소)

기계어는 직접적으로 프로그래밍하기가 매우 힘들다. 수치화 된 각각의 연산 부호들을 해독하는 일이 인간에게는 매우 힘든 일이기 때문이다. 예를 들어, 
EAX 와 EBX 레지스터의 값을 더해 다시 EAX 레지스터에 대입하는 문장은 아래와 같이 기계어로 나타내진다.  

\begin{quote}
   03 C3
\end{quote}
이는 정말로 이해하기 힘들다. 다행이도 \emph{어셈블러(assembler)} \index{어셈블러} 라 불리는 프로그램이 위와 같은 지루한 일을 프로그래머가 안해도 되게
해준다. 

\subsection{어셈블리어}\index{어셈블리어|(}
어셈블리 언어 프로그램은 고급 언어 프로그램 처럼 문자 형태로 저장된다. 각각의 어셈블리 명령들은 하나의 기계 명령에 대응된다. 예를 들어서 위에
예로 든 덧셈 연산은 어셈블리어로 아래와 같이 표현된다.

\begin{CodeQuote}
   add eax, ebx
\end{CodeQuote}
이를 통해 단순히 숫자로 나타낸 기계어 보다 훨씬 알기 쉽게 명령을 나타낼 수 있다.
{\code add} 는 덧셈을 한다라는 명령을 인간이 알기 쉬운 문자열로로 대응 시킨
\emph{연상 기호(mnemonic)} \index{연상 기호} 이다. 어셈블리어의 일반적인 형식은 아래와 같다

\begin{CodeQuote}
  {\em mnemonic operand(s)}
\end{CodeQuote}

\emph{어셈블러(assembler)} \index{어셈블러} 는 어셈블리 명령들로 작성된 파일을 읽어 들어서 기계어로 작성된 코드로 변환해 주는 프로그램이다. 
\emph{컴파일러(Compiler)} \index{컴파일러} 는 위와 비슷한 작업을 하지만 다만 고급 언어로 작성된 코드를 기계어로 변환해 준다.
\MarginNote{컴퓨터 과학자들은 컴파일러 자체를 어떻게 작성할지에 대해 수년간 연구하였다! } 모든 어셈블리 문장은 하나의 기계어 문장을 직접
적으로 나타낸다. 한편 고급 언어의 경우 하나의 문장이 \emph{훨씬} 많고 복잡한 기계 명령을 나타낸다. 

어셈블리어와 고급언어의 또다른 차이점은 모든 CPU 가 각각의 고유한 기계어를 가지고 있고 또 고유의 어셈블리어를 가지고 있다는 점이다. 다른 컴퓨터 아키텍쳐
에 어셈블리 프로그램을 포팅(porting) 하는 이는 고급 언어에서 처리하는 것 보다 훨씬 어려운 일이다. 

이 책에서는 예제들은 Netwide Assembler 또는 NASM \index{NASM} 이라 불리는 어셈블리어를 사용할 것이다. 이는 인터넷에서 무료로 받을 수 있다. 
(머리말 참조)
다른 어셈블러들로는 마이크로소프트 사의 MASM \index{MASM} 이나 볼랜드 사의 TASM \index{TASM} 을 들 수 있다. NASM 과 MASM/TASM 의 
문법은 약간 다르다. 

\subsection{피연산자}

기계 코드 명령들에 따라 피연산자의 형태와 개수가 다르다. 그러나 대부분 각각의 명령들은 고정된 수의 피연산자 수(보통 0 - 3)를 갖는다. 
피연산자로 올 수 있는 것들은 아래와 같다.

\begin{description}
\item[레지스터:]
이 피연산자들은 CPU 레지스터들에 직접적으로 접근할 수 있다. 

\item[메모리:]
이는 메모리에 저장된 데이터를 가리킨다. 이 때, 메모리의 주소값은 명령에 직접적으로 사용하거나, 레지스터에 저장하여 사용할 수도 있다.
언제나 세그먼트 최상단 부터의 오프셋 값으로 나타낸다. 

\item[즉시 피연산자:]
\index{즉시 피연산자}
즉시 피연산자(immediate) 는 명령 자체에 있는 고정된 값들이다. 이들은 명령 자체에 저장된다(즉, 데이터 세그먼트가 아닌 코드 세그먼트에 저장됨)

\item[묵시적 피연산자:]
묵시적 피연산자(implied) 는 정확히 나타나지 않는다. 예를 들어서 증가 명령은 레지스터나 메모리에 1 을 더한다. 이 때 1 을 묵시적 피연산자 라고 부른다. 
\end{description}
\index{어셈블리어|)}

\subsection{기초 명령}

어셈블리어에서 가장 기본이 되는 명령은 {\code MOV} \index{MOV} 명령이다. 이는 특정한 지점의 데이터를 다른 지점으로 이동시킨다. (이는 고급 언어의
대입 연산자와 같다) 이 명령은 두 개의 피연산자를 필요로 한다
\begin{CodeQuote}
  mov {\em dest, src}
\end{CodeQuote}
{\em src} 에 있는 데이터가 {\em dest\/} 로 복사된다. 이 때, 두 피연산자가 모두 메모리이면 안된다. 이는 어셈블리어의 한 가지 흠
이라 볼 수 있다. 어셈블리에서는 명령문에 따라 특정한 기준 없이 규칙이 달라지는 경우가 있다. 두 개의 피연산자의 크기가 같아야 한다. 
예를 들어 AX (16 비트)에 있는 값은 BL (8비트) 로 저장될 수 없다. 

여기 예제가 있다. (세미 콜론은 주석의 시작을 나타낸다\index{주석}):

\begin{AsmCodeListing}[frame=none, numbers=none]
      mov    eax, 3   ; eax 레지스터에 3 을 대입한다 (이 때 3 은 즉시 피연산자 이다)
      mov    bx, ax   ; ax 레지스터의 값을 bx 레지스터에 대입한다.
\end{AsmCodeListing}

{\code ADD} \index{ADD} 명령은 두 개의 정수를 더할 때 사용된다.

\begin{AsmCodeListing}[frame=none, numbers=none]
      add    eax, 4   ; eax = eax + 4
      add    al, ah   ; al = al + ah 
\end{AsmCodeListing}

{\code SUB} \index{SUB} 명려은 한 정수에서 다른 정수를 뺄 때 사용된다.

\begin{AsmCodeListing}[frame=none, numbers=none]
      sub    bx, 10   ; bx = bx - 10
      sub    ebx, edi ; ebx = ebx - edi
\end{AsmCodeListing}

{\code INC} \index{INC} 와 {\code DEC} \index{DEC} 는 1 을 증가 또는 감소 시키는 명령이다. 1 이 묵시적 피연산자이므로 {\code INC} 와 {\code DEC} 의
기계 코드는 동일한 작업을 수행하는 {\code ADD} 와 {\code SUB} 의 기계 코드의 크기 보다 작다. 

\begin{AsmCodeListing}[frame=none, numbers=none]
      inc    ecx      ; ecx++
      dec    dl       ; dl--
\end{AsmCodeListing}

\subsection{지시어\index{지시어|(}}

\emph{지시어(directive)} 는 CPU 가 아닌 어셈블러를 위해 만들어 진 것이다. 이것들은 주로 어셈블러로 하여금 무언가를 하게 하거나 아니면
어셈블러에게 무언가를 알려주는 역할을 한다. 지시어는 기계 코드로 변환되지 않는다. 지시어들의 주 용도로는:

\begin{list}{$\bullet$}{\setlength{\itemsep}{0pt}}
\item  상수를 정의한다
\item 데이터를 저장할 메모리를 정의한다
\item 메모리를 세그먼드로 묶는다
\item 조건적으로 소스 코드를 포함시킨다
\item 다른 파일들을 포함시킨다.
\end{list}

NASM 은 C 와 비슷한 전처리기 명령들이 있다. 그러나 NASM 의 전처리기 지시어는 C 에서의 \# 이 아닌 \% 을 사용한다. 

\subsubsection{equ 지시어\index{지시어!equ}}

{\code equ} 지시어는 \emph{심볼(symbol)} 을 정의할 때 사용된다. 심볼은 어셈블리 프로그래밍시 사용되는 상수를 뜻한다. 이는 아래와 같이 사용한다

\begin{quote}
  \code {\em symbol} equ {\em value}
\end{quote}

한 번 정의된 심볼의 값은 \emph{절대로} 재정의 될 수 없다. 

\subsubsection{The \% 정의 지시어\index{지시어!\%정의}}

이 지시어는 C 에서의 {\code \#define} 지시어와 비슷하다. 이는 C 에서 처럼 상수 매크로를 정의할 때 사용된다
\begin{AsmCodeListing}[frame=none, numbers=none]
%define SIZE 100
      mov    eax, SIZE
\end{AsmCodeListing}

위의 나온 코드는 {\code SIZE} 라는 이름의 매크로를 정의하고 이를 {\code MOV} 명령에서 어떻게 사용하는지 보여주고 있다. 매크로는 심볼들 보다 좀 더 
유연한다. 왜냐하면 매크로는 심볼들과는 다르게 재정의 될 수 있고 단순한 수가 아니여도 되기 때문이다. 

\subsubsection{데이터 지시어\index{지시어!데이터|(}}

\begin{table}[t]
\centering
\begin{tabular}{||c|c||} \hline
{\bf Unit} & {\bf Letter} \\
\hline
바이트 & B \\
워드& W \\
더블워드 & D \\
쿼드워드 & Q \\
10 바이트 & T \\
\hline
\end{tabular}
\caption{{\code RESX} 와 {\code DX} 지시어를 위한 글자 
         \label{tab:size-letters} }
\end{table}

데이터 지시어들은 데이터 세그먼트에서 메모리 상의 공간을 정의하는데 사용된다. 메모리 공간이 정의되는데에는 2 가지 방법이 있다. 첫 번째 방법은
오직 데이터를 위한 공간들만 정의한다. 두 번째 방법은 초기의 값들을 위한 방들을 정의한다. 첫 번째 방법의 경우 {\code RES{\em
X}}\index{지시어!RES\emph{X}} 지시어를 사용한다. {\em X} 에는 저장될 데이터의 크기를 나타내는 문자가 들어간다. 이는 표~\ref{tab:size-letters}를
참조하여라. 

두 번째 방법(초기값도 정의하는) 은 {\code D{\em X}} 지시어\index{지시어!D\emph{X}} 를 사용한다. {\em X} 에는 역시 {\code RES{\em X}} 지시어
사용시 들어가는 문자가 들어가게 된다. 

메모리 위치를 \emph{라벨(label)} 로 표시하는 것은 매우 흔한 일이다. \index{라벨} 라벨은 코드 상에서 특정한 메모리 위치를 가리킬 수 있다. 아래 
예제를 참조하면 :

\begin{AsmCodeListing}[frame=none, numbers=none]
L1    db     0        ; L1 로 라벨 붙혀진 바이트가 0 으로 초기화 됨
L2    dw     1000     ; L2 로 라벨 붙혀진 워드가 1000 으로 초기화 됨
L3    db     110101b  ; L3 로 라벨 붙혀진 바이트가 이진수 110101(십진수로 53)로 초기화 됨
L4    db     12h      ; L4 로 라벨 붙혀진 바이트가 16진수 12 로 초기화 됨
L5    db     17o      ; L5 로 라벨 붙혀진 바이트가 8진수 17 로 초기화됨
L6    dd     1A92h    ; L6 로 라벨 붙혀진 더블워드가 16진수 1A92 로 초기화 됨
L7    resb   1        ; L7 로 라벨 붙혀진 초기화 되지 않은 1 바이트
L8    db     "A"      ; L8 로 라벨 붙혀진 아스키 코드 A(65) 로 초기화 됨
\end{AsmCodeListing}

큰따옴표와 작은따옴표는 모두 같은 것으로 취급된다. 데이터가 연속적으로 정의되면 그 데이터들은 메모리 상에 연속되게 존재하게 된다. 
따라서, 메모리 상에서 L2 는 L1 바로 다음으로 위치하게 된다. 메모리 상에서의 순서 또한 정의 될 수 있다. 

\begin{AsmCodeListing}[frame=none, numbers=none]
L9    db     0, 1, 2, 3              ; 4 바이트를 정의한다
L10   db     "w", "o", "r", 'd', 0   ; C 문자열 "word" 를 정의한다
L11   db     'word', 0               ; L10 과 같다
\end{AsmCodeListing}

{\code DD} \index{지시어!DD} 지시어는 정수와 단일 정밀도 부동 소수점 \footnote{단일정밀도 부동 소수점은 C 언어의 {\code float} 와 같다} 상수들을 
정의할 수 있다. 그러나 {\code DQ}\index{지시어!DQ} 의 경우 오직 2배 정밀도 부동 소수점 상수만 정의할 수 있다.

크기가 큰 것들을 위해선 NASM 의 {\code TIMES} \index{지시어!TIMES} 지시어를 이용하면 된다. 이 지시어는 피연산자를 특정한 횟수 만큼 반복한다. 
예를 들면

\begin{AsmCodeListing}[frame=none, numbers=none]
L12   times 100 db 0                 ; 이는 100 개의 (db 0) 을 나열하는 것과 같다
L13   resw   100                     ; 100 개의 워드를 위한 공간을 정의한다.
\end{AsmCodeListing}
\index{지시어!데이터|)}
\index{지시어|)}

\index{라벨|(}
아까 라벨이 코드 상의 데이터를 가리키는데 쓰인다는 말을 상기해보자. 라벨이 사용되는 용도는 2 가지가 있다. 만약 라벨을 그냥 이용한다면
이는 데이터의 주소 (즉, 오프셋) 으로 생각된다. 만약 라벨이 대괄호({\code []}) 속에 사용된다면 이는 그 주소에 위치한 데이터를 나타낸다. 
다시 말해 라벨을 어떤 데이터에 대한 \emph{포인터(pointer)} 라 하면 대괄호는 C 언어의 \* 와 같은 역할을 하는 것이다. 
(MASM/TASM 에서는 다른 방법을 사용한다) 32비트 모드에서는 주소들의 크기는 32비트 이다. 아래 예제를 보면: 

\begin{AsmCodeListing}[frame=none]
      mov    al, [L1]      ; AL 에 L1 에 위치한 데이터를 대입한다.
      mov    eax, L1       ; EAX = L1 에 위치한 바이트의 주소 
      mov    [L1], ah      ; L1 에 위치한 바이트에 ah 를 대입한다
      mov    eax, [L6]     ; L6 에 위치한 더블워드를 EAX 에 대입한다
      add    eax, [L6]     ; EAX = EAX + L6 에 위치한 더블워드
      add    [L6], eax     ; L6 에 위치한 더블워드 += EAX
      mov    al, [L6]      ; L6 에 위치한 더블워드의 하위 비트를 AL 에 대입한다 
\end{AsmCodeListing}
7 번째 행을 보면 NASM 의 중요한 특징을 알 수 있다. 어셈블러는 라벨이 어떠한 데이터를 가리키고 있는지 \emph{전혀} 신경쓰지 않는다. 이는 모두
프로그래머에 달려 있기에 언제나 라벨을 정확히 사용하는데 신중을 기해야 할 것이다. 나중에 데이터의 주소값을 레지스터에 저장하여 C 언어의
포인터 처럼 쓰는 연산을 많이 하게 된다. 이 때도 이 포인터가 정확하게 사용되고 있는지에 대해선 어셈블러가 전혀 신경쓰지 않는다. 이 때문에
어셈블리를 이용한 프로그래밍은 C 보다도 오류가 잦아지게 된다. 

다음과 같은 명령을 고려하자.
\begin{AsmCodeListing}[frame=none, numbers=none]
      mov    [L6], 1             ; L6 에 위치한 데이터에 1 을 저장한다
\end{AsmCodeListing}

위 문장은 {\code 명령의 크기가 정의되지 않았습니다} 라는 오류를 발생한다. 왜? 왜냐하면 어셈블러가 1 을 바이트로 저장할지, 워드로 저장할지, 
더블워드로 저장할지 모르기 때문이다. 이 문제를 해결하기 위해선 아래와 같이 해주면 된다. 

\begin{AsmCodeListing}[frame=none, numbers=none]
      mov    dword [L6], 1       ; L6 에 1 을 저장한다
\end{AsmCodeListing}
\index{DWORD}
이를 통해 어셈블러는 1 을 메모리 상의 {\code L6} 에서 시작하는 부분 부터 더블워드로 저장하라는 것을 알 수 있다. 그 외에도 {\code BYTE}\index{BYTE}, 
{\code WORD}\index{WORD}, {\code QWORD}\index{QWORD}, {\code TWORD}\footnote{{\code TWORD} 는 메모리 상의 10 바이트를 정의한다
부동 소수점 보조 프로세서가 이 데이타 형식을 사용한다.}\index{TWORD} 가 있다.
\index{라벨|)}

\subsection{입출력 \index{I/O|(}}

입출력(I/O)은 매우 시스템 종속적인 작업들이다. 이는 컴퓨터 시스템의 하드웨어와의 소통을 필요로 한다. C 와 같은 고급 언어에선 I/O 를 위한 표준 라이브러리
를 제공한다. 어셈블리 언어에서는 C 와 같은 라이브러리가 제공되지 않는다. 따라서, 어셈블리에선 직접적으로 하드웨어에 접근하거나
(이는 보호 모드에서 특별한 권한을 가지는 작업이기도 한다) 운영체제가 제공하는 저수준의 루틴들을 사용해야 한다. 

\index{I/O!어셈블리 I/O 라이브러리|(} 
어셈블리 루틴과 함께 C 언어를 사용하는 것은 매우 흔한 일이다. 왜냐하면 어셈블리 코드가 표준 C 입출력 라이브러리 루틴을 사용할 수 있기 때문이다.
그러나, C 가 사용하는 루틴에게 정보를 전달하는데에는 규칙이 있다는 점을 알아야 한다. 그 규칙들을 여기서 직접 다루기에는 복잡하다 (나중에 다룰 것이다).
저자는 입출력을 단순화 하기 위해 복잡한 C 규칙을 숨기고 훨씬 간단한 인터페이스를 이용하는 루틴을 제공한다. 표 ~\ref{tab:asmio} 에 
저자가 개발한 루틴들에 대해 설명하고 있다. 읽기 루틴만을 제외한 모든 루틴들은 모든 레지스터의 값을 저장한다. 이 루틴들을 이용하기 위해서는 이
루틴들에 대한 정보가 기록된 파일들을 소스에 포함시켜 어셈블러가 사용할 수 있게 해야 한다. NASM 에서 파일을 포함하기 위해선 {\code \%include} 전처리기
지시어를 사용하면 된다. 저자가 만든 I/O 루틴을 \footnote{ {\code asm\_io.inc} (그리고  {\code asm\_io.inc}  가 필요로 하는 {\code asm\_io} 파일) 은 이
책의 웹 페이지인 {\code http://www.drpaulcarter.com/pcasm} 에서 다운로드가 가능하다} NASM 에 포함 시키기 위해서는 아래와 같이 하면 된다.

\begin{AsmCodeListing}[frame=none, numbers=none]
%include "asm_io.inc"
\end{AsmCodeListing}

\begin{table}[t]
\centering
\begin{tabular}{lp{3.5in}}
{\bf print\_int} & EAX 에 저장된 값을 화면에 출력한다.  
                   \\
{\bf print\_char} & AL 에 저장된 아스키 코드 값에 대응
                    하는 문자를 화면에 표시한다 \\
{\bf print\_string} & EAX 에 저장된 문자열의 {\em 주소값} 에 해당하는
                     문자열을 화면에 출력한다. 문자열은 반드시 C 형식
                     이어야 한다({\em i.e.} null 로 종료). \\
{\bf print\_nl} & 화면에 개행문자를 출력한다. \\
{\bf read\_int} & 키보드로 부터 정수를 입력받은 다음 EAX 
                 레지스터에 저장한다. \\
{\bf read\_char} & 키보드로 부터 한 개의 문자를 입력 받은 후
                  그 문자의 아스키코드값을 EAX 레지스터에 저장한다. \\
\end{tabular}
\caption{어셈블리 I/O 루틴 \label{tab:asmio} \index{I/O!asm\_io library!print\_int}
\index{I/O!asm\_io library!print\_char} \index{I/O!asm\_io library!print\_string} 
\index{I/O!asm\_io library!print\_nl} \index{I/O!asm\_io library!read\_int}
\index{I/O!asm\_io library!read\_char}}
\end{table}

위 I/O 루틴 중 출력 루틴을 사용하고 싶다면 EAX 레지스터에 적당한 값을 대입하고 {\code CALL} 명령을 사용해 호출한다. {\code CALL} 명령은 고급 언어에서의
함수를 호출하는 것과 동일하다. 이 명령 사용시 코드의 다른 부분으로 실행을 분기했다가 그 루틴이 종료된다면 다시 원래 호출했던 부분으로 돌아오게(리턴)
된다. 아래의 예제 프로그램은 이러한 I/O 루틴의 호출을 보여준다. 

\subsection{디버깅\index{디버깅|(}}
저자의 라이브러리는 프로그램을 디버깅하는데 유용한 루틴들을 포함하고 있다. 이 디버깅 루틴은 컴퓨터의 상태를 변경하지 않고서 화면에 그대로 상태를 
출력하게 해준다. 이 루틴들을 CPU 의 현 상태를 저장하고 서브 루틴을 호출하는 \emph{매크로} 이다. 이 매크로들은 {\code asm\_io.inc} 파일에 정의되어 
있다. 매크로들은 보통 명령들 처럼 사용된다. 이 때 매크로들의 피연산자 들은 반점(,)을 통해 구분한다. 

저자의 라이브러리에는 4 개의 디버깅 루틴이 있는데 이는 각각 {\code dump\_regs}, {\code
dump\_mem}, {\code dump\_stack}, {\code dump\_math} 이다. 이 들은 레지스터, 메모리, 스택, 그리고 수학 보조프로세서에 들어있는 값들을 모두 보여준다

\begin{description}

\item[dump\_regs]
\index{I/O!asm\_io library!dump\_regs} 
이 매크로는 레지스터에 들어있는 값들을 화면에({\code stdout}) 16진수로 출력한다. 또한 플래그 레지스터의  비트 세트\footnote{Chapter~2 에서 다룰 것이다} 
를 화면에 출력한다. 예를 들어서 제로 플래그가 1 이라면 \emph{ZF} 가 화면에 출력된다. 0 이라면 화면에 출력되지 않는다. 이 매크로는 한 개의 정수를 인자로
가지는데 이를 통해 서로 다른 {\code dump\_regs} 명령의 출력들을 구분 할 수 있게 한다. 

\item[dump\_mem]
\index{I/O!asm\_io library!dump\_mem} 
이 매크로는 메모리의 일부 영역의 값들을 16 진수나 아스키 문자로 출력한다. 이 매크로는 반점으로 구분되는 3 개의 매개 변수를 가진다.
첫 번째는 {\code dump\_regs} 매크로 처럼 출력을 구분하기 위한 정수, 두 번째는 출력하기 위한 메모리의 주소, 마지막은 주소 뒤에 출력될 16 바이트 
패러그래프의 수 이다.  요청된 메모리 주소 이전의 패러그래프 경계 부터 메모리 값들이 출력 될 것이다. 

\item[dump\_stack]
\index{I/O!asm\_io library!dump\_stack} 
이 매크로는 CPU 의 스택에 저장된 값들을 출력한다. (스택에 대해서는 Chapter~4 에서 다룰 것이다) 스택은 더블워드로 구성되어 있으며 이 매크로를 통해
스택에 들어 있는 값들을 출력할 수 있다. 3 개의 반점으로 구분되는 인자를 가지며 첫 번째는 {\code dump\_regs} 에서 처럼 출력을 구분하기 위한 정수, 두 번째는 
{\code EBP} 레지스터에 저장된 주소 \emph{아래} 부터 출력할 더블워드의 수, 세 번째는 {\code EBP} 레지스터 주소 \emph{위}로 출력할 더블워드의 수 이다. 

\item[dump\_math]
\index{I/O!asm\_io library!dump\_math} 
이는 수치 부프로세서의 레지스터에 저장되어 있는 값들을 출력한다. {\code dump\_regs} 에서 처럼 결과를 구분 하기 위한 정수를 매개 변수로 입력받는다.
\end{description}
\index{디버깅|)}
\index{I/O!asm\_io library|)} 
\index{I/O|)}

\section{프로그램 만들기}

요즘에는 어셈블리어 언어만을 이용하여 프로그램을 만드는 것은 흔치 않은 일이다. 어셈블리는 오직 중요한 일부 루틴에서만 이용하게 된다. 왜냐하면, 어셈블러
언어 보다 고급 언어로 작성하는 일이 \emph{훨씬} 편하기 때문이다. 또한 어셈블리 언어를 이용시 프로그램을 다른 플랫폼에 프로그램을 포팅하는 작업이 매우
어렵게 된다. 사실상, 어셈블리어를 사용하는 일은 드물다. 

하지만 우리는 왜 어셈블리 언어를 배워야 하는가?

\begin{enumerate}
\item 때때로 컴파일러가 생성한 코드 보다 어셈블리어로 직접 쓴 코드가 훨씬 
      빠르고 작을 수 있다. 
\item 어셈블리어를 통해 고급언어 사용시 힘들거나 불가능한 시스템 하드웨어의
      직접적인 접근을 할 수 있다. 
\item 어셈블리 언어를 공부함을 통해서 컴퓨터의 작동 원리를 깊게 파악 할 수 있다. 
      
\item 어셈블리 언어를 공부함을 통해 컴파일러와 C 와 같은 고급 언어가 어떻게
      동작하는지 파악하는데 도움이 된다.
\end{enumerate}
특히 마지막 두 가지 이유를 보면 우리는 어셈블리로 직접 프로그래밍을 하지 않아도 많은 도움을 받을 수 있다는 사실을 알 수 있다. 사실 저자인 나도 어셈블리어
만으로는 거의 프로그램을 짜지 않지만 어셈블리에서 얻은 아이디어를 매일 사용하고 있다. 

\subsection{첫 프로그램}

\begin{figure}[t]
\begin{lstlisting}[frame=tlrb]{}
int main()
{
  int ret_status;
  ret_status = asm_main();
  return ret_status;
}
\end{lstlisting}
\caption{{\code driver.c} 코드\label{fig:driverProg} \index{C driver}}
\end{figure}
우리가 처음에 작성할 프로그램들은 그림 ~\ref{fig:driverProg} 와 같은 단순한 C 드라이버 프로그램 들이다. 이는 단순히 {\code asm\_main}  함수
만을 호출한다. 이 루틴은 우리가 나중에 어셈블리로 쓸 것이다. C 드라이버 루틴을 이용하면 몇 가지 장점들이 있다. 첫 번째로 C 시스템은 보호 모드에서 프로그램이
잘 돌아가도록 모든 것을 설정해 준다. C 에 의해 모든 세그먼트와 이에 대응 되는 세그먼트 레지스터들은 초기화 된다. 어셈블리에서도 걱정할 필요는
없다. 두 번째로 어셈블리 코드를 사용하여 C 라이브러리를 사용할 수 있다. 저자의 I/O 루틴이 이 사실을 이용한다. 이들은 C 의 I/O 함수를 사용한다 (
{\code printf} 등등) 아래에는 간단한 어셈블리 프로그램을 소개했다. 

\begin{AsmCodeListing}[label=first.asm]
; file: first.asm
; 최초의 어셈블리 프로그램. 이 프로그램은 2 개의 정수를 입력받아
; 그 합을 출력한다. 
;
; 실행 가능한 프로그램을 만들려면 DJGPP 를 이용해라 : 
; nasm -f coff first.asm
; gcc -o first first.o driver.c asm_io.o

%include "asm_io.inc"
;
; 초기화 된 데이터는 .data 세그먼트에 들어간다.

segment .data
;
; 아래 라벨들은 출력을 위한 문자열들을 가리킨다. 
;
prompt1 db    "Enter a number: ", 0       ; 널 종료 문자임을 잊지 말라!
prompt2 db    "Enter another number: ", 0
outmsg1 db    "You entered ", 0
outmsg2 db    " and ", 0
outmsg3 db    ", the sum of these is ", 0

;
; 초기화 되지 않은 데이터는 .bss 세그먼트에 들어간다. 
;
segment .bss
;
; 이 라벨들은 입력 값들을 저장하기 위한 더블워드를 가리킨다. 
input1  resd 1
input2  resd 1

;
; 코드는 .text 세그먼트에 들어간다.
;
segment .text
        global  _asm_main
_asm_main:
        enter   0,0               ; 셋업(set up) 루틴
        pusha

        mov     eax, prompt1      ; prompt 를 출력
        call    print_string

        call    read_int          ; 정수를 읽는다. 
        mov     [input1], eax     ; input1 에 저장.

        mov     eax, prompt2      ; prompt 를 출력
        call    print_string

        call    read_int          ; 정수를 읽는다. 
        mov     [input2], eax     ; input2 에 저장. 

        mov     eax, [input1]     ; eax = input1 에 위치한 dword
        add     eax, [input2]     ; eax += input2 에 위치한 dword
        mov     ebx, eax          ; ebx = eax

        dump_regs 1                ; 레지스터의 값을 출력
        dump_mem  2, outmsg1, 1    ; 메모리를 출력
;
; 아래 단계별로 메세지를 출력한다. 
;
        mov     eax, outmsg1
        call    print_string      ; 첫 번째 메세지를 출력
        mov     eax, [input1]     
        call    print_int         ; input1 을 출력
        mov     eax, outmsg2
        call    print_string      ; 두 번째 메세지를 출력
        mov     eax, [input2]
        call    print_int         ; input2 를 출력
        mov     eax, outmsg3
        call    print_string      ; 세 번째 메세지를 출력
        mov     eax, ebx
        call    print_int         ; 합을 출력 (ebx)
        call    print_nl          ; 개행문자를 출력

        popa
        mov     eax, 0            ; C 로 리턴된다. 
        leave                     
        ret
\end{AsmCodeListing}

~13 행에서 프로그램의 데이터 세그먼트({\code .data})\index{데이터 세그먼트}에 저장될 메모리 공간을 정의하였다. 이 세그먼트에는 오직 초기화 된 데이터 만이
올 수 있다. ~17 에서 21 행에서는 몇 개의 문자열들이 선언되었다. 이들은 C 라이브러리를 통해 출력되기 때문에 반드시 \emph{널(NULL)} 문자로 끝마쳐져야 한다
(아스키 코드 값이 0). 참고로 {\code 0} 과 {\code '0'} 에는 매우 큰 차이가 있다는 사실을 알아두라. 

초기화 되지 않은 데이터는 반드시 bss 세그먼트({\code .bss}) 에서 선언되어야 한다. 이 세그먼트는 초기 유닉스 기반의 어셈블러에서 사용된 ``심볼을 통해
시작된 블록(block started by symbol)." 연산자의 이름에서 따온 것이다. 이것에 대해선 나중에 다루겠다.  

코드 세그먼트\index{코드 세그먼트}는 역사적으로 {\code .text} 라 불려왔다. 이는 명령어들이 저장되는 곳이다. (~38 행) 의 메인 루틴의 라벨에 접두어 \_ (under
score)가 사용되었음을 유의하자. 이는 \emph{C 함수 호출 규약(C calling convention)} \index{호출 규약!C} 의 일부분이다. 이 규약은 코드 컴파일 시에 C 가
사용하는 규칙들을 명시해 준다. 이는  어셈블리어와 C 를 같이 사용시에 이 호출 규약을 아는 것이 매우 중요하다. 나중에 이 호출 규약에 대해 속속 살펴 볼 것이다.
그러나, 지금은 오직 모든 C 심볼들 ({\em i.e.}, 함수와 전역 변수) 에는 C 컴파일러에 의해 \_ 가 맨 앞에 붙는 다는 사실만 알면 된다. (이 규칙은 오직 DOS/Windows
에만 해당하며 리눅스 C 컴파일러에서는 C 심볼 이름에 아무것도 붙이지 않는다) 

37 행의 {\code global}{\index{지시어!전역}} 지시어는 어셈블러로 하여금 {\code \_asm\_main} 라벨을 전역으로 생성하라고 알려준다. C 에서와는 달리 라벨들은
\emph{내부 지역(internal scope)}이 기본으로 된다. 이 뜻은 오직 같은 모듈에서만 그 라벨을 사용할 수 있다는 뜻이다. 하지만 \emph {전역(global)} 지시어를 이용하면 
라벨을 \emph {외부 지역(external scope)} 에서도 사용할 수 있게 된다. 이러한 형식의 라벨은 프로그램의 어떠한 다른 모듈에서도 접근이 가능하다. {\code asm\_io}
모듈은 {\code print\_int} 등을 모두 전역으로 선언한다. 이 때문에 우리가 {\code first.asm} 모듈에서 이를 사용할 수 있었다. 

\subsection{컴파일러 의존성}

위에 나온 어셈블리 코드는 오직 무료 GNU\footnote{무료 소프트웨어 재단(Free Software Foundation)} 의 DJGPP \index{컴파일러!DJGPP} C/C++ 컴파일러
에서만 돌아간다\footnote{\code http://www.delorie.com/djgpp} 이 컴파일러는 인터넷을 통해 무료로 다운로드 할 수 있다. 이 컴파일러는 386 이상의 
PC 를 필요로 하며, DOS 나 Windows 95/98, NT 등에서 잘 실행된다. 이 컴파일러는 COFF(Common Object File Forma) 형식의 목적 파일(Object File) 을 
이용한다. 이 형식으로 어셈블하기 위해서는 {\code nasm} 과 함께 {\code -f~coff} 스위치를 사용해야 한다. (위 소스코드 주석이 잘 나와 있다) 
생성되는 목적 파일의 확장자는 {\code o} 가 된다. 

리눅스 C 컴파일러 또한 GNU 컴파일러이다. \index{컴파일러!gcc} 위 코드를 리눅스에서 실행되게 바꾸려면 단순히 ~37 에서 38 행의 \_ 를 지워주기만
하면 된다. 리눅스에서 목적파일은 ELF(링크 및 실행 가능한 포맷, Excutable and Linkable Format) 형식을 이용한다. 리눅스에선 {\code -f~elf} 스위치를
이용하면 된다. 또한 목적파일의 확장자는 {\code o} 이다 . \MarginNote{이 컴파일러를 위한 예제 파일들은 저자의 홈페이지에서 다운 받을 수 
있다. }

볼랜드 C/C++ \index{컴파일러!볼랜드} 는 또다른 유명한 컴파일러이다. 이는 목적파일을 마이크로소프트의 OMF 형식을 이용한다. 볼랜드사 컴파일러를
위해선 {\code -f~obj} 스위치를 이용하면 된다. OMF 형식은 다른 목적 파일 포맷들과는 달리 다른 \emph{세그먼트} 지시어를 사용한다.  ~13 행의 데이터
세그먼트는 아래와 같이 바뀌어야 한다. 목적 파일의 확장자는 {\code obj} 이다.

\begin{CodeQuote}
segment \_DATA public align=4 class=DATA use32
\end{CodeQuote}
26 행의 bss 세그먼트는 아래와 같이 바뀐다:
\begin{CodeQuote}
segment \_BSS public align=4 class=BSS use32
\end{CodeQuote}
36행의 텍스트 세그먼트는 아래와 같이 바뀐다. 
\begin{CodeQuote}
segment \_TEXT public align=1 class=CODE use32
\end{CodeQuote}
뿐만 아니라 36 행에 아래와 같은 새로운 행이 덧붙여져야 한다. 
\begin{CodeQuote}
group DGROUP \_BSS \_DATA
\end{CodeQuote}

마이크로소프트 C/C++ \index{컴파일러!마이크로소프트} 컴파일러는 목적 파일으로 OMF 나 Win32 형식을 이용할 수 있다. (OMF 형식을 이용한다 해도
그 정보를 내부적으로는 Win32 로 바꾸어서 처리한다) Win32 형식을 이용하면 DJGPP 나 리눅스에서 처럼 세그먼트를 정의할 수 있다. {\code -f~win32} 스위치
를 통해 위 모드로 출력할 수 있다. 목적 파일의 확장자는 {\code obj} 가 된다. 

\subsection{코드를 어셈블 하기}

커맨드 라인에 아래와 같이 입력한다.

\begin{CodeQuote}
nasm -f {\em object-format} first.asm
\end{CodeQuote}
이 때, {\em object-format} 부분에는 {\em coff\/}, {\em elf\/}, {\em obj}, {\em win32} 중 하나가 올 수 있으며 사용할 C 컴파일러에 따라 달라진다. 
(물론 리눅스나 볼랜드나에 따라서도 소스파일의 내용이 바뀌어야 함을 잊지 말아야한다)

\subsection{C 코드를 컴파일 하기}

DJGPP 를 사용한다면:
\begin{CodeQuote}
gcc -c driver.c
\end{CodeQuote}
{\code -c} 스위치는 단지 컴파일 만을 하란 뜻이다. 아직 링크는 하지 않는다. 리눅스나 볼랜드, 마이크로소프트 컴파일러도 동일한 스위치를 사용한다. 

\subsection{목적 파일 링크하기 \label{seq:linking} \index{링킹|(}}

링킹은 기계어 코드와 목적 파일과 라이브러리 파일에 들어있는 데이터를 하나로 합쳐서 실행가능한 파일을 만드는 과정이다. 아래에 나타나있듯이 
이 과정은 복잡한다. 

C 코드는 표준 C 라이브러리와 특별한 \emph{개시 코드(startup code)}\index{개시 코드} 를 필요로 한다. 이는 C 컴파일러가 정확한 매개 변수와 함께 링커를 호출
하고 직접적으로 링커를 호출하는 작업을 훨씬 쉽게 하게 해준다. 예를 들어서 위 프로그램을 DJGPP 으로 링크 할 때, \index{컴파일러!DJGPP}

\begin{CodeQuote}
gcc -o first driver.o first.o asm\_io.o
\end{CodeQuote}
를 쓰면 {\code first.exe} (리눅스에선 그냥 {\code first}) 라는 실행 가능한 파일이 만들어 진다. 

볼랜드 컴파일러 \index{컴파일러!볼랜드} 는 
\begin{CodeQuote}
bcc32 first.obj driver.obj asm\_io.obj
\end{CodeQuote}
와 같이 하면 된다. 볼랜드는 실행 가능한 파일의 이름을 정할 때 나열된 첫 번째 파일의 이름을 따오게 된다. 따라서, 위 경우 프로그램의 이름은 {\code first.exe} 
가 된다. 

컴파일과 링크 과정을 하나로 같이 묶어서 할 수 도 있다. 예를 들어서 
\begin{CodeQuote}
gcc -o first {\em driver.c} first.o asm\_io.o
\end{CodeQuote}
그러면 {\code gcc} 는 {\code driver.c} 를 컴파일 한 후 링크할 것이다. 
\index{링킹|)}

\subsection{어셈블리 리스트 파일 이해 \index{listing file|(}}

{\code -l {\em listing-file}} 스위치는 {\code nasm} 으로 하여금 주어진 이름의 리스트 파일을 생성하라고 명령한다. 
이 파일은 코드가 어떻게 어셈블 되었는지 보여준다.
아래 ~17 에서 18 행 (데이타 세그먼트) 이 리스트 파일에 어떻게 나타났는지 보여준다. (리스트 파일의 행 번호를 잘 보면 실제 소스 파일의 행 번호와 다름을 
알 수 있다) 

\begin{Verbatim}[xleftmargin=\AsmMargin]
48 00000000 456E7465722061206E-     prompt1 db    "Enter a number: ", 0
49 00000009 756D6265723A2000
50 00000011 456E74657220616E6F-     prompt2 db    "Enter another number: ", 0
51 0000001A 74686572206E756D62-
52 00000023 65723A2000
 \end{Verbatim}
 첫 번재 열은 각 행의 번호를 표시하고 두 번째 열은 세그먼트의 데이타의 오프셋을 16 진수로 나타낸다. 세 번째 열은 저장될 16 진수 값을 보여준다. 위 경우 그 값은
 아스키코드 값과 대응이 된다. 마지막으로 소스 파일의 코드가 나타나게 된다. 두 번째 열에 나타나는 오프셋은 실제 프로그램에서의 데이터의 오프셋과 다를
 가능성이 \emph{매우} 높다. 각각의 모듈은 데이터 세그먼트(를 포함한 다른 세그먼트들에서도) 에서의 자신만의 라벨을 가지고 있다.  링크 과정에선 
 (~\ref{seq:linking}참조) 에선 모든 데이터 세그먼트의 라벨들의 정의가 하나의 데이터 세그먼트로 통합된다. 그리고 마지막으로 정리된 오프셋들은 링커에 의해 
 계산된다. 

아래 소스코드의 ~54 에서 56 행의 텍스트 세그먼트의 리스트 파일에서 해당되는 부분을 보여주고 있다. 

\begin{Verbatim}[xleftmargin=\AsmMargin]
94 0000002C A1[00000000]          mov     eax, [input1]
95 00000031 0305[04000000]        add     eax, [input2]
96 00000037 89C3                  mov     ebx, eax
\end{Verbatim}
3 번째 열은 어셈블리에 의해 만들어진 기계어 코드를 보여준다. 보통 명령어들의 완전한 코드는 아직 계산되지 못한다. 예를 들어서 ~94 행에서 {\code input1} 의
오프셋(혹은 주소) 은 코드가 링크 되기 전 까지는 확실히 알 수 없다. 어셈블러는 {\code mov} 명령의 연산 부호(Opcode) 는 계산할 수 있지만 (이는 A1 에 
나타남) 대괄호 안의 정확한 값은 알 수 없다. 위 경우 어셈블러는 위 파일에서 정의된 bss 세그먼트의 시작을 임시적으로 오프셋 0 으로 하였다. 이것이
프로그램의 마지막 bss 세그먼트의 시작을 의미한다는 것이 \emph{아니} 라는 사실을 명심하여라. 코드가 링크가 된다면 링커는 정확한 오프셋 값을 그 자리에
집어 넣을 것이다. 다른 명령들의 경우, 예를 들어 ~96 행을 본다면 어떠한 라벨도 참조하지 않음을 알 수 있다. 여기에선 어셈블러가 정확한 기계어 코드를 
계산할 수 있다. 

\index{리스팅 파일|)}

\subsubsection{빅, 리틀 엔디안 표현 \index{엔디안|(}}
~95 행을 자세히 본다면 기계어 코드의 대괄호 속에 있는 오프셋 값이 매우 이상하다는 것을 알 수 있다. {\code input2} 라벨은 오프셋 4 (파일에 정의되어 있는 대로)
이다. 그러나, 메모리에서의 오프셋은 00000004 가 아니라 04000000 이다! 왜냐하면 프로세서가 정수를 메모리 상에 다른 순서로 저장하기 때문이다. 정수를 저장하는
방법에는 대표적으로 2 가지 방법이 있다. \emph{빅 엔디안(Big endian)} 과 \emph{리틀 엔디안(Little endian)} 이 그것이다. 빅 엔디안은 우리가 생각하는 그대로
정수를 저장한다. 가장 큰 바이트(\emph{i.e.} 가장 상위 바이트) 가 먼저 저장되고 가장 하위 바이트가 나중에 저장된다. 예를 들어서 dword 00000004 는 4 개의 바이
트 00~00~00~04 로 저장된다. IBM 메인 프레임, 대부분의 RISC 프로세서, 그리고 모토롤라 프로세서 들은 빅 엔디안 방법을 이용한다. 그러나 인텔 기반의 프로세서
는 모두 리틀 엔디안 방식을 이용한다. 여기서는 하위 바이트가 먼저 저장된다. 따라서 00000004 는 메모리 상에 04~00~00~00 으로 저장된다. 이 형식은 CPU
에 따라 정해져 있어서 바뀔 수 없다. 보통 프로그래머는 어떤 방식을 사용하느냐를 구별한 필요가 없다. 그러나 이 사실을 아는 것이 중요한 이유가 몇 가지 있는데

\begin{enumerate}
\item 이진 데이터가 서로 다른 컴퓨터로 전송될 때 (파일 또는
      네트워크를 통해서 ).
\item 멀티바이트(multibyte) 정수 데이터가 메모리에 쓰이고  
      개개의 바이트로 읽어 질 때(\emph{역도 같음})
\end{enumerate}

엔디안 표현은 배열의 원소들의 순서에 대해서는 영향을 주지 않는다. 배열의 첫 번째 원소는 언제나 최하위의 주소를 갖는다. 이는 문자열(단지 문자들의 배열인)
에서도 해당된다. 엔디안 표현은 단지 배열의 개개의 원소들에 대해서만 영향을 줄 뿐이다. 
\index{엔디안|)}

\begin{figure}[t]
\begin{AsmCodeListing}[label=skel.asm]
%include "asm_io.inc"
segment .data
;
; 초기화된 데이터는 여기 데이타 세그먼트에 들어간다. 
;

segment .bss
;
; 초기화 되지 않은 데이터는 여기 bss 세그먼트에 들어간다. 
;

segment .text
        global  _asm_main
_asm_main:
        enter   0,0               ; 셋업 루틴
        pusha

;
; 코드는 텍스트 세그먼트에 들어간다. 이 주석 앞 뒤의 코드를
; 수정하지 마세요 
;

        popa
        mov     eax, 0            ; C 로 리턴된다. 
        leave                     
        ret
\end{AsmCodeListing}
\caption{뼈대 프로그램\label{fig:skel}}
\end{figure}

\section{뼈대 파일 \index{뼈대 파일}}

그림 ~\ref{fig:skel} 은 뼈대 파일(Skeleton file) 을 보여준다. 어셈블리 프로그램 코딩시 매번 같은 내용을 작성하기보단 위 코드를 기준으로 시작하면
편리하다. 








\chapter{基本組合語言}

\section{整形工作方式 \index{整形|(}}

\subsection{整形表示法 \index{整形!表示法|(}}

\index{整形!無符號|(}
整形有兩種類型:有符號和無符號。無符號整形(即此類型沒有負數)以一種非常直接的二進位方式來表示。數字200作為一個
無符號整形數將被表示為11001000(或十六進位C8)。 \index{整形!無符號|)}

\index{整形!有符號|(}
有符號整形(即此類型可能為正數也可能為負數)以一種更複雜的方式來表示。例如,考慮
$-56$。$+56$當作一個位元組來考慮時將被表示為00111000。在紙上,你可以將$-56$表示為$-111000$,但是在電腦記憶體中如何以一個位元組來表示,如何儲存這個負號呢?

有三種普遍的技術被用來在電腦記憶體中表示有符號整形。所有的方法都把整形的最大有效位元當作一個\emph{符號位元}來使用。\index{整形!符號位元}如果數為正數,則這一位為0;為負數,這一位就為1。
\index{整形!有符號|)}
\subsubsection{原碼 \index{整形!表示法!原碼}}

第一種方法是最簡單的,它被稱為\emph{原碼}。它用兩部分來表示一個整形。第一部分是符號位元,第二部分是整形的原碼。所以56表示成位元組形式為$\underline{0}0111000$
(符號位元加了下劃線)而$-56$將表示為
$\underline{1}0111000$。最大的一個位元組的值將是
$\underline{0}1111111$或$+127$,而最小的一個位元組的值將是
$\underline{1}1111111$或$-127$。要得到一個數的相反數,只需要將符號位元變反。
這個方法很簡單直接,但是它有它的缺點。首先,0會有兩個可能的值:$+0$
($\underline{0}0000000$) 和 $-0$
($\underline{1}0000000$)。因為0不是正數,也不是負數,所以這些表示法都應該把它表示成一樣。這樣會把CPU的運算邏輯弄得很複雜
。第二,普通的運算同樣
是麻煩的。如果10加$-56$,這個將改變為10減去56。同樣,這將會複雜化CPU的邏輯。

\subsubsection{反碼 \index{整形!表示法!反碼}}

第二種方法稱為\emph{反碼}表示法。一個數的反碼可以通過將這個數的每一位求反得到。
(另外一個得到的方法是:新的位值等於$1 - $老的位值。) 例如:
$\underline{0}0111000$
($+56$)的反碼是$\underline{1}1000111$。在反碼表示法中,計算一個數的反碼等價於求反。因此,
$-56$就可以表示為$\underline{1}1000111$。注意,符號位元在反碼中是自動改變的,你需要兩次求反碼來得到原始的數值。就像第一種方法一樣,0有兩種表示:
$\underline{0}0000000$ ($+0$)和$\underline{1}1111111$
($-0$)。用反碼表示的數值的運算同樣是麻煩的。

這有一個小訣竅來得到一個十六進位數值的反碼,而不需要將它轉換成二進位。這個訣竅就是用F(或十進位中的15)減去每一個十六進位位。這個方法假定數中的每一位的數值是由4位元二進位組成的。這有一個例子:$+56$ 用十六進位表示為38。要得到反碼,用F減去每一位,得到C7。這個結果是與上面的結果吻合的。

\subsubsection{補數 \index{整形!表示法!補數|(}
               \index{補數|(}}

前面描述的兩個方法用在早期的電腦中。現代的電腦使用第三種方法稱為\emph{補數}表示法。一個數的補數可以由下面兩步得到:
\begin{enumerate}
\item 找到該數的反碼
\item 將第一步的結果加1
\end{enumerate}
這有一個使用$\underline{0}0111000$ (56)的例子。首先,經計算得到反碼:$\underline{1}1000111$ 。然後加1:
\[
\begin{array}{rr}
 & \underline{1}1000111 \\
+&                    1 \\ \hline
 & \underline{1}1001000
\end{array}
\]

在補數表示法中,計算一個補數等價於對一個數求反。因此,$\underline{1}1001000$
是$-56$的補數。要得到原始數值需兩次求反。令人驚訝的是補數不符合這個規定。通過對$\underline{1}1001000$的反碼加1得到補數
。
\[
\begin{array}{rr}
 & \underline{0}0110111 \\
+&                    1 \\ \hline
 & \underline{0}0111000
\end{array}
\]

當在兩個補數運算元間進行加法操作時,最左邊的位相加可能會產生一個進位。這個進位是\emph{不}被使用的。記住在電腦中的所有資料都是有固定大小的(根據位元數多少)。兩個位元組相加通常得到一個位元組的結果
(就像兩個字相加得到一個字,{\em 等\/}。)
這個特性對於補數表示法來說是非常重要的。例如,把0作為一個位元組來考慮它的補數形式($\underline{0}0000000$)。計算它的補數形式得到總數:
\[
\begin{array}{rr}
 & \underline{1}1111111 \\
+&                    1 \\ \hline
c& \underline{0}0000000
\end{array}
\]
其中$c$代表一個進位。(稍後將展示如何偵查到這個進位,但是它在這的結果中不儲存。)因此,在補數表示法中0只有一種表示。這就使得補數的運算比前面的方法簡單。

\begin{table}
\centering
\begin{tabular}{||c|c||}
\hline
數值 & 十六進位表示 \\
\hline
0 & 00 \\
1 & 01 \\
127 & 7F \\
-128 & 80 \\
-127 & 81 \\
-2 & FE \\
-1 & FF \\
\hline
\end{tabular}
\caption{補數表示法 \label{tab:twocomp}}
\end{table}

使用補數表示法,一個有符號的位元組可以用來代表從$-128$
到$+127$的數值。表~\ref{tab:twocomp}
展示一些可選的值。如果使用了16位,那麼可以表示從$-32,768$\\
到$+32,767$的有符號數值。$+32,767$可以表示為7FFF, $-32,768$ 為8000,
-128為FF80而-1為FFFF。32位的補數大約可以表示$-20$億到$+20$億的數值範圍。


CPU對某一的位元組(或字,雙字)具體表示多少
並不是很清楚。組合語言並沒有類型的概念,而高階語言有。資料解釋成什麼取決於使用在這個資料上的指令。到底十六進位數FF被看成一個有符號數
$-1$
還是無符號數$+255$取決於程式師。C語言定義了有符號和無符號整形。這就使C編譯器能決定使用正確的指令來使用資料。

\index{補數|)} \index{整形!表示法!補數|)} 

\subsection{正負號延伸 \index{整形!正負號延伸|(}}

在組合語言中,所有資料都有一個指定的大小。為了與其他資料一起使用而改變資料大小是不常用的。減小它的大小是最簡單的。

\subsubsection{減小數據的大小}

要減小資料的大小,只需要簡單地將多餘的有效位移位即可。這是一個普通的例子:
\begin{AsmCodeListing}[numbers=none,frame=none]
      mov    ax, 0034h      ; ax = 52 (以十六位元儲存)
      mov    cl, al         ; cl = ax的低八位
\end{AsmCodeListing}

當然,如果數字不能以更小的大小來正確描述,那麼減小資料的大小將不能工作。例如,如果{\code AX}是
0134h (或十進位的308) ,那麼上面的代碼仍然將
{\code CL}置為34h。這種方法對於有符號和無符號數都能工作。考慮有符號數:如果{\code AX}是FFFFh (也就是$-1$),那麼{\code CL} 將會是FFh (一個位元組表示的$-1$)。然而,注意如果在{\code AX}裏的值是無符號的,這個就不正確了!

無符號數的規則是:為了能轉換正確,所有需要移除的位元都必須是0。有符號數的規則是:需要移除的位元必須要麼都是1,要麼都是0。另外,沒有移除的第一個比特位的值必須等於移除的位的第一位。這一位將會是變小的值的新的符號位元。這一位元與原始符號位元相同是非常重要的!

\subsubsection{增大數據的大小}

增大資料的大小比減小資料的大小更複雜。考慮十六進位位元組:FF。如果它擴展成一個字,那麼這個字的值應該是多少呢?它取決於如何解釋FF。如果FF是一個無符號位元組(十進位中為),那麼這個字就應該是00FF;但是,如果它是一個有符號位元組(十進位中為$-1$),那麼這個字就應該為
FFFF。

一般說來,擴展一個無符號數,你需將所有的新位置為0.因此,FF就變成了00FF。但是,擴展一個有符號數,你必須\emph{擴展}符號位元。\index{整形!符號位元}
這就意味著所有的新位元通過複製符號位元得到。因為FF的符號位元為1,所以新的位必須全為1,從而得到FFFF。如果有符號數5A
(十進位中為90)被擴展了,那麼結果應該是005A。

80386提供了好幾條指令用於數的擴展。謹記電腦是不清楚一個數是有符號的或是無符號的。這取決於程式師是否用了正確的指令。

對於無符號數,你可以使用{\code MOV}指令簡單地將高位置0。例如,將一個在AL中的無符號位元組擴展到AX中:
\begin{AsmCodeListing}[numbers=none,frame=none]
      mov    ah, 0   ; 輸出高8位為0
\end{AsmCodeListing}
但是,使用{\code MOV}指令把一個在AX中的無符號字轉換成在EAX中的無符號雙字是不可能的。為什麼不可以呢?因為在{\code MOV}指令中沒有方法指定EAX的高16位。80386通過提供一個新的指令{\code MOVZX}來解決這個問題。\index{MOVZX} 這個指令有兩個運算元。目的運算元
(第一個運算元)必須是一個16或32位的寄存器。源運算元(第二個運算元)可以是一個8或16位元的寄存器或記憶體中的一個字。另一個限制是目的運算元必須大於源運算元。(許多指令要求源和目的運算元必須是一樣的大小。) 這兒有幾個例子:
\begin{AsmCodeListing}[numbers=none,frame=none]
      movzx  eax, ax      ; 將ax擴展成eax
      movzx  eax, al      ; 將al擴展成eax
      movzx  ax, al       ; 將al擴展成ax
      movzx  ebx, ax      ; 將ax擴展成ebx
\end{AsmCodeListing}

對於有符號數,在任何情況下,沒有一個簡單的方法來使用{\code MOV}
指令。8086提供了幾條用來擴展有符號數的指令。{\code CBW} \index{CBW}
(Convert Byte to
Word(位元組轉換成字))指令將AL正負號擴展成AX。運算元是不顯示的。{\code
CWD} \index{CWD} (Convert Word to Double
word(字轉換成雙字))指令將AX正負號擴展成DX:AX。 \\
DX:AX表示法表示將DX和AX寄存器當作一個32位寄存器來看待,其中高16位在DX中,低16位在AX中。(記住8086沒有32位寄存器!)
80386加了好幾條新的指令。{\code CWDE} \index{CWDE} (Convert Word to
Double word
Extended(字轉換成擴展的雙字))指令將AX正負號擴展成EAX。{\code CDQ}
\index{CDQ} (Convert Double word to Quad
word(雙字擴展成四字))指令將EAX正負號擴展成EDX:EAX\index{寄存器!EDX:EAX}
(64位!). 最後, {\code MOVSX} \index{MOVSX}指令像{\code
MOVZX}指令一樣工作,除了它使用有符號數的規則外。


\subsubsection{C編程中的應用}

無符號和有符號數的擴展\MarginNote{ANSI C並沒有定義\\{\code char}類型是有符號\\的,還是無符號的,\\它取決於各個編譯\\器的決定。這就是\\為什麼在圖~\ref{fig:charExt}中\\明確指定類型的原因。
}同樣發生在C中。 C中的變數可以被聲明成有符號的,或無符號的({\code
int}是有符號的)。考慮在圖~\ref{fig:charExt}中的代碼。在第3行中,變數{\code
a}使用了無符號數的規則(使用{\code
MOVZX})進行了擴展,但是在第4行,變數{\code
b}使用了有符號數的規則(使用 {\code MOVSX})進行了擴展。

\begin{figure}[t]
\begin{lstlisting}[frame=tlrb]{}
unsigned char uchar = 0xFF;
signed char   schar = 0xFF;
int a = (int) uchar;     /* a = 255 (0x000000FF) */
int b = (int) schar;     /* b = -1  (0xFFFFFFFF) */
\end{lstlisting}
\caption{}
\label{fig:charExt}
\end{figure}

這有一個直接與這個主題相關的普遍的C編程中的一個bug。考慮在圖~\ref{fig:IObug}中的代碼。
{\code fgetc()}的原型{\samepage 是:
\begin{CodeQuote}
int fgetc( FILE * );
\end{CodeQuote}
一個可能的問題:}為什麼這個函式返回一個{\code
int}類型,然後又被當作字元類型被讀呢?
原因是它一般確實是返回一個{\code char}
類型的值(使用0擴展成一個{\code
int}類型的值)。但是,有一個值它可能不會以字元的形式返回:{\code
EOF}。 這是一個宏,通常被定義為 $-1$。因此,{\code
fgetc()}不是返回一個通過擴展成{\code int}類型得到的{\code
char}類型的值 (在十六進位中表示為{\code 000000{\em xx}}),就是{\code
EOF} (在十六進位中表示為{\code FFFFFFFF})。

\begin{figure}[t]
\lstset{escapeinside=`',language=Pascal,%
}
\begin{lstlisting}[stepnumber=0,frame=tlrb]{}
char ch;
while( (ch = fgetc(fp)) != EOF ) {
  /* `對ch做一些事情' */
}
\end{lstlisting}
\caption{} \label{fig:IObug}
\end{figure}

圖~\ref{fig:IObug}中的程式的基本的問題是
{\code fgetc()}返回一個{\code int}類型,但是這個值以{\code char}的形式儲存。C將會切去較高順序的位元來使{\code int}類型的值適合{\code char}類型。唯一的問題是數
(十六進位) {\code 000000FF}和{\code FFFFFFFF}都會被切成位元組{\code FF}。因此,while迴圈不能區別從檔中讀到的位元組{\code FF}和檔的結束。

實際上,在這種情況下代碼會怎麼做,取決於{\code
char}是有符號的,還是無符號的。為什麼?因為在第2行{\code ch}是與
{\code EOF}進行比較。 因為{\code EOF}是一個{\code
int}類型的值\footnote{這是一個普通的誤解:在檔的最後有一個EOF字元。
這是 \emph{不} 正確的!},{\code ch}將會擴展成一個{\code
int}類型,以便於這兩個值在相同大小下比較\footnote{對於這個要求的原因將在下面介紹。}。
就像圖~\ref{fig:charExt}展示的一樣,變數是有符號的還是無符號的是非常重要的。

如果{\code char}是無符號的,那麼{\code FF}就被擴展成 {\code
000000FF}。這個拿去與{\code EOF} ({\code FFFFFFFF})比較,它們並不相等。因此,迴圈不會結束。

如果{\code char}是有符號的,{\code FF}就被擴展成{\code
FFFFFFFF}。這就導致比較相等,迴圈結束。但是,因為位元組{\code FF}可能是從檔中讀到的,迴圈就可能過早地被結束了。

這個問題的解決辦法是定義{\code ch} 變數為
{\code int}類型,而不是 {\code char}類型。當做了這個改變,在第2行就不會有切去和擴展操作執行了。在迴圈休內,對值進行切去操作是很安全的,因為{\code ch}在這兒 \emph{必須}實際上已經是一個簡單的位元組了。

\index{整形!正負號延伸|)} \index{整形!表示法|)}

\subsection{補數運算 \index{補數!運算|(}}

就像我們早些時候看到的,{\code add}指令執行加法操作,而{\code
sub}指令的執行減法操作。在FLAGS寄存器中的兩位能被這些指令設置,它們是:\emph{overflow(溢出位)}
和\emph{carry
flag(進位元標誌位元)}。如果操作的正確結果太大了以致於不匹配有符號數運算的目的運算元,溢出標誌位元將被置位元。如果在加法中的最高有效位有一個進位或在減法中的最高有效位有一個借位,進位元標誌位元將被置位元。因此,它可以用來檢查無符號數運算的溢出。進位元標誌位元在有符號數運算中的使用將看起來非常簡單。補數的一個最大的優點是加法和減法的規則實際上就與無符號整形的一樣。因此,
{\code add}和{\code sub}可以同時被用在有符號和無符號整形上。
\[
\begin{array}{rrcrr}
 & 002\mathrm{C} & & & 44\\
+& \mathrm{FFFF} & &+&(-1)\\ \cline{1-2} \cline{4-5}
 & 002\mathrm{B} & & & 43
\end{array}
\]
這兒有一個進位產生,但是它不是結果的一部分。

\index{整形!乘法|(} \index{MUL|(} \index{IMUL|(}
這有兩個不同的乘法和除法指令。首先,使用{\code MUL}或{\code IMUL}
指令來進行乘法運算。 {\code MUL}指令用於無符號數之間相乘,而 {\code
IMUL}指令用於有符號數之間相乘。為什麼需要兩個不同的指令呢?無符號數和有符號數補數的乘法規則是不同的。為什麼會這樣?考慮位元組FF乘以它本身產生一個字的結果。使用無符號乘法這就是255乘上255,得65025
(或十六進位的FE01)。使用有符號數乘法這就是 $-1$ 乘上 $-1$,得1
(或十六進位的0001)。

這兒有乘法指令的幾種格式。最老的格式是像這樣的:
\begin{AsmCodeListing}[numbers=none,frame=none]
      mul   source
\end{AsmCodeListing}
\emph{source}要麼是一個寄存器,要麼是一個指定的記憶體。它不可以是一個立即數。實際上,乘法怎麼執行取決於源運算元的大小。如果運算元大小是一個位元組,它乘以在AL寄存器中的位元組,而結果被儲存到了16位元寄存器AX中。如果源運算元是16位,它乘以在AX中的字,而32位元的結果被儲存到了DX:AX。如果源運算元是32位的,它乘以在EAX中的數,而結果被儲存到了EDX:EAX\index{寄存器!EDX:EAX}。
\index{MUL|)}

\begin{table}[t]
\centering
\begin{tabular}{|c|c|c|l|}
\hline { \bf dest} & { \bf source1 } & {\bf source2} &\multicolumn{1}{c|}{\bf 操作} \\ \hline
            & reg/mem8        &               & AX = AL*source1 \\
            & reg/mem16       &               & DX:AX = AX*source1 \\
            & reg/mem32       &               & EDX:EAX = EAX*source1 \\
reg16       & reg/mem16       &               & dest *= source1 \\
reg32       & reg/mem32       &               & dest *= source1 \\
reg16       & immed8          &               & dest *= immed8 \\
reg32       & immed8          &               & dest *= immed8 \\
reg16       & immed16         &               & dest *= immed16 \\
reg32       & immed32         &               & dest *= immed32 \\
reg16       & reg/mem16       & immed8        & dest = source1*source2 \\
reg32       & reg/mem32       & immed8        & dest = source1*source2 \\
reg16       & reg/mem16       & immed16       & dest = source1*source2 \\
reg32       & reg/mem32       & immed32       & dest = source1*source2 \\
\hline
\end{tabular}
\caption{{\code imul}指令 \label{tab:imul}}
\end{table}

{\code IMUL}指令擁有與{\code MUL}指令相同的格式,但是同樣增加了其他一些指令格式。這有兩個和三個運算元的格式:
\begin{AsmCodeListing}[numbers=none,frame=none]
      imul   dest (目的運算元), source1(源運算元1)
      imul   dest (目的運算元), source1(源運算元1), source2(源運算元2)
\end{AsmCodeListing}
表~\ref{tab:imul}展示可能的組合。
 \index{IMUL|)} \index{整形!乘法|)}

\index{整形!除法|(} \index{DIV} 兩個除法運算符是{\code DIV}和{\code
IDIV}。它們分別執行無符號整形和有符號整形的除法。普遍的格式是:
\begin{AsmCodeListing}[numbers=none,frame=none]
      div   source
\end{AsmCodeListing}
如果源運算元為8位,那麼AX就除以這個運算元。商儲存在AL中,而餘數儲存在AH中。如果源運算元為16位,那麼DX:AX就除以這個運算元。商儲存在AX中,而餘數儲存在DX中。如果源運算元為32位,那麼
EDX:EAX\index{寄存器!EDX:EAX}就除以這個運算元,同時商儲存在EAX中,餘數儲存在EDX中。{\code
IDIV} \index{IDIV}指令以同樣的方法進行工作。這沒有像{\code
IMUL}指令一樣的特殊的{\code IDIV}指令。
如果商太大了,以致於不匹配它的寄存器,或除數為0,那麼這個程式將被中斷和中止。一個普遍的錯誤是在進行除法之前忘記了初始化DX或EDX。
\index{整形!除法|)}

{\code NEG} \index{NEG}指令通過計算它的單一的運算元補數來得到這個運算元的相反數。它的運算元可以是任意的8位,16位或32位元寄存器或記憶體區域。

\subsection{程式例子}
\index{math.asm|(}
\begin{AsmCodeListing}[label=math.asm]
%include "asm_io.inc"
segment .data         ; 輸出字串
prompt          db    "Enter a number: ", 0
square_msg      db    "Square of input is ", 0
cube_msg        db    "Cube of input is ", 0
cube25_msg      db    "Cube of input times 25 is ", 0
quot_msg        db    "Quotient of cube/100 is ", 0
rem_msg         db    "Remainder of cube/100 is ", 0
neg_msg         db    "The negation of the remainder is ", 0

segment .bss
input   resd 1

segment .text
        global  _asm_main
_asm_main:
        enter   0,0               ; 開始運行程式
    pusha

        mov     eax, prompt
        call    print_string

        call    read_int
        mov     [input], eax

        imul    eax               ; edx:eax = eax * eax
        mov     ebx, eax          ; 保存結果到ebx中
        mov     eax, square_msg
        call    print_string
        mov     eax, ebx
        call    print_int
        call    print_nl

        mov     ebx, eax
        imul    ebx, [input]      ; ebx *= [input]
        mov     eax, cube_msg
        call    print_string
        mov     eax, ebx
        call    print_int
        call    print_nl

        imul    ecx, ebx, 25      ; ecx = ebx*25
        mov     eax, cube25_msg
        call    print_string
        mov     eax, ecx
        call    print_int
        call    print_nl

        mov     eax, ebx
        cdq                       ; 通過正負號延伸初始化edx
        mov     ecx, 100          ; 不可以被立即數除
        idiv    ecx               ; edx:eax / ecx
        mov     ecx, eax          ; 保存商到ecx中
        mov     eax, quot_msg
        call    print_string
        mov     eax, ecx
        call    print_int
        call    print_nl
        mov     eax, rem_msg
        call    print_string
        mov     eax, edx
        call    print_int
        call    print_nl

        neg     edx               ; 求這個餘數的相反數
        mov     eax, neg_msg
        call    print_string
        mov     eax, edx
        call    print_int
        call    print_nl

        popa
        mov     eax, 0            ; 返回到C中
        leave
        ret
\end{AsmCodeListing}
\index{math.asm|)}

\subsection{擴充精度運算 \label{sec:ExtPrecArith} \index{整形!擴充精度數|(}}

組合語言同樣提供允許你執行大於雙字的數的加減法的指令。 這些指令使用了進位元標誌位元。就像上面規定的,{\code ADD}
\index{ADD}和{\code SUB} \index{SUB}指令在進位元或借位產生時分別修改了進位元標誌位元。儲存在進位元標誌位元裏的資訊可以用來做大的數位的加減法,通過將這些運算元分成小的雙字(或更小) 塊。

{\code ADC} \index{ADC}和{\code SBB} \index{SBB}指令使用了進位元標誌位元裏的資訊。{\code ADC}指令執行下麵的操作:
\begin{center}
{\code \emph{operand1} = \emph{operand1} + carry flag + \emph{operand2} }
\end{center}
{\code SBB}執行下麵的操作:
\begin{center}
{\code \emph{operand1} = \emph{operand1} - carry flag - \emph{operand2} }
\end{center}
這些如何使用?考慮在EDX:EAX\index{整形!EDX:EAX}和EBX:ECX中的64位整形的總數。下面的代碼將總數儲存到EDX:EAX中:
\begin{AsmCodeListing}[frame=none]
      add    eax, ecx       ; 低32位相加
      adc    edx, ebx       ; 高32位帶以前總數的進位相加
\end{AsmCodeListing}
減法也是一樣的。下面的代碼用EDX:EAX減去EBX:ECX:
\begin{AsmCodeListing}[frame=none]
      sub    eax, ecx       ; 低32位相減
      sbb    edx, ebx       ; 高32位帶借位相減
\end{AsmCodeListing}

對於\emph{實際上}大的數字,可以使用一個迴圈(看
小節~\ref{sec:control})。對於一個求和的迴圈,對於每一次重複(替代所有的,除了第一次重複)使用{\code
ADC}指令將會非常便利。通過在迴圈開始之前使用{\code CLC}
\index{CLC}(CLear
Carry(清除進位元))指令初始化進位元標誌位元為0,可以使這個操作正確執行。如果進位元標誌位元為0,那麼{\code
ADD}和 {\code ADC}指令就沒有區別了。這個在減法中也是一樣的。
\index{整形!擴充精度數|)} \index{補數!運算|)}

\section{控制結構}
\label{sec:control} 高階語言提供高級的控制結構(\emph{例如},
\emph{if}和\emph{while}語句)來控制執行的順序。組合語言並沒有提供像這樣的複雜控制結構。它使用聲名狼藉的\emph{goto}來替代,如果使用不恰當可能會導致非常複雜的代碼。但是,它\emph{是}能夠寫出結構化的組合語言程式。基本的步驟是使用熟悉的高階語言控制結構來設計程式的邏輯,然後將這個設計翻譯成恰當的組合語言(就像一個編譯器要做的一樣)。

\subsection{比較 \index{整形!比較|(} \index{CMP|(}}
%TODO: Make a table of all the FLAG bits

\index{整形!FLAGS|(}
控制結構決定做什麼是基於資料的比較的。在組合語言中,比較的結果儲存在FLAGS寄存器中,以便以後使用。80x86提供{\code
CMP}指令來執行比較操作。FLAGS寄存器根據{\code
CMP}指令的兩個運算元的不同來設置。具體的操作是相減,然後FLAGS根據結果來設置,但是結果是\emph{不}在任何地方儲存的。如果你需要結果,可以使用SUB來代替{\code
CMP}指令。

\index{整形!無符號|(}
對於無符號整形,有兩個標誌位元(在FLAGS寄存器裏的位)
是非常重要的:零標誌位元(zero flag(ZF))
\index{寄存器!FLAGS!ZF}和進位元標誌位元(carry flag(CF))
\index{寄存器!FLAGS!CF}。 如果比較的結果是0的話,零標誌位元將置成(1)
。進位元標誌位元在減法中當作一個借位來使用。考慮這個比較:
\begin{AsmCodeListing}[frame=none, numbers=none]
      cmp    vleft, vright
\end{AsmCodeListing}
{\code
vleft~-~vright}的差別被計算出來,然後相應地設置標誌位元。如果{\code
CMP}執行後得到差別為0,即{\code
vleft~=~vright}那麼ZF就被置位了(\emph{也就是:} 1),但是CF不被置位
(\emph{也就是:} 0)。如果{\code
vleft~>~vright},那麼ZF就不被置位而且CF也不被置位(沒有借位)。如果
{\code vleft~<~vright},那麼ZF就不被置位,而CF就被置位了(有借位)。
\index{整形!無符號|)}

\index{整形!有符號|(}
對於有符號整形,有三個標誌位元非常重要:零標誌位元(zero flag
\index{寄存器!FLAGS!ZF} (ZF)),溢出標誌位元(overflow
flag\index{寄存器!FLAGS!OF}(OF))和符號標誌位元(sign
flag\index{寄存器!FLAGS!SF} (SF))。 \MarginNote{如果{\code vleft~>~vright},\\為什麼SF~=~OF?\\因為如果沒有溢出\\,那麼差別將是一\\個正確的值,而且\\肯定是非負的。因此,\\SF~=~OF~=~0。但是,\\如果有溢出,那麼差\\別將不是一個正\\確的值(而且事實上\\將會是個負數)。因此\\,SF~=~OF~=~1。}如果一個操作的結果上溢(下溢),那麼溢出標誌位元將被置位元。如果一個操作的結果為負數,那麼符號標誌位元將被置位元。如果{\code
vleft~=~vright},那麼ZF將被置位元(正好與無符號整形一樣)。 如果{\code
vleft~>~vright},那麼ZF不被設置,而且 SF~=~OF。如果{\code
vleft~<~vright},那麼ZF不被設置而且SF~$\neq$~OF。
\index{整形!有符號|)}

不要忘記其他的指令同樣會改變FLAGS寄存器,不僅僅{\code CMP}可以。
\index{CMP|)} \index{整形!比較|)} \index{整形!FLAGS|)}
\index{整形|)}

\subsection{分支指令}

分支指令可以將執行控制權轉移到一個程式的任意一點上。換言之,它們像\emph{goto}一樣運作。有兩種類型的分支:
無條件的和有條件的。一個無條件的分支就跟goto一樣,它總會產生分支。一個有條件分支可能也可能不產生分支,它取決於在FLAGS寄存器裏的標誌位元。如果一個有條件分支沒有產生分支,控制權將傳遞到下一指令。

\index{JMP|(} {\code JMP}
(\emph{jump}的簡稱)指令產生無條件分支。它唯一的參數通常是一個指向分支指向的指令的\emph{代碼標號}。彙編器和連接器將用指令的正確位址來替代這個標號。這又是一個乏味的運算元,通過這個,彙編器使得程式師的日子不好過。能認識到在{\code
JMP}指令後的指令不會被執行,除非另一條分支指令指向它,是非常重要的。

這兒有jump指令的幾個變更形式:
\begin{description}

\item[SHORT] 這個跳轉類型局限在一小範圍內。它僅僅可以在記憶體中向上或向下移動128位元組。這個類型的好處是相對於其他的,它使用較少的記憶體。它使用一個有符號位元組來儲存跳轉的\emph{位移}。位移表示向前或向後移動的位元組數(位移須加上EIP)。為了指定一個短跳轉,需在{\code JMP}指令裏的變數之前使用關鍵字{\code SHORT}。

\item[NEAR] 這個跳轉類型是無條件和有條件分支的缺省類型,它可以用來跳到一段中的任意地方。事實上,80386支援兩種類型的近跳轉。其中一個的位移使用兩個位元組。它就允許你向上或向下移動32,000個位元組。另一種類型的位移使用四個位元組,當然它就允許你移動到代碼段中的任意位置。四位元組類型是386保護模式的缺省類型。兩個位元組類型可以通過在{\code JMP}指令裏的變數之前放置關鍵字{\code WORD}來指定。

\item[FAR] 這個跳轉類型允許控制轉移到另一個代碼段。在386保護模式下,這種事情是非常鮮見的。
\end{description}

有效的代碼標號遵守與資料變數一樣的規則。代碼標號通過在代碼段裏把它們放在它們標記的聲明前面來定義它們。有一個冒號放在變數定義的地方的結尾處。這個冒號\emph{不}是名字的一部分。
\index{JMP|)}

\index{有條件分支|(}
\begin{table}[t]
\center
\begin{tabular}{|ll|}
\hline
JZ  & 如果ZF被置位元了,就分支 \\
JNZ & 如果ZF沒有被置位元了,就分支 \\
JO  & 如果OF被置位元了,就分支 \\
JNO & 如果OF沒有被置位元了,就分支 \\
JS  & 如果SF被置位元了,就分支 \\
JNS & 如果SF沒有被置位元了,就分支 \\
JC  & 如果CF被置位元了,就分支 \\
JNC & 如果CF沒有被置位元了,就分支 \\
JP  & 如果PF被置位元了,就分支 \\
JNP & 如果PF沒有被置位元了,就分支 \\
\hline
\end{tabular}
\caption{簡單條件分支 \label{tab:SimpBran} \index{JZ} \index{JNZ}
        \index{JO} \index{JNO} \index{JS} \index{JNS} \index{JC} \index{JNC}
        \index{JP} \index{JNP}}
\end{table}

條件分支有許多不同的指令。它們都使用一個代碼標號作為它們唯一的運算元。最簡單的就是看FLAGS寄存器裏的一個標誌位元來決定是否要分支。看表~\ref{tab:SimpBran}得到關於這些指令的列表。(PF是\emph{奇偶標誌位元(parity
flag)}
\index{寄存器!FLAGS!PF},它表示結果中的低8位1的位數值為奇數個或偶數個。)

下麵的虛擬碼:
\begin{Verbatim}
if ( EAX == 0 )
  EBX = 1;
else
  EBX = 2;
\end{Verbatim}
可以寫成彙編形式,如:
\begin{AsmCodeListing}[frame=none]
      cmp    eax, 0            ; 置標誌位元(如果eax - 0 = 0,ZF就被置位)
      jz     thenblock         ; 如果ZF被置位了,就跳轉到thenblock
      mov    ebx, 2            ; IF結構的ELSE部分
      jmp    next              ; 跳過IF結構中的THEN部分
thenblock:
      mov    ebx, 1            ; IF結構的THEN部分
next:
\end{AsmCodeListing}

其他比較使用在表~\ref{tab:SimpBran}裏的條件分支並不是很容易。為了舉例說明,考慮下面的虛擬碼:
\begin{Verbatim}
if ( EAX >= 5 )
  EBX = 1;
else
  EBX = 2;
\end{Verbatim}
如果EAX大於或等於5,ZF可能被置位或不置位,而SF將等於OF。這是測試這些條件的彙編代碼
(假定EAX是有符號的):
\begin{AsmCodeListing}[frame=none]
      cmp    eax, 5
      js     signon            ; 如果SF = 1,就跳轉到signon
      jo     elseblock         ; 如果OF = 1而且SF = 0,就跳轉到elseblock
      jmp    thenblock         ; 如果SF = 0而且OF = 0,就跳轉到thenblock
signon:
      jo     thenblock         ; 如果SF = 1而且OF = 1,就跳轉到thenblock
elseblock:
      mov    ebx, 2
      jmp    next
thenblock:
      mov    ebx, 1
next:
\end{AsmCodeListing}

\begin{table}
\center
\begin{tabular}{|ll|ll|}
\hline
\multicolumn{2}{|c|}{\textbf{有符號}} & \multicolumn{2}{c|}{\textbf{無符號}} \\
\hline
JE & 如果{\code vleft = vright},則分支 & JE & 如果{\code vleft = vright},則分支 \\
JNE & 如果{\code vleft $\neq$ vright},則分支 & JNE & 如果{\code vleft $\neq$ vright},則分支 \\
JL, JNGE & 如果{\code vleft < vright},則分支 & JB, JNAE & 如果{\code vleft < vright},則分支 \\
JLE, JNG & 如果{\code vleft $\leq$ vright},則分支 & JBE, JNA & 如果{\code vleft $\leq$ vright},則分支 \\
JG, JNLE & 如果{\code vleft > vright},則分支 & JA, JNBE & 如果{\code vleft > vright},則分支 \\
JGE, JNL & 如果{\code vleft $\geq$ vright},則分支 & JAE, JNB & 如果{\code vleft $\geq$ vright},則分支 \\
\hline
\end{tabular}
\caption{有符號和無符號的比較指令 \label{tab:CompBran} \index{JE}
\index{JNE}
         \index{JL} \index{JNGE} \index{JLE} \index{JNG} \index{JG} \index{JNLE} \index{JGE}
         \index{JNL}}
\end{table}

上面的代碼使用起來非常不便。幸運的是,80x86提供了額外的分支指令使這種類型的測試條件\emph{更}容易些。每個版本都分為有符號和無符號兩種。表~\ref{tab:CompBran}展示了這些指令。等於或不等於分支(JE和JNE)對於有符號和無符號整形是相同的。(事實上,JE和JZ,
JNE和JNZ基本上完全相同 。)
每個其他的分支指令都有兩個同義字。例如:看JL (jump less than)和 JNGE
(jump not greater than or equal to)。有相同的指令這是因為:
\[ x < y \Longrightarrow \mathbf{not}( x \geq y ) \]
無符號分支使用A代表\emph{大於}而B代表\emph{小於},替換了L和G。

使用這些新的指令,上面的虛擬碼可以更容易地翻譯成組合語言:
\begin{AsmCodeListing}[frame=none]
      cmp    eax, 5
      jge    thenblock
      mov    ebx, 2
      jmp    next
thenblock:
      mov    ebx, 1
next:
\end{AsmCodeListing}
\index{有條件分支|)}

\subsection{迴圈指令}

80x86提供了幾條專門為實現像\emph{for}一樣的迴圈而設計的指令。每一個這樣的指令帶有一個代碼標號作為它們唯一的運算元。
\begin{description}
\item[LOOP]
\index{迴圈}
 ECX自減,如果 ECX $\neq$ 0,分支到代碼標號指向的位址
\item[LOOPE, LOOPZ]
\index{LOOPE} \index{LOOPZ}
 ECX自減(FLAGS寄存器沒有被修改),如果
                    ECX $\neq$ 0 而且 ZF = 1,則分支
\item[LOOPNE, LOOPNZ]
\index{LOOPNE} \index{LOOPNZ}
 ECX自減(FLAGS沒有改變),如果 ECX $\neq$ 0
                      而且 ZF = 0,則分支
\end{description}

最後兩個迴圈指令對於連續的查找迴圈是非常有用的。下麵的虛擬碼:
\begin{lstlisting}[stepnumber=0]{}
sum = 0;
for( i=10; i >0; i-- )
  sum += i;
\end{lstlisting}
\noindent 可以翻譯在組合語言,如:
\begin{AsmCodeListing}[frame=none]
      mov    eax, 0          ; eax是總數(sum)
      mov    ecx, 10         ; ecx是i
loop_start:
      add    eax, ecx
      loop   loop_start
\end{AsmCodeListing}

\section{翻譯標準的控制結構}

這一小節講述在高階語言裏的標準控制結構如何應用到組合語言中。

\subsection{If語句 \index{if語句|(}}
下麵的虛擬碼:
\lstset{escapeinside=`',language=Pascal,%
}
\begin{lstlisting}[stepnumber=0]{}
if ( `條件' )
  then_block;
else
  else_block;
\end{lstlisting}
\noindent 可以像這樣被應用:
\begin{AsmCodeListing}[frame=none]
      ; 設置FLAGS的代碼
      jxx    else_block    ; 選擇xx,如果條件為假,則分支
      ; then模組的代碼
      jmp    endif
else_block:
      ; else模組的代碼
endif:
\end{AsmCodeListing}

如果沒有else語句的話,那麼{\code else\_block}分支可以用{\code endif}分支取代。
\begin{AsmCodeListing}[frame=none]
      ; 設置FLAGS的代碼
      jxx    endif          ; 選擇xx,如果條件為假,則分支
      ; then模組的代碼
endif:
\end{AsmCodeListing}
\index{if語句|)}

\subsection{While迴圈\index{while迴圈|(}}
\emph{while}迴圈是一個頂端測試迴圈:
\lstset{escapeinside=`',language=Pascal,%
}
\begin{lstlisting}[stepnumber=0]{}
while( `條件' ) {
  `循環體';
}
\end{lstlisting}
\noindent 這個可以翻譯成:
\begin{AsmCodeListing}[frame=none]
while:
      ; 基於條件的設置FLAGS的代碼
      jxx    endwhile       ; 選擇xx,如果條件為假,則分支
      ; 循環體
      jmp    while
endwhile:
\end{AsmCodeListing}
\index{while迴圈|)}

\subsection{Do while迴圈\index{do while迴圈|(}}
\emph{do while}迴圈是一個末端測試迴圈:
\lstset{escapeinside=`',language=Pascal,%
}
\begin{lstlisting}[stepnumber=0]{}
do {
  `循環體';
} while( `條件' );
\end{lstlisting}
\noindent 這個可以翻譯成:
\begin{AsmCodeListing}[frame=none]
do:
      ; 循環體
      ; 基於條件的設置FLAGS的代碼
      jxx    do          ; 選擇xx,如果條件為假,則分支
\end{AsmCodeListing}
\index{do while迴圈|)}


\begin{figure}[t]
\lstset{escapeinside=`',language=Pascal,%
}
\begin{lstlisting}[frame=tlrb]{}
  unsigned guess;   /* `當前對素數的猜測'  */
  unsigned factor;  /* `猜測數的可能的因數'  */
  unsigned limit;   /* `查找這個值以下的素數'  */

  printf("Find primes up to: ");
  scanf("%u", &limit);
  printf("2\n");    /* `把頭兩個素數當特殊的事件處理'  */
  printf("3\n");
  guess = 5;        /* `初始化猜測數' */
  while ( guess <= limit ) {
    /* `查找一個猜測數的因數' */
    factor = 3;
    while ( factor*factor < guess &&
            guess % factor != 0 )
     factor += 2;
    if ( guess % factor != 0 )
      printf("%d\n", guess);
    guess += 2;    /* `只考慮奇數' */
  }
\end{lstlisting}
\caption{}\label{fig:primec}
\end{figure}

\section{例子:查找素數}
這一小節是一個查找素數的程式。根據回憶,素數是一個只能被1和它本身整除的數。沒有公式來做這件事情。這個程式使用的基本方法是在一個給定的範圍內查找所有奇數的因數\footnote{2是唯一的偶數素數。}。
如果一個奇數沒有找到一個因數,那麼它就是一個素數。圖~\ref{fig:primec}
展示了用C寫的基本的演算法。

這是組合語言版:
\index{prime.asm|(}
\begin{AsmCodeListing}[label=prime.asm]
%include "asm_io.inc"
segment .data
Message         db      "Find primes up to: ", 0

segment .bss
Limit           resd    1               ; 查找這個值以下的素數
Guess           resd    1               ; 當前對素數的猜測

segment .text
        global  _asm_main
_asm_main:
        enter   0,0               ; 程式開始運行
        pusha

        mov     eax, Message
        call    print_string
        call    read_int             ; scanf("%u", & limit );
        mov     [Limit], eax

        mov     eax, 2               ; printf("2\n");
        call    print_int
        call    print_nl
        mov     eax, 3               ; printf("3\n");
        call    print_int
        call    print_nl

        mov     dword [Guess], 5     ; Guess = 5;
while_limit:                         ; while ( Guess <= Limit )
        mov     eax,[Guess]
        cmp     eax, [Limit]
        jnbe    end_while_limit      ; 因為數位為無符號數,所以使用jnbe

        mov     ebx, 3               ; ebx等於factor = 3;
while_factor:
        mov     eax,ebx
        mul     eax                  ; edx:eax = eax*eax
        jo      end_while_factor     ; 如果結果不匹配eax
        cmp     eax, [Guess]
        jnb     end_while_factor     ; if !(factor*factor < guess)
        mov     eax,[Guess]
        mov     edx,0
        div     ebx                  ; edx = edx:eax % ebx
        cmp     edx, 0
        je      end_while_factor     ; if !(guess % factor != 0)

        add     ebx,2                ; factor += 2;
        jmp     while_factor
end_while_factor:
        je      end_if               ; if !(guess % factor != 0)
        mov     eax,[Guess]          ; printf("%u\n")
        call    print_int
        call    print_nl
end_if:
        add     dword [Guess], 2     ; guess += 2
        jmp     while_limit
end_while_limit:

        popa
        mov     eax, 0            ; 返回到C中
        leave
        ret
\end{AsmCodeListing}
\index{prime.asm|)}


% -*-latex-*-
\chapter{Operaciones con bits}
\section{Operaciones de desplazamientos\index{operaciones con bits!desplazamientos|(}}

El lenguaje ensamblador le permite al programador manipular bits
individuales de los datos. Una operaci�n com�n es llamada un
\emph{desplazamiento}. Una operaci�n de desplazamiento mueve la posici�n
de los bits de alg�n dato. Los desplazamientos pueden  ser hacia la
izquierda (hacia el bit m�s significativo) o hacia la derecha (el bit
menos significativo).

\subsection{Desplazamientos l�gicos\index{operaciones con
bits!desplazamientos!desplazamientos l�gicos|(}}

Un desplazamiento l�gico es el tipo m�s simple de desplazamiento.
Desplaza de una manera muy directa. La Figura~\ref{fig:logshifts}
muestra un ejemplo del desplazamiento de un byte.

\begin{figure}[h]
\centering
\begin{tabular}{l|c|c|c|c|c|c|c|c|}
\cline{2-9}
Original      & 1 & 1 & 1 & 0 & 1 & 0 & 1 & 0 \\
\cline{2-9}
Desplazado a la izquierda & 1 & 1 & 0 & 1 & 0 & 1 & 0 & 0 \\
\cline{2-9}
Desplazado a la derecha & 0 & 1 & 1 & 1 & 0 & 1 & 0 & 1 \\
\cline{2-9}
\end{tabular}
\caption{Desplazamientos l�gicos\label{fig:logshifts}}
\end{figure}

Observe que los nuevos bits que entran son siempre cero. Se usan las
instrucciones {\code SHL} \index{SHL} y {\code SHR}\index{SHR} para
realizar los desplazamientos a la izquierda y derecha respectivamente.
Estas instrucciones le permiten a uno desplazar cualquier n�mero de
posiciones. El n�mero de posiciones puede ser o una constante o puede
estar almacenado en el registro {\code CL}. El �ltimo bit desplazado se
almacena en la bandera de carry. A continuaci�n, algunos ejemplos:
\begin{AsmCodeListing}[frame=none]
      mov    ax, 0C123H
      shl    ax, 1           ; desplaza un bit a la izquierda,   
                             ; ax = 8246H, CF = 1
      shr    ax, 1           ; desplaza un bit a la derecha,  
                             ; ax = 4123H, CF = 0
      shr    ax, 1           ; desplaza un bit a la derecha,  
                             ; ax = 2091H, CF = 1
      mov    ax, 0C123H
      shl    ax, 2           ; desplaza dos bit a la izquierda,  
                             ;ax = 048CH, CF = 1
      mov    cl, 3
      shr    ax, cl          ; desplaza tres bit a la derecha,
                             ; ax = 0091H, CF = 1
\end{AsmCodeListing}

\subsection{Uso de los desplazamientos}

Los usos m�s comunes de las operaciones de desplazamientos son las
multiplicaciones y divisiones r�pidas. Recuerde que en el sistema decimal
la multiplicaci�n y divisi�n por una potencia de 10 es s�lo un
desplazamiento de los d�gitos. Lo mismo se aplica para las potencias de
dos en binario. Por ejemplo para duplicar el n�mero binario $1011_2$ (o
11 en decimal), al desplazar una vez a la izquierda obtenemos $10110_2$
(o 22). El cociente de una divisi�n por una potencia de dos es el
resultado de un desplazamiento a la derecha. Para dividir por 2 solo haga
un desplazamiento a la derecha; para dividir por 4 ($2^2$) desplace los
bits 2 lugares; para dividir por 8 desplace 3 lugares a la derecha etc.
Las instrucciones de desplazamiento son muy elementales y son
\emph{mucho} m�s r�pidas que las instrucciones correspondientes {\code
MUL}\index{MUL} y {\code DIV} \index{DIV}.

Los desplazamientos l�gicos se pueden usar para multiplicar y dividir
valores sin signo. Ellos no funcionan, en general, para valores con
signo.  Considere el valor de dos bytes FFFF ($-1$). Si �ste se desplaza
l�gicamente a la derecha una vez �el resultado es 7FFF que es $+32,767$!
Se pueden usar otro tipo de desplazamientos para valores con signo.
\index{operaciones con bits!desplazamientos!desplazamiento l�gicos|)}

\subsection{Desplazamientos aritm�ticos\index{operaciones con
bits!desplazamientos!desplazamientos aritm�ticos|(}}

Estos desplazamientos est�n dise�ados para permitir que n�meros con signo
se puedan multiplicar y dividir r�pidamente por potencias de 2. Ellos
aseguran que el bit de signo se trate correctamente.
\begin{description}
\item[SAL] \index{SAL} 
(\emph{Shift aritmetic left}).  Esta instrucci�n es solo sin�nimo para
{\code SHL}. Se ensambla con el mismo c�digo de m�quina que SHL. Como el
bit de signo no se cambia por el desplazamiento, el resultado ser�
correcto.  SAR \item[SAR] \index{SAR} (\emph{Shift Arithmetic Right}).
Esta es una instrucci�n nueva que no desplaza el bit de signo (el bit m�s
significativo) de su operando. Los otros bits se desplazan como es normal
excepto que los bits nuevos que entran por la derecha son copias del bit
de signo (esto es, si el bit de signo es 1, los nuevos bits son tambi�n
1). As�, si un byte se desplaza con esta instrucci�n, s�lo los 7 bits
inferiores se desplazan. Como las otras instrucciones de desplazamiento,
el �ltimo bit que sale se almacena en la bandera de carry.
\end{description}

\begin{AsmCodeListing}[frame=none]
      mov    ax, 0C123H
      sal    ax, 1           ; ax = 8246H, CF = 1
      sal    ax, 1           ; ax = 048CH, CF = 1
      sar    ax, 2           ; ax = 0123H, CF = 0
\end{AsmCodeListing}
\index{operaciones con bits!desplazamientos!desplazamientos aritm�ticos|)}

\subsection{Desplazamientos de rotaci�n\index{operaciones con
bits!desplazamientos!rotaciones|(}}

Los desplazamientos de rotaci�n trabajan como los desplazamientos l�gicos
excepto que los bits perdidos en un extremo del dato se desplazan al otro
lado. As�, el dato es tratado como si fuera una estructura circular. Las
dos rotaciones m�s simples son  {\code ROL} \index{ROL} y {\code ROR}
\index{ROR}, que hacen rotaciones a la izquierda y a  la derecha
respectivamente. Tal como los otros desplazamientos, estos
desplazamientos dejan una copia del �ltimo bit rotado en la bandera de
carry.
\begin{AsmCodeListing}[frame=none]
      mov    ax, 0C123H
      rol    ax, 1           ; ax = 8247H, CF = 1
      rol    ax, 1           ; ax = 048FH, CF = 1
      rol    ax, 1           ; ax = 091EH, CF = 0
      ror    ax, 2           ; ax = 8247H, CF = 1
      ror    ax, 1           ; ax = C123H, CF = 1
\end{AsmCodeListing}

Hay dos instrucciones de rotaci�n adicionales que desplazan los bits en
el dato y la bandera de carry llamadas {\code RCL}\index{RCL} y {\code
RCR}.  \index{RCR}. Por ejemplo, si el registro {\code AX} rota con estas
instrucciones, los 17 bits se desplazan y la bandera de carry se rota.
\begin{AsmCodeListing}[frame=none]
      mov    ax, 0C123H
      clc                    ; borra la bandera de carry (CF = 0)
      rcl    ax, 1           ; ax = 8246H, CF = 1
      rcl    ax, 1           ; ax = 048DH, CF = 1
      rcl    ax, 1           ; ax = 091BH, CF = 0
      rcr    ax, 2           ; ax = 8246H, CF = 1
      rcr    ax, 1           ; ax = C123H, CF = 0
\end{AsmCodeListing}
\index{operaciones con bits!desplazamientos!rotaciones|)}

\subsection{Aplicaci�n simple\label{sect:AddBitsExample}}

A continuaci�n est� un fragmento de  c�digo que cuenta el n�mero de bits
que est�n ``encendidos'' (1) en el registro EAX. 
%TODO: show how the ADC instruction could be used to remove the jnc
\begin{AsmCodeListing}
      mov    bl, 0           ; bl contendr� el n�mero de bits prendidos
      mov    ecx, 32         ; ecx es el contador del bucle
count_loop:
      shl    eax, 1          ; desplaza los bits en la bandera de carry
      jnc    skip_inc        ; si CF == 0, va a skip_inc
      inc    bl
skip_inc:
      loop   count_loop
\end{AsmCodeListing}
El c�digo anterior destruye el valor original de {\code EAX} ({\code EAX}
es cero al final del bucle). Si uno desea conservar el valor de EAX, la
l�nea~4 deber�a ser reemplazada con {\code rol eax,1}.
\index{operaciones con bits!desplazamientos|)}

\section{Operaciones booleanas entre bits}

Hay cuatro operadores booleanos b�sicos \emph{AND}, \emph{OR}, \emph{XOR}
y \emph{NOT}. Una \emph{tabla de verdad} muestra el resultado de cada
operaci�n por cada posible valor de los operandos.

\subsection{La operaci�n \emph{AND} \index{operaciones con bits!AND}}

\begin{table}[t]
\centering
\begin{tabular}{|c|c|c|}
\hline
\emph{X} & \emph{Y} & \emph{X} AND \emph{Y} \\
\hline \hline
0 & 0 & 0 \\
0 & 1 & 0 \\
1 & 0 & 0 \\
1 & 1 & 1 \\
\hline
\end{tabular}
\caption{La operaci�n AND \label{tab:and} \index{AND}}
\end{table}

El resultado del \emph{AND} de dos bits es 1 solo si ambos bits son 1, si
no el resultado es cero como muestra el Cuadro~\ref{tab:and}

\begin{figure}[t]
\centering
\begin{tabular}{rcccccccc}
    & 1 & 0 & 1 & 0 & 1 & 0 & 1 & 0 \\
AND & 1 & 1 & 0 & 0 & 1 & 0 & 0 & 1 \\
\hline
    & 1 & 0 & 0 & 0 & 1 & 0 & 0 & 0
\end{tabular}
\caption{ANDdo un byte \label{fig:and}}
\end{figure}

Los procesadores tienen estas operaciones como instrucciones que act�an
independientemente en todos los bits del dato en paralelo. Por ejemplo,
si el contenido de {\code AL} y {\code BL} se les opera con \emph{AND},
la operaci�n se aplica a cada uno de los 8 pares de bits correspondientes
en los dos registros como muestra la Figura~\ref{fig:and}. A continuaci�n
un c�digo de ejemplo:
\begin{AsmCodeListing}[frame=none]
      mov    ax, 0C123H
      and    ax, 82F6H          ; ax = 8022H
\end{AsmCodeListing}

\subsection{La operaci�n \emph{OR} \index{operaciones con bits!OR}}

\begin{table}[t]
\centering
\begin{tabular}{|c|c|c|}
\hline
\emph{X} & \emph{Y} & \emph{X} OR \emph{Y} \\
\hline \hline
0 & 0 & 0 \\
0 & 1 & 1 \\
1 & 0 & 1 \\
1 & 1 & 1 \\
\hline
\end{tabular}
\caption{La operaci�n OR\label{tab:or} \index{OR}}
\end{table}

El \emph{O} inclusivo entre dos bits es 0 solo si ambos bits son 0, si no
el resultado es 1 como se muestra  en el Cuadro~\ref{tab:or} . A
continuaci�n un c�digo de ejemplo:

\begin{AsmCodeListing}[frame=none]
      mov    ax, 0C123H
      or     ax, 0E831H          ; ax = E933H
\end{AsmCodeListing}

\subsection{La operaci�n \emph{XOR} \index{operaciones con bits!XOR}}

\begin{table}
\centering
\begin{tabular}{|c|c|c|}
\hline
\emph{X} & \emph{Y} & \emph{X} XOR \emph{Y} \\
\hline \hline
0 & 0 & 0 \\
0 & 1 & 1 \\
1 & 0 & 1 \\
1 & 1 & 0 \\
\hline
\end{tabular}
\caption{La operaci�n XOR \label{tab:xor}\index{XOR}}
\end{table}

El \emph{O} exclusivo entre 2 bits es 0 si y solo si ambos bits son
iguales, sino el resultado es 1 como muestra el Cuadro~\ref{tab:xor}.
Sigue un c�digo de ejemplo:

\begin{AsmCodeListing}[frame=none]
      mov    ax, 0C123H
      xor    ax, 0E831H          ; ax = 2912H
\end{AsmCodeListing}

\subsection{La operaci�n \emph{NOT} \index{operciones con bits!NOT}}

\begin{table}[t]
\centering
\begin{tabular}{|c|c|}
\hline
\emph{X} & NOT \emph{X} \\
\hline \hline
0 & 1 \\
1 & 0 \\
\hline
\end{tabular}
\caption{La operaci�n NOT \label{tab:not}\index{NOT}}
\end{table}

La operaci�n \emph{NOT} es \emph{unaria} (act�a sobre un solo operando,
no como las operaciones \emph{binarias} como \emph{AND}).  El \emph{NOT}
de un bit es el valor opuesto del bit como se muestra en el
Cuadro~\ref{tab:not}. Sigue un c�digo de ejemplo:

\begin{AsmCodeListing}[frame=none]
      mov    ax, 0C123H
      not    ax                 ; ax = 3EDCH
\end{AsmCodeListing}

Observe que \emph{NOT} halla el complemento a 1. A diferencia de las
otras operaciones entre bits, la instrucci�n {\code NOT} no cambian
ninguno de los bits en el registro {\code FLAGS}.

\subsection{La instrucci�n {\code TEST} \index{TEST}}

La instrucci�n {\code TEST} realiza una operaci�n \emph{AND}, pero no
almacena el resultado. Solo fija las banderas del registro {\code FLAGS}
dependiendo del resultado obtenido (muy parecido a lo que hace la
instrucci�n {\code CMP} con la resta que solo fija las banderas). Por
ejemplo, si el resultado fuera cero, {\code ZF} se fijar�a.

\begin{table}
\begin{tabular}{lp{3in}}
Prende el bit \emph{i} & \emph{OR} el n�mero con $2^i$ (es el n�mero con
�nicamente el bit \emph{i}-�simo prendido) \\
Apaga el bit \emph{i} & \emph{AND} el n�mero binario con s�lo el bit
\emph{i} apagado . Este operando es a menudo llamado \emph{m�scara} \\
Complementa el bit \emph{i} & \emph{XOR} el n�mero con $2^i$
\end{tabular}
\caption{Usos de las operaciones booleanas \label{tab:bool}}
\end{table}

\subsection{Usos de las operaciones con bits\index{operaciones con bits!ensamblador|(}}

Las operaciones con bits son muy �tiles para manipular bits individuales
sin modificar los otros bits. El Cuadro~\ref{tab:bool}  muestra los 3
usos m�s comunes de estas operaciones. Sigue un ejemplo de c�mo
implementar estas ideas.
\begin{AsmCodeListing}[frame=none]
      mov    ax, 0C123H
      or     ax, 8           ; prende el bit 3,   ax = C12BH
      and    ax, 0FFDFH      ; apaga el bit 5,  ax = C10BH
      xor    ax, 8000H       ; invierte el bit 31,   ax = 410BH
      or     ax, 0F00H       ; prende el nibble,  ax = 4F0BH
      and    ax, 0FFF0H      ; apaga nibble, ax = 4F00H
      xor    ax, 0F00FH      ; invierte nibbles,  ax = BF0FH
      xor    ax, 0FFFFH      ; complemento a uno ,  ax = 40F0H
\end{AsmCodeListing}

La operaci�n \emph{AND} se puede usar tambi�n para hallar el residuo de
una divisi�n por una potencia de dos. Para encontrar el residuo de una
divisi�n  por $2^i$, efect�a un AND entre el dividendo y una m�scara
igual a $2^i-1$,\emph{AND} el n�mero con una m�scara igual a $2^i - 1$.
Esta m�scara contendr� unos desde el bit 0 hasta el bit $i-1$. Son solo
estos bits los que contienen el residuo. El resultado de la \emph{AND}
conservar� estos bits y dejar� cero los otros. A continuaci�n un
fragmento de  c�digo que encuentra el cociente y el residuo de la
divisi�n de 100 por 16.
\begin{AsmCodeListing}[frame=none]
      mov    eax, 100        ; 100 = 64H
      mov    ebx, 0000000FH  ; m�cara = 16 - 1 = 15 or F
      and    ebx, eax        ; ebx = residuo = 4
      shr    eax, 4          ; eax = cociente de eax/2^4 = 6
\end{AsmCodeListing}
Usando el registro {\code CL} es posible modificar arbitrariamente bits.
El siguiente es un ejemplo que fija (prende) un bit arbitrario en {\code
EAX}. El n�mero del bit a prender se almacena en {\code BH}.
\begin{AsmCodeListing}[frame=none]
      mov    cl, bh          ; 
      mov    ebx, 1
      shl    ebx, cl         ; se desplaza a la derecha cl veces
      or     eax, ebx        ; prende el bit
\end{AsmCodeListing}
Apagar un bit es solo un poco m�s dif�cil.
\begin{AsmCodeListing}[frame=none]
      mov    cl, bh          ; 
      mov    ebx, 1
      shl    ebx, cl         ; se desplaza a la derecha cl veces
      not    ebx             ; invierte los bits
      and    eax, ebx        ; apaga el bit
\end{AsmCodeListing}
El c�digo para complementar un bit arbitrario es dejado como ejercicio al 
lector.

Es com�n ver esta instrucci�n en un programa 80x86.
\begin{AsmCodeListing}[frame=none,numbers=none]
      xor    eax, eax         ; eax = 0
\end{AsmCodeListing}
Un n�mero XOR con sigo mismo, el resultado es siempre cero. Esta
instrucci�n se usa porque su c�digo de m�quina es m�s peque�o que la
instrucci�n MOV equivalente. 

\index{operaciones con bits!ensamblador|)}

\begin{figure}[t]
\begin{AsmCodeListing}
      mov    bl, 0           ; bl contendr� el n�mero de bits prendidos
      mov    ecx, 32         ; ecx es el bucle contador
count_loop:
      shl    eax, 1          ; se desplaza el bit en la bandera de carry
      adc    bl, 0           ; a�ade solo la bandera de carry a bl
      loop   count_loop
\end{AsmCodeListing}
\caption{Contando bits con {\code ADC}\label{fig:countBitsAdc}}
\end{figure}

\section{Evitando saltos condicionales}
\index{predicci�n de ramificaciones|(} 

Los procesadores modernos usan t�cnicas muy sofisticadas para ejecutar el
c�digo  tan r�pido como sea posible. Una t�cnica com�n se conoce como
\emph{ejecuci�n especulativa}\index{ejecuci�n especulativa}.  Esta
t�cnica usa las capacidades de procesamiento paralelo  de la CPU para
ejecutar m�ltiples instrucciones a la vez. Las instrucciones
condicionales tienen un problema con esta idea. El procesador, en
general, no sabe si se realiza o no la ramificaci�n.Si se efect�a, se
ejecutar� un conjunto de instrucciones diferentes que si no se efect�a
(el salto). El procesador trata de predecir si ocurrir� la ramificaci�n o
no. Si la predicci�n fue err�nea el procesador ha perdido su tiempo
ejecutando un c�digo err�neo.

\index{predicci�n de ramificaciones|)}

Una manera de evitar este problema, es evitar usar ramificaciones
condicionales cuando es posible. El c�digo de ejemplo en
\ref{sect:AddBitsExample} muestra un programa muy simple donde uno podr�a
hacer esto. En el ejemplo anterior,  los bits ``encendidos'' del registro
EAX se cuentan. Usa una ramificaci�n para saltarse la instrucci�n {\code
INC}. La figura~\ref{fig:countBitsAdc} muestra c�mo se puede quitar la
ramificaci�n usando la instrucci�n {\code ADC}\index{ADC} para sumar el
bit de carry directamente.

Las instrucciones {\code SET\emph{xx}}\index{SET\emph{xx}} suministran
una manera de suprimir ramificaciones en ciertos casos. Esta instrucci�n
fija el valor de un registro o un lugar de memoria de 8 bits  a cero,
basado en el estudio del registro FLAGS. Los caracteres luego de {\code
SET} son los mismos caracteres usados en los saltos condicionales. Si la
condici�n correspondiente de {\code SET\emph{xx}} es verdadero, el
resultado almacenado es uno, si es falso se almacena cero. Por ejemplo,
\begin{AsmCodeListing}[frame=none,numbers=none]
      setz   al        ; AL = 1 if Z flag is set, else 0
\end{AsmCodeListing}
Usando estas instrucciones, uno puede desarrollar algunas t�cnicas
ingeniosas que calculan valores sin ramificaciones.

Por ejemplo, considere el problema de encontrar el mayor de dos valores.
La aproximaci�n normal para resolver este problema ser�a el uso de {\code
CMP} y usar un salto condicional y proceder con el valor que fue m�s
grande. El programa de ejemplo de abajo muestra c�mo se puede encontrar
el mayor sin ninguna ramificaci�n.


\begin{AsmCodeListing}
; Archivo: max.asm
%include "asm_io.inc"
segment .data

message1 db "Digite un n�mero: ",0
message2 db "Digite otro n�mero: ", 0
message3 db "El mayor n�mero es: ", 0

segment .bss

input1  resd    1        ; primer n�mero ingresado 

segment .text
        global  _asm_main
_asm_main:
        enter   0,0               ; 
        pusha

        mov     eax, message1     ; imprime el primer mensaje
        call    print_string
        call    read_int          ; ingresa el primer n�mero
        mov     [input1], eax

        mov     eax, message2     ; imprime el segundo mensaje
        call    print_string
        call    read_int          ; ingresa el segundo n�mero (en  eax)

        xor     ebx, ebx          ; ebx = 0
        cmp     eax, [input1]     ; compara el segundo y el primer n�mero
        setg    bl                ; ebx = (input2 > input1) ?          1 : 0
        neg     ebx               ; ebx = (input2 > input1) ? 0xFFFFFFFF : 0
        mov     ecx, ebx          ; ecx = (input2 > input1) ? 0xFFFFFFFF : 0
        and     ecx, eax          ; ecx = (input2 > input1) ?     input2 : 0
        not     ebx               ; ebx = (input2 > input1) ?          0 : 0xFFFFFFFF
        and     ebx, [input1]     ; ebx = (input2 > input1) ?          0 : input1
        or      ecx, ebx          ; ecx = (input2 > input1) ?     input2 : input1

        mov     eax, message3     ; imprime los resultado
        call    print_string
        mov     eax, ecx
        call    print_int
        call    print_nl

        popa
        mov     eax, 0            ; retorna a C
        leave                     
        ret
\end{AsmCodeListing}

El truco es crear una m�scara de bits que se pueda usar para seleccionar
el valor mayor.  La instrucci�n {\code SETG}\index{SETG} en la l�nea 30
fija BL a 1. Si la segunda entrada es mayor o 0 en otro caso. Esta no es
la m�scara deseada. Para crear la m�scara de bits requerida la l�nea 31
usa la instrucci�n {\code NEG}\index{NEG} en el registro EBX. (Observe
que se borr� EBX primero). Si EBX es 0 no hace nada; sin embargo si EBX
es 1, el resultado es la representaci�n en complemento a dos de -1 o
0xFFFFFFFF. Esta es la m�scara que se necesita.  El resto del c�digo usa
esta m�scara para seleccionar la entrada correcta como e la mayor.

Un truco alternativo es usar la instrucci�n {\code DEC}. En el c�digo de
arriba, si NEG se reemplaza con un {\code DEC}, de nuevo el resultado
ser� 0 o 0xFFFFFFFF.  Sin embargo, los valores son invertidos que cuando
se usa la instrucci�n {\code NEG}.

\section{Manipulando bits en C\index{operaciones con bits!C|(}}

\subsection{Las operacones entre bits de C}

A diferencia de algunos lenguajes de alto nivel C suministra operadores
para operaciones entre bits. La operaci�n {\code AND} se representa con
el operador {\code \&}\footnote{�Este operador es diferente del operador
binario \&\&  y del unario \&!}.  La operaci�n \emph{OR} se representa
por el operador binario {\code |}.  La operaci�n \emph{XOR} se representa
con el operador binario {\code \verb|^| }. Y la operaci�n \emph{NOT} se
representa con el operador unario {\code \verb|~| }.

Los desplazamientos son realizados por C con los operadores binarios
{\code \verb|<<| } y {\code \verb|>>| }. El operador {\code \verb|<<| }
realiza desplazamientos a la izquierda y el operador {\code \verb|>>| }
hace desplazamientos a la derecha. Estos operadores toman 2 operandos. El
de la derecha es el valor a desplazar y el de la izquierda es el n�mero
de bits a desplazar. Si el valor a desplazar es un tipo sin signo, se
realiza un desplazamiento l�gico. Si el valor es con signo (como {\code
int}), entonces se usa un desplazamiento aritm�tico a continuaci�n, un
ejemplo en C del uso de estos operadores:
\begin{lstlisting}{}
short int s;          /* se asume que short int es de 16 bits */
short unsigned u;
s = -1;               /* s = 0xFFFF (complemento a dos) */
u = 100;              /* u = 0x0064 */
u = u | 0x0100;       /* u = 0x0164 */
s = s & 0xFFF0;       /* s = 0xFFF0 */
s = s ^ u;            /* s = 0xFE94 */
u = u << 3;           /* u = 0x0B20 (desplazamiento l�gico) */
s = s >> 2;           /* s = 0xFFA5 (desplazamiento aritm�tico) */
\end{lstlisting}

\subsection{Usando las operaciones entre bits en C}

Los operadores entre bits se usan en C para los mismos prop�sitos que en
lenguaje ensamblador. Ellos le permiten a uno manipular bits individuales
y se pueden usar para multiplicaciones y divisiones r�pidas. De hecho, un
compilador de C inteligente usar� desplazamientos autom�ticamente para
multiplicaciones como {\code X*=2},  
\begin{table}
\centering
\begin{tabular}{|c|l|}
\hline
Macro & \multicolumn{1}{c|}{Meaning} \\
\hline \hline
{\code S\_IRUSR} & el propietario puede leer \\
{\code S\_IWUSR} & el propietario puede escribir \\
{\code S\_IXUSR} & el propietario puede ejecutar \\
\hline
{\code S\_IRGRP} & el grupo de propietario puede leer \\
{\code S\_IWGRP} & el grupo del propietario puede escribir \\
{\code S\_IXGRP} & el grupo del propietario puede ejecutar\\
\hline
{\code S\_IROTH} & los otros pueden leer \\
{\code S\_IWOTH} & los otros pueden escribir \\
{\code S\_IXOTH} & los otros pueden ejecutar \\
\hline
\end{tabular}
\caption{Macros POSIX para permisos de archivos \label{tab:posix}}
\end{table}
Muchos API\footnote{Aplication Programming Interface} (Como \emph{POSIX}
\footnote{Significa Portable Operatting System Interface for Computer
Enviroments. Una norma desarrollada por el IEEE basado en UNIX.} y Win
32).  tienen funciones que usan operandos que tienen datos codificados
como bits. Por ejemplo, los sistemas POSIX mantienen los permisos de los
archivos para 3 tipos diferentes de usuarios (un mejor nombre ser�a
\emph{propietario}), \emph{grupo} y \emph{otros}.  A cada tipo de usuario
se le puede  conceder permisos para leer, escribir o ejecutar un archivo.
Para cambiar los permisos de un archivo requiere que el programador de C
manipule bits individuales. POSIX define varios macros para ayudar (vea
el Cuadro~\ref{tab:posix}).  La funci�n {\code chmod} se puede usar para
establecer los permisos de un archivo.  Esta funci�n toma dos par�metros,
una cadena con el nombre del archivo sobre el que se va a actuar y un
entero\footnote{Actualmente un par�metro de tipo {\code mode\_t} que es
un typedef a un tipo integral.} Con los bits apropiados  para los
permisos deseados. Por ejemplo, el c�digo de abajo fija los permisos para
permitir que el propietario del archivo leerlo y escribirlo, a los
usuarios en e, grupo leer en archivo y que los otros no tengan acceso.
\begin{lstlisting}[stepnumber=0]{}
chmod("foo", S_IRUSR | S_IWUSR | S_IRGRP );
\end{lstlisting}

La funci�n POSIX {\code stat} se puede usar para encontrar los bits de
permisos actuales de un archivo. Usada con la funci�n {\code chmod}, es
posible modificar algunos de los permisos sin cambiar los otros. A
continuaci�n un ejemplo que quita el acceso de la escritura a los otros y
a�ade el accesode lectura. Los otros permisos no son alterados Los otros
permisos no se alteran.
\begin{lstlisting}{}
struct stat file_stats;    /* estructura usada por stat() */
stat("foo", &file_stats);  /* lee la informaci�n del archivo. 
                              file_stats.st_mode holds permission bits */
chmod("foo", (file_stats.st_mode & ~S_IWOTH) | S_IRUSR);
\end{lstlisting}
\index{operaciones con bits!C|)}

\section{Representaciones Littel Endian y Big Endian\index{endianess|(}}

El Cap�tulo~1 introdujo el concepto de las representaciones big y little
endian de datos multibyte. Sin embargo, el autor ha encontrad que este
tema confunde a muchas personas. Esta secci�n cubre el t�pico con m�s
detalle.

El lector recordar� que  lo endian se refiere al orden en que los bytes
individuales se almacenan en memoria (\emph{no} bits) de un elemento
multibyte se almacena en memoria. Big endian es el m�todo m�s directo.
Almacena el byte m�s significativo primero, luego el siguiente byte en
peso y as� sucesivamente. En otras palabras los bits de \emph{m�s peso}
se almacenan primero. Little endian almacena los bytes en el orden
contrario (primero el menos significativo). La familia de procesadores
X86 usa la representaci�n little endian.

Como un ejemplo, considere la palabra doble $12345678_{16}$. En la
representaci�n big endian, los bytes se almacenar�an como 12~34~56~78. En
la representaci�n little endian los bytes se almacenar�an como
78~56~34~12.

El lector probablemente se preguntar� as� mismo, �por qu� cualquier
dise�ador sensato de circuitos integrados usar�a la representaci�n little
endian? �Eran los ingenieros de Intel s�dicos para infligir esta confusa
representaci�n a multitud de programadores? Parecer�a que la CPU no tiene
que hacer trabajo extra para almacenar los bytes hacia atr�s en la
memoria (e invertirlos cuando los lee de la memoria). La respuesta es que
la CPU no hace ning�n trabajo extra para leer y escribir en la memoria
usando el formato little endian. Uno sabe que la CPU est� compuesta de
muchos circuitos electr�nicos que simplemente trabajan con bits. Los bits
(y bytes) no est�n en un orden en particular en la CPU.

Considere el registro de 2 bytes {\code AX}. Se puede descomponer en
registros de un byte ({\code AH} y {\code AL}). Hay circuitos en la CPU
que mantienen los valores de {\code AH} y {\code AL}. Los circuitos no
est�n en un orden particular en la CPU. Esto significa que los circuitos
para {\code AH} no est�n antes o despu�s que los circuitos para {\code
AL}.Una instrucci�n mov que y el valor de AX en memoria  el valor de Al y
luego AH Quiere decir que, no es dif�cil para la CPU hacer que almacene
{\code AH} primero.

\begin{figure}[t]
\begin{lstlisting}[stepnumber=0,frame=tblr]{}
  unsigned short word = 0x1234;   /* se asume sizeof(short) == 2 */
  unsigned char * p = (unsigned char *) &word;

  if ( p[0] == 0x12 )
    printf("M�quina Big Endian\n");
  else
    printf("M�quina Little Endian\n");
\end{lstlisting}
\caption{C�mo determinar lo Endianness \label{fig:determineEndian}}
\end{figure}

El mismo argumento se aplica a los bits individuales dentro de un byte,
no hay realmente ning�n orden en los circuitos de la CPU (o la memoria).
Sin embargo, ya que los bits individuales no se  pueden direccionar
directamente en la CPU o en la memoria, no hay manera de conocer que
orden  parece que conservaran internamente en la CPU.


El c�digo en C  en la Figura~\ref{fig:determineEndian} muestra c�mo se
puede determinar lo endian de una CPU. El apuntador  \lstinline|p| trata
la variable  \lstinline|word| como dos elementos de un arreglo de
caracteres. As�,  \lstinline|p[0]| eval�a el primer byte de
\lstinline|word| en la memoria que depende de lo endian en la CPU.

\subsection{Cuando tener cuidado con Little and Big Endian}

Para la programaci�n t�pica, lo endian de la CPU no es importante. La
mayor�a de las veces esto es importante cuando se transfiere datos
binarios entre sistemas de c�mputo diferente. Esto es ocurre normalmente
usando un medio de datos f�sico (como un disco) o una red.
\MarginNote{Ahora con los conjuntos de caracteres multibyte como UNICODE
\index{UNICODE}, lo endian es importante a�n para texto. UNICODE soporta
ambos tipos de representaci�n y tiene un mecanismo para especificar cu�l
se est� usando para representar los datos.} Ya que el c�digo ASCII es de
1 byte la caracter�stica endian no le es importante.

Todos los encabezados internos de TCP/IP almacena enteros en big Endian
(llamado \emph{orden de byte de la red}). Y las bibliotecas de TCP/IP
suministran funciones de C para tratar la cuesti�n endian de una manera
port�til. Por ejemplo la funci�n \lstinline|htonl()| convierte una
palabra doble ( o long int) del formato del \emph{host} al de \emph{red}.
La funci�n \lstinline|ntohl()| hace la transformaci�n
inversa.\footnote{Ahora, invertir lo endian de un entero simplemente
coloca al rev�s los bytes; as� convertir de big a little o de little a
big es la misma operaci�n. As�, ambas funciones hacen la misma cosa.}
Para un sistema big endian, las dos funciones s�lo retornan su entrada
sin cambio alguno. Esto le permite a uno escribir programas  de red que
compilar�n y se ejecutar�n correctamente en cualquier sistema sin
importar lo endian. Para m�s informaci�n sobre lo endian  programaci�n de
redes vea el excelente libro de W. Richard Steven \emph{UNIX Network
Programming}.

\begin{figure}[t]
\begin{lstlisting}[frame=tlrb]{}
unsigned invert_endian( unsigned x )
{
  unsigned invert;
  const unsigned char * xp = (const unsigned char *) &x;
  unsigned char * ip = (unsigned char *) & invert;

  ip[0] = xp[3];   /* invierte los bytes individuales */
  ip[1] = xp[2];
  ip[2] = xp[1];
  ip[3] = xp[0];

  return invert;   /* retorna los bytes invertidos */
}
\end{lstlisting}
\caption{Funci�n invert\_endian \label{fig:invertEndian}\index{endianess!invert\_endian}}
\end{figure}

La Figura~\ref{fig:invertEndian} muestra una funci�n de C que invierte lo
endian de una palabra doble. El 486 ha introducido una nueva instrucci�n
de m�quina llamada {\code BSWAP}\index{BSWAP} que invierte los bytes de
cualquier registro de 32 bits. Por ejemplo:
\begin{AsmCodeListing}[frame=none,numbers=none]
      bswap   edx          ; intercambia los bytes de edx
\end{AsmCodeListing}
Esta instrucci�n no se puede usar en los registros de 16 bits, sin
embargo la instrucci�n {\code XCHG}\index{XCHG} se puede usar para
intercambiar los bytes en los registros de 16 bits que se pueden
descomponer en registros de 8 bits. Por ejemplo:
\begin{AsmCodeListing}[frame=none,numbers=none]
      xchg    ah,al        ; intercambia los bytes de ax
\end{AsmCodeListing}
\index{endianess|)}

\section{Contando bits\index{contando bits|(}}

Al principio se dio una t�cnica directa para contar el n�mero de bits que
est�n ``encendidos'' en una palabra doble. Esta secci�n mira otros
m�todos menos directos de hacer esto, como un ejercicio que usa las
operaciones de bits discutidas en este cap�tulo.

\begin{figure}[t]
\begin{lstlisting}[frame=tblr]{}
int count_bits( unsigned int data )
{
  int cnt = 0;

  while( data != 0 ) {
    data = data & (data - 1);
    cnt++;
  }
  return cnt;
}
\end{lstlisting}
\caption{Contando bits: m�todo uno \label{fig:meth1}}
\end{figure}

\subsection{M�todo uno\index{contando bits!m�todo uno|(}}

El primer m�todo es muy simple, pero no obvio. La figura~\ref{fig:meth1}
muestra el c�digo.

�C�mo trabaja este m�todo? En cada iteraci�n el bucle, se apaga un bit de
data {\code dato}. Cuando todos los bits se han apagado (cuando el {\code
dato} es cero), el bucle finaliza. El n�mero de iteraciones requerido
para hacer el {\code dato} cero es igual al n�mero de bits en el valor
original de data {\code data}.

La l�nea~6 es donde se apaga un bit del {\code dato}. �C�mo se hace esto?
Considere la forma general de la representaci�n en binario del {\code
dato} y el 1 del extremo derecho de esta representaci�n. Por definici�n
cada bit despu�s de este 1 debe ser cero. Ahora, �Cu�l ser�  la
representaci�n de {\code data -1}? Los bits a la izquierda del 1 del
extremo derecho ser�n los mismos que para {\code data}, pero en el punto
del 1 del extremo derecho ellos ser�n el  complemento de los bits
originales de {\code data}. Por ejemplo:\\
\begin{tabular}{lcl}
{\code data}     & = & xxxxx10000 \\
{\code data - 1} & = & xxxxx01111
\end{tabular}\\
donde X es igual para ambos n�meros. Cuando se hace {\code data}
\emph{AND} {\code data -1}, el resultado ser� cero el 1 del extremo
derecho en {\code data} y deja todos los otros bits sin cambio.

\begin{figure}[t]
\begin{lstlisting}[frame=tlrb]{}
static unsigned char byte_bit_count[256];  /* lookup table */

void initialize_count_bits()
{
  int cnt, i, data;

  for( i = 0; i < 256; i++ ) {
    cnt = 0;
    data = i;
    while( data != 0 ) {	/* m�todo uno */
      data = data & (data - 1);
      cnt++;
    }
    byte_bit_count[i] = cnt;
  }
}

int count_bits( unsigned int data )
{
  const unsigned char * byte = ( unsigned char *) & data;

  return byte_bit_count[byte[0]] + byte_bit_count[byte[1]] +
         byte_bit_count[byte[2]] + byte_bit_count[byte[3]];
}
\end{lstlisting}
\caption{M�todo dos \label{fig:meth2}}
\end{figure}
\index{contando bits!m�todo uno|)}

\subsection{M�todo dos\index{contando bits!m�todo dos|(}}

Una b�squeda en una tabla se puede usar para contar bits de una palabra
doble arbitraria. La aproximaci�n directa ser�a precalcular el n�mero de
bits para cada palabra doble y almacenar esto en un arreglo. Sin embargo,
hay dos problemas relacionados con esta aproximaci�n. �Hay alrededor de
\emph{4 mil millones} de palabras dobles! Esto significa que el arreglo
ser� muy grande e iniciarlo consumir�a mucho tiempo. (de hecho, a menos
que uno vaya a utilizar realmente el arreglo de m�s que 4 mil millones de
veces, se tomar� m�s tiempo iniciando el arreglo que el que se requerir�a
para calcular la cantidad de bits usando el m�todo uno).

Un m�todo m�s realista  ser�a precalcular la cantidad de bits para todos
los valores posibles de un byte y almacenar esto en un arreglo. Entonces
la palabra doble se puede dividir en 4 bytes. Se hallan los b y se suman
para encontrar la cantidad de bits de la palabra doble original. La
figura~\ref{fig:meth2} muestra la implementaci�n de esta aproximaci�n.

La funci�n {\code initialize\_count\_bits} debe ser llamada antes, del
primer llamado a la funci�n {\code count\_bits} . Esta funci�n inicia el
arreglo global {\code byte\_bit\_count}. La funci�n {\code count\_bits}
mira la variable {\code data} no como una palabra doble, sino como un
arreglo de 4 bytes. El apuntador {\code dword} act�a como un apuntador
a este arreglo de 4 bytes. As� {\code dword [0]} es uno de los bytes en
{\code data} ( el byte menos significativo o el m�s significativo
dependiendo si es little o big endian respectivamente). Claro est� uno
podr�a usar una instrucci�n como:
\begin{lstlisting}[stepnumber=0]{}
(data >> 24) & 0x000000FF
\end{lstlisting}
\noindent Para encontrar el byte m�s significativo y hacer algo parecido
con los otros bytes; sin embargo, estas construcciones ser�n m�s lentas
que una referencia al arreglo.

Un �ltimo punto, se podr�a usar f�cilmente un bucle {\code for} para
calcular la suma en las l�neas~22 y 23. Pero el bucle {\code for}
incluir�a el trabajo extra de iniciar el �ndice del bucle, comparar el
�ndice luego de cada iteraci�n e incrementar el �ndice. Calcular la suma
como la suma expl�cita de 4 valores ser� m�s r�pido. De hecho un
compilador inteligente podr�a convertir la versi�n del bucle {\code for}
a la suma expl�cita. Este proceso de reducir o eliminar iteraciones de
bucles es una t�cnica de optimizaci�n conocida como \emph{loop
unrolling}.
\index{contando bits!m�todo dos|)}

\subsection{M�todo tres\index{contando bits!m�todo tres|(}}

\begin{figure}[t]
\begin{lstlisting}[frame=tlrb]{}
int count_bits(unsigned int x )
{
  static unsigned int mask[] = { 0x55555555,
                                 0x33333333,
                                 0x0F0F0F0F,
                                 0x00FF00FF,
                                 0x0000FFFF };
  int i;
  int shift;   /* n�mero de posiciones a desplazarse a la derecha */

  for( i=0, shift=1; i < 5; i++, shift *= 2 )
    x = (x & mask[i]) + ( (x >> shift) & mask[i] );
  return x;
}
\end{lstlisting}
\caption{M�todo tres \label{fig:method3}}
\end{figure}

Hay otro m�todo ingenioso de contar bits que est�n en un dato. Este
m�todo literalmente a�ade los unos y ceros del dato unido. Esta suma debe
ser igual al n�mero de unos en el dato. Por ejemplo considere calcular
los unos en un byte almacenado en una variable llamada {\code data}. El
primer paso es hacer la siguiente operaci�n:
\begin{lstlisting}[stepnumber=0]{}
data = (data & 0x55) + ((data >> 1) & 0x55);
\end{lstlisting}
�Qu� hace esto? La constante hexadecimal {\code 0X55} es $01010101$ en
binario. En el primer operando de la suma {\code data} es \emph{AND} con
�l, los bits en las posiciones pares  se sacan. El Segundo operando
({\code data \verb|>>| 1 \& 0x55}), primero mueve todos los bits de
posiciones pares a impares y usa la misma m�scara para sacar estos mismos
bits. Ahora, el primer operando contiene los bits pares y el segundo los
bits impares de {\code data}. Cuando estos dos operandos se suman, se
suman los bits pares e impares de {\code data}. Por ejemplo si data es
$10110011_2$, entonces:\\
\begin{tabular}{rcr|l|l|l|l|}
\cline{4-7}
{\code data \&} $01010101_2$          &    &   & 00 & 01 & 00 & 01 \\
+ {\code (data \verb|>>| 1) \&} $01010101_2$ & or & + & 01 & 01 & 00 & 01 \\
\cline{1-1} \cline{3-7}
                                      &    &   & 01 & 10 & 00 & 10 \\
\cline{4-7}
\end{tabular}

La suma de la derecha muestra los bits sumados. Los bits del byte se
dividen en 4 campos de 2 bits para mostrar que se realizan 4 sumas
independientes. Ya que la mayor�a de estas sumas pueden ser dos, no hay
posibilidad de que la suma desborde este campo y da�e otro de los campos
de la suma.

Claro est�, el n�mero total de bits no se ha calculado a�n. Sin embargo
la misma t�cnica que se us� arriba se puede usar para calcular el total
en una serie de pasos similares. El siguiente paso podr�a ser:
\begin{lstlisting}[stepnumber=0]{}
data = (data & 0x33) + ((data >> 2) & 0x33);
\end{lstlisting}
Continuando con el ejemplo de arriba (recuerde que {\code data} es ahora
$01100010_2$):\\
\begin{tabular}{rcr|l|l|}
\cline{4-5}
{\code data \&} $00110011_2$          &    &   & 0010 & 0010 \\
+ {\code (data \verb|>>| 2) \&} $00110011_2$ & or & + & 0001 & 0000 \\
\cline{1-1} \cline{3-5}
                                      &    &   & 0011 & 0010 \\
\cline{4-5}
\end{tabular}\\
Ahora hay 2 campos de 4 bits que se suman independientemente.

El pr�ximo paso es sumar estas dos sumas unidas para conformar el
resultado final:
\begin{lstlisting}[stepnumber=0]{}
data = (data & 0x0F) + ((data >> 4) & 0x0F);
\end{lstlisting} 

Usando el ejemplo de arriba (con {\code data} igual a $00110010_2$):\\
\begin{tabular}{rcrl}
{\code data \&} $00001111_2$          &    &   & 00000010 \\
+ {\code (data \verb|>>| 4) \&} $00001111_2$ & or & + & 00000011 \\
\cline{1-1} \cline{3-4}
                                      &    &   & 00000101 \\
\end{tabular}\\
Ahora {\code data} es 5 que es el resultado correcto. La
Figura~\ref{fig:method3} muestra una implementaci�n de este m�todo que
cuenta los bits en una palabra doble.  Usa un bucle {\code for} para
calcular la suma. Podr�a ser m�s r�pido deshacer el bucle; sin embargo,
el bucle clarifica c�mo el m�todo generaliza a diferentes tama�os de
datos.
\index{contando bits!m�todo tres|)}
\index{contando bits|)}

%-*- latex -*-
\chapter{Subprograms}

This chapter looks at using subprograms to make modular programs and to
interface with high level languages (like C). Functions and procedures are
high level language examples of subprograms.

The code that calls a subprogram and the subprogram itself must agree
on how data will be passed between them. These rules on how data will
be passed are called \emph{calling conventions}. \index{calling
convention} A large part of this chapter will deal with the standard C
calling conventions that can be used to interface assembly subprograms
with C programs. This (and other conventions) often pass the addresses
of data (\emph{i.e.} pointers) to allow the subprogram to access the
data in memory.

\section{Indirect Addressing\index{indirect addressing|(}}

Indirect addressing allows registers to act like pointer variables. To
indicate that a register is to be used indirectly as a pointer, it is
enclosed in square brackets ({\code []}). For example:
\begin{AsmCodeListing}[frame=none]
      mov    ax, [Data]     ; normal direct memory addressing of a word
      mov    ebx, Data      ; ebx = & Data
      mov    ax, [ebx]      ; ax = *ebx
\end{AsmCodeListing}
Because AX holds a word, line~3 reads a word starting at the address stored 
in EBX. If AX was replaced with AL, only a single byte would be read. It is
important to realize that registers do not have types like variables do in
C. What EBX is assumed to point to is completely determined by what
instructions are used. Furthermore, even the fact that EBX is a pointer is
completely determined by the what instructions are used. If EBX is used
incorrectly, often there will be no assembler error; however, the program
will not work correctly. This is one of the many reasons that assembly
programming is more error prone than high level programming.

All the 32-bit general purpose (EAX, EBX, ECX, EDX) and index (ESI, EDI)
registers can be used for indirect addressing. In general, the 16-bit 
and 8-bit registers can not be.
\index{indirect addressing|)}

\section{Simple Subprogram Example\index{subprogram|(}}

A subprogram is an independent unit of code that can be used from different
parts of a program. In other words, a subprogram is like a function in C. A
jump can be used to invoke the subprogram, but returning presents a problem.
If the subprogram is to be used by different parts of the program, it must
return back to the section of code that invoked it. Thus, the jump back from
the subprogram can not be hard coded to a label. The code below shows how this
could be done using the indirect form of the {\code JMP} instruction. This 
form of the instruction uses the value of a register to determine where to
jump to (thus, the register acts much like a \emph{function pointer} in C.)
Here is the first program from chapter~1 rewritten to use a subprogram.
\begin{AsmCodeListing}[label=sub1.asm]
; file: sub1.asm
; Subprogram example program
%include "asm_io.inc"

segment .data
prompt1 db    "Enter a number: ", 0       ; don't forget null terminator
prompt2 db    "Enter another number: ", 0
outmsg1 db    "You entered ", 0
outmsg2 db    " and ", 0
outmsg3 db    ", the sum of these is ", 0

segment .bss
input1  resd 1
input2  resd 1

segment .text
        global  _asm_main
_asm_main:
        enter   0,0               ; setup routine
        pusha

        mov     eax, prompt1      ; print out prompt
        call    print_string

        mov     ebx, input1       ; store address of input1 into ebx
        mov     ecx, ret1         ; store return address into ecx
        jmp     short get_int     ; read integer
ret1:
        mov     eax, prompt2      ; print out prompt
        call    print_string

        mov     ebx, input2
        mov     ecx, $ + 7        ; ecx = this address + 7
        jmp     short get_int

        mov     eax, [input1]     ; eax = dword at input1
        add     eax, [input2]     ; eax += dword at input2
        mov     ebx, eax          ; ebx = eax

        mov     eax, outmsg1
        call    print_string      ; print out first message
        mov     eax, [input1]     
        call    print_int         ; print out input1
        mov     eax, outmsg2
        call    print_string      ; print out second message
        mov     eax, [input2]
        call    print_int         ; print out input2
        mov     eax, outmsg3
        call    print_string      ; print out third message
        mov     eax, ebx
        call    print_int         ; print out sum (ebx)
        call    print_nl          ; print new-line

        popa
        mov     eax, 0            ; return back to C
        leave                     
        ret
; subprogram get_int
; Parameters:
;   ebx - address of dword to store integer into
;   ecx - address of instruction to return to
; Notes:
;   value of eax is destroyed
get_int:
        call    read_int
        mov     [ebx], eax         ; store input into memory
        jmp     ecx                ; jump back to caller
\end{AsmCodeListing}

The {\code get\_int} subprogram uses a simple, register-based calling
convention. It expects the EBX register to hold the address of the
DWORD to store the number input into and the ECX register to hold the
code address of the instruction to jump back to. In lines~25 to 28,
the {\code ret1} label is used to compute this return address. In
lines~32 to 34, the {\code \$} operator is used to compute the return
address. The {\code \$} operator returns the current address for the
line it appears on. The expression {\code \$ + 7} computes the address
of the {\code MOV} instruction on line~36.

Both of these return address computations are awkward. The first method
requires a label to be defined for each subprogram call. The second method
does not require a label, but does require careful thought. If a near jump
was used instead of a short jump, the number to add to {\code \$} would not
be 7! Fortunately, there is a much simpler way to invoke subprograms. This
method uses the \emph{stack}.

\section{The Stack\index{stack|(}}

Many CPU's have built-in support for a stack. A stack is a Last-In First-Out
(\emph{LIFO}) list. The stack is an area of memory that is organized in this
fashion. The {\code PUSH} instruction adds data to the stack and the
{\code POP} instruction removes data. The data removed is always the last
data added (that is why it is called a last-in first-out list).

The SS segment register specifies the segment that contains the stack (usually
this is the same segment data is stored into). The ESP register contains the
address of the data that would be removed from the stack. This data is said
to be at the \emph{top} of the stack. Data can only be added in double word
units. That is, one can not push a single byte on the stack.

The {\code PUSH} instruction inserts a double word\footnote{Actually
words can be pushed too, but in 32-bit protected mode, it is better to
work with only double words on the stack.} on the stack by subtracting
4 from ESP and then stores the double word at {\code [ESP]}. The
{\code POP} instruction reads the double word at {\code [ESP]} and
then adds 4 to ESP. The code below demostrates how these instructions
work and assumes that ESP is initially {\code 1000H}.
\begin{AsmCodeListing}[frame=none]
      push   dword 1    ; 1 stored at 0FFCh, ESP = 0FFCh
      push   dword 2    ; 2 stored at 0FF8h, ESP = 0FF8h
      push   dword 3    ; 3 stored at 0FF4h, ESP = 0FF4h
      pop    eax        ; EAX = 3, ESP = 0FF8h
      pop    ebx        ; EBX = 2, ESP = 0FFCh
      pop    ecx        ; ECX = 1, ESP = 1000h
\end{AsmCodeListing}

The stack can be used as a convenient place to store data temporarily. It is
also used for making subprogram calls, passing parameters and local
variables.

The 80x86 also provides a {\code PUSHA} instruction that pushes the values
of EAX, EBX, ECX, EDX, ESI, EDI and EBP registers (not in this order). The
{\code POPA} instruction can be used to pop them all back off.
\index{stack|)}

\section{The CALL and RET Instructions\index{subprogram!calling|(}}
\index{CALL|(}
\index{RET|(}
The 80x86 provides two instructions that use the stack to make calling
subprograms quick and easy. The CALL instruction makes an
unconditional jump to a subprogram and \emph{pushes} the address of
the next instruction on the stack. The RET instruction
\emph{pops off} an address and jumps to that address. When using these
instructions, it is very important that one manage the stack correctly
so that the right number is popped off by the RET instruction!

The previous program can be rewritten to use these new instructions by 
changing lines~25 to 34 to be:
\begin{AsmCodeListing}[numbers=none]
      mov    ebx, input1
      call   get_int

      mov    ebx, input2
      call   get_int
\end{AsmCodeListing}
and change the subprogram {\code get\_int} to:
\begin{AsmCodeListing}[numbers=none]
get_int:
      call   read_int
      mov    [ebx], eax
      ret
\end{AsmCodeListing}

There are several advantages to CALL and RET:
\begin{itemize}
\item It is simpler!
\item It allows subprograms calls to be nested easily. Notice that
{\code get\_int} calls {\code read\_int}. This call pushes another address
on the stack. At the end of {\code read\_int}'s code is a RET that pops
off the return address and jumps back to {\code get\_int}'s code. Then when
{\code get\_int}'s RET is executed, it pops off the return address that 
jumps back to {\code asm\_main}. This works correctly because of the LIFO
property of the stack.
\end{itemize}

Remember it is \emph{very} important to pop off all data that is pushed
on the stack. For example, consider the following:
\begin{AsmCodeListing}[frame=none]
get_int:
      call   read_int
      mov    [ebx], eax
      push   eax
      ret                  ; pops off EAX value, not return address!!
\end{AsmCodeListing}
This code would not return correctly!
\index{RET|)}
\index{CALL|)}

\section{Calling Conventions\index{calling convention|(}}

When a subprogram is invoked, the calling code and the subprogram (the
\emph{callee}) must agree on how to pass data between them. High-level
languages have standard ways to pass data known as \emph{calling 
conventions}. For high-level code to interface with assembly language, the
assembly language code must use the same conventions as the high-level
language. The calling conventions can differ from compiler to compiler or
may vary depending on how the code is compiled (\emph{e.g.} if
optimizations are on or not). One universal convention is that the code will
be invoked with a {\code CALL} instruction and return via a {\code RET}.

All PC C compilers support one calling convention that will be
described in the rest of this chapter in stages. These conventions
allow one to create subprograms that are \emph{reentrant}. A reentrant
subprogram may be called at any point of a program safely (even inside
the subprogram itself).

\subsection{Passing parameters on the stack\index{stack|(}\index{stack!parameters|(}}

Parameters to a subprogram may be passed on the stack. They are pushed onto
the stack before the {\code CALL} instruction. Just as in C, if the
parameter is to be changed by the subprogram, the \emph{address} of the 
data must be passed, not the \emph{value}. If the parameter's size is less
than a double word, it must be converted to a double word before being pushed.

The parameters on the stack are not popped off by the subprogram, instead
they are accessed from the stack itself. Why?
\begin{itemize}
\item Since they have to be pushed on the stack before the {\code CALL}
instruction, the return address would have to be popped off first (and
then pushed back on again).
\item Often the parameters will have to be used in several places in the
subprogram. Usually, they can not be kept in a register for the entire
subprogram and would have to be stored in memory. Leaving them on the
stack keeps a copy of the data in memory that can be accessed at any
point of the subprogram.
\end{itemize}

\begin{figure}
\centering
\begin{tabular}{l|c|}
\cline{2-2}
&  \\ \cline{2-2}
ESP + 4 & Parameter \\ \cline{2-2}
ESP     & Return address \\ \cline{2-2}
 & \\ \cline{2-2}
\end{tabular}
\caption{}
\label{fig:stack1}
\end{figure}
Consider \MarginNote{When using indirect addressing, the 80x86 processor 
accesses different segments depending on what registers are used in the
indirect addressing expression. ESP (and EBP) use the stack segment while
EAX, EBX, ECX and EDX use the data seg\-ment. However, this is usually 
unimportant for most protected mode programs, because for them the data 
and stack segments are the same.}
a subprogram that is passed a single parameter on the stack. When
the subprogram is invoked, the stack looks like Figure~\ref{fig:stack1}.
The parameter can be accessed using indirect addressing ({\code [ESP+4]}
\footnote{It is legal to add a constant to a register when using indirect
addressing. More complicated expressions are possible too. This topic is covered
in the next chapter}).
\begin{figure}
\centering
\begin{tabular}{l|c|}
\cline{2-2}
&  \\ \cline{2-2}
ESP + 8 & Parameter \\ \cline{2-2}
ESP + 4 & Return address \\ \cline{2-2}
ESP     & subprogram data \\ \cline{2-2}
\end{tabular}
\caption{}
\label{fig:stack2}
\end{figure}

\begin{figure}[t]
\begin{AsmCodeListing}[frame=single]
subprogram_label:
      push   ebp           ; save original EBP value on stack
      mov    ebp, esp      ; new EBP = ESP
; subprogram code
      pop    ebp           ; restore original EBP value
      ret
\end{AsmCodeListing}
\caption{General subprogram form \label{fig:subskel1}}
\end{figure}

If the stack is also used inside the subprogram to store data, the
number needed to be added to ESP will change. For example,
Figure~\ref{fig:stack2} shows what the stack looks like if a DWORD is
pushed the stack. Now the parameter is at {\code ESP + 8} not {\code
ESP + 4}. Thus, it can be very error prone to use ESP when referencing
parameters. To solve this problem, the 80386 supplies another register
to use: EBP. This register's only purpose is to reference data on the
stack. The C calling convention mandates that a subprogram first save
the value of EBP on the stack and then set EBP to be equal to ESP.
This allows ESP to change as data is pushed or popped off the stack
without modifying EBP. At the end of the subprogram, the original
value of EBP must be restored (this is why it is saved at the start of
the subprogram.)  Figure~\ref{fig:subskel1} shows the general form of
a subprogram that follows these conventions.

\begin{figure}[t]
\centering
\begin{tabular}{ll|c|}
\cline{3-3}
&  & \\ \cline{3-3}
ESP + 8 & EBP + 8 & Parameter \\ \cline{3-3}
ESP + 4 & EBP + 4 & Return address \\ \cline{3-3}
ESP     & EBP     & saved EBP \\ \cline{3-3}
\end{tabular}
\caption{}
\label{fig:stack3}
\end{figure}


Lines 2 and 3 in Figure~\ref{fig:subskel1} make up the general \emph{prologue}
of a subprogram. Lines 5 and 6 make up the \emph{epilogue}. 
Figure~\ref{fig:stack3} shows what the stack looks like immediately
after the prologue. Now the parameter can be access with {\code [EBP + 8]}
at any place in the subprogram without worrying about what else has
been pushed onto the stack by the subprogram.

After the subprogram is over, the parameters that were pushed on the
stack must be removed. The C calling convention \index{calling
convention!C} specifies that the caller code must do this. Other
conventions are different. For example, the Pascal calling convention
\index{calling convention!Pascal} specifies that the subprogram must
remove the parameters.  (There is another form of the RET \index{RET}
instruction that makes this easy to do.) Some C compilers support this
convention too. The {\code pascal} keyword is used in the prototype
and definition of the function to tell the compiler to use this
convention. In fact, the {\code stdcall} convention \index{calling
convention!stdcall} that the MS Windows API C functions use also works
this way.  What is the advantage of this way? It is a little more
efficient than the C convention. Why do all C functions not use this
convention, then? In general, C allows a function to have varying
number of arguments (\emph{e.g.}, the {\code printf} and {\code scanf}
functions). For these types of functions, the operation to remove the
parameters from the stack will vary from one call of the function to
the next. The C convention allows the instructions to perform this
operation to be easily varied from one call to the next. The Pascal
and stdcall convention makes this operation very difficult. Thus, the
Pascal convention (like the Pascal language) does not allow this type
of function. MS Windows can use this convention since none of its API
functions take varying numbers of arguments.

\begin{figure}[t]
\begin{AsmCodeListing}[frame=single]
      push   dword 1        ; pass 1 as parameter
      call   fun
      add    esp, 4         ; remove parameter from stack
\end{AsmCodeListing}
\caption{Sample subprogram call \label{fig:subcall}}
\end{figure}

Figure~\ref{fig:subcall} shows how a subprogram using the C calling
convention would be called. Line~3 removes the parameter from the
stack by directly manipulating the stack pointer. A {\code POP}
instruction could be used to do this also, but would require the
useless result to be stored in a register. Actually, for this
particular case, many compilers would use a {\code POP ECX}
instruction to remove the parameter. The compiler would use a {\code
POP} instead of an {\code ADD} because the {\code ADD} requires more
bytes for the instruction. However, the {\code POP} also changes ECX's
value! Next is another example program with two subprograms that use
the C calling conventions discussed above. Line~54 (and other lines)
shows that multiple data and text segments may be declared in a single
source file. They will be combined into single data and text segments
in the linking process. Splitting up the data and code into separate
segments allow the data that a subprogram uses to be defined close by
the code of the subprogram.
\index{stack!parameters|)}

\begin{AsmCodeListing}[label=sub3.asm]
%include "asm_io.inc"

segment .data
sum     dd   0

segment .bss
input   resd 1

;
; pseudo-code algorithm
; i = 1;
; sum = 0;
; while( get_int(i, &input), input != 0 ) {
;   sum += input;
;   i++;
; }
; print_sum(num);
segment .text
        global  _asm_main
_asm_main:
        enter   0,0               ; setup routine
        pusha

        mov     edx, 1            ; edx is 'i' in pseudo-code
while_loop:
        push    edx               ; save i on stack
        push    dword input       ; push address of input on stack
        call    get_int
        add     esp, 8            ; remove i and &input from stack

        mov     eax, [input]
        cmp     eax, 0
        je      end_while

        add     [sum], eax        ; sum += input

        inc     edx
        jmp     short while_loop

end_while:
        push    dword [sum]       ; push value of sum onto stack
        call    print_sum
        pop     ecx               ; remove [sum] from stack

        popa
        leave                     
        ret

; subprogram get_int
; Parameters (in order pushed on stack)
;   number of input (at [ebp + 12])
;   address of word to store input into (at [ebp + 8])
; Notes:
;   values of eax and ebx are destroyed
segment .data
prompt  db      ") Enter an integer number (0 to quit): ", 0

segment .text
get_int:
        push    ebp
        mov     ebp, esp

        mov     eax, [ebp + 12]
        call    print_int

        mov     eax, prompt
        call    print_string
        
        call    read_int
        mov     ebx, [ebp + 8]
        mov     [ebx], eax         ; store input into memory

        pop     ebp
        ret                        ; jump back to caller

; subprogram print_sum
; prints out the sum
; Parameter:
;   sum to print out (at [ebp+8])
; Note: destroys value of eax
;
segment .data
result  db      "The sum is ", 0

segment .text
print_sum:
        push    ebp
        mov     ebp, esp

        mov     eax, result
        call    print_string

        mov     eax, [ebp+8]
        call    print_int
        call    print_nl

        pop     ebp
        ret
\end{AsmCodeListing}


\subsection{Local variables on the stack\index{stack!local variables|(}}

The stack can be used as a convenient location for local variables. This is
exactly where C stores normal (or \emph{automatic} in C lingo) variables.
Using the stack for variables is important if one wishes subprograms to be
reentrant. A reentrant subprogram will work if it is invoked at any place,
including the subprogram itself. In other words, reentrant subprograms
can be invoked \emph{recursively}. Using the stack for variables also saves
memory. Data not stored on the stack is using memory from the beginning of
the program until the end of the program (C calls these types of variables
\emph{global} or \emph{static}). Data stored on the stack only use memory
when the subprogram they are defined for is active.

\begin{figure}[t]
\begin{AsmCodeListing}[frame=single]
subprogram_label:
      push   ebp                ; save original EBP value on stack
      mov    ebp, esp           ; new EBP = ESP
      sub    esp, LOCAL_BYTES   ; = # bytes needed by locals
; subprogram code
      mov    esp, ebp           ; deallocate locals
      pop    ebp                ; restore original EBP value
      ret
\end{AsmCodeListing}
\caption{General subprogram form with local variables\label{fig:subskel2}}
\end{figure}

\begin{figure}[t]
\begin{lstlisting}[frame=tlrb]{}
void calc_sum( int n, int * sump )
{
  int i, sum = 0;

  for( i=1; i <= n; i++ )
    sum += i;
  *sump = sum;
}
\end{lstlisting}
\caption{C version of sum \label{fig:Csum}}
\end{figure}

\begin{figure}[t]
\begin{AsmCodeListing}[frame=single]
cal_sum:
      push   ebp
      mov    ebp, esp
      sub    esp, 4               ; make room for local sum

      mov    dword [ebp - 4], 0   ; sum = 0
      mov    ebx, 1               ; ebx (i) = 1
for_loop:
      cmp    ebx, [ebp+8]         ; is i <= n?
      jnle   end_for

      add    [ebp-4], ebx         ; sum += i
      inc    ebx
      jmp    short for_loop

end_for:
      mov    ebx, [ebp+12]        ; ebx = sump
      mov    eax, [ebp-4]         ; eax = sum
      mov    [ebx], eax           ; *sump = sum;

      mov    esp, ebp
      pop    ebp
      ret
\end{AsmCodeListing}
\caption{Assembly version of sum\label{fig:Asmsum}}
\end{figure}

Local variables are stored right after the saved EBP value in the stack.
They are allocated by subtracting the number of bytes required from ESP
in the prologue of the subprogram. Figure~\ref{fig:subskel2} shows the 
new subprogram skeleton. The EBP register is used to access local variables.
Consider the C function in Figure~\ref{fig:Csum}. Figure~\ref{fig:Asmsum}
shows how the equivalent subprogram could be written in assembly.

\begin{figure}[t]
\centering
\begin{tabular}{ll|c|}
\cline{3-3}
ESP + 16 & EBP + 12 & {\code sump} \\ \cline{3-3}
ESP + 12 & EBP + 8  & {\code n} \\ \cline{3-3}
ESP + 8  & EBP + 4  & Return address \\ \cline{3-3}
ESP + 4  & EBP      & saved EBP \\ \cline{3-3}
ESP      & EBP - 4  & {\code sum} \\ \cline{3-3}
\end{tabular}
\caption{}
\label{fig:SumStack}
\end{figure}

Figure~\ref{fig:SumStack} shows what the stack looks like after the
prologue of the program in Figure~\ref{fig:Asmsum}. This section of
the stack that contains the parameters, return information and local
variable storage is called a \emph{stack frame}. Every invocation of
a C function creates a new stack frame on the stack.

\begin{figure}[t]
\begin{AsmCodeListing}[frame=single]
subprogram_label:
      enter  LOCAL_BYTES, 0     ; = # bytes needed by locals
; subprogram code
      leave
      ret
\end{AsmCodeListing}
\caption{General subprogram form with local variables using 
{\code ENTER} and {\code LEAVE}\label{fig:subskel3}}
\end{figure}

\MarginNote{Despite the fact that {\code ENTER} and {\code LEAVE} simplify
the prologue and epilogue they are not used very often. Why? Because
they are slower than the equivalent simplier instructions! This is an
example of when one can not assume that a one instruction sequence is
faster than a multiple instruction one.} 
The prologue and epilogue of a subprogram can be simplified by using
two special instructions that are designed specifically for this
purpose. The {\code ENTER} instruction performs the prologue code and the
{\code LEAVE} performs the epilogue. The {\code ENTER} instruction
takes two immediate operands. For the C calling convention, the second
operand is always 0. The first operand is the number bytes needed by
local variables. The {\code LEAVE} instruction has no
operands. Figure~\ref{fig:subskel3} shows how these instructions are
used. Note that the program skeleton (Figure~\ref{fig:skel}) also uses
{\code ENTER} and {\code LEAVE}.
\index{stack!local variables|)}
\index{stack|)}
\index{calling convention|)}
\index{subprogram!calling|)}

\section{Multi-Module Programs\index{multi-module programs|(}}

A \emph{multi-module program} is one composed of more than one object
file.  All the programs presented here have been multi-module
programs. They consisted of the C driver object file and the assembly
object file (plus the C library object files). Recall that the linker
combines the object files into a single executable program. The linker
must match up references made to each label in one module (\emph{i.e.}
object file) to its definition in another module. In order for module
A to use a label defined in module B, the {\code extern} directive
must be used. After the {\code extern} \index{directive!extern}
directive comes a comma delimited list of labels. The directive tells
the assembler to treat these labels as \emph{external} to the
module. That is, these are labels that can be used in this module, but
are defined in another. The {\code asm\_io.inc} file defines the
{\code read\_int}, \emph{etc.} routines as external.

In assembly, labels can not be accessed externally by default. If a
label can be accessed from other modules than the one it is defined
in, it must be declared \emph{global} in its module. The {\code
global} \index{directive!global} directive does this. Line~13 of the
skeleton program listing in Figure~\ref{fig:skel} shows the {\code
\_asm\_main} label being defined as global. Without this declaration,
there would be a linker error. Why? Because the C code would not be
able to refer to the \emph{internal} {\code \_asm\_main} label.

Next is the code for the previous example, rewritten to use two modules. The
two subprograms ({\code get\_int} and {\code print\_sum}) are in a separate
source file than the {\code \_asm\_main} routine.

\begin{AsmCodeListing}[label=main4.asm,commandchars=\\\{\}]
%include "asm_io.inc"

segment .data
sum     dd   0

segment .bss
input   resd 1

segment .text
        global  _asm_main
\textit{        extern  get_int, print_sum}
_asm_main:
        enter   0,0               ; setup routine
        pusha

        mov     edx, 1            ; edx is 'i' in pseudo-code
while_loop:
        push    edx               ; save i on stack
        push    dword input       ; push address on input on stack
        call    get_int
        add     esp, 8            ; remove i and &input from stack

        mov     eax, [input]
        cmp     eax, 0
        je      end_while

        add     [sum], eax        ; sum += input

        inc     edx
        jmp     short while_loop

end_while:
        push    dword [sum]       ; push value of sum onto stack
        call    print_sum
        pop     ecx               ; remove [sum] from stack

        popa
        leave                     
        ret
\end{AsmCodeListing}

\begin{AsmCodeListing}[label=sub4.asm,commandchars=\\\{\}]
%include "asm_io.inc"

segment .data
prompt  db      ") Enter an integer number (0 to quit): ", 0

segment .text
\textit{        global  get_int, print_sum}
get_int:
        enter   0,0

        mov     eax, [ebp + 12]
        call    print_int

        mov     eax, prompt
        call    print_string
        
        call    read_int
        mov     ebx, [ebp + 8]
        mov     [ebx], eax         ; store input into memory

        leave
        ret                        ; jump back to caller

segment .data
result  db      "The sum is ", 0

segment .text
print_sum:
        enter   0,0

        mov     eax, result
        call    print_string

        mov     eax, [ebp+8]
        call    print_int
        call    print_nl

        leave
        ret
\end{AsmCodeListing}

The previous example only has global \index{directive!global} code
labels; however, global data labels work exactly the same way.
\index{multi-module programs|)}

\section{Interfacing Assembly with C\index{interfacing with C|(}\index{calling convention!C|(}}

Today, very few programs are written completely in assembly. Compilers are
very good at converting high level code into efficient machine code. Since
it is much easier to write code in a high level language, it is more popular.
In addition, high level code is \emph{much} more portable than assembly!

When assembly is used, it is often only used for small parts of the code.
This can be done in two ways: calling assembly subroutines from C or inline
assembly. Inline assembly allows the programmer to place assembly statements
directly into C code. This can be very convenient; however, there are 
disadvantages to inline assembly. The assembly code must be written
in the format the compiler uses. No compiler at the moment supports NASM's
format. Different compilers require different formats. Borland and Microsoft
require MASM format. DJGPP and Linux's gcc require GAS\footnote{GAS is the
assembler that all GNU compiler's use. It uses the AT\&T syntax which is
very different from the relatively similar syntaxes of MASM, TASM and NASM.}
format. The technique of calling an assembly subroutine is much more
standardized on the PC.

Assembly routines are usually used with C for the following reasons:
\begin{itemize}
\item Direct access is needed to hardware features of the computer that
      are difficult or impossible to access from C.
\item The routine must be as fast as possible and the programmer can
      hand optimize the code better than the compiler can.
\end{itemize}

The last reason is not as valid as it once was. Compiler technology has
improved over the years and compilers can often generate very efficient code
(especially if compiler optimizations are turned on). The disadvantages of
assembly routines are: reduced portability and readability.

Most of the C calling conventions have already been specified. However, there
are a few additional features that need to be described.

\subsection{Saving registers\index{calling convention!C!registers|(}}
First, 
\MarginNote{The {\code register} keyword can be used in a C variable
declaration to suggest to the compiler that it use a register for this
variable instead of a memory location. These are known as register
variables. Modern compilers do this automatically without requiring any
suggestions.}
C assumes that a subroutine maintains the values of the
following registers: EBX, ESI, EDI, EBP, CS, DS, SS, ES. This does not
mean that the subroutine can not change them internally. Instead, it
means that if it does change their values, it must restore their 
original values before the subroutine returns. The EBX, ESI and EDI values
must be unmodified because C uses these registers for \emph{register
variables}. Usually the stack is used to save the original values of these
registers.

\begin{figure}[t]
\begin{AsmCodeListing}[frame=single]
segment .data
x            dd     0
format       db     "x = %d\n", 0

segment .text
...
      push   dword [x]     ; push x's value
      push   dword format  ; push address of format string
      call   _printf       ; note underscore!
      add    esp, 8        ; remove parameters from stack
\end{AsmCodeListing}
\caption{Call to {\code printf} \label{fig:Cprintf}}
\end{figure}

\begin{figure}[t]
\centering
\begin{tabular}{l|c|}
\cline{2-2}
EBP + 12 & value of {\code x} \\ \cline{2-2}
EBP + 8  & address of format string \\ \cline{2-2}
EBP + 4  & Return address \\ \cline{2-2}
EBP      & saved EBP \\ \cline{2-2}
\end{tabular}
\caption{Stack inside {\code printf}\label{fig:CprintfStack}}
\end{figure}
\index{calling convention!C!registers|)}

\subsection{Labels of functions\index{calling convention!C!labels|(}}
Most C compilers prepend a single underscore({\code \_}) character at
the beginning of the names of functions and global/static
variables. For example, a function named {\code f} will be assigned
the label {\code \_f}. Thus, if this is to be an assembly routine, it
\emph{must} be labelled {\code \_f}, not {\code f}. The Linux gcc
compiler does \emph{not} prepend any character.  Under Linux ELF
executables, one simply would use the label {\code f} for the C
function {\code f}.  However, DJGPP's gcc does prepend an
underscore. Note that in the assembly skeleton program
(Figure~\ref{fig:skel}), the label for the main routine is {\code
\_asm\_main}.
\index{calling convention!C!labels|)}

\subsection{Passing parameters\index{calling convention!C!parameters|(}}
Under the C calling convention, the arguments of a function are pushed on
the stack in the \emph{reverse} order that they appear in the function
call.

Consider the following C statement: \verb|printf("x = %d\n",x);|
Figure~\ref{fig:Cprintf} shows how this would be compiled (shown in
the equivalent NASM format). Figure~\ref{fig:CprintfStack} shows what
the stack looks like after the prologue inside the {\code printf}
function. The {\code printf} function is one of the C library
functions that can take any number of arguments. The rules of the C
calling conventions were specifically written to allow these types of
functions. \MarginNote{It is not necessary to use assembly to process
an arbitrary number of arguments in C. The {\code stdarg.h} header
file defines macros that can be used to process them portably. See any
good C book for details.} Since the address of the format string is
pushed last, its location on the stack will \emph{always} be at
{\code EBP + 8} no matter how many parameters are passed to the
function. The {\code printf} code can then look at the format string
to determine how many parameters should have been passed and look for
them on the stack.

Of course, if a mistake is made, \verb|printf("x = %d\n")|, the
{\code printf} code will still print out the double word value at 
{\code [EBP + 12]}. However, this will not be {\code x}'s value!
\index{calling convention!C!parameters|)}

\subsection{Calculating addresses of local variables\index{stack!local variables|(}}

Finding the address of a label defined in the {\code data} or {\code
bss} segments is simple. Basically, the linker does this. However,
calculating the address of a local variable (or parameter) on the
stack is not as straightforward. However, this is a very common need
when calling subroutines. Consider the case of passing the address of
a variable (let's call it {\code x}) to a function (let's call it
{\code foo}).  If {\code x} is located at EBP $-$ 8 on the stack, one
cannot just use:
\begin{AsmCodeListing}[numbers=none,frame=none]
      mov    eax, ebp - 8
\end{AsmCodeListing}
Why? The value that {\code MOV} stores into EAX must be computed by
the assembler (that is, it must in the end be a constant). However,
there is an instruction that does the desired calculation. It is
called \index{LEA|(} {\code LEA}  (for \emph{Load Effective Address}). The following
would calculate the address of {\code x} and store it into EAX:
\begin{AsmCodeListing}[numbers=none,frame=none]
      lea    eax, [ebp - 8]
\end{AsmCodeListing}
Now EAX holds the address of {\code x} and could be pushed on the
stack when calling function {\code foo}. Do not be confused, it looks
like this instruction is reading the data at
[EBP\nolinebreak$-$\nolinebreak8]; however, this is \emph{not}
true. The {\code LEA} instruction \emph{never} reads memory! It only
computes the address that would be read by another instruction and
stores this address in its first register operand. Since it does not
actually read any memory, no memory size designation (\emph{e.g.}
{\code dword}) is needed or allowed.

\index{LEA|)}
\index{stack!local variables|)}

\subsection{Returning values\index{calling convention!C!return values|(}}

Non-void C functions return back a value. The C calling conventions
specify how this is done. Return values are passed via registers. All
integral types ({\code char}, {\code int}, {\code enum}, \emph{etc.})
are returned in the EAX register. If they are smaller than 32-bits,
they are extended to 32-bits when stored in EAX. (How they are
extended depends on if they are signed or unsigned types.) 64-bit values
are returned in the EDX:EAX\index{register!EDX:EAX} register pair. Pointer
values are also stored in EAX. Floating point values are stored in the
ST0 register of the math coprocessor. (This register is discussed in
the floating point chapter.)
\index{calling convention!C!return values|)}
\index{calling convention!C|)}

\subsection{Other calling conventions\index{calling convention|(}}

The rules above describe the standard C calling convention that is
supported by all 80x86 C compilers. Often compilers support other
calling conventions as well. When interfacing with assembly language
it is \emph{very} important to know what calling convention the
compiler is using when it calls your function. Usually, the default is
to use the standard calling convention; however, this is not always
the case\footnote{The Watcom C\index{compiler!Watcom} compiler is an
example of one that does \emph{not} use the standard convention by
default. See the example source code file for Watcom for details}.
Compilers that use multiple conventions often have command line
switches that can be used to change the default convention.  They also
provide extensions to the C syntax to explicitly assign calling
conventions to individual functions. However, these extensions are not
standardized and may vary from one compiler to another.

The GCC compiler allows different calling conventions. The convention
of a function can be explicitly declared by using the {\code
\_\_attribute\_\_} extension\index{compiler!gcc!\_\_attribute\_\_}. For example,
to declare a void function that uses the standard calling convention
\index{calling convention!C} named {\code f} that takes a single
{\code int} parameter, use the following syntax for its prototype:
\begin{lstlisting}[stepnumber=0]{}
void f( int ) __attribute__((cdecl));
\end{lstlisting}
GCC also supports the \emph{standard call} \index{calling
convention!stdcall} calling convention. The function above could be
declared to use this convention by replacing the {\code cdecl} with
{\code stdcall}. The difference in {\code stdcall} and {\code cdecl}
is that {\code stdcall} requires the subroutine to remove the
parameters from the stack (as the Pascal calling convention
does). Thus, the {\code stdcall} convention can only be used with
functions that take a fixed number of arguments (\emph{i.e.} ones not
like {\code printf} and {\code scanf}).

GCC also supports an additional attribute called {\code regparm}
\index{calling convention!register} that tells the compiler to use
registers to pass up to 3 integer arguments to a function instead of
using the stack. This is a common type of optimization that many
compilers support.

Borland and Microsoft use a common syntax to declare calling
conventions.  They add the {\code \_\_cdecl}\index{calling
convention!\_\_cdecl} and {\code \_\_stdcall}\index{calling
convention!\_\_stdcall} keywords to C. These keywords act as function
modifiers and appear immediately before the function name in a
prototype. For example, the function {\code f} above would be defined
as follows for Borland and Microsoft:
\begin{lstlisting}[stepnumber=0]{}
void __cdecl f( int );
\end{lstlisting}

There are advantages and disadvantages to each of the calling
conventions.  The main advantages of the {\code cdecl}\index{calling
convention!C} convention is that it is simple and very flexible. It
can be used for any type of C function and C compiler. Using other
conventions can limit the portability of the subroutine. Its main
disadvantage is that it can be slower than some of the others and use
more memory (since every invocation of the function requires code to
remove the parameters on the stack).

The advantages of the {\code stdcall}\index{calling
convention!standard call} convention is that it uses less memory than
{\code cdecl}. No stack cleanup is required after the {\code CALL}
instruction. Its main disadvantage is that it can not be used with
functions that have variable numbers of arguments.

The advantage of using a convention that uses registers to pass integer
parameters is speed. The main disadvantage is that the convention is more
complex. Some parameters may be in registers and others on the stack.

\index{calling convention|)}

\subsection{Examples}

Next is an example that shows how an assembly routine can be interfaced to
a C program. (Note that this program does not use the assembly skeleton
program (Figure~\ref{fig:skel}) or the driver.c module.)

\LabelLine{main5.c}
\begin{lstlisting}{}
#include <stdio.h>
/* prototype for assembly routine */
void calc_sum( int, int * ) __attribute__((cdecl));

int main( void )
{
  int n, sum;

  printf("Sum integers up to: ");
  scanf("%d", &n);
  calc_sum(n, &sum);
  printf("Sum is %d\n", sum);
  return 0;
}
\end{lstlisting}
\LabelLine{main5.c}

\begin{AsmCodeListing}[label=sub5.asm, commandchars=\\\%|]
; subroutine _calc_sum
; finds the sum of the integers 1 through n
; Parameters:
;   n    - what to sum up to (at [ebp + 8])
;   sump - pointer to int to store sum into (at [ebp + 12])
; pseudo C code:
; void calc_sum( int n, int * sump )
; {
;   int i, sum = 0;
;   for( i=1; i <= n; i++ )
;     sum += i;
;   *sump = sum;
; }

segment .text
        global  _calc_sum
;
; local variable:
;   sum at [ebp-4]
_calc_sum:
        enter   4,0               ; make room for sum on stack
        push    ebx               ; IMPORTANT! \label%line:pushebx|

        mov     dword [ebp-4],0   ; sum = 0
        dump_stack 1, 2, 4        ; print out stack from ebp-8 to ebp+16 \label%line:dumpstack|
        mov     ecx, 1            ; ecx is i in pseudocode
for_loop:
        cmp     ecx, [ebp+8]      ; cmp i and n
        jnle    end_for           ; if not i <= n, quit

        add     [ebp-4], ecx      ; sum += i
        inc     ecx
        jmp     short for_loop

end_for:
        mov     ebx, [ebp+12]     ; ebx = sump
        mov     eax, [ebp-4]      ; eax = sum
        mov     [ebx], eax

        pop     ebx               ; restore ebx
        leave
        ret
\end{AsmCodeListing}

\begin{figure}[t]
\begin{Verbatim}[frame=single]
Sum integers up to: 10
Stack Dump # 1
EBP = BFFFFB70 ESP = BFFFFB68
 +16  BFFFFB80  080499EC
 +12  BFFFFB7C  BFFFFB80
  +8  BFFFFB78  0000000A
  +4  BFFFFB74  08048501
  +0  BFFFFB70  BFFFFB88
  -4  BFFFFB6C  00000000
  -8  BFFFFB68  4010648C
Sum is 55
\end{Verbatim}
\caption{Sample run of sub5 program \label{fig:dumpstack}}
\end{figure}

Why is line~\ref{line:pushebx} of {\code sub5.asm} so important?
Because the C calling convention requires the value of EBX to be
unmodified by the function call. If this is not done, it is very
likely that the program will not work correctly.

Line~\ref{line:dumpstack} demonstrates how the {\code dump\_stack} macro
works. Recall that the first parameter is just a numeric label, and the
second and third parameters determine how many double words to display below
and above EBP respectively. Figure~\ref{fig:dumpstack} shows an example run
of the program. For this dump, one can see that the address of the dword
to store the sum is BFFFFB80 (at EBP~+~12); the number to sum up to is 0000000A
(at EBP~+~8); the return address for the routine is 08048501 (at EBP~+~4);
the saved EBP value is BFFFFB88 (at EBP); the value of the local variable is
0 at (EBP~-~4); and finally the saved EBX value is 4010648C (at EBP~-~8).

The {\code calc\_sum} function could be rewritten to return the sum as its
return value instead of using a pointer parameter. Since the sum is an
integral value, the sum should be left in the EAX register. Line~11 of the
{\code main5.c} file would be changed to:
\begin{lstlisting}[stepnumber=0]{}
  sum = calc_sum(n);
\end{lstlisting}
Also, the prototype of {\code calc\_sum} would need be altered. Below is
the modified assembly code:
\begin{AsmCodeListing}[label=sub6.asm]
; subroutine _calc_sum
; finds the sum of the integers 1 through n
; Parameters:
;   n    - what to sum up to (at [ebp + 8])
; Return value:
;   value of sum
; pseudo C code:
; int calc_sum( int n )
; {
;   int i, sum = 0;
;   for( i=1; i <= n; i++ )
;     sum += i;
;   return sum;
; }
segment .text
        global  _calc_sum
;
; local variable:
;   sum at [ebp-4]
_calc_sum:
        enter   4,0               ; make room for sum on stack

        mov     dword [ebp-4],0   ; sum = 0
        mov     ecx, 1            ; ecx is i in pseudocode
for_loop:
        cmp     ecx, [ebp+8]      ; cmp i and n
        jnle    end_for           ; if not i <= n, quit

        add     [ebp-4], ecx      ; sum += i
        inc     ecx
        jmp     short for_loop

end_for:
        mov     eax, [ebp-4]      ; eax = sum

        leave
        ret
\end{AsmCodeListing}

\subsection{Calling C functions from assembly}

\begin{figure}[t]
\begin{AsmCodeListing}[frame=single]
segment .data
format       db "%d", 0

segment .text
...
      lea    eax, [ebp-16]
      push   eax
      push   dword format
      call   _scanf
      add    esp, 8
...
\end{AsmCodeListing}
\caption{Calling {\code scanf} from assembly\label{fig:scanf}}
\end{figure}

One great advantage of interfacing C and assembly is that allows
assembly code to access the large C library and user-written functions.
For example, what if one wanted to call the {\code scanf} function to
read in an integer from the keyboard? Figure~\ref{fig:scanf} shows
code to do this. One very important point to remember is that {\code
scanf} follows the C calling standard to the letter. This means that it
preserves the values of the EBX, ESI and EDI registers; however, the
EAX, ECX and EDX registers may be modified! In fact, EAX will definitely
be changed, as it will contain the return value of the {\code scanf} call.
For other examples of using interfacing with C, look at the code in
{\code asm\_io.asm} which was used to create {\code asm\_io.obj}.
\index{interfacing with C|)}

\section{Reentrant and Recursive Subprograms\index{recursion|(}}

\index{subprogram!reentrant|(}
A reentrant subprogram must satisfy the following properties:
\begin{itemize}
\item It must not modify any code instructions. In a high level language
this would be difficult, but in assembly it is not hard for a program to
try to modify its own code. For example:
\begin{AsmCodeListing}[frame=none, numbers=none]
      mov    word [cs:$+7], 5      ; copy 5 into the word 7 bytes ahead
      add    ax, 2                 ; previous statement changes 2 to 5!
\end{AsmCodeListing}
This code would work in real mode, but in protected mode operating systems 
the code segment is marked as read only. When the first line above executes,
the program will be aborted on these systems. This type of programming is
bad for many reasons. It is confusing, hard to maintain and does not allow
code sharing (see below).

\item It must not modify global data (such as data in the {\code data} and
the {\code bss} segments). All variables are stored on the stack.

\end{itemize}

There are several advantages to writing reentrant code.
\begin{itemize}
\item A reentrant subprogram can be called recursively.
\item A reentrant program can be shared by multiple processes. On many
multi-tasking operating systems, if there are multiple instances of a
program running, only \emph{one} copy of the code is in memory. Shared
libraries and DLL's (\emph{Dynamic Link Libraries}) use this idea as well.
\item Reentrant subprograms work much better in \emph{multi-threaded}
\footnote{A multi-threaded program has multiple threads of execution. That
is, the program itself is multi-tasked.} pro\-grams. Windows 9x/NT and most
UNIX-like operating systems (Solaris, Linux, \emph{etc.}) support 
multi-threaded programs.
\end{itemize}
\index{subprogram!reentrant|)}

\subsection{Recursive subprograms}

These types of subprograms call themselves. The recursion can be either
\emph{direct} or \emph{indirect}. Direct recursion occurs when a subprogram,
say {\code foo}, calls itself inside {\code foo}'s body. Indirect recursion
occurs when a subprogram is not called by itself directly, but by another
subprogram it calls. For example, subprogram {\code foo} could call
{\code bar} and {\code bar} could call {\code foo}.

Recursive subprograms must have a \emph{termination condition}. When
this condition is true, no more recursive calls are made. If a
recursive routine does not have a termination condition or the condition
never becomes true, the recursion will never end (much like an infinite
loop).

\begin{figure}
\begin{AsmCodeListing}[frame=single]
; finds n!
segment .text
      global _fact
_fact:
      enter  0,0

      mov    eax, [ebp+8]    ; eax = n
      cmp    eax, 1
      jbe    term_cond       ; if n <= 1, terminate
      dec    eax
      push   eax
      call   _fact           ; eax = fact(n-1)
      pop    ecx             ; answer in eax
      mul    dword [ebp+8]   ; edx:eax = eax * [ebp+8]
      jmp    short end_fact
term_cond:
      mov    eax, 1
end_fact:
      leave
      ret
\end{AsmCodeListing}
\caption{Recursive factorial function\label{fig:factorial}}
\end{figure}

\begin{figure}
\centering
%\includegraphics{factStack.eps}
\input{factStack.latex}
\caption{Stack frames for factorial function\label{fig:factStack}}
\end{figure}

Figure~\ref{fig:factorial} shows a function that calculates factorials
recursively. It could be called from C with:
\begin{lstlisting}[stepnumber=0]{}
x = fact(3);         /* find 3! */
\end{lstlisting}
Figure~\ref{fig:factStack} shows what the stack looks like at its deepest
point for the above function call.

\begin{figure}[t]
\begin{lstlisting}[frame=tlrb]{}
void f( int x )
{
  int i;
  for( i=0; i < x; i++ ) {
    printf("%d\n", i);
    f(i);
  }
}
\end{lstlisting}
\caption{Another example (C version)\label{fig:rec2C}}
\end{figure}

\begin{figure}
\begin{AsmCodeListing}[frame=single]
%define i ebp-4
%define x ebp+8          ; useful macros
segment .data
format       db "%d", 10, 0     ; 10 = '\n'
segment .text
      global _f
      extern _printf
_f:
      enter  4,0           ; allocate room on stack for i

      mov    dword [i], 0  ; i = 0
lp:
      mov    eax, [i]      ; is i < x?
      cmp    eax, [x]
      jnl    quit

      push   eax           ; call printf
      push   format
      call   _printf
      add    esp, 8

      push   dword [i]     ; call f
      call   _f
      pop    eax

      inc    dword [i]     ; i++
      jmp    short lp
quit:
      leave
      ret
\end{AsmCodeListing}
\caption{Another example (assembly version)\label{fig:rec2Asm}}
\end{figure}

Figures~\ref{fig:rec2C} and \ref{fig:rec2Asm} show another more
complicated recursive example in C and assembly, respectively. What is
the output is for {\code f(3)}? Note that the {\code ENTER} instruction
creates a new {\code i} on the stack for each recursive call. Thus, each
recursive instance of {\code f} has its own independent variable {\code i}.
Defining {\code i} as a double word in the {\code data} segment would not
work the same. 
\index{recursion|)}

\subsection{Review of C variable storage types}

C provides several types of variable storage.
\begin{description}
\item[global] 
\index{storage types!global}
These variables are defined outside of any function and
are stored at fixed memory locations (in the {\code data} or {\code
bss} segments) and exist from the beginning of the program until the
end. By default, they can be accessed from any function in the program;
however, if they are declared as {\code static}, only the functions in
the same module can access them (\emph{i.e.} in assembly terms, the
label is internal, not external).

\item[static] 
\index{storage types!static}
These are \emph{local} variables of a function that are
declared {\code static}. (Unfortunately, C uses the keyword {\code
static} for two different purposes!) These variables are also stored
at fixed memory locations (in {\code data} or {\code bss}), but can
only be directly accessed in the functions they are defined in. 

\item[automatic] 
\index{storage types!automatic}
This is the default type for a C variable defined inside a
function. These variables are allocated on the stack when the function
they are defined in is invoked and are deallocated when the function
returns. Thus, they do not have fixed memory locations.

\item[register] 
\index{storage types!register}
This keyword asks the compiler to use a register for
the data in this variable. This is just a \emph{request}. The compiler
does \emph{not} have to honor it. If the address of the variable is
used anywhere in the program it will not be honored (since registers
do not have addresses). Also, only simple integral types can be
register values.  Structured types can not be; they would not fit in a
register! C compilers will often automatically make normal automatic
variables into register variables without any hint from the programmer.

\item[volatile] 
\index{storage types!volatile}
This keyword tells the compiler that the value of the
variable may change any moment. This means that the compiler can not
make any assumptions about when the variable is modified. Often a
compiler might store the value of a variable in a register temporarily
and use the register in place of the variable in a section of code. It
can not do these types of optimizations with {\code volatile}
variables. A common example of a volatile variable would be one could
be altered by two threads of a multi-threaded program. Consider the 
following code:
\begin{lstlisting}{}
x = 10;
y = 20;
z = x;
\end{lstlisting}
If {\code x} could be altered by another thread, it is possible that the
other thread changes {\code x} between lines~1 and 3 so that {\code z}
would not be 10. However, if the {\code x} was not declared volatile, the
compiler might assume that {\code x} is unchanged and set {\code z} to 10.

Another use of {\code volatile} is to keep the compiler from using a
register for a variable. 

\end{description}
\index{subprogram|)}

% -*-latex-*-
\chapter{Arrays}
\index{arrays|(}
\section{Introduction}

An \emph{array} is a contiguous block of list of data in memory. Each element
of the list must be the same type and use exactly the same number of bytes
of memory for storage. Because of these properties, arrays allow efficient
access of the data by its position (or index) in the array. The address
of any element can be computed by knowing three facts:
\begin{itemize}
\item The address of the first element of the array.
\item The number of bytes in each element
\item The index of the element
\end{itemize}

It is convenient to consider the index of the first element of the array
to be zero (just as in C). It is possible to use other values for the
first index, but it complicates the computations.

\subsection{Defining arrays\index{arrays!defining|(}}

\begin{figure}[t]
\begin{AsmCodeListing}[frame=single]
segment .data
; define array of 10 double words initialized to 1,2,..,10
a1           dd   1, 2, 3, 4, 5, 6, 7, 8, 9, 10
; define array of 10 words initialized to 0
a2           dw   0, 0, 0, 0, 0, 0, 0, 0, 0, 0
; same as before using TIMES
a3           times 10 dw 0
; define array of bytes with 200 0's and then a 100 1's
a4           times 200 db 0
             times 100 db 1

segment .bss
; define an array of 10 uninitialized double words
a5           resd  10
; define an array of 100 uninitialized words
a6           resw  100
\end{AsmCodeListing}
\caption{Defining arrays\label{fig:DataArrays}}
\end{figure}

\subsubsection{Defining arrays in the {\code data} and {\code bss} segments
               \index{arrays!defining!static}}

To define an initialized array in the {\code data} segment, use the
normal {\code db}, {\code dw}, \emph{etc.}
\index{directive!D\emph{X}}directives. NASM also provides a useful directive
named {\code TIMES} \index{directive!TIMES} that can be used to repeat a statement many times
without having to duplicate the statements by hand.
Figure~\ref{fig:DataArrays} shows several examples of these.

To define an uninitialized array in the {\code bss} segment, use the
{\code resb}, {\code resw}, \emph{etc.} \index{directive!RES\emph{X}}
directives. Remember that these directives have an operand that
specifies how many units of memory to
reserve. Figure~\ref{fig:DataArrays} also shows examples of these
types of definitions.

\begin{figure}[t]
\centering
\begin{tabular}{l|c|ll|c|}
\cline{2-2} \cline{5-5}
EBP - 1  & char    & \hspace{2em} &           & \\
\cline{2-2}
         & unused  &              &           & \\
\cline{2-2}
EBP - 8  & dword 1 &              &           & \\
\cline{2-2}
EBP - 12 & dword 2 &              &           & word \\
\cline{2-2}
         &         &              &           & array \\
         &         &              &           & \\
         & word    &              &           & \\
         & array   &              & EBP - 100 & \\
\cline{5-5}
         &         &              & EBP - 104 & dword 1 \\
\cline{5-5}
         &         &              & EBP - 108 & dword 2 \\
\cline{5-5}
         &         &              & EBP - 109 & char \\
\cline{5-5}
EBP - 112 &        &              &           & unused \\
\cline{2-2} \cline{5-5}
\end{tabular}
\caption{Arrangements of the stack\label{fig:StackLayouts}}
\end{figure}

\subsubsection{Defining arrays as local variables on the stack\index{arrays!defining!local variable}}

There is no direct way to define a local array variable on the
stack. As before, one computes the total bytes required by \emph{all}
local variables, including arrays, and subtracts this from ESP (either
directly or using the {\code ENTER} instruction). For example, if a
function needed a character variable, two double word integers and a
50 element word array, one would need $1 + 2 \times 4 + 50 \times 2 =
109$ bytes. However, the number subtracted from ESP should be a
multiple of four (112 in this case) to keep ESP on a double word
boundary. One could arrange the variables inside this 109 bytes in
several ways. Figure~\ref{fig:StackLayouts} shows two possible ways. The
unused part of the first ordering is there to keep the double words on
double word boundaries to speed up memory accesses.
\index{arrays!defining|)}

\subsection{Accessing elements of arrays\index{arrays!accessing|(}}

There is no {\code [ ]} operator in assembly language as in C. To
access an element of an array, its address must be computed. Consider
the following two array definitions:
\begin{AsmCodeListing}[frame=none, numbers=none]
array1       db     5, 4, 3, 2, 1     ; array of bytes
array2       dw     5, 4, 3, 2, 1     ; array of words
\end{AsmCodeListing}
Here are some examples using this arrays:
\begin{AsmCodeListing}[frame=none]
      mov    al, [array1]             ; al = array1[0]
      mov    al, [array1 + 1]         ; al = array1[1]
      mov    [array1 + 3], al         ; array1[3] = al
      mov    ax, [array2]             ; ax = array2[0]
      mov    ax, [array2 + 2]         ; ax = array2[1] (NOT array2[2]!)
      mov    [array2 + 6], ax         ; array2[3] = ax
      mov    ax, [array2 + 1]         ; ax = ??
\end{AsmCodeListing}
In line~5, element 1 of the word array is referenced, not element 2. Why?
Words are two byte units, so to move to the next element of a word array,
one must move two bytes ahead, not one. Line~7 will read one byte from the
first element and one from the second. In C, the compiler looks at the type
of a pointer in determining how many bytes to move in an expression that
uses pointer arithmetic so that the programmer does not have to. However,
in assembly, it is up to the programmer to take the size of array elements
in account when moving from element to element.

\begin{figure}[t]
\begin{AsmCodeListing}[frame=single,commandchars=\\\{\}]
      mov    ebx, array1           ; ebx = address of array1
      mov    dx, 0                 ; dx will hold sum
      mov    ah, 0                 ; ?
      mov    ecx, 5
lp:
      mov    al, [ebx]             ; al = *ebx
      add    dx, ax                ; dx += ax (not al!) \label{line:SumArray1}
      inc    ebx                   ; bx++
      loop   lp
\end{AsmCodeListing}
\caption{Summing elements of an array (Version 1)\label{fig:SumArray1}}
\end{figure}

\begin{figure}[t]
\begin{AsmCodeListing}[frame=single,commandchars=\\\{\}]
      mov    ebx, array1           ; ebx = address of array1
      mov    dx, 0                 ; dx will hold sum
      mov    ecx, 5
lp:
\textit{      add    dl, [ebx]             ; dl += *ebx}
\textit{      jnc    next                  ; if no carry goto next}
\textit{      inc    dh                    ; inc dh}
\textit{next:}
      inc    ebx                   ; bx++
      loop   lp
\end{AsmCodeListing}
\caption{Summing elements of an array (Version 2)\label{fig:SumArray2}}
\end{figure}

\begin{figure}[t]
\begin{AsmCodeListing}[frame=single,commandchars=\\\{\}]
      mov    ebx, array1           ; ebx = address of array1
      mov    dx, 0                 ; dx will hold sum
      mov    ecx, 5
lp:
\textit{      add    dl, [ebx]             ; dl += *ebx}
\textit{      adc    dh, 0                 ; dh += carry flag + 0}
      inc    ebx                   ; bx++
      loop   lp
\end{AsmCodeListing}
\caption{Summing elements of an array (Version 3)\label{fig:SumArray3}}
\end{figure}

Figure~\ref{fig:SumArray1} shows a code snippet that adds all the
elements of {\code array1} in the previous example code. In
line~\ref{line:SumArray1}, AX is added to DX. Why not AL? First, the
two operands of the {\code ADD} instruction must be the same
size. Secondly, it would be easy to add up bytes and get a sum that
was too big to fit into a byte. By using DX, sums up to 65,535 are
allowed. However, it is important to realize that AH is being added
also.  This is why AH is set to zero\footnote{Setting AH to zero is
implicitly assuming that AL is an unsigned number. If it is signed,
the appropriate action would be to insert a {\code CBW} instruction
between lines~6 and 7} in line~3.

Figures~\ref{fig:SumArray2} and \ref{fig:SumArray3} show two alternative
ways to calculate the sum. The lines in italics replace lines~6 and 7
of Figure~\ref{fig:SumArray1}.

\subsection{More advanced indirect addressing\index{indirect addressing!arrays|(}}

Not surprisingly, indirect addressing is often used with arrays. The most
general form of an indirect memory reference is:
\begin{center}
{\code [ \emph{base reg} + \emph{factor}*\emph{index reg} + 
      \emph{constant}]}
\end{center}
where:
\begin{description}
\item[base reg] is one of the registers EAX, EBX, ECX, EDX, EBP, ESP, ESI
                or EDI.
\item[factor] is either 1, 2, 4 or 8. (If 1, factor is omitted.)
\item[index reg] is one of the registers EAX, EBX, ECX, EDX, EBP, ESI, EDI.
                 (Note that ESP is not in list.)
\item[constant] is a 32-bit constant. The constant can be a label (or
                a label expression).
\end{description}

\subsection{Example}
Here is an example that uses an array and passes it to a function. It uses the
{\code array1c.c} program (listed below) as a driver, not the 
{\code driver.c} program. \index{array1.asm|(}
\begin{AsmCodeListing}[label=array1.asm]
%define ARRAY_SIZE 100
%define NEW_LINE 10

segment .data
FirstMsg        db   "First 10 elements of array", 0
Prompt          db   "Enter index of element to display: ", 0
SecondMsg       db   "Element %d is %d", NEW_LINE, 0
ThirdMsg        db   "Elements 20 through 29 of array", 0
InputFormat     db   "%d", 0

segment .bss
array           resd ARRAY_SIZE

segment .text
        extern  _puts, _printf, _scanf, _dump_line
        global  _asm_main
_asm_main:
        enter   4,0		; local dword variable at EBP - 4
        push    ebx
        push    esi

; initialize array to 100, 99, 98, 97, ...

        mov     ecx, ARRAY_SIZE
        mov     ebx, array
init_loop:
        mov     [ebx], ecx
        add     ebx, 4
        loop    init_loop

        push    dword FirstMsg         ; print out FirstMsg
        call    _puts
        pop     ecx

        push    dword 10
        push    dword array
        call    _print_array           ; print first 10 elements of array
        add     esp, 8

; prompt user for element index
Prompt_loop:
        push    dword Prompt
        call    _printf
        pop     ecx

        lea     eax, [ebp-4]      ; eax = address of local dword
        push    eax
        push    dword InputFormat
        call    _scanf
        add     esp, 8
        cmp     eax, 1               ; eax = return value of scanf
        je      InputOK

        call    _dump_line  ; dump rest of line and start over
        jmp     Prompt_loop          ; if input invalid

InputOK:
        mov     esi, [ebp-4]
        push    dword [array + 4*esi]
        push    esi
        push    dword SecondMsg      ; print out value of element
        call    _printf
        add     esp, 12

        push    dword ThirdMsg       ; print out elements 20-29
        call    _puts
        pop     ecx

        push    dword 10
        push    dword array + 20*4     ; address of array[20]
        call    _print_array
        add     esp, 8

        pop     esi
        pop     ebx
        mov     eax, 0            ; return back to C
        leave                     
        ret

;
; routine _print_array
; C-callable routine that prints out elements of a double word array as
; signed integers.
; C prototype:
; void print_array( const int * a, int n);
; Parameters:
;   a - pointer to array to print out (at ebp+8 on stack)
;   n - number of integers to print out (at ebp+12 on stack)

segment .data
OutputFormat    db   "%-5d %5d", NEW_LINE, 0

segment .text
        global  _print_array
_print_array:
        enter   0,0
        push    esi
        push    ebx

        xor     esi, esi                  ; esi = 0
        mov     ecx, [ebp+12]             ; ecx = n
        mov     ebx, [ebp+8]              ; ebx = address of array
print_loop:
        push    ecx                       ; printf might change ecx!

        push    dword [ebx + 4*esi]       ; push array[esi]
        push    esi
        push    dword OutputFormat
        call    _printf
        add     esp, 12                   ; remove parameters (leave ecx!)

        inc     esi
        pop     ecx
        loop    print_loop

        pop     ebx
        pop     esi
        leave
        ret
\end{AsmCodeListing}

\LabelLine{array1c.c}
\begin{lstlisting}{}
#include <stdio.h>

int asm_main( void );
void dump_line( void );

int main()
{
  int ret_status;
  ret_status = asm_main();
  return ret_status;
}

/*
 * function dump_line
 * dumps all chars left in current line from input buffer
 */
void dump_line()
{
  int ch;

  while( (ch = getchar()) != EOF && ch != '\n')
    /* null body*/ ;
}
\end{lstlisting}
\LabelLine{array1c.c}
\index{array1.asm|)}
\index{indirect addressing!arrays|)}
\index{arrays!accessing|)}

\subsubsection{The {\code LEA} instruction revisited\index{LEA|(}}

The {\code LEA} instruction can be used for other purposes than just
calcuating addresses. A fairly
common one is for fast computations. Consider the following:
\begin{AsmCodeListing}[numbers=none,frame=none]
      lea    ebx, [4*eax + eax]
\end{AsmCodeListing}
This effectively stores the value of $5 \times \mathtt{EAX}$ into EBX.
Using {\code LEA} to do this is both easier and faster than using
{\code MUL}\index{MUL}. However, one must realize that the expression
inside the square brackets \emph{must} be a legal indirect address.
Thus, for example, this instruction can not be used to multiple by 6
quickly.
\index{LEA|)}


\subsection{Multidimensional Arrays\index{arrays!multidimensional|(}}

Multidimensional arrays are not really very different than the plain
one dimensional arrays already discussed. In fact, they are represented 
in memory as just that, a plain one dimensional array.

\subsubsection{Two Dimensional Arrays\index{arrays!multidimensional!two dimensional|(}}
Not surprisingly, the simplest multidimensional array is a two dimensional
one. A two dimensional array is often displayed as a grid of elements. Each
element is identified by a pair of indices. By convention, the first index
is identified with the row of the element and the second index the column.

Consider an array with three rows and two columns defined as: 
\begin{lstlisting}[stepnumber=0]{}
  int a[3][2];
\end{lstlisting}
The C compiler would reserve room for a 6 ($= 2 \times 3$) integer array and
map the elements as follows:

\parbox{\textwidth}{
\vspace{0.5em}
\centering
\begin{tabular}{||l|c|c|c|c|c|c||}
\hline
Index & 0 & 1 & 2 & 3 & 4 & 5 \\
\hline
Element & a[0][0] & a[0][1] & a[1][0] & a[1][1] & a[2][0] & a[2][1]  \\
\hline
\end{tabular}
\vspace{0.5em}
}
\noindent What the table attempts to show is that the element referenced as 
{\code a[0][0]} is stored at the beginning of the 6 element one
dimensional array. Element {\code a[0][1]} is stored in the next
position (index~1) and so on. Each row of the two dimensional array is
stored contiguously in memory. The last element of a row is followed
by the first element of the next row. This is known as the
\emph{rowwise} representation of the array and is how a C/C++ compiler would
represent the array.

\begin{figure}[t]
\begin{AsmCodeListing}[]
   mov    eax, [ebp - 44]          ; ebp - 44 is i's location
   sal    eax, 1                   ; multiple i by 2
   add    eax, [ebp - 48]          ; add j
   mov    eax, [ebp + 4*eax - 40]  ; ebp - 40 is the address of a[0][0]
   mov    [ebp - 52], eax          ; store result into x (at ebp - 52)
\end{AsmCodeListing}
\caption{ Assembly for \lstinline|x = a[i][j]| \label{fig:aij}}
\end{figure}

How does the compiler determine where {\code a[i][j]} appears in the rowwise
representation? A simple formula will compute the index from {\code i} and
{\code j}. The formula in this case is $2i + j$. It's not too hard to see how
this formula is derived. Each row is two elements long; so, the first element
of row $i$ is at position $2i$. Then the position of column $j$ is found by
adding $j$ to $2i$. This analysis also shows how the formula is generalized 
to an array with {\code N} columns: $N \times i + j$. Notice that the formula
does \emph{not} depend on the number of rows.

As an example, let us see how \emph{gcc} compiles the following code (using the
array {\code a} defined above):
\begin{lstlisting}[stepnumber=0]{}
  x = a[i][j];
\end{lstlisting}
Figure~\ref{fig:aij} shows the assembly this is translated into.
Thus, the compiler essentially converts the code to:
\begin{lstlisting}[stepnumber=0]{}
  x = *(&a[0][0] + 2*i + j);
\end{lstlisting}
and in fact, the programmer could write this way with the same result.

There is nothing magical about the choice of the rowwise representation of the
array. A columnwise representation would work just as well: 

\parbox{\textwidth}{
\vspace{0.5em}
\centering
\begin{tabular}{||l|c|c|c|c|c|c||}
\hline
Index & 0 & 1 & 2 & 3 & 4 & 5 \\
\hline
Element & a[0][0] & a[1][0] & a[2][0] & a[0][1] & a[1][1] & a[2][1]  \\
\hline
\end{tabular}
\vspace{0.5em}
}
\noindent In the columnwise representation, each column is stored contiguously. 
Element {\code [i][j]} is stored at position $i + 3j$. Other languages
(FORTRAN, for example) use the columnwise representation. This is
important when interfacing code with multiple languages.
\index{arrays!multidimensional!two dimensional|)}

\subsubsection{Dimensions Above Two}
For dimensions above two, the same basic idea is applied. Consider a three
dimensional array:
\begin{lstlisting}[stepnumber=0]{}
  int b[4][3][2];
\end{lstlisting}
This array would be stored like it was four two dimensional arrays each of size
{\code [3][2]} consecutively in memory. The table below shows how it starts out:

\parbox{\textwidth}{
\vspace{0.5em}
\centering
\begin{tabular}{||l|c|c|c|c|c|c||}
\hline
Index & 0 & 1 & 2 & 3 & 4 & 5  \\
\hline
Element & b[0][0][0] & b[0][0][1]  & b[0][1][0] & b[0][1][1] & b[0][2][0]
&  b[0][2][1]  \\
\hline
\hline
Index & 6 & 7 & 8 & 9 & 10 & 11 \\
\hline
Element & b[1][0][0] & b[1][0][1] & b[1][1][0] & b[1][1][1]  & b[1][2][0] 
& b[1][2][1] \\
\hline
\end{tabular}
\vspace{0.5em}
}
\noindent The formula for computing the position of {\code b[i][j][k]}
is $6i + 2j + k$. The 6 is determined by the size of the {\code
[3][2]} arrays. In general, for an array dimensioned as {\code
a[L][M][N]} the position of element {\code a[i][j][k]} will be $M\times N\times i 
+ N \times j + k$. Notice again that the first
dimension ({\code L}) does not appear in the formula.

For higher dimensions, the same process is generalized. For an $n$ dimensional
array of dimensions $D_1$ to $D_n$, the position of element denoted by the
indices $i_1$ to $i_n$ is given by the formula:
\begin{displaymath}
D_2 \times D_3 \cdots \times D_n \times i_1 + D_3 \times D_4 \cdots \times D_n 
\times i_2 + \cdots + D_n \times i_{n-1} + i_n
\end{displaymath}
or for the \"{u}ber math geek, it can be written more succinctly as:
\begin{displaymath}
\sum_{j=1}^{n} \: \left( \prod_{k=j+1}^{n} D_k \right) \: i_j
\end{displaymath}
\MarginNote{This is where you can tell the author was a physics major. (Or was the
reference to FORTRAN a giveaway?)}
The first dimension, $D_1$, does not appear in the formula.

For the columnwise representation, the general formula would be:
\begin{displaymath}
i_1 + D_1 \times i_2 + \cdots + D_1 \times D_2 \times \cdots \times D_{n-2} 
\times i_{n-1} + D_1 \times D_2 \times \cdots \times D_{n-1} \times i_n
\end{displaymath}
or in \"{u}ber math geek notation:
\begin{displaymath}
\sum_{j=1}^{n} \: \left( \prod_{k=1}^{j-1} D_k \right) \: i_j
\end{displaymath}
In this case, it is the last dimension, $D_n$, that does not appear in the
formula.

\subsubsection{Passing Multidimensional Arrays as Parameters in C\index{arrays!multidimensional!parameters|(}}

The rowwise representation of multidimensional arrays has a direct
effect in C programming. For one dimensional arrays, the size of the
array is not required to compute where any specific element is located
in memory. This is not true for multidimensional arrays.  To access
the elements of these arrays, the compiler must know all but the first
dimension. This becomes apparent when considering the prototype of a
function that takes a multidimensional array as a parameter. The
following will not compile:
\begin{lstlisting}[stepnumber=0]{}
  void f( int a[ ][ ] );  /* no dimension information */
\end{lstlisting}
However, the following does compile:
\begin{lstlisting}[stepnumber=0]{}
  void f( int a[ ][2] );
\end{lstlisting}
Any two dimensional array with two columns can be passed to this function.
The first dimension is not required\footnote{A size can be specified here,
but it is ignored by the compiler.}.

Do not be confused by a function with this prototype:
\begin{lstlisting}[stepnumber=0]{}
  void f( int * a[ ] );
\end{lstlisting}
This defines a single dimensional array of integer pointers (which incidently
can be used to create an array of arrays that acts much like a two dimensional
array).

For higher dimensional arrays, all but the first dimension of the array must
be specified for parameters. For example, a four dimensional array parameter
might be passed like:
\begin{lstlisting}[stepnumber=0]{}
  void f( int a[ ][4][3][2] );
\end{lstlisting}
\index{arrays!multidimensional!parameters|)}
\index{arrays!multidimensional|)}

\section{Array/String Instructions}
\index{string instructions|(} 

The 80x86 family of processors provide several instructions that are
designed to work with arrays. These instructions are called
\emph{string instructions}. They use the index registers (ESI and EDI)
to perform an operation and then to automatically increment or
decrement one or both of the index registers. The \emph{direction
flag} (DF) \index{register!FLAGS!DF} in the FLAGS register determines
where the index registers are incremented or decremented. There are
two instructions that modify the direction flag:
\begin{description}
\item[CLD] \index{CLD} clears the direction flag. In this state, the index registers
           are incremented.
\item[STD] \index{STD} sets the direction flag. In this state, the index registers are
           decremented.
\end{description}
A \emph{very} common mistake in 80x86 programming is to forget to explicitly
put the direction flag in the correct state. This often leads to code that
works most of the time (when the direction flag happens to be in the desired
state), but does not work \emph{all} the time.

\begin{figure}[t]
\centering
{\code
\begin{tabular}{|lp{1.5in}|lp{1.5in}|}
\hline
LODSB & AL = [DS:ESI]\newline ESI = ESI $\pm$ 1 & 
STOSB & [ES:EDI] = AL\newline EDI = EDI $\pm$ 1 \\
\hline
LODSW & AX = [DS:ESI]\newline ESI = ESI $\pm$ 2 & 
STOSW & [ES:EDI] = AX\newline EDI = EDI $\pm$ 2 \\
\hline
LODSD & EAX = [DS:ESI]\newline ESI = ESI $\pm$ 4 & 
STOSD & [ES:EDI] = EAX\newline EDI = EDI $\pm$ 4 \\
\hline
\end{tabular}
}
\caption{Reading and writing string instructions\label{fig:rwString}
         \index{LODSB} \index{STOSB} \index{LODSW} \index{LODSD} \index{STOSD}}
\end{figure}

\begin{figure}[t]
\begin{AsmCodeListing}[frame=single]
segment .data
array1  dd  1, 2, 3, 4, 5, 6, 7, 8, 9, 10

segment .bss
array2  resd 10

segment .text
      cld                   ; don't forget this!
      mov    esi, array1
      mov    edi, array2
      mov    ecx, 10
lp:
      lodsd
      stosd
      loop  lp
\end{AsmCodeListing}
\caption{Load and store example\label{fig:lodEx}}
\end{figure}

\subsection{Reading and writing memory}

The simplest string instructions either read or write memory or
both. They may read or write a byte, word or double word at a time.
Figure~\ref{fig:rwString} shows these instructions with a short
pseudo-code description of what they do. There are several points to
notice here. First, ESI is used for reading and EDI for writing. It is
easy to remember this if one remembers that SI stands for \emph{Source
Index} and DI stands for \emph{Destination
Index}. \index{register!ESI} \index{register!EDI} Next, notice that
the register that holds the data is fixed (either AL, AX or
EAX). Finally, note that the storing instructions use ES to detemine
the segment to write to, not DS. In protected mode programming this is
not usually a problem, since there is only one data segment and ES
should be automatically initialized to reference it (just as DS
is). However, in real mode programming, it is \emph{very} important
for the programmer to initialize ES to the correct segment
\index{register!segment} selector value\footnote{Another complication
is that one can not copy the value of the DS register into the ES
register directly using a single {\code MOV} instruction. Instead, the
value of DS must be copied to a general purpose register (like AX) and
then copied from that register to ES using two {\code MOV}
instructions.}. Figure~\ref{fig:lodEx} shows an example use of these
instructions that copies an array into another.

\begin{figure}[t]
\centering
{\code
\begin{tabular}{|lp{2.5in}|}
\hline
MOVSB & byte [ES:EDI] = byte [DS:ESI] \newline ESI = ESI $\pm$ 1 \newline
        EDI = EDI $\pm$ 1 \\
\hline
MOVSW & word [ES:EDI] = word [DS:ESI] \newline ESI = ESI $\pm$ 2 \newline
        EDI = EDI $\pm$ 2 \\
\hline
MOVSD & dword [ES:EDI] = dword [DS:ESI] \newline ESI = ESI $\pm$ 4 \newline
        EDI = EDI $\pm$ 4 \\
\hline
\end{tabular}
}
\caption{Memory move string instructions\label{fig:movString} \index{MOVSB}
         \index{MOVSW} \index{MOVSD}}
\end{figure}

\begin{figure}[t]
\begin{AsmCodeListing}[frame=single]
segment .bss
array  resd 10

segment .text
      cld                   ; don't forget this!
      mov    edi, array
      mov    ecx, 10
      xor    eax, eax
      rep stosd
\end{AsmCodeListing}
\caption{Zero array example\label{fig:zeroArrayEx}}
\end{figure}

The combination of a {\code LODSx} and {\code STOSx} instruction (as in
lines~13 and 14 of Figure~\ref{fig:lodEx}) is very common. In fact, this
combination can be performed by a single {\code MOVSx} string instruction.
Figure~\ref{fig:movString} describes the operations that these 
instructions perform. Lines~13 and 14 of Figure~\ref{fig:lodEx} could be
replaced with a single {\code MOVSD} instruction with the same effect. The
only difference would be that the EAX register would not be used at all
in the loop.

\subsection{The {\code REP} instruction prefix\index{REP|(}}

The 80x86 family provides a special instruction prefix\footnote{A
instruction prefix is not an instruction, it is a special byte that is
placed before a string instruction that modifies the instructions
behavior. Other prefixes are also used to override segment defaults of
memory accesses} called {\code REP} that can be used with the above string
instructions. This prefix tells the CPU to repeat the next string instruction
a specified number of times. The ECX register is used to count the iterations
(just as for the {\code LOOP} instruction). Using the {\code REP} prefix, 
the loop in Figure~\ref{fig:lodEx} (lines~12 to 15) could be replaced with
a single line:
\begin{AsmCodeListing}[frame=none, numbers=none]
      rep movsd
\end{AsmCodeListing}
Figure~\ref{fig:zeroArrayEx} shows another example that zeroes out the
contents of an array.
\index{REP|)}

\begin{figure}[t]
\centering
{\code
\begin{tabular}{|lp{3.5in}|}
\hline
CMPSB & compares byte [DS:ESI] and byte [ES:EDI] \newline ESI = ESI $\pm$ 1 
        \newline EDI = EDI $\pm$ 1 \\
\hline
CMPSW & compares word [DS:ESI] and word [ES:EDI] \newline ESI = ESI $\pm$ 2 
        \newline EDI = EDI $\pm$ 2 \\
\hline
CMPSD & compares dword [DS:ESI] and dword [ES:EDI] \newline ESI = ESI $\pm$ 4 
        \newline EDI = EDI $\pm$ 4 \\
\hline
SCASB & compares AL and [ES:EDI] \newline EDI $\pm$ 1 \\
\hline
SCASW & compares AX and [ES:EDI] \newline EDI $\pm$ 2 \\
\hline
SCASD & compares EAX and [ES:EDI] \newline EDI $\pm$ 4 \\
\hline
\end{tabular}
}
\caption{Comparison string instructions\label{fig:cmpString}
         \index{CMPSB} \index{CMPSW} \index{CMPSD} \index{SCASB}
         \index{SCASW} \index{SCASD}}
\end{figure}

\begin{figure}[t]
\begin{AsmCodeListing}[frame=single,commandchars=\\\{\}]
segment .bss
array        resd 100

segment .text
      cld
      mov    edi, array    ; pointer to start of array
      mov    ecx, 100      ; number of elements
      mov    eax, 12       ; number to scan for
lp:
      scasd    \label{line:scasdSrchStrEx}
      je     found
      loop   lp
 ; code to perform if not found
      jmp    onward
found:
      sub    edi, 4         ; edi now points to 12 in array\label{line:subSrchStrEx}
 ; code to perform if found
onward:
\end{AsmCodeListing}
\caption{Search example\label{fig:srchStrEx}}
\end{figure}

\subsection{Comparison string instructions}

Figure~\ref{fig:cmpString} shows several new string instructions that
can be used to compare memory with other memory or a register. They
are useful for comparing or searching arrays. They set the FLAGS
register just like the {\code CMP} instruction. The {\code CMPSx}
\index{CMPSB} \index{CMPSW} \index{CMPSD} instructions compare
corresponding memory locations and the {\code SCASx} \index{SCASB}
\index{SCASW} \index{SCASD} scan memory locations for a specific
value.

Figure~\ref{fig:srchStrEx} shows a short code snippet that searches
for the number 12 in a double word array. The {\code SCASD} instruction on
line~\ref{line:scasdSrchStrEx} always adds 4 to EDI, even if the value
searched for is found. Thus, if one wishes to find the address of the 12
found in the array, it is necessary to subtract 4 from EDI (as 
line~\ref{line:subSrchStrEx} does).

\begin{figure}[t]
\centering
\begin{tabular}{|l|p{4in}|}
\hline
{\code REPE}, {\code REPZ} & repeats instruction while Z flag is set or
                             at most ECX times \\
\hline
{\code REPNE}, {\code REPNZ} & repeats instruction while Z flag is cleared or
                             at most ECX times \\
\hline
\end{tabular}
\caption{{\code REPx} instruction prefixes\label{fig:repx} \index{REPE} 
          \index{REPNE}}
\end{figure}

\begin{figure}
\begin{AsmCodeListing}[frame=single,commandchars=\\\{\}]
segment .text
      cld
      mov    esi, block1        ; address of first block
      mov    edi, block2        ; address of second block
      mov    ecx, size          ; size of blocks in bytes
      repe   cmpsb              ; repeat while Z flag is set
      je     equal              ; if Z set, blocks equal \label{line:cmpBlocksEx}
   ; code to perform if blocks are not equal
      jmp    onward
equal:
   ; code to perform if equal
onward:
\end{AsmCodeListing}
\caption{Comparing memory blocks\label{fig:cmpBlocksEx}}
\end{figure}

\subsection{The {\code REPx} instruction prefixes}

There are several other {\code REP}-like instruction prefixes that can be
used with the comparison string instructions. Figure~\ref{fig:repx} shows
the two new prefixes and describes their operation. {\code REPE} \index{REPE} and
{\code REPZ} are just synonyms for the same prefix (as are {\code REPNE} \index{REPNE}
and {\code REPNZ}). If the repeated comparison string instruction stops
because of the result of the comparison, the index register or registers
are still incremented and ECX decremented; however, the FLAGS register
still holds the state that terminated the repetition. 
\MarginNote{Why can one not just look to see if ECX is zero after the
repeated comparison?} Thus, it is possible
to use the Z flag to determine if the repeated comparisons stopped because
of a comparison or ECX becoming zero.

Figure~\ref{fig:cmpBlocksEx} shows an example code snippet that determines
if two blocks of memory are equal. The {\code JE} on 
line~\ref{line:cmpBlocksEx} of the example checks to see the result of the
previous instruction. If the repeated comparison stopped because it found
two unequal bytes, the Z flag will still be cleared and no branch is made;
however, if the comparisons stopped because ECX became zero, the Z flag
will still be set and the code branches to the {\code equal} label.

\subsection{Example}

This section contains an assembly source file with several functions that
implement array operations using string instructions. Many of the functions
duplicate familiar C library functions.

\index{memory.asm|(}
\begin{AsmCodeListing}[label=memory.asm]
global _asm_copy, _asm_find, _asm_strlen, _asm_strcpy

segment .text
; function _asm_copy
; copies blocks of memory
; C prototype
; void asm_copy( void * dest, const void * src, unsigned sz);
; parameters:
;   dest - pointer to buffer to copy to
;   src  - pointer to buffer to copy from
;   sz   - number of bytes to copy

; next, some helpful symbols are defined

%define dest [ebp+8]
%define src  [ebp+12]
%define sz   [ebp+16]
_asm_copy:
        enter   0, 0
        push    esi
        push    edi

        mov     esi, src        ; esi = address of buffer to copy from
        mov     edi, dest       ; edi = address of buffer to copy to
        mov     ecx, sz         ; ecx = number of bytes to copy

        cld                     ; clear direction flag 
        rep     movsb           ; execute movsb ECX times

        pop     edi
        pop     esi
        leave
        ret


; function _asm_find
; searches memory for a given byte
; void * asm_find( const void * src, char target, unsigned sz);
; parameters:
;   src    - pointer to buffer to search
;   target - byte value to search for
;   sz     - number of bytes in buffer
; return value:
;   if target is found, pointer to first occurrence of target in buffer
;   is returned
;   else
;     NULL is returned
; NOTE: target is a byte value, but is pushed on stack as a dword value.
;       The byte value is stored in the lower 8-bits.
; 
%define src    [ebp+8]
%define target [ebp+12]
%define sz     [ebp+16]

_asm_find:
        enter   0,0
        push    edi

        mov     eax, target     ; al has value to search for
        mov     edi, src
        mov     ecx, sz
        cld

        repne   scasb           ; scan until ECX == 0 or [ES:EDI] == AL

        je      found_it        ; if zero flag set, then found value
        mov     eax, 0          ; if not found, return NULL pointer
        jmp     short quit
found_it:
        mov     eax, edi          
        dec     eax              ; if found return (DI - 1)
quit:
        pop     edi
        leave
        ret


; function _asm_strlen
; returns the size of a string
; unsigned asm_strlen( const char * );
; parameter:
;   src - pointer to string
; return value:
;   number of chars in string (not counting, ending 0) (in EAX)

%define src [ebp + 8]
_asm_strlen:
        enter   0,0
        push    edi

        mov     edi, src        ; edi = pointer to string
        mov     ecx, 0FFFFFFFFh ; use largest possible ECX
        xor     al,al           ; al = 0
        cld

        repnz   scasb           ; scan for terminating 0

;
; repnz will go one step too far, so length is FFFFFFFE - ECX,
; not FFFFFFFF - ECX
;
        mov     eax,0FFFFFFFEh
        sub     eax, ecx        ; length = 0FFFFFFFEh - ecx

        pop     edi
        leave
        ret

; function _asm_strcpy
; copies a string
; void asm_strcpy( char * dest, const char * src);
; parameters:
;   dest - pointer to string to copy to
;   src  - pointer to string to copy from
; 
%define dest [ebp + 8]
%define src  [ebp + 12]
_asm_strcpy:
        enter   0,0
        push    esi
        push    edi

        mov     edi, dest
        mov     esi, src
        cld
cpy_loop:
        lodsb                   ; load AL & inc si
        stosb                   ; store AL & inc di
        or      al, al          ; set condition flags
        jnz     cpy_loop        ; if not past terminating 0, continue

        pop     edi
        pop     esi
        leave
        ret
\end{AsmCodeListing}

\LabelLine{memex.c}
\begin{lstlisting}{}
#include <stdio.h>

#define STR_SIZE 30
/* prototypes */

void asm_copy( void *, const void *, unsigned ) __attribute__((cdecl));
void * asm_find( const void *, 
                 char target, unsigned ) __attribute__((cdecl));
unsigned asm_strlen( const char * ) __attribute__((cdecl));
void asm_strcpy( char *, const char * ) __attribute__((cdecl));

int main()
{
  char st1[STR_SIZE] = "test string";
  char st2[STR_SIZE];
  char * st;
  char   ch;

  asm_copy(st2, st1, STR_SIZE);   /* copy all 30 chars of string */
  printf("%s\n", st2);

  printf("Enter a char: ");  /* look for byte in string */
  scanf("%c%*[^\n]", &ch);
  st = asm_find(st2, ch, STR_SIZE);
  if ( st )
    printf("Found it: %s\n", st);
  else
    printf("Not found\n");

  st1[0] = 0;
  printf("Enter string:");
  scanf("%s", st1);
  printf("len = %u\n", asm_strlen(st1));

  asm_strcpy( st2, st1);     /* copy meaningful data in string */
  printf("%s\n", st2 );

  return 0;
}
\end{lstlisting}
\LabelLine{memex.c}
\index{memory.asm|)}
\index{string instructions|)}
\index{arrays|)}














% -*-latex-*-
\chapter{부동 소수점}\index{부동 소수점|(}

\section{부동 소수점 표현}\index{부동 소수점!표현|(}

\subsection{정수가 아닌 이진수}

우리가 첫 번째 장에서 기수법에 대해 이야기 하였을 때, 오직 정수들에 대해서만 이야기 하였다.
그러나, 명백히도 십진법과 같이 다른 진법 체계에서도 정수가 아닌 수를 표현할 수 있다. 십진법에선
소수점 다음으로 나타나는 숫자들은 10 에 음수 승을 취하게 된다. 

\[ 0.123 = 1 \times 10^{-1} + 2 \times 10^{-2} + 3 \times 10^{-3} \]

물론, 이진수들에 대해서도 동일하게 적용된다. 
\[ 0.101_2 = 1 \times 2^{-1} + 0 \times 2^{-2} + 1 \times 2^{-3} = 0.625 \]

~1 장에서 정수 이진수를 십진수로 바꾸는데 사용했던 방법을 동일하게 적용하여 아래와 같이
정수가 아닌 이진수를 십진수로 바꿀 수 있다. 

\[ 110.011_2 = 4 + 2 + 0.25 + 0.125 = 6.375 \]

십진수를 이진수로 바꾸는 것도 크게 어렵진 않다. 보통, 십진수를 두 개의 부분으로 나눈다:
정수 부분과 소수 부분. 정수 부분은 ~1 장에서 다루었던 방법으로 이진수로 바꾼다. 소수 부분은
아래의 설명한 방법을 통해 이진수로 변경될 수 있다. 

\begin{figure}[t]
\centering
\fbox{
\begin{tabular}{p{2in}p{2in}}
\begin{eqnarray*}
0.5625 \times 2 & = & 1.125 \\
0.125 \times 2 & = & 0.25 \\
0.25 \times 2 & = & 0.5 \\
0.5 \times 2 & = & 1.0 \\
\end{eqnarray*}
&
\begin{eqnarray*}
\mbox{first bit} & = & 1 \\
\mbox{second bit} & = & 0 \\
\mbox{third bit} & = & 0 \\
\mbox{fourth bit} & = & 1 \\
\end{eqnarray*}
\end{tabular}
}
\caption{0.5625 를 이진수로 바꾸기\label{fig:binConvert1}}
\end{figure}

예를 들어서 각각의 비트를 $a,b,c, \ldots$ 로 나타낸 이진수를 생각하자. 
이는 아래와 같이 나타난다:

\[ 0.abcdef\ldots \]
위 수에 2 를 곱한 수는 아래와 같이 이진법으로 나타난다. 

\[ a.bcdef\ldots \]
첫 번째 비트가 이제 1 의 자리에 있음을 유의해라. $a$ 를 $0$ 으로 바꾼다면
\[ 0.bcdef\ldots \]
그리고 2 를 다시 곱한다면 
\[ b.cdef\ldots \]
이제, 두 번째 비트 ($b$) 가 1 의 자리에 오게 된다. 이 방법을 여러번 반복하여
필요한 비트 수 만큼까지 값을 찾을 수 있다. 그림 ~\ref{fig:binConvert1} 는
0.5625 를 이진수로 변환하는 예를 보여주고 있다. 이 방법은 소수 부분의 값이
0 이 될 때 중단을 한다. 

\begin{figure}[t]
\centering
\fbox{\parbox{2in}{
\begin{eqnarray*}
0.85 \times 2 & = & 1.7 \\
0.7 \times 2 & = &  1.4 \\
0.4 \times 2 & = &  0.8 \\
0.8 \times 2 & = &  1.6 \\
0.6 \times 2 & = &  1.2 \\
0.2 \times 2 & = &  0.4 \\
0.4 \times 2 & = &  0.8 \\
0.8 \times 2 & = &  1.6 \\
\end{eqnarray*}
}}
\caption{0.85 를 이진수로 바꾸기\label{fig:binConvert2}}
\end{figure}

또다른 예제에선 23.85 를 이진수로 바꾸었다. 정수 부분은
이진수로 바꾸기가 매우 쉽다 ($23 = 10111_2$). 하지만 소수 부분 ($0.85$)은
어떨까? 그림 ~\ref{fig:binConvert2} 는 소수 부분을 이진수로 바꾸는
처음 몇 부분의 과정을 보여주고 있다. 사실, 이 수를 완전히 이진수로 바꾸기 위해선
무한번의 연산이 필요하다. 이 말은 0.85 는 순환 이진 소수라는 것이다.
\footnote{이는 전혀 놀라운 일이 아닌데, 어떤 수는 특정한 진법을 통해 
표현한다면 무한 소수가 될 수 있지만 다른 진법에서는 유한 소수가 될 수 있다.
예를 들어서 $\frac{1}{3}$ 을 생각해 보면 십진법으로 표현시 3 이 무한히 반복 되지만
3 진법으로 표현한다면 $0.1_3$ 과 같이 깔끔하게 나타날 수 있다.}
위 계산에서 나타나는 숫자들은 특정한 규칙들이 있다. 이를 잘 살펴본다면 
$0.85 = 0.11\overline{0110}_2$ 임을 알 수 있다. 따라서, $23.85 = 10111.11\overline{0110}_2$
이다. 

따라서, 위 계산 결과를 통해 알 수 있는 점은 23.85 는 유한 개의 이진 비트를 사용하여
수를 \emph{정확히} 나타낼 수 없다라는 점이다. (이는 $\frac{1}{3}$ 이 십진법으로
유한개의 자리수를 가지는 소수로 정확하게 나타낼 수 없는 점과 같다) 이 장에서는
C 에서의 {\code float} 과 {\code double} 변수들이 이진수로 어떻게 저장되는지 살펴 볼 것이다.
따라서 23.85 와 같은 값들은 이 변수에 정확하게 저장될 수 없다. 오직 23.85 의 근사값 만이
변수에 저장될 것이다. 

하드웨어를 단순화 하기 위해 부동 소수점 수들은 일정한 형식으로 저장된다. 이 형식은
과학적 기수법 표현 (다만 이진수에서는 10 의 멱수를 사용하지 않고 2 의 멱수를 사용한다)
을 사용한다. 예를 들어 23.85, 혹은 $10111.11011001100110\ldots_2$ 은 다음과 같이
저장된다. 

\[ 1.011111011001100110\ldots \times 2^{100} \]
(지수 (100) 도 이진수로 나타난다.). \emph{정규화} 된 부동 소수점 수는
아래의 같은 꼴을 가진다. 

\[ 1.ssssssssssssssss \times 2^{eeeeeee} \]
이 때 $1.sssssssssssss$ 은\emph{가수(significand)} 라 하고 $eeeeeeee$ 은 
\emph{지수(exponent)} 라 한다. 

\subsection{IEEE 부동 소수점 표현}\index{부동 소수점!표현!IEEE|(}

IEEE (전기 전자 기술자 협회, Institute of Electrical and Electronic Engineers) 는 
부동 소수점 수들을 저장하기 위한 구체적인 이진 형식을 만든 국제적인 조직이다.
이 형식은 오늘날 만들어진 대부분 (모든 컴퓨터는 아니다!) 의 컴퓨터에서 사용된다. 
많은 경우 이 형식은 컴퓨터 하드웨어 자체에서 지원된다. 예를 들어 인텔의 수치
보조처리기(numeric coprocessor) 이다. (이는 Pentium 부터 모든 인텔 CPU 들에게
추가되었다) IEEE 는 정밀도가 다른 두 개의 형식을 지원한다. 이는 단일 정밀도(single precision)
와 2배 정밀도(double precision) 이다. 단일 정밀도는 C 언어의 {\code float} 변수들이
사용하며 2배 정밀도는 {\code double} 변수들이 사용한다.

인털의 수치 보조처리기는 또한 더 높은 정밀도를 가지는 \emph{확장 정밀도(extended precision)}를
사용한다. 사실, 보조처리기에 있는 모든 데이터들은 이 정밀도로 표현된다. 보조처리기는
이를 메모리에 저장할 때에 단일 또는 2배 정밀도를 가지는 수들로 자동으로 변환한다.
\footnote{ 일부 컴파일러 (예를 들어 Borland) 의 {\code long double} 형은 이 확장 정밀도를
이용한다. 그러나 대부분의 경우 {\code double} 과 {\code long double} 형 모두 
2배 정밀도를 사용한다. (이는 ANSI C 에서 허용된다)} 확장 정밀도는 IEEE 의 float 이나 double에 비해
다른 형식을 사용하며 여기서는 이야기 하지 않도록 하겠다. 

\subsubsection{IEEE 단일 정밀도}\index{부동 소수점!표현!단일 정밀도|(}

\begin{figure}[t]
\fbox{
\centering
\parbox{5in}{
\begin{tabular}{|c|c|c|}
\multicolumn{1}{p{0.3cm}}{31} &
\multicolumn{1}{p{2.5cm}}{30 \hfill 23} &
\multicolumn{1}{p{6cm}}{22 \hfill 0} \\
\hline
s & e & f \\
\hline
\end{tabular}
\\[0.4cm]
\begin{tabular}{cp{4.5in}}
s & 부호 비트 - 0 = 양수, 1 = 음수 \\
e & 편향 지수(8 비트) = 실제 지수값 + 7F (십진수로 127). 
    지수의 값이 00 과 FF 일 때는 특별한 의미를 가진다.  \\
f & 가수 - 1. 다음의 처음 23 비트 
\end{tabular}
}}
\caption{IEEE 단일 정밀도\label{fig:IEEEsingle}}
\end{figure}

단일 정밀도 부동 소수점 형식은 수를 32 비트로 표현한다. 이는 대체로 십진수로 7 자리 정도
정확하다. 부동 소수점 수들은 정수들에 비해 훨씬 복접한 형식으로 저장된다.
그림 ~\ref{fig:IEEEsingle} 은 IEEE 단일 정밀도 수의 기본적인 형식을 보여준다. 
이 형식에선 약간의 이상한 점이 있다. 부동 소수점 수들은 음수를 위해 2 의 보수 표현법을 사용하지
않는다. 이들은 부호 있는 크기 표현(signed magnitude) 을 사용한다. 31 째 비트는 위에 나와있듯이
수의 부호를 정한다. 

이진 지수는 그 값 그대로 저장되지 않는다. 그 대신 7F 를 더해서 23 ~ 30 번째 비트에 더해진다.
따라서 이 \emph{편향 된 지수(biased exponent)} 는 언제나 음수가 아니다. 

소수 부분은 언제나 정규화 된 가수 부분이라 생각한다 ($1.sssssssss$ 형식으로). 언제나 첫 번째 비트가
1 이므로, 1 은 \emph{저장되지 않는다}. 이를 통해 비트 하나를 추가적으로 저장할 수 있게 되어 정확성을
향상 시킬 수 있다. 이 아이디어는 \emph{숨겨진 1 표현법(hidden one representation)} 이라 한다.\index{부동 소수점!표현!숨겨진 1}

그렇다면 23.85 는 어떻게 저장될까? 
 \MarginNote{언제나 명심해 둘 것은 프로그램이 41 BE CC CD 를 어떻게 해석하느냐에 따라
 그 값이 달라질 수 있다는 사실이다! 예를 들어 단일 정밀도 부동 소수점 수라고 생각한다면 
 이 값은 23.850000381 을 나타나게 된다. 하지만 더블워드 정수라고 생각한다면 이는 
1,103,023,309 를 의미한다. CPU 는 어는 것이 이 수에 대한 정확한 표현인지 알 수 없다. }
먼저 이 수는 양수이므로 부호 비트는 0 이다. 지수 값은 4 이므로, 편향된 지수 값은
$7\mathrm{F} + 4 = 83_{16}$ 가 된다. 마지막으로 가수 부분은 01111101100110011001100 이다.
(맨 처음의 1 은 숨겨져 있다는 사실을 명심해라) 이를 모두 합치면 (참고로 이해를 돕기 위해서
부호 비트와 가수 부분은 밑줄이 처져 있고, 비트 들은 4 비트 니블들로 나뉘어 나타냈다)
% 오타 faction -> fraction
\[ \underline{0}\,100\;0001\;1
   \,\underline{011\;1110\;1100\;1100\;1100\;1100}_2 = 41 \mathrm{BE} 
\mathrm{CC} \mathrm{CC}_{16} \]
이는 23.85 를 정확하게 표현한 것은 아니다 (왜냐하면 이진법에서 순환소수로 나타나기 때문).
만일 위 수를 십진수로 바꾼다면 그 값은 대략 23.849998474 가 될 것이다. 이 값은
23.85 와 매우 가깝지만 정확히 같지는 않다. 사실, C 에선 23.85 를 위와 같이 나타내지 않는다.
C 에선, 정확한 이진 표현에서 단일 정밀도 형식으로 변환 시, 마지막으로 잘라낸 비트가 1 이므로 마지막 비트는 1 로 반올림 된다. 
따라서 23.85 는 41 BE CC CD 로 나타나게 된다. 이는 십진수로 바꾸면 23.850000381 이므로 좀더
23.85 를 가깝게 표현할 수 있다.  

-23.85 는 어떻게 저장될까? 단순히 부호 비트만 바꾸면 된다. 따라서 C1 BE CC CD 가 된다. 
2의 보수 표현법을 취하지 \emph{않는다}!

\begin{table}[t]
\fbox{
\begin{tabular}{lp{3.1in}}
$e=0 \quad\mathrm{,}\quad f=0$ & 이면 0 을 나타낸다. (이는 정규화 
                         될 수 없다)  +0 와 -0 가 있음을 유의해라. \\
$e=0 \quad\mathrm{,}\quad f \neq 0$ & 는 \emph{비정규화 된 수} 임을
                              나타낸다. 이것은 다음 단원에서 다룬다. \\
$e=\mathrm{FF} \quad\mathrm{,}\quad f=0$ 
& 는 ($\infty$) 을 나타낸다. 음의 무한과 양의 무한이 있을 수 
있다. \\
$e=\mathrm{FF} \quad\mathrm{,}\quad f\neq 0$ 
& \emph{NaN}(Not A Number) 라 부르는 정의되지 않은 값을 나타낸다.
\end{tabular}
}
\caption{\emph{f} 와 \emph{e} 의 특별한 값들\label{tab:floatSpecials}}
\end{table}

\emph{e} 와 \emph{f} 의 특정한 조합은 IEEE float 에서 특별한 의미를 가진다.
표 ~\ref{tab:floatSpecials} 는 이러한 특별한 값들을 보여준다. 무한대는 오버 플로우(over flow)나
0 으로의 나눗셈에 의해 발생된다. 정의되지 않은 값(NaN) 은 올바르지 못한 연산, 예를 들면
음수의 제곱근을 구한다던지, 두 개의 무한대를 더할 때 \emph{등등} 발생된다.

정규화된 단일 정밀도 수들은 $1.0 \times 2^{-126}$ ($\approx 1.1755 \times 10^{-35}$) 에서 
$1.11111\ldots \times 2^{127}$ ($\approx 3.4028 \times 10^{35}$) 까지의 값을 가질 수 있다.

\subsubsection{비정규화 된 수\index{부동 소수점!표현!비정규화|(}}

비정규화 된 수(Denormalized number)들은 너무 값이 작아서 정규화 할 수 없는 수들을 나타내기 위해 사용된다.
(\emph{i.e.} $1.0 \times 2^{-126}$ 미만) 예를 들어 $1.001_2 \times 2^{-129}$ ($\approx 1.6530 \times 10^{-39}$) 라는
수를 생각하자. 이 수를 정규화 하기에는 지수가 너무 작다. 그러나, 이 수는 비정규화 된 꼴:
$0.01001_2 \times 2^{-127}$ 로 나타낼 수 있다. 이 수를 저장하기 위해 편향된 지수는
0 으로 맞춰 진다. (표 ~\ref{tab:floatSpecials} 참조) 그리고 $2^{-127}$ 와 곱해진
소수 부분은 맨 앞 자리의 비트를 포함한 완전한 수로 나타나게 된다. 따라서, $1.001 \times 2^{-129}$는:
\[ \underline{0}\,000\;0000\;0
   \,\underline{001\;0010\;0000\;0000\;0000\;0000} \]
\index{부동 소수점!표현!비정규화|)}
\index{부동 소수점!표현!단일 정밀도|)}


\subsubsection{IEEE 2배 정밀도\index{부동 소수점!표현!2배 정밀도|(}}

\begin{figure}[t]
\centering
\begin{tabular}{|c|c|c|}
\multicolumn{1}{p{0.3cm}}{63} &
\multicolumn{1}{p{3cm}}{62 \hfill 52} &
\multicolumn{1}{p{7cm}}{51 \hfill 0} \\
\hline
s & e & f \\
\hline
\end{tabular}
\caption{IEEE 2배 정밀도\label{fig:IEEEdouble}}
\end{figure}

IEEE 2배 정밀도 표현은 64 비트를 사용하며, 십진법으로 대략 15 자리 정도 정확하게
나타낼 수 있다. 그림 ~\ref{fig:IEEEdouble} 에서 나타나듯이 기본적인 형식은
단일 정밀도와 매우 유사하다. 다만 편향된 지수와 가수 부분이 각각 11 비트와 
52 비트로 늘어났다.  

편향된 지수 부분의 크기가 더 커졌으므로 두 가지 변화를 야기하게 된다. 먼저
편향된 지수 값을 구할 때 진짜 지수에 3FF(1023) 을 더한다 (단일 정밀도의 7F 가 아니다).
두 번째로 진짜 지수가 가질 수 있는 값의 범위가 넓어 졌다.
2배 정밀도가 표현할 수 있는 수의 범위는 대략  $10^{-308}$ to $10^{308}$ 정도 된다. 

또한 가수 부분이 더 커졌으므로 유효 숫자의 자리수가 더 늘어나게 되었다. 

예를 들어 23.85 를 다시 생각해 보자. 편향된 지수는 16 진수로 $4 + \mathrm{3FF} = 403$가
될 것이다. 따라서 2배 정밀도 표현으로는 :
\[ \underline{0}\,100\;0000\;0011\;\underline{0111\;1101\;1001\;1001\;1001\;
   1001\;1001\;1001\;1001\;1001\;1001\;1001\;1010} \]
혹은 40 37 D9 99 99 99 99 9A 로 나타나게 된다. 이 수를 십진수로 변환한다면 
그 값은 23.8500000000000014 가 되어 (무려 12 개의 0 이 있다!) 23.85 를 훨씬 정밀하게
표현할 수 있게 된다.

2배 정밀도도 단일 정밀도 표현과 같이 특별한 값들이 있다. 
\footnote{유일한 차이점은 무한대와 NaN 의 경우 인데, 편향된 지수 값이 7FF 가 아니라
FF 가 된다. }
비정규화 된 수도 또한 매우 비슷하다. 유일한 차이점은 비정규화 된 수는$2^{-127}$ 가 아닌 $2^{-1023}$ 
가 된다. 

\index{부동 소수점!표현!2배 정밀도|)}
\index{부동 소수점!표현!IEEE|)}
\index{부동 소수점!표현|)}

\section{부동 소수점 수들의 산술 연산\index{부동 소수점!산술 연산|(}}

컴퓨터에서의 부동 소수점 수들의 산술 연산은 수들이 연속된 수학에서와는 다르다. 
수학에서는 모든 수들이 정확한 값을 가졌다고 생각된다. 위에서 다루었듯이 
컴퓨터가 다루는 많은 수들은 유한개의 비트를 통해 정확히 나타낼 수 없다. 모든 연산들은
제한된 정밀도를 가진다. 이 단원에서의 예제들은 단순화를 위해 8 비트의 유효 숫자를 가진다고
생각한다. 

\subsection{덧셈}
두 개의 부동 소수점 수들을 더하기 위해선 지수가 반드시 같아야 한다. 만일, 이 지수들이 같지 않다면
수의 가수를 쉬프트 하여 큰 지수에 맞추어야 한다. 예를 들어 $10.375 + 6.34375 = 16.71875$
를 이진법에서 수행한다고 하자. 

\[
\begin{array}{rr}
 & 1.0100110 \times 2^3 \\
+& 1.1001011 \times 2^2 \\ \hline
\end{array}
\]
이 두 수들은 지수값이 같지 않으므로 가수 부분을 쉬프트 하여 두 수의 지수가
같게 한 뒤 더한다.

\[
\begin{array}{rr@{.}l}
 &  1&0100110 \times 2^3 \\
+&  0&1100110 \times 2^3 \\ \hline
 & 10&0001100 \times 2^3
\end{array}
\]
이 때 $1.1001011 \times 2^2$ 에 쉬프트 연산을 취하면서 마지막 비트를 없애고
반올림을 하여 $0.1100110 \times 2^3$ 가 되었음을 유의하자. 덧셈의 결과는 
$10.0001100 \times 2^3$ (혹은 $1.00001100 \times 2^4$) 로 $10000.110_2$
혹은 16.75 와 같이 같다. 이는 정확한 계산 결과인 16.71875 와 같지 \emph{않다}! 이는 단지
덧셈 과정에서 반올림에 의해 오차가 나타난 근사값에 불과하다. 

컴퓨터 (혹은 계산기)에서의 부동 소수점 연산은 언제나 근사임을 명심해야 한다.
수학에서의 법칙들은 컴퓨터에서의 부동 소수점들에 적용될 수 없다. 수학은 언제나
컴퓨터가 절대로 만들어 낼 수 없는 무한대의 정밀도를 가정한다. 예를 들어서
수학에선 $(a + b) - b = a$ 라 말하지만 컴퓨터에서는 이 식이 성립하지 않을 수 있다!

\subsection{뺄셈}
뺄셈은 덧셈과 매우 유사하며, 같은 문제를 가지고 있다. 예를 들어서 $16.75 - 15.9375 = 0.8125$
를 생각해보자. 

\[
\begin{array}{rr}
 & 1.0000110 \times 2^4 \\
-& 1.1111111 \times 2^3 \\ \hline
\end{array}
\]
$1.1111111 \times 2^3$ 를 쉬프트 하면 $1.0000000 \times 2^4$ (반올림 됨)
\[
\begin{array}{rr}
 & 1.0000110 \times 2^4 \\
-& 1.0000000 \times 2^4 \\ \hline
 & 0.0000110 \times 2^4
\end{array}
\]
$0.0000110 \times 2^4 = 0.11_2 = 0.75$ 이므로 원 답과 일치하지 않는다. 

\subsection{곱셈과 나눗셈}

곱셈의 경우 유효 숫자 부분은 곱해지고, 지수 부분은 더해 진다. 예를 들어 
$10.375 \times 2.5 = 25.9375$ 을 생각해 보면 

\[
\begin{array}{rr@{}l}
 &  1.0&100110 \times 2^3 \\
\times &  1.0&100000 \times 2^1 \\ \hline
 &     &10100110 \\
+&   10&100110   \\ \hline
 &   1.1&0011111000000 \times 2^4
\end{array}
\]
당연하게도 실제 결과는 8 비트로 반올림 되어 
\[1.1010000 \times 2^4 = 11010.000_2 = 26 \]

나눗셈은 좀더 복잡하지만 반올림으로 인하여 같은 문제를 가진다. 

\subsection{프로그래밍을 위한 조언}

이 단원의 핵심적인 내용은 부동 소수점 연산이 정확하지 않다는 것이다. 
프로그래머는 언제나 이를 고려해야 한다. 프로그래머들이 가장 많이
범하는 실수로는 연산이 정확하다고 가정 하에 부동 소수점 수들을 비교하는 것이다.
예를 들어 \lstinline|f(x)| 라는 이름의 함수를 고려하자. 이 함수는 복잡한 연산을 수행한다.
우리의 프로그램의 목적은 이 함수의 근을 찾는 것이다. 
\footnote{함수의 근이라 하면 $f(x) = 0$ 을 만족하는 값 $x$ 를 말한다.} 여러분은 아마
\lstinline|x| 가 근인지 확인하기 위해 아래의 식을 이용할 것이다. 

\begin{lstlisting}[stepnumber=0]{}
  if ( f(x) == 0.0 )
\end{lstlisting}
그러나 만일 \lstinline|f(x)| 가 $1 \times 10^{-30}$ 을 리턴하면? 이는
진짜 근의 훌륭한 근사값이다. 그러나, 위 식은 성립하지 않는다. 
아마도 \lstinline|f(x)| 에서의 반올림으로 인한 오차로 인해 어떠한 IEEE 부동 소수점 형식으로 표현된
\lstinline|x|값도 정확히 0 을 리턴하지 않을 것이다. 

위 보다 훨씬 낳은 방법은 아래와 같다. 

\begin{lstlisting}[stepnumber=0]{}
  if ( fabs(f(x)) < EPS )
\end{lstlisting}
\lstinline|EPS| 는 매우 작은 양수 값인 매크로로 정의되어 있다. (예를 들면 $1 \times 10^{-10}$)
이는 \lstinline|f(x)| 가 0 에 매우 가까울 때 성립이 된다. 통상적으로 두 개의 부동 소수점 값
(\lstinline|x|) 을 다른 (\lstinline|y|) 값에 비교 할 때 아래와 같이 사용한다: 

\begin{lstlisting}[stepnumber=0]{}
  if ( fabs(x - y)/fabs(y) < EPS )
\end{lstlisting}
\index{부동 소수점!산술 연산|)}

\section{수치 부프로세서}
\index{부동 소수점 부프로세서|(}
\subsection{하드웨어}
\index{부동 소수점 부프로세서!하드웨어|(}
가장 초기의 인털 프로세서들은 부동 소수점 연산을 지원하는 하드웨어가 없었다.
이 말은 부동 소수점 연산을 할 수 없었다는 것이 아니다. 이는 단지 부동 소수점 수들의
연산을 위해 많은 수의 비-부동 소수점 명령들을 수행해야 했다는 뜻이다. 이러한
초기의 시스템에선 인텔은 부수적인\emph{수치 부프로세서(math coprocessor)}를 제공하였다.
이 수치 부프로세서는 부동 소수점 수들의 연산을 위한 소프트웨어 상의 프로시저의 비해
훨씬 빠른 속도로 명령을 수행하였다 (초기의 프로세서의 경우 대략 10 배 이상!). 8086/8088을
위한 부프로세서는 8087 이라 불리었다. 80286 에선 80287이, 80386 에선 80387 이 부프로세서 였다. 
80486DX 프로세서에선 80486에 수치 부프로세서가 통합되었다. 
\footnote{그러나 80486SX 에는 통합된 부프로세서가 \emph{없었다}. 이를 위해 독립적인 80487SX
칩이 있었다.} 펜티엄 이후 80x86 세대의 모든 프로세서들에게는 수치 부프로세서가 통합되었다. 그러나, 
독립적으로 떨어져 있었을 때 처럼 프로그램 된다. 심지어 부프로세서가 없던 초기의 시스템들은 
수학 부프로세서를 에뮬레이트 할 수 있는 소프트웨어가 있었다. 이 에뮬레이터는 프로그램이
부프로세서 명령을 실행하면 자동적으로 활성화 되어 소프트웨어 프로시저를 통해 부프로세서가 
내놓았을 같은 값을 내놓았다. (비록 매우 느렸지만)

수치 부프로세서는 8 개의 부동 소수점 레지스터가 있다. 각각의 레지스터는 80 비트의 데이터를 보관한다.
부동 소수점 수들은 이 레지스터들에 \emph{언제나} 80 비트 확장 정밀도로 저장된다. 이 레지스터들은 각각
{\code ST0}, {\code ST1}, {\code ST2}, $\ldots$ {\code ST7} 로 이름 붙여졌다. 부동 소수점 레지스터들은
주  CPU 에서의 정수 레지스터들과는 달리 다른 용도로 사용된다. 부동 소수점 레지스터들은
\emph{스택(stack)}으로 구성되어 있다. 스택이 \emph{후입 선출} (LIFO) 리스트임을
상기하자. {\code ST0} 는 언제나 스택의 최상단의 값을 가리킨다. 모든 새로운 수들은 스택의 최상단에 들어간다.
기존의 수들은 새로운 수를 위한 공간을 만들기 위해 스택 한 칸 내려간다. 

수치 부프로세서에는 상태 레지스터(status register) 가 있다. 이는 몇 가지의 플래그를 가진다.
비교 연산에서는 오직 4 개의 플래그들이 사용된다:C$_0$, C$_1$, C$_2$, C$_3$. 이것들의
사용은 나중에 다룰 것이다. 

\index{부동 소수점 부프로세서!하드웨어|)}

\subsection{명령}

보통의 CPU 명령들과 부프로세서의 명령들을 쉽게 구별하기 위해 모든 부프로세서의 연상기호들은
{\code F} 로 시작한다. 

\subsubsection{데이터를 불러오고 저장하기}\index{부동 소수점 부프로세서!불러오고 저장하기|(}
부프로세서의 레지스터 스택의 최상위 데이터에 값을 저장하는 몇 가지 연산들이 있다.\\
\begin{tabular}{lp{4in}}
{\code FLD \emph{source}} \index{FLD} & 
는 메모리에서 부동 소수점 값을 불러와 스택 최상단에 저장한다. \emph{source} 는
단일 혹은 2배 정밀도 값, 아니면 부프로세서의 레지스터 여야 한다. \\ 
{\code FILD \emph{source}} \index{FILD} &
는 \emph{정수}를 메모리로 부터 읽어들어, 부동 소수점 형식으로 변환한 뒤
그 결과를 스택 최상단에 저장한다. \emph{source} 는 워드, 더블워드, 쿼드워드
중 하나여야 한다. \\
{\code FLD1} \index{FLD1} &
1 을 스택 최상단에 저장한다.  \\
{\code FLDZ} \index{FLDZ} &
0 을 스택 최상단에 저장한다. \\
\end{tabular}


스택으로 부터의 데이터를 메모리에 저장하는 명령들도 몇 가지 있다. 이러한 명령들의 일부는
스택에서 수를 \emph{팝(pop)} (\emph{i.e.} 제거하다) 한 후에 그 값을 저장한다. 


\begin{tabular}{lp{4in}}
{\code FST \emph{dest}} \index{FST} &
스택의 최상단({\code ST0}) 을 메모리에 저장한다. \emph{dest} 은 
단일 혹은 2배 정밀도 수 이거나 부프로세서 레지스터가 될 수 있다. \\
{\code FSTP \emph{dest}} \index{FSTP} &
은 스택의 최상단 값을 {\code FST} 와 같이 메모리에 저장한다. 그러나
수가 저장된 다음에는 그 값은 스택에서 팝 된다. \emph{dest} 은
단일 혹은 2배 정밀도 수 이거나 부프로세서 레지스터가 될 수 있다. \\
{\code FIST \emph{dest}} \index{FIST} &
는 스택의 최상단의 값을 정수로 변환한 뒤 메모리에 저장한다. \emph{dest}는
워드 혹은 더블워드가 되어야만 한다. 스택 그 자체로는 바뀌지 않는다. 부동 소수점
수가 정수로 어떻게 변환되느냐넨 부프로세서의 \emph{제어 워드(control word)}에 
몇 개의 비트값에 따라 달라진다. 이는 특별한 (비-부동 소수점) 워드 레지스터로 부프로세서가
어떻게 작동할지 제어한다. 기본적으로 제어 워드는 초기화 되어 있는데 이 때에는
수에 가장 가까운 정수로 변환되어 진다. 그러나, {\code FSTCW} (저장 제어 워드
Store Control Word) 와 {\code FLDCW} (불러오기 제어 워드, Load Control Word) 명령들을 통해
어떻게 변환 할지를 바꿀 수 있다. \index{FSTCW} \index{FLDCW} \\

{\code FISTP \emph{dest}} \index{FIST} &
{\code FIST} 와 두 가지 빼고 동일하다. 하나는, 스택의 최상단이 팝 된다는 것이고,
다른 하나는 \emph{dest} 가 쿼드워드도 될 수 있다는 것이다. 
\end{tabular}

스택 자체에서 데이터를 이동, 혹은 제거할 수 있는 명령이 두 개가 있다. \\

\begin{tabular}{lp{4in}}
{\code FXCH ST\emph{n}} \index{FXCH}  &
스택에서의 {\code ST0} 와 {\code ST\emph{n}} 의 값을 바꾼다. 
(이 때, \emph{n} 은 1 부터 7 까지의 값을 가질 수 있다.) \\
{\code FFREE ST\emph{n}} \index{FFREE} &
레지스터가 비었거나 미사용중임을 표시하여 스택에서의 레지스터의 값을 비운다. 
\end{tabular}
\index{부동 소수점 부프로세서!불러오고 저장하기|)}

\subsubsection{덧셈과 뺄셈}\index{부동 소수점 부프로세서!덧셈과 뺄셈|(}

각각의 덧셈 명령들은 {\code ST0} 과 다른 피연산자의 합을 계산한다. 그 결과는 언제나
부프로세서 레지스터에 저장된다. \\
\begin{tabular}{p{1.5in}p{3.5in}}
{\code FADD \emph{src}} \index{FADD} &
{\code ST0 += \emph{src}}. \emph{src} 는 어떤 부프로세서 레지스터이거나 
메모리 상의 단일 혹은 2배 정밀도 수여도 상관 없다. \\
{\code FADD \emph{dest}, ST0} &
{\code \emph{dest} += ST0}. \emph{dest}는 임의의 부프로세서 레지스터 이다. \\
{\code FADDP \emph{dest}} 또는 \newline {\code FADDP \emph{dest}, STO} \index{FADDP} &
{\code \emph{dest} += ST0} 하고 팝 한다. \emph{dest} 는 임의의 
부프로세서 레지스터 이다. \\
{\code FIADD \emph{src}} \index{FIADD} &
{\code ST0 += (float) \emph{src}}. 정수를 {\code ST0} 에 더한다. 
\emph{src} 는 메모리 상의 워드 혹은 더블워드 값이다. 
\end{tabular}

\begin{figure}[t]
\begin{AsmCodeListing}[frame=single]
segment .bss
array        resq SIZE
sum          resq 1

segment .text
      mov    ecx, SIZE
      mov    esi, array
      fldz                  ; ST0 = 0
lp:
      fadd   qword [esi]    ; ST0 += *(esi)
      add    esi, 8         ; 다음 double 로 이동
      loop   lp
      fstp   qword sum      ; sum 에 결과를 저장
\end{AsmCodeListing}
\caption{배열의 원소 합 구하는 예제\label{fig:addEx}}
\end{figure}

두 개의 피연산자에 대해 덧셈 명령은 한 가지 이지만, 뺄셈 명령은 2 가지의 서로 다른 뺄셈 명령이
존재할 수 있다. 왜냐하면 피연산자의 순서가 뺄셈에서는 중요하기 때문이다. (\emph{i.e.} $a + b = b + a$, 이지만
$a - b \neq b - a$ 은 아니다!) 각 명령에 대해 뺄셈을 역으로 수행하는 명령이 있다. 이 역명령들은 끝에 언제나
{\code R} 이나 {\code RP} 가 붙는다. 그림 ~\ref{fig:addEx} 는 배열의 double 형 원소들을 모두 더하는 코드를 보여주고
있다. ~10 에서 13 행 까지 우리는 메모리 피연산의 크기를 확실히 지정해야 한다. 그렇지 않는 다면 어셈블러는 메모리
피연산자가 float(더블워드) 인지 double(쿼드워드) 인자 알 수 없기 때문이다. 

\begin{tabular}{p{1.5in}p{3.5in}}
{\code FSUB \emph{src}} \index{FSUB} &
{\code ST0 -= \emph{src}}. \emph{src} 는 임의의 부프로세서 레지스터 거나
메모리 상의 단일 혹은 2배 정밀도 수 이면 된다. \\
{\code FSUBR \emph{src}} \index{FSUBR} &
{\code ST0 = \emph{src} - ST0}. \emph{src} 는 임의의 부프로세서 레지스터 거나
메모리 상의 단일 혹은 2배 정밀도 수 이면 된다. \\
{\code FSUB \emph{dest}, ST0} &
{\code \emph{dest} -= ST0}. \emph{dest} 는 임의의 부프로세서 레지스터 이다. \\
{\code FSUBR \emph{dest}, ST0} &
{\code \emph{dest} = ST0 - \emph{dest}}.\emph{dest} 는 임의의 부프로세서 
레지스터 이다. \\
{\code FSUBP \emph{dest}} 혹은 \newline {\code FSUBP \emph{dest}, STO} \index{FSUBP} &
{\code \emph{dest} -= ST0} 한 후 팝 한다. \emph{dest} 는 임의의 부프로세서
레지스터 이다. \\
{\code FSUBRP \emph{dest}} or \newline {\code FSUBRP \emph{dest}, STO} \index{FSUBRP} &
{\code \emph{dest} = ST0 - \emph{dest}} 한 후 팝 한다. \emph{dest} 는 임의의 
부프로세서 레지스터 이다. \\
{\code FISUB \emph{src}} \index{FISUB} &
{\code ST0 -= (float) \emph{src}}.{\code ST0} 에서 정수를 뺀다. 
\emph{src} 는 반드시 메모리 상의 워드 혹은 더블워드 여야 한다. \\
{\code FISUBR \emph{src}} \index{FISUBR} &
{\code ST0 = (float) \emph{src} - ST0}.{\code ST0} 에서 정수를 뺀다.
\emph{src} 는 반드시 메모리 상의 워드 혹은 더블워드 여야 한다.
\end{tabular}

\index{부동 소수점 부프로세서!덧셈과 뺄셈|)}

\subsubsection{곱셈과 나눗셈}\index{부동 소수점 부프로세서!곱셈과 나눗셈|(}

곰셉 명령은 덧셈 명령과 완벽히 동일하다. 

\begin{tabular}{p{1.5in}p{3.5in}}
{\code FMUL \emph{src}} \index{FMUL} &
{\code ST0 *= \emph{src}}. \emph{src} 는 임의의 부프로세서 레지스터 거나
메모리 상의 단일 혹은 2배 정밀도 수 이면 된다. \\
{\code FMUL \emph{dest}, ST0} &
{\code \emph{dest} *= ST0}.\emph{dest} 는 임의의 부프로세서 레지스터 이다. \\
{\code FMULP \emph{dest}} 혹은 \newline {\code FMULP \emph{dest}, STO} \index{FMULP} &
{\code \emph{dest} *= ST0} 한 후 팝 한다. \emph{dest} 는 임의의 부프로세서
레지스터 이다. \\
{\code FIMUL \emph{src}} \index{FMUL} &
{\code ST0 *= (float) \emph{src}}. {\code ST0} 에 정수를 곱한다.
\emph{src} 는 메모리 상의 워드 혹은 더블워드 여야 한다. 
\end{tabular}

놀랍지 않게도 나눗셈 명령은 뺄셈 명령과 거의 동일하다. 다만 0 으로 나눈다면
무한대가 된다.\\

\begin{tabular}{p{1.5in}p{3.5in}}
{\code FDIV \emph{src}} \index{FDIV} &
{\code ST0 /= \emph{src}}. \emph{src} 는 임의의 부프로세서 레지스터 거나
메모리 상의 단일 혹은 2배 정밀도 수 이면 된다. \\
{\code FDIVR \emph{src}} \index{FDIVR} &
{\code ST0 = \emph{src} / ST0}. \emph{src} 는 임의의 부프로세서 레지스터 거나
메모리 상의 단일 혹은 2배 정밀도 수 이면 된다. \\
{\code FDIV \emph{dest}, ST0} &
{\code \emph{dest} /= ST0}. \emph{dest} 는 임의의 부프로세서 레지스터 이다. \\
{\code FDIVR \emph{dest}, ST0} &
{\code \emph{dest} = ST0 / \emph{dest}}. \emph{dest} 는 임의의 부프로세서
레지스터 이다. \\
{\code FDIVP \emph{dest}} 혹은 \newline {\code FDIVP \emph{dest}, STO} \index{FDIVP} &
{\code \emph{dest} /= ST0} 한 후 팝 한다. \emph{dest} 는 임의의 부프로세서
레지스터 이다. \\
{\code FDIVRP \emph{dest}} 혹은 \newline {\code FDIVRP \emph{dest}, STO} \index{FDIVRP} &
{\code \emph{dest} = ST0 / \emph{dest}} 한 후 팝한다. \emph{dest} 는 임의의
부프로세서 레지스터 이다. \\
{\code FIDIV \emph{src}} \index{FIDIV} &
{\code ST0 /= (float) \emph{src}}. {\code ST0}를 정수로 나눈다. 
\emph{src} 는 메모리 상의 워드 혹은 더블워드 여야 한다. \\
{\code FIDIVR \emph{src}} \index{FIDIVR} &
{\code ST0 = (float) \emph{src} / ST0}. 정수를 {\code ST0} 로 나눈다.
\emph{src} 는 메모리 상의 워드 혹은 더블워드 여야 한다. 
\end{tabular}
\index{부동 소수점 부프로세서!곱셈과 나눗셈|)}
\subsubsection{비교\index{부동 소수점 부프로세서!비교|(}}

부프로세서는 부동 소수점 수들에 대한 비교 명령도 수행한다. {\code FCOM} 계열의
명령들이 이 작업을 한다. \\

\begin{tabular}{lp{4in}}
{\code FCOM \emph{src}} \index{FCOM} & 
{\code ST0} 와 {\code \emph{src}} 를 비교. \emph{src} 는 임의의 
부프로세서 레지스터 거나 메모리 상의 단일 혹은 2배 정밀도 수 이면 된다. \\
{\code FCOMP \emph{src}} \index{FCOMP} & 
{\code ST0} 와 {\code \emph{src}} 를 비교한 후 팝 한다.\emph{src} 는
임의의 부프로세서 레지스터 거나 메모리 상의 단일 혹은 2배 정밀도 수 이면 된다.\\
{\code FCOMPP} \index{FCOMPP} & 
{\code ST0} 와 {\code ST1} 를 비교 한 후 팝을 두 번 한다. \\
{\code FICOM \emph{src}} \index{FICOM} & 
{\code ST0} 와 {\code (float) \emph{src}} 를 비교한다. \emph{src} 는 메모리 상의  
워드 혹은 더블워드 정수 이다. \\
{\code FICOMP \emph{src}} \index{FICOMP} & 
{\code ST0} 와 {\code (float)\emph{src}} 를 비교 한 후 팝 한다. 
\emph{src} 는 메모리 상의 워드 혹은 더블워드 정수 이다.  \\
{\code FTST } \index{FTST} &
{\code ST0} 와 0 을 비교한다.
\end{tabular}

\begin{figure}[t]
\begin{AsmCodeListing}[frame=single]
;     if ( x > y )
;
      fld    qword [x]       ; ST0 = x
      fcomp  qword [y]       ; STO 와 y 를 비교
      fstsw  ax              ; 플래그 레지스터에 C 비트를 이동 
      sahf
      jna    else_part       ;  x>y 가 아니면 else_part 로 분기
then_part:
      ; then 을 위한 부분
      jmp    end_if
else_part:
      ; else 를 위한 부분
end_if:
\end{AsmCodeListing}
\caption{비교 예제\label{fig:compEx}}
\end{figure}

이 명령들은 부프로세서 상태 레지스터의 C$_0$, C$_1$, C$_2$, C$_3$ 비트들을
바꾼다. 불행이도 CPU 가 직접 이러한 비트에 접근하는 것은 불가능 하다. 조건 분기 명령은
부프로세서의 상태 레지스터가 아닌 플래그 레지스터를 사용한다.그러나 상태 워드의 비트들을
이에 대응되는 플래그 레지스터의 비트에 옮기는 것은 아래의 몇 가지 새로운 명령을
이용한다면 쉽다
\\

\begin{tabular}{lp{4in}}
{\code FSTSW \emph{dest}} \index{FSTSW} & 
부프로세서의 상태 워드를 메모리의 워드나 AX 레지스터에 저장한다.\\
{\code SAHF} \index{SAHF} & 
AH 레지스터를 플래그 레지스터에 저장한다. \\
{\code LAHF} \index{LAHF} & 
플래그 레지스터의 비트들과 함게 AH 를 불러온다. \\
\end{tabular}

\begin{figure}
\begin{AsmCodeListing}[frame=single]
global _dmax

segment .text
; function _dmax
; 두 개의 double 인자 들 중 큰 것을 리턴한다.
; C 원형
; double dmax( double d1, double d2 )
; 인자들:
;   d1   - 첫 번째 double
;   d2   - 두 번째 double
; 리턴값:
;  d1 과 d2 중 큰 것 (ST0 에 대입하여)
%define d1   ebp+8
%define d2   ebp+16
_dmax:
        enter   0, 0

        fld     qword [d2]
        fld     qword [d1]          ; ST0 = d1, ST1 = d2
        fcomip  st1                 ; ST0 = d2
        jna     short d2_bigger
        fcomp   st0                 ; d2 를 팝 한다.
        fld     qword [d1]          ; ST0 = d1
        jmp     short exit
d2_bigger:                          ; d2 가 더 크면 아무것도 안함
exit:
        leave
        ret
\end{AsmCodeListing}
\caption{{\code FCOMIP} 예제 \label{fig:fcomipEx}}
\index{FCOMIP}
\end{figure}

그림 ~\ref{fig:compEx} 는 코드 예제를 보여준다. ~5 행과 6 행은 부프로세서의
상태 워드의 C$_0$, C$_1$, C$_2$, C$_3$ 비트들을 플래그 레지스터에 대입한다. 
전달한 비트들은 두 개의 \emph{부호가 없는} 정수의 비교 결과와 동일한다. 이는
우리가 ~7 행에서 {\code JNA} 를 사용한 이유이다. 

펜티엄 프로(그리고 그 이후의 프로세서들 (펜티엄 II, III)) 는 두 개의 새로운 비교
명령을 지원하는데 이는 CPU 의 플래그 레지스터의 값을 직접적으로 변경할 수 있다. 

\begin{tabular}{lp{4in}}
{\code FCOMI \emph{src}} \index{FCOMI} & 
{\code ST0} 와 {\code \emph{src}} 를 비교한다. \emph{src} 는 반드시
부프로세서 레지스터 여야 한다. \\
{\code FCOMIP \emph{src}} \index{FCOMIP} & 
{\code ST0} 와 {\code \emph{src}} 를 비교한 후 팝 한다.\emph{src} 는
반드시 부프로세서 레지스터 여야 한다. \\
\end{tabular}
그림 ~\ref{fig:fcomipEx} 은 두 개의 double 값 중 큰 것을 찾는 {\code FCOMIP} 를 사용한
예제 서브루틴을 보여주고 있다. 이 명령들을 정수 비교 함수들과 혼동하지 말라 ({\code FICOM} 과
{\code FICOMP}).

\index{부동 소수점 부프로세서!비교|)}

\subsubsection{잡다한 명령들}
%FINIT?

\begin{figure}
\begin{AsmCodeListing}[frame=single]
segment .data
x            dq  2.75          ; double 형식으로 변환
five         dw  5

segment .text
      fild   dword [five]      ; ST0 = 5
      fld    qword [x]         ; ST0 = 2.75, ST1 = 5
      fscale                   ; ST0 = 2.75 * 32, ST1 = 5
\end{AsmCodeListing}
\caption{{\code FSCALE} 예제\label{fig:fscaleEx}}
\index{FSCALE}
\end{figure}

이 단원은 부프로세서가 제공하는 여러 잡다한 명령들에 대해 다루어 보도록 
한다. 

\begin{tabular}{lp{4in}}
{\code FCHS} \index{FCHS} & 
{\code ST0 = - ST0}, {\code ST0} 의 부호를 바꾼다. \\
{\code FABS} \index{FABS} & 
$\mathtt{ST0} = |\mathtt{ST0}|$ ,{\code ST0} 의 절대값을 취한다.\\
{\code FSQRT} \index{FSQRT} &
$\mathtt{ST0} = \sqrt{\mathtt{STO}}$, {\code ST0} 의 제곱근을 구한다. \\
{\code FSCALE} \index{FSCALE} &
$\mathtt{ST0} = \mathtt{ST0} \times 2^{\lfloor \mathtt{ST1} \rfloor}$
,{\code ST0} 에 2 의 멱수를 빠르게 곱한다. {\code ST1} 는 스택에저 제거되지 않는다.
그림 ~\ref{fig:fscaleEx} 는 이 명령을 어떻게 사용하는지에 대한 예제를 보여준다. 

\end{tabular}

\subsection{예제}

\subsection{2차 방정식의 근의 공식\index{quad.asm|(}}
우리의 첫 번째 예제는 2차 방정식의 근의 공식을 어셈블리로 나타낸 것이다.

\[ a x^2 + b x + c = 0 \]
근의 공식을 통해 두 개의 근을 구할 수 있다. $x$: $x_1$ 와 $x_2$.
\[ x_1, x_2 = \frac{-b \pm \sqrt{b^2 - 4 a c}}{2 a} \]
제곱근 안의 식 ($b^2 - 4 a c$) 은 \emph{판별식(discriminant)} 라 부른다.
이 값은 근이 어떠한 성질을 띄는지 판별하는데 유용하게 쓰인다. 

\begin{enumerate}
\item 실수인 중근을 가진다. $b^2 - 4 a c = 0$
\item 두 개의 실수근을 가진다. $b^2 - 4 a c > 0$
\item 두 개의 복소근을 가진다. $b^2 - 4 a c < 0$
\end{enumerate}

아래 어셈블리 서브루틴을 사용하는 작은 C 프로그램을 나타냈다. 

\LabelLine{quadt.c}
\begin{lstlisting}{}
#include <stdio.h>

int quadratic( double, double, double, double *, double *);

int main()
{
  double a,b,c, root1, root2;

  printf("Enter a, b, c: ");
  scanf("%lf %lf %lf", &a, &b, &c);
  if (quadratic( a, b, c, &root1, &root2) )
    printf("roots: %.10g %.10g\n", root1, root2);
  else
    printf("No real roots\n");
  return 0;
}
\end{lstlisting}
\LabelLine{quadt.c}

여기 어셈블리 서브루틴이 있다:
\begin{AsmCodeListing}[label=quad.asm,commentchar=$]
; 함수 quadratic
; 2차 방정식의 근을 찾는다.  
;       a*x^2 + b*x + c = 0
; C 원형:
;   int quadratic( double a, double b, double c,
;                  double * root1, double *root2 )
; 인자:
;   a, b, c - 2차 방정식의 각 계수들 (위를 참조)
;   root1   - 첫 번째 근을 저장할 double 을 가리키는 포인터 
;   root2   - 두 번째 근을 저장할 double 을 가리키는 포인터 
; 리턴값:
;  실근을 찾으면 1 을 리턴, 아니면 0 리턴

%define a               qword [ebp+8]
%define b               qword [ebp+16]
%define c               qword [ebp+24]
%define root1           dword [ebp+32]
%define root2           dword [ebp+36]
%define disc            qword [ebp-8]
%define one_over_2a     qword [ebp-16]

segment .data
MinusFour       dw      -4

segment .text
        global  _quadratic
_quadratic:
        push    ebp
        mov     ebp, esp
        sub     esp, 16         ; 2 개의 더블을 할당한다 (disc 와 one_over_2a)
        push    ebx             ; 원래의 ebx 값을 저장

        fild    word [MinusFour]; 스택 -4
        fld     a               ; 스택: a, -4
        fld     c               ; 스택: c, a, -4
        fmulp   st1             ; 스택: a*c, -4
        fmulp   st1             ; 스택: -4*a*c
        fld     b
        fld     b               ; 스택: b, b, -4*a*c
        fmulp   st1             ; 스택: b*b, -4*a*c
        faddp   st1             ; 스택: b*b - 4*a*c
        ftst                    ; 0 과 비교
        fstsw   ax
        sahf
        jb      no_real_solutions ; 만일 disc < 0, 이면 해가 없다.
        fsqrt                   ; 스택: sqrt(b*b - 4*a*c)
        fstp    disc            ; 저장 후 스택을 팝 한다.
        fld1                    ; 스택: 1.0
        fld     a               ; 스택: a, 1.0
        fscale                  ; 스택: a * 2^(1.0) = 2*a, 1
        fdivp   st1             ; 스택: 1/(2*a)
        fst     one_over_2a     ; 스택: 1/(2*a)
        fld     b               ; 스택: b, 1/(2*a)
        fld     disc            ; 스택: disc, b, 1/(2*a)
        fsubrp  st1             ; 스택: disc - b, 1/(2*a)
        fmulp   st1             ; 스택: (-b + disc)/(2*a)
        mov     ebx, root1
        fstp    qword [ebx]     ;  *root1 에 저장
        fld     b               ; 스택: b
        fld     disc            ; 스택: disc, b
        fchs                    ; 스택: -disc, b
        fsubrp  st1             ; 스택: -disc - b
        fmul    one_over_2a     ; 스택: (-b - disc)/(2*a)
        mov     ebx, root2
        fstp    qword [ebx]     ; *root2 에 저장
        mov     eax, 1          ; 리턴값은 1
        jmp     short quit

no_real_solutions:
        mov     eax, 0          ; 리턴값은 0

quit:
        pop     ebx
        mov     esp, ebp
        pop     ebp
        ret
\end{AsmCodeListing}
\index{quad.asm|)}

\subsection{파일로 부터 배열을 읽기\index{read.asm|(}}
이번 예제에서는 어셈블리가 파일로 부터 double 을 읽어들인다. 아래는
짧은 C 테스트 프로그램이다. 

\LabelLine{readt.c}
\begin{lstlisting}[escapeinside=~~]{}
/*
 * ~이 프로그램은 32 비트 read\_doubles() 어셈블리 프로시저를 테스트 한다.~
 * ~이는 stdin 으로 부터 double 을 읽어 들인다. (파일로 부터 읽기 위해 ~
 * ~ 리다이렉션(redirection)을 사용한다.)~
 */
#include <stdio.h>
extern int read_doubles( FILE *, double *, int );
#define MAX 100

int main()
{
  int i,n;
  double a[MAX];

  n = read_doubles(stdin, a, MAX);

  for( i=0; i < n; i++ )
    printf("%3d %g\n", i, a[i]);
  return 0;
}
\end{lstlisting}
\LabelLine{readt.c}

아래는 어셈블리 루틴이다. 
\begin{AsmCodeListing}[label=read.asm]
segment .data
format  db      "%lf", 0        ; format for fscanf()

segment .text
        global  _read_doubles
        extern  _fscanf

%define SIZEOF_DOUBLE   8
%define FP              dword [ebp + 8]
%define ARRAYP          dword [ebp + 12]
%define ARRAY_SIZE      dword [ebp + 16]
%define TEMP_DOUBLE     [ebp - 8]

;
; _read_doubles 함수
; C 원형:
;   int read_doubles( FILE * fp, double * arrayp, int array_size );
; 이 함수는 텍스트 파일로 부터 double 을 읽어 들여서 EOF 가 나올 때 까지나 
; 배열이 꽉 찰 때 까지 배열에 저장한다.
; 인자:
;   fp         - 읽어 들일 FILE 포인터 (읽기 가능해야 한다)
;   arrayp     - 읽어 들인 값을 저장할 double 배열을 가리키는 포인터
;   array_size - 배열의 원소의 개수 
; 리턴:
;  배열에 저장된 double 의 개수 (EAX 에 저장됨)

_read_doubles:
        push    ebp
        mov     ebp,esp
        sub     esp, SIZEOF_DOUBLE      ; 스택에 하나의 double 을 정의 

        push    esi                     ; esi 저장
        mov     esi, ARRAYP             ; esi = ARRAYP
        xor     edx, edx                ; edx = 배열 원소 위치 (처음이 0)

while_loop:
        cmp     edx, ARRAY_SIZE         ; edx < ARRAY_SIZE 인가?
        jnl     short quit              ; 아니면 루프 루프 종료 
;
; TEMP_DOUBLE 로 double 값을 읽어 들이기 위해 fscanf() 를 호출
; fscanf() 는 edx 값을 바꿀 수 있으므로 저장해 놓는다. 
;
        push    edx                     ; edx 저장
        lea     eax, TEMP_DOUBLE
        push    eax                     ; &TEMP_DOUBLE 푸시
        push    dword format            ; &format 푸시
        push    FP                      ; 파일 포인터 푸시 
        call    _fscanf
        add     esp, 12
        pop     edx                     ; edx 복원
        cmp     eax, 1                  ; fscanf 가 1 을 리턴했는가?
        jne     short quit              ; 아니면 루프 종료
;
; TEMP_DOUBLE 를 ARRAYP[edx] 에 대입
; (8 바이트의 double 은 두 개의 4 바이트 레지스터를 통해 대입된다.)
;
        mov     eax, [ebp - 8]
        mov     [esi + 8*edx], eax      ; 하위 4 바이트를 복사 
        mov     eax, [ebp - 4]
        mov     [esi + 8*edx + 4], eax  ; 상위 4 바이트를 복사 

        inc     edx
        jmp     while_loop

quit:
        pop     esi                     ; esi 를 복원한다. 

        mov     eax, edx                ; 리턴값을 eax 에 저장 

        mov     esp, ebp
        pop     ebp
        ret 
\end{AsmCodeListing}
\index{read.asm|)}

\subsection{소수 찾기\index{prime2.asm|(}}

마지막 예제는 소수를 찾는 프로그램이다. 이미 앞에서 한 번 다루어 보았지만 
이번 것은 좀더 속도가 향상되었다. 이번 프로그램은 앞에서 찾은 소수를 배열에 저장해
놓고 특정한 수가 소수 인지를 판별하기 위해 특정한 수 이전의 모든 짝수로 나누어 보는
것이 아니라 배열에 저장해 놓은 소수들만으로 나누어 본다. 

또다른 차이점은 특정한 수가 소수 인지를 판별하기 위해 그 수의 제곱근 이하의 소수들만으로
나누어 본다는 것이다. 이 때문에 부프로세서의 제어 워드를 변경하여 제곱근을 정수로 저장할 때
반올림 하기 보단 내림을 해야한다. 이는 제어 워드의 10 과 11 번째 비트를 변경하면 된다. 
이 비트들은 RC (반올림 제어, Rounding Control) 비트로 불린다. 만일 두 비트 모두 0 이라면 (기본값)
부프로세서는 정수로 변환시에 반올림을 한다. 두 비트가 모두 1 이라면 부프로세서는 정수로
변환시 내림을 한다. 유의해야 할 점은 기존의 제어 워드를 저장하고 리턴하기 전에 복원해야 한다는
것이다. 

아래는 C 드라이버 프로그램 이다
\LabelLine{fprime.c}
\begin{lstlisting}[escapeinside=~~]{}
#include <stdio.h>
#include <stdlib.h>
/*
 * ~ find\_primes 함수 ~
 * ~ 지정된 개수의 소수들을 찾는다~
 * ~인자들:~
 *   a -~ 배열을 보관할 배열~
 *   n - ~찾을 소수의 개수~
 */
extern void find_primes( int * a, unsigned n );

int main()
{
  int status;
  unsigned i;
  unsigned max;
  int * a;

  printf("How many primes do you wish to find? ");
  scanf("%u", &max);

  a = calloc( sizeof(int), max);

  if ( a ) {

    find_primes(a,max);

    /*~찾은 마지막 20 개의 소수를 출력 ~*/
    for(i= ( max > 20 ) ? max - 20 : 0; i < max; i++ )
      printf("%3d %d\n", i+1, a[i]);

    free(a);
    status = 0;
  }
  else {
    fprintf(stderr, "Can not create array of %u ints\n", max);
    status = 1;
  }

  return status;
}
\end{lstlisting}
\LabelLine{fprime.c}

아래는 어셈블리 루틴이다.

\begin{AsmCodeListing}[label=prime2.asm]
segment .text
        global  _find_primes
;
; find_primes 함수
; 지정한 수 만큼의 소수를 찾는다 
; 인자:
;   array  - 소수를 보관할 배열
;   n_find - 찾을 소수의 개수
; C 원형:
;extern void find_primes( int * array, unsigned n_find )
;
%define array         ebp + 8
%define n_find        ebp + 12
%define n             ebp - 4           ; 현재까지 찾은 소수의 개수 
%define isqrt         ebp - 8           ; guess 의 제곱근의 내림한 값 
%define orig_cntl_wd  ebp - 10          ; 원래 제어 워드 값
%define new_cntl_wd   ebp - 12          ; 새로운 제워 워드 값

_find_primes:
        enter   12,0                    ; 지역 변수를 위한 공간을 할당 

        push    ebx                     ; 가능한 레지스터 변수들을 저장 
        push    esi

        fstcw   word [orig_cntl_wd]     ; 현재의 제어 워드를 저장 
        mov     ax, [orig_cntl_wd]
        or      ax, 0C00h               ; 반올림 비트를 11 로 한다. (내림)
        mov     [new_cntl_wd], ax
        fldcw   word [new_cntl_wd]

        mov     esi, [array]            ; esi 는 배열을 가리킨다. 
        mov     dword [esi], 2          ; array[0] = 2
        mov     dword [esi + 4], 3      ; array[1] = 3
        mov     ebx, 5                  ; ebx = guess = 5
        mov     dword [n], 2            ; n = 2
;
; 이 바깥 루프는 각 루프 마다 새로운 소수를 하나씩 찾는다. 이 새로운 소수는
; 배열의 끝부분에 추가된다. 기존의 소수 찾는 프로그램들과는 달리 이 함수는 
; 소수 판별을 위해 그 수 미만의 모든 홀수들로 나누어 보지 않는다.
; 오직 이미 찾은 소수들 만으로 나누어 본다. (이 때문에 이 소수들을 배열에
; 저장한 것이다.)
;
while_limit:
        mov     eax, [n]
        cmp     eax, [n_find]           ; while ( n < n_find )
        jnb     short quit_limit

        mov     ecx, 1                  ; ecx 는 원소의 위치를 가리키는데 사용 
        push    ebx                     ; guess 를 스택에 저장 
        fild    dword [esp]             ;guess 를 부프로세서 스택에 불러온다
        pop     ebx                     ; 스택에서 guess 를 뺀다.
        fsqrt                           ; sqrt(guess) 값을 구한다. 
        fistp   dword [isqrt]           ; isqrt = floor(sqrt(quess))
;
; 이 내부의 루프는 guess (ebx)를 이전에 찾은 소수들로 나누어서
; guess 의 소수 인수를 찾을 때 까지 혹은  
; 소수 인수가 floor(sqrt(guess)) 보다 클 때 까지 나눈다. 
;(floor 은 내림 함수)
while_factor:
        mov     eax, dword [esi + 4*ecx]        ; eax = array[ecx]
        cmp     eax, [isqrt]                    ; while ( isqrt < array[ecx] 
        jnbe    short quit_factor_prime
        mov     eax, ebx
        xor     edx, edx
        div     dword [esi + 4*ecx]     
        or      edx, edx                        ; && guess % array[ecx] != 0 )
        jz      short quit_factor_not_prime
        inc     ecx                             ; 다음 소수로 시도한다 
        jmp     short while_factor

;
; 새 소수를 찾음!
;
quit_factor_prime:
        mov     eax, [n]
        mov     dword [esi + 4*eax], ebx        ; guess 를 배열의 끝 부분에 추가 
        inc     eax
        mov     [n], eax                        ; inc n

quit_factor_not_prime:
        add     ebx, 2                          ; 다음 홀수를 시도해 본다 
        jmp     short while_limit

quit_limit:

        fldcw   word [orig_cntl_wd]             ; 제어 워드를 복원 
        pop     esi                             ; 레지스터 변수들을 복원 
        pop     ebx

        leave
        ret 
\end{AsmCodeListing}
\index{prime2.asm|)}
\index{부동 소수점 부프로세서|)}
\index{부동 소수점|)}
% -*-latex-*-
\chapter{�ṹ����C++}

\section{�ṹ��\index{�ṹ��|(}}

\subsection{���}

��C�����еĽṹ����������ص����ݼ��ϵ�һ����ϱ����С�������м����ŵ㣺
\begin{enumerate}
\item ͨ��չʾ�����ڽṹ���ڵ������ǽ�����������ʹ�������������ˡ�
\item ��ʹ�������ݸ�������ü򵥡����浥���ش��ݶ����������ͨ������һ����Ԫ�����ݶ��������
\item �������˴���� \index{�ֲ���}\emph{�ֲ���}\footnote{���Կ��κβ���ϵͳ���й��������ڴ�����е������������IJ��֡�} ��
\end{enumerate}

�ӻ�����ԵĹ۵㿴���ṹ�������Ϊ��ӵ��\emph{��ͬ}��С��Ԫ�ص����顣�������������Ԫ�صĴ�С����������һ���ġ������֪���������ʼ��ַ��ÿ��Ԫ�صĴ�С����Ҫ��Ԫ�ص��±꣬��������Ծ��ܼ�������Ԫ�صĵ�ַ��

�ṹ���е�Ԫ�صĴ�С����һ��Ҫ��һ����(����ͨ��������Dz�һ����)����Ϊ���ԭ�򣬽ṹ���е�ÿ��Ԫ�ر��������ָ��������Ҫ��ÿ��Ԫ��һ��\emph{���}(��������)�������Ǹ�һ�������±ꡣ

�ڻ�������У��ṹ���е�Ԫ�ؿ���ͨ���ͷ��������е�Ԫ��һ���ķ��������ʡ�Ϊ�˷���һ��Ԫ�أ������֪���ṹ�����ʼ��ַ�����Ԫ������ڽṹ���\emph{���ƫ�Ƶ�ַ}�����ǣ������鲻һ�����ǣ�������ͨ��Ԫ�ص��±��������ƫ�Ƶ�ַ���ṹ���Ԫ�صĵ�ַ��Ҫͨ������������ֵ��

���磬��������Ľṹ�壺
\lstset{escapeinside=`',language=Pascal,%
}
\begin{lstlisting}[stepnumber=0]{}
struct S {
  short int x;    /* `2���ֽڵ�����' */
  int       y;    /* `4���ֽڵ�����' */
  double    z;    /* `8���ֽڵĸ�����' */
};
\end{lstlisting}

\begin{figure}
\centering
\begin{tabular}{r|c|}
\multicolumn{1}{c}{ƫ�Ƶ�ַ} & \multicolumn{1}{c}{ Ԫ�� } \\
\cline{2-2}
0 & {\code x} \\
\cline{2-2}
2 & \\
  & {\code y} \\
\cline{2-2}
6 & \\
  & \\
  & {\code z} \\
  & \\
\cline{2-2}
\end{tabular}
\caption{�ṹ��S \label{fig:structPic1}}
\end{figure}

ͼ~\ref{fig:structPic1}չʾ��һ��{\code S}�ṹ������ڵ����ڴ�������δ���ġ�ANSI C��׼�涨�ṹ���е�Ԫ�����ڴ��д����˳�����{\code struct}�����е�˳����һ���ġ���ͬ���涨��һ��Ԫ����ǡ���ڽṹ�����ʼ��ַ��(\emph{Ҳ����˵}ƫ�Ƶ�ַΪ0)����ͬ����{\code stddef.h}ͷ�ļ��ж�������һ�����õĺ�{\code offsetof()}��\index{�ṹ��!offsetof()}�������������ͷ��ؽṹ��������Ԫ�ص�ƫ�Ƶ�ַ�������Я��������������һ���ǽṹ��\emph{����}�ı��������ڶ�������Ҫ�õ�ƫ�Ƶ�ַ��Ԫ��������ˣ�ͼ~\ref{fig:structPic1}�еģ�{\code offsetof(S, y)}�Ľ������2��

%TODO: talk about definition of offsetof() ??

\subsection{�ڴ��ַ����}

\begin{figure}
\centering
\begin{tabular}{r|c|}
\multicolumn{1}{c}{ƫ�Ƶ�ַ} & \multicolumn{1}{c}{ Ԫ�� } \\
\cline{2-2}
0 & {\code x} \\
\cline{2-2}
2 & \emph{unused} \\
\cline{2-2}
4 & \\
  & {\code y} \\
\cline{2-2}
8 & \\
  & \\
  & {\code z} \\
  & \\
\cline{2-2}
\end{tabular}
\caption{�ṹ��S \label{fig:structPic2}}

\end{figure}
\index{�ṹ��!��ַ����|(}
�����\emph{gcc}����������ʹ��{\code offsetof}�����õ�{\code y}��ƫ�Ƶ�ַ����ô���ǽ��ҵ�������4��������2��Ϊʲô�أ�\MarginNote{����һ��һ����ַ���������4�Ժ󣬵�ַ�Ǵ���˫�ֽ��ϵġ�} ���\emph{gcc}(���������������)����ȱʡ����£������Ƕ�����˫�ֽ��ϵġ���32λ����ģʽ�£���������Ǵ�˫�ֽ翪ʼ����ģ���ôCPU�ܿ��ٵض�ȡ�ڴ档ͼ~\ref{fig:structPic2}չʾ�����ʹ��\emph{gcc}����ô{\code S}�ṹ�����ڴ�������δ���ġ��������ڽṹ���в���������û��ʹ�õ��ֽڣ�������{\code y}(��{\code z})������˫�ֽ��ϡ���ͱ�������C�ж���Ľṹ�壬ʹ��{\code offsetof}����ƫ��������Ԫ���Լ��������Լ���ƫ��Ϊʲô��һ���õ��뷨��

��Ȼ�����ֻ���ڻ�������ʹ�ýṹ�壬����Ա�����Լ�����ƫ�Ƶ�ַ�����ǣ��������Ҫʹ��C�ͻ��Ľӿڼ�������ô�ڻ������C������Լ������μ���ṹ��Ԫ�ص�ƫ�Ƶ�ַ�Ƿdz���Ҫ�ģ�һ���鷳�ĵط��Dz�ͬ��C������������Ԫ�ص�ƫ�Ƶ�ַ�Dz�ͬ�ġ����磺���������Ѿ�֪���ģ�
\emph{gcc}�����������ṹ��{\code S}��ͼ~\ref{fig:structPic2}�����ǣ�Borland�ı������������ṹ����ͼ~\ref{fig:structPic1}��C�������ṩ��ָ�����ݶ���ķ��������ǣ�ANSI C��׼��û��ָ�����Ǹ������ɣ���˲�ͬ�ı�����ʹ�ò�ͬ�ķ���������ڴ��ַ���롣



%Borland's compiler has a flag, {\code -a}, that can be
%used to define the alignment used for all data. Compiling with {\code -a 4}
%tells \emph{bcc} to use double word alignment. Microsoft's compiler
%provides a {\code \#pragma pack} directive that can be used to set
%the alignment (consult Microsoft's documentation for details). Borland's
%compiler also supports Microsoft's pragma

\emph{gcc}\index{������!gcc!\_\_attribute\_\_}��������һ�����Ǹ��ӵķ�����ָ����ַ���롣��������ʹ��������﷨��ָ���������͵ĵ�ַ���롣���磬����һ�У�
\begin{lstlisting}[stepnumber=0]{}
  typedef short int unaligned_int __attribute__((aligned(1)));
\end{lstlisting}
\noindent������һ����Ϊ{\code unaligned\_int}�������ͣ������õ����ֽڽ���뷽ʽ��(�ǵģ�������{\code
\_\_attribute\_\_}��������Ŷ�����Ҫ�ģ�){\code aligned}�IJ���1������������2�ij˷�ֵ�������������ʾ���õ����������뷽ʽ��(2Ϊ�ֱ߽磬4��ʾ˫�ֽ磬
\emph{�ȵȡ�})����ṹ�����{\code y}Ԫ�ظ�Ϊ{\code unaligned\_int}���ͣ���ô\emph{gcc}������{\code y}��ƫ�Ƶ�ַΪ2.���ǣ�{\code z}��Ȼ����ƫ�Ƶ�ַ8��λ�ã���Ϊ˫�������͵�ȱʡ���뷽ʽΪ˫�ֶ��롣Ҫ��{\code z}��ƫ�Ƶ�ַΪ6����ô��Ҫ�ı��������Ͷ��塣

\begin{figure}[t]
\lstset{escapeinside=`',language=Pascal,%
}
\begin{lstlisting}[frame=tlrb,stepnumber=0]{}
struct S {
  short int x;    /* `2���ֽڵ�����' */
  int       y;    /* `4���ֽڵ�����' */
  double    z;    /* `8���ֽڵĸ�����'   */
} __attribute__((packed));
\end{lstlisting}
\caption{ʹ��\emph{gcc}��ѹ���ṹ�� \label{fig:packedStruct}\index{������!gcc!\_\_attribute\_\_}}
\end{figure}

\emph{gcc}������ͬ��������\emph{ѹ��}һ���ṹ�塣�����߱�����ʹ�þ�����С�Ŀռ�����������ṹ�塣ͼ~\ref{fig:packedStruct}չʾ��{\code S}��������ַ��������塣������ʽ�µ�{\code S}��ʹ�ÿ��ܵ����ٵ��ֽ�����14���ֽڡ�

Microsoft��Borland�ı�������֧��ʹ��{\code \#pragma}ָʾ���ķ�����ָ�����뷽ʽ��\index{������!Microsoft!pragma pack}
\begin{lstlisting}[stepnumber=0]{}
#pragma pack(1)
\end{lstlisting}
�����ָʾ�����߱����������ֽڽ�Ķ��뷽ʽ��ѹ���ṹ���е�Ԫ�ء�(\emph{Ҳ����˵}��û�ж�������ռ�)�����е�1������2��4��8��16���棬�ֱ�����ָ�����뷽ʽΪ�ֱ߽磬˫�ֽ磬���ֽ�ͽڱ߽硣���ָʾ���ڱ���һ��ָʾ����Ϊ��Ч֮ǰ������Ч����Ϳ��ܻᵼ��һЩ���⣬��Ϊ��Щָʾ��ͨ��ʹ����ͷ�ļ��С�������ͷ�ļ��ڰ����ṹ�������ͷ�ļ�֮ǰ�������������У���ô��Щ�ṹ��ķ��÷�ʽ��������ȱʡ�ķ��÷�ʽ��ͬ���⽫���·dz����صIJ��Ҵ��󡣳����еIJ�ͬģ�齫�Ὣ�ṹ��Ԫ��\emph{����}�ڲ�ͬ�ĵط���

\begin{figure}[t]
\lstset{escapeinside=`',language=Pascal,%
}
\begin{lstlisting}[frame=tlrb,stepnumber=0]{}
#pragma pack(push)    /* `������뷽ʽ��״ֵ̬' */
#pragma pack(1)       /* `����Ϊ�ֽڽ�'   */

struct S {
  short int x;    /* `2���ֽڵ�����' */
  int       y;    /* `4���ֽڵ�����' */
  double    z;    /* `8���ֽڵĸ�����'   */
};

#pragma pack(pop)     /* `�ָ�ԭʼ�Ķ��뷽ʽ' */
\end{lstlisting}
\caption{ʹ��Microsoft��Borland��ѹ���ṹ�� \label{fig:msPacked}\index{������!Microsoft!pragma pack}}
\end{figure}

��һ������������������⡣Microsoft��Borland��֧��������������浱ǰ���뷽ʽ״ֵ̬�����ָ�����ͼ~\ref{fig:msPacked}չʾ�����ʹ�����ַ�����
\index{�ṹ��!��ַ����|)}

\subsection{λ��s\index{�ṹ��!λ��|(}}

\begin{figure}[t]
\lstset{escapeinside=`',language=Pascal,%
}
\begin{lstlisting}[frame=tlrb,stepnumber=0]{}
struct S {
  unsigned f1 : 3;   /* `3���'  */
  unsigned f2 : 10;  /* `10���' */
  unsigned f3 : 11;  /* `11���' */
  unsigned f4 : 8;   /* `8���'  */
};
\end{lstlisting}
\caption{���������� \label{fig:bitStruct}}
\end{figure}

λ��������ָ���ṹ���еij�Ա�Ĵ�СΪֻʹ��ָ���ı���λ��������λ���Ĵ�С����һ��Ҫ��8�ı�����һ��λ���Ա�Ķ����\lstinline|unsigned int|��\lstinline|int|�ij�Ա������һ����ֻ���ڶ���ĺ���������ð�ź�λ���Ĵ�С��ͼ~\ref{fig:bitStruct}չʾ��һ�����ӡ���������һ��32λ�ı�������������ļ�������ɣ�
\begin{center}
\begin{tabular}{|c|c|c|c|}
\multicolumn{1}{c}{8������} & \multicolumn{1}{c}{11������} &
\multicolumn{1}{c}{10������} & \multicolumn{1}{c}{3������} \\
\hline \hspace{2em} f4 \hspace{2em} & \hspace{3em} f3 \hspace{3em}
& \hspace{3em} f2 \hspace{3em} & f1 \\
\hline
\end{tabular}
\end{center}
��һ��λ��ָ������˫�ֵ������Чλ����\footnote{ʵ���ϣ�ANSI/ISO C��׼��ʵ������η��ñ���λ������˱�������������ԡ����ǣ��ձ��C������(\emph{gcc}��\emph{Microsoft}��
\emph{Borland})�������������ñ���λ��}

���ǣ�����㿴����Щ����λʵ�������ڴ�������δ���ģ���ͻᷢ�ָ�ʽ��������˼򵥡��ѵ㷢���ڵ�λ���Խ�ֽڽ�ʱ����Ϊ��little endian�������ϵ��ֽڽ����෴��˳�򴢴浽�ڴ��С����磬{\code S}�ṹ�����ڴ��н�������ʾ��
\begin{center}
\begin{tabular}{|c|c||c|c||c||c|}
\multicolumn{1}{c}{5������} & \multicolumn{1}{c}{3������} &
\multicolumn{1}{c}{3������} & \multicolumn{1}{c}{5������} &
\multicolumn{1}{c}{8������} & \multicolumn{1}{c}{8������} \\ \hline
f2l & f1 &  f3l  & f2m & \hspace{1em} f3m \hspace{1em}
& \hspace{1.5em} f4 \hspace{1.5em} \\
\hline
\end{tabular}
\end{center}
\emph{f2l}������ʾ\emph{f2}λ���ĩβ�������λ(\emph{Ҳ����}����������Чλ)��\emph{f2m}������ʾ\emph{f2}����������Чλ��˫��ֱ�ߵĵط���ʾ�ֽڽ硣����㽫���е��ֽڷ���\emph{f2}��\emph{f3}λ�����½�ϵ���ȷ��λ�á�

\begin{figure}[t]
\centering
\begin{tabular}{|c*{8}{|p{1.3em}}|}
\hline �ֽ� $\backslash$ λ & 7 & 6 & 5 & 4 & 3 & 2 & 1 & 0 \\
\hline 0 & \multicolumn{8}{c|}{������(08h) } \\ \hline 1 &
\multicolumn{3}{c|}{�߼���Ԫ \# } & \multicolumn{5}{c|}{LBA��msb} \\
\hline 2 & \multicolumn{8}{c|}{�߼����ַ���м䲿��} \\
\hline 3 & \multicolumn{8}{c|}{�߼����ַ��lsb} \\
\hline 4 & \multicolumn{8}{c|}{���ݵij���} \\
\hline 5 & \multicolumn{8}{c|}{������} \\ \hline
\end{tabular}
\caption{SCSI�������ʽ\label{fig:scsi-read}}
\end{figure}

\begin{figure}[t]
\lstset{escapeinside=`',language=Pascal,%
}
\begin{lstlisting}[frame=lrtb]{}
#define MS_OR_BORLAND (defined(__BORLANDC__) \
                        || defined(_MSC_VER))

#if MS_OR_BORLAND
#  pragma pack(push)
#  pragma pack(1)
#endif

struct SCSI_read_cmd {
  unsigned opcode : 8;
  unsigned lba_msb : 5;
  unsigned logical_unit : 3;
  unsigned lba_mid : 8;    /* `�м�ı���λ' */
  unsigned lba_lsb : 8;
  unsigned transfer_length : 8;
  unsigned control : 8;
}
#if defined(__GNUC__)
   __attribute__((packed))
#endif
;

#if MS_OR_BORLAND
#  pragma pack(pop)
#endif
\end{lstlisting}
\caption{SCSI�������ʽ�Ľṹ\label{fig:scsi-read-struct}\index{������!gcc!\_\_attribute\_\_}
         \index{������!Microsoft!pragma pack}}
\end{figure}

�����ڴ�ķ��÷�ʽͨ�������Ǻ���Ҫ��������������Ҫ���͵������л�ӳ����д���(ʵ�������λ���Ƿdz���ͬ��)��Ӳ���豸�Ľӿ�ʹ�������ı���λ�Ƿdz��ձ�ģ���ʱʹ��λ���������Ƿdz����õġ�

\begin{figure}[t]
\centering
\begin{tabular}{|c||c||c||c||c|c||c|}
\multicolumn{1}{c}{8������} & \multicolumn{1}{c}{8������} &
\multicolumn{1}{c}{8������} & \multicolumn{1}{c}{8������} &
\multicolumn{1}{c}{3������} & \multicolumn{1}{c}{5������} &
\multicolumn{1}{c}{8������} \\ \hline ������ & ����ij��� & lba\_lsb
& lba\_mid &
logical\_unit  & lba\_msb & opcode \\
\hline
\end{tabular}
\caption{{\code SCSI\_read\_cmd}��λ��ͼ \label{fig:scsi-read-map}}
\end{figure}
\index{SCSI|(}
SCSI\footnote{Small Computer Systems Interface��С�ͼ����ϵͳ�ӿڣ�һ��Ӳ�̣�\emph{�ȵ�}�Ĺ�ҵ��׼}����һ�����ӡ�SCSI�豸��ֱ�Ӷ����ָ��Ϊ����һ�������ֽڵ���Ϣ���豸����ʽָ��Ϊͼ~\ref{fig:scsi-read}�еĸ�ʽ��ʹ��λ��������������ѵ���
\emph{�߼������ַ(logical block address)}�����ڴ������п�Խ��������ͬ���ֽڡ���ͼ~\ref{fig:scsi-read}�У�����Կ�����������big endian�ĸ�ʽ����ġ�ͼ~\ref{fig:scsi-read-struct}չʾ��һ����ͼ�����б������й����Ķ��塣ǰ���ж�����һ���꣬��δ�������Microsoft��Borland������������ʱ��������Ϊ�档���ܱȽϻ��ҵIJ�����11�е�14�С����ȣ�����ܻ���Ϊʲô\lstinline|lba_mid|��
\lstinline|lba_lsb|λ��Ҫ�ֿ������壬�����Ƕ����һ��16λ����ԭ������������big endian˳�򴢴�ġ�������������һ��16λ������little endian˳�������档��Σ�\lstinline|lba_msb|�� \lstinline|logical_unit|λ�������ƺ������ˣ����ǣ�������������������DZ������������˳�����ڷš�ͼ~\ref{fig:scsi-read-map}չʾ����Ϊһ��48λ��ʵ�壬����λ��ͼ�������ġ�(�ֽڽ�ͬ������˫��ֱ������ʾ��)�������ڴ�������little endian�ĸ�ʽ�����棬��ô����λ����Ҫ��ĸ�ʽ�����С�
(ͼ~\ref{fig:scsi-read}).

\begin{figure}[t]
\lstset{escapeinside=`',language=Pascal,%
}
\begin{lstlisting}[frame=lrtb]{}
struct SCSI_read_cmd {
  unsigned char opcode;
  unsigned char lba_msb : 5;
  unsigned char logical_unit : 3;
  unsigned char lba_mid;    /* `�м�ı���λ' */
  unsigned char lba_lsb;
  unsigned char transfer_length;
  unsigned char control;
}
#if defined(__GNUC__)
   __attribute__((packed))
#endif
;
\end{lstlisting}
\caption{��һ��SCSI�������ʽ�Ľṹ\label{fig:scsi-read-struct2}
         \index{������!gcc!\_\_attribute\_\_}\index{������!Microsoft!pragma pack}}
\end{figure}

���ǵø���һ�㣬����֪��\lstinline|SCSI_read_cmd|�Ķ�����Microsoft
C�������в�����ȫ��ȷ���������\lstinline|sizeof(SCSI_read_cmd)|����ʽ����ֵ�ˣ�
Microsoft
C������8,������6��������ΪMicrosoft������ʹ��λ���������������λ��Ʊ���ͼ����Ϊ���е�λ�򶼱�����Ϊ\lstinline|unsigned|���ͣ����Ա������ڽṹ���ĩβ���������ֽ�ʹ������Ϊһ��˫�����͵�����������������ͨ����\lstinline|unsignedshort|������е�λ�������������������ڣ�Microsoft����������Ҫ�����κε�����ֽڣ���Ϊ�����ֽ��������ֽ������͵�������\footnote{���ҵIJ�ͬ���͵�λ�򽫵��·dz����ҵ���Ϊ��������Ҫ�Լ�ȥʵ�顣}��������ı䣬�����ı�����Ҳ����ȷ������ͼ~\ref{fig:scsi-read-struct2}չʾ������һ�ֶ��壬�������е����ֱ������Ϲ�������ͨ��ʹ��\lstinline|unsignedchar|�����˳�2λ�������������λ������⡣
\index{SCSI|)}

�������ǰ������۷dz����ҵĶ��ߣ��벻Ҫ���١����������ǻ��ҵģ�ͨ��������ȫ�ر���ʹ��λ�������λ�������ֶ��ؼ����޸ı���λ�����߷����ܱ���һЩ���ҡ�

\index{�ṹ��!λ��|)}

%TODO:discuss alignment issues and struct size issues

\subsection{�ڻ��������ʹ�ýṹ��}

�����������ۣ��ڻ�������з��ʽṹ��������ڷ������顣��Ϊһ���򵥵����ӣ�����һ�������д����һ�������򣺽�0д�뵽{\code S}�ṹ���{\code y}�С��ٶ���������ԭ���������ģ�
\begin{lstlisting}[stepnumber=0]{}
void zero_y( S * s_p );
\end{lstlisting}
\noindent ���������£�
\begin{AsmCodeListing}
%define      y_offset  4
_zero_y:
      enter  0,0
      mov    eax, [ebp + 8]      ; �Ӷ�ջ�еõ�s_p(�ṹ���ָ��)
      mov    dword [eax + y_offset], 0
      leave
      ret
\end{AsmCodeListing}

C�����������һ���ṹ�嵱����ֵ���ݸ����������ǣ�ͨ���ⶼ��һ�������⡣������ֵ������ʱ���ڽṹ���е��������ݶ����븴�Ƶ���ջ�У�Ȼ���ڳ��������ó���ʹ�á���һ���ṹ��ָ����������и��ߵ�Ч�ʡ�

C����ͬ������һ���ṹ��������Ϊһ�������ķ���ֵ�������ԣ�һ���ṹ�岻��ͨ�����浽{\code EAX}�Ĵ����������ء���ͬ�ı�����������������ķ���Ҳ��ͬ��һ���������ձ�ʹ�õĽ�����������ڲ���д����������Я��һ���ṹ��ָ����������ָ������������ֵ���뵽�ṹ���У�����ṹ�����ڵ��õij������涨��ġ�

����������(����NASM)��������Ļ������ж���ṹ�������֧�֡���������������õ�����ϸ����Ϣ��

% add section on structure return values for functions

\index{�ṹ��|)}

\section{������Ժ�C++\index{C++|(}}

C++���������C���Ե�һ����չ��ʽ������C���Ժͻ�����ԽӿڵĻ�������ͬ��������C++�����ǣ���һЩ������Ҫ������ͬ����ӵ��һЩ������Ե�֪ʶ�����ܺ���������C++�е�һЩ��չ���֡���һ�ڼٶ����Ѿ���һ����C++����֪ʶ��

\subsection{���غ��������ָı�\index{C++!���ָı�|(}}
\label{subsec:mangling}
\begin{figure}
\centering
\begin{lstlisting}[frame=tlrb]{}
#include <stdio.h>

void f( int x )
{
  printf("%d\n", x);
}

void f( double x )
{
  printf("%g\n", x);
}
\end{lstlisting}
\caption{������Ϊ{\code f()}�ĺ���\label{fig:twof}}
\end{figure}

C++������ͬ�ĺ���(�����Ա����)ʹ��ͬ���ĺ����������塣����ֹһ����������ͬһ��������ʱ����Щ�����ͳ�Ϊ\emph{���غ���}����C�����У�����������������ʹ�õĺ�������һ������ô������������һ��������Ϊ�������ӵ�Ŀ���ļ��У�һ�����������ҵ��������塣���磬����ͼ~\ref{fig:twof}�еĴ��롣�ȼ۵Ļ����뽫����������Ϊ{\code \_f}�ı�ţ����������Ǵ���ġ�

C++ʹ�ú�Cһ�������ӹ��̣�����ͨ��ִ��\emph{���ָı�}���޸�������Ǻ����ķ������������������ij�ֳ����ϣ�CҲ���Ѿ�ʹ�������ָıࡣ�����������ı��ʱ������C��������������һ���»��ߡ����ǣ�C���Խ���ͬ���ķ������ı�ͼ~\ref{fig:twof}�е���������������ô�������һ������C++ʹ��һ�����߼��ĸı���̣�Ϊ��Щ��������������ͬ�ı�š����磺ͼ~\ref{fig:twof}�еĵ�һ����������DJGPPָ��Ϊ���{\code \_f\_\_Fi}�����ڶ���������ָ��Ϊ{\code \_f\_\_Fd}�������ͱ������κε����Ӵ���
% check to make sure that DJGPP does still but an _ at beginning for C++

���ҵ��ǣ�������C++����θı����ֲ�û��һ����׼�����Ҳ�ͬ�ı������ı������Ҳ��һ�������磬Borland C++��ʹ�ñ��{\code @f\$qi}��{\code @f\$qd}����ʾͼ~\ref{fig:twof}�е��������������ǣ����򲢲�����ȫ����ġ��ı������ֱ���ɺ�����\emph{ǩ��}��һ��������ǩ����ͨ����Я���IJ�����˳�������������ġ�ע�⣬��Я����һ��{\code int}�����ĺ��������ĸı����ֵ�ĩβ����һ��\emph{i}(����DJGPP ��Borland����һ��)����Я����һ��{\code double}�����ĺ��������ĸı����ֵ�ĩβ����һ��\emph{d}�������һ����Ϊ{\code f}�ĺ���������ԭ�����£�
\begin{lstlisting}[stepnumber=0]{}
  void f( int x, int y, double z);
\end{lstlisting}
\noindent DJGPP������������ָı��{\code \_f\_\_Fiid}��Borland��������ı��
{\code @f\$qiid}��

�����ķ������Ͳ�\emph{����}����ǩ����һ���֣������Ҳ������뵽���ĸı������С������ʵ��������C++�е�һ�����ع���ֻ��ǩ��Ψһ�ĺ����������ء��������ܿ����ģ������C++�ж������������ֺ�ǩ����һ���ĺ�������ô���ǽ��õ�ͬ����ǩ�������⽫����һ�����Ӵ���ȱʡ����£����е�C++��������������ָı࣬��������Щû�����صĺ�����������һ���ļ�ʱ����������û�з���֪��һ���ض��ĺ���������������������е����ָıࡣ��ʵ�ϣ��ͺ���ǩ���ķ���һ����������ͬ��ͨ������������������ı�ȫ�ֱ����ı���������ˣ��������һ���ļ��ж�����һ��ȫ�ֱ���Ϊijһ����Ȼ����ͼ����һ���ļ�����һ�������������ʹ��������ô������һ�����Ӵ���C++������Ա���Ϊ\emph{���Ͱ�ȫ����}��
\index{C++!���Ͱ�ȫ����}��ͬ����¶����һ�����͵Ĵ���ԭ�Ͳ�һ�¡�����һ��ģ���к����Ķ��������һ��ģ��ʹ�õĺ���ԭ�Ͳ�һ��ʱ���ͷ������ִ�����C�У�����һ���dz��ѵ��Գ��������⡣C�����ܲ�׽�����ִ��󡣳��򽫱���������ӣ����ǽ�����δ����IJ���������������õĴ���Ὣ�ͺ���������һ��������ѹ��ջ��һ������C++�У���������һ�����Ӵ���

��C++�������﷨����һ����������ʱ����ͨ���鿴���ݸ������IJ�����������Ѱ��ƥ��ĺ���\footnote{���ƥ�䲢��һ��Ҫ�Ǿ�ȷƥ�䣬��������ͨ��ǿ��ת�Ͳ���������ƥ�䡣������̵Ĺ��򳬳��˱���ķ�Χ������һ��C++�������õ�����ϸ����Ϣ��}��������ҵ���һ��ƥ��ĺ�������ôͨ��ʹ�ñ����������ָı������������һ��{\code CALL}��������ȷ�ĺ�����

��Ϊ��ͬ�ı�����ʹ�ò�ͬ�����ָı�������Բ�ͬ�����������C++������ܲ��������ӵ�һ�𡣵�����ʹ��һ��Ԥ�����C++��ʱ�������ʵ�Ƿdz���Ҫ�ģ����������д��һ������C++������ʹ�õĻ�������ô������֪��Ҫʹ�õ�C++������ʹ�õ����ָı����(��ʹ�����潫���͵ļ���)��

������ѧ�����ܻ�ѯ����ͼ~\ref{fig:twof}�еĴ��뵽���ܲ�����Ԥ�ڰ㹤������ΪC++�ı������к����ĺ���������ô
{\code printf}�����ı࣬�����������������һ�������{\code \_printf}����{\code CALL}���á�����һ���dz���ȷ�ĵ��ǣ����{\code printf}��ԭ�ͱ��򵥵ط������ļ��Ŀ�ʼ���֣���ô��ͽ�������ԭ��Ϊ��
\begin{lstlisting}[stepnumber=0]{}
  int printf( const char *, ...);
\end{lstlisting}
\noindent DJGPP��������ı�Ϊ{\code
\_printf\_\_FPCce}��({\code F}��ʾ\emph{function������}��{\code P}��ʾ\emph{pointer��ָ��}��{\code C}��ʾ\emph{const������}�� {\code c}��ʾ
\emph{char}��{\code e}��ʾʡ�Ժš�)��ô���������������C���е�{\code printf}��������Ȼ��������һ�ַ�����C++������������C���롣���Ƿdz���Ҫ�ģ���Ϊ��������\emph{����}�dz����õľɵ�C���롣�������������������C�����⣬C++ͬ�����������ʹ���������C�ı�Լ���Ļ����롣

\index{C++!extern ""C""|(}
C++��չ��{\code extern}�ؼ��֣�����������ָ�������εĺ�����ȫ�ֱ���ʹ�õ�������CԼ������C++�����У�����Щ������ȫ�ֱ���ʹ����\emph{C����}�����磬Ϊ������{\code printf}ΪC���ӣ���ʹ�������ԭ�ͣ�
\begin{lstlisting}[language=C++,stepnumber=0]{}
extern "C" int printf( const char *, ... );
\end{lstlisting}
\noindent ��͸��߱�������Ҫ�����������ʹ��C++�����ָı���򣬶�ʹ��C��������������ǣ�����������ˣ���ô{\code printf}�����������ء�����ṩ��һ�����׵ķ�������C++�ͻ�����ӿ��ϣ�ʹ��C���Ӷ���һ��������Ȼ����ʹ��C����Լ����

Ϊ�˷��㣬C++ͬ���������庯����ȫ�ֱ������C���ӡ�ͨ��������ȫ�ֱ������þ��������ű�ʾ��
\lstset{escapeinside=`',language=Pascal,%
}
\begin{lstlisting}[stepnumber=0,language=C++]{}
extern "C" {
  /* `C���ӵ�ȫ�ֱ����ͺ���ԭ��' */
}
\end{lstlisting}

��������˵����C/C++�������е�ANSI Cͷ�ļ�����ᷢ����ÿ��ͷ�ļ����涼���������������
\begin{lstlisting}[stepnumber=0,language=C++]{}
#ifdef __cplusplus
extern "C" {
#endif
\end{lstlisting}
\noindent �����ڵײ���һ�������վ��������ŵ�ͬ���Ľṹ��C++�����������˺�{\code \_\_cplusplus}(��\emph{����}��ͷ���»���)������Ĵ���Ƭ�������C++�����룬��ô����ͷ�ļ��ͱ�һ��{\code extern~"C"}��Χ�����ˣ��������ʹ��C�����룬�Ͳ���ִ���κβ���
(��Ϊ����{\code extern~"C"}��C������������һ���﷨����)������Ա����ʹ��ͬ���ļ��������ڻ������д���һ���ܱ�C��C++ʹ�õ�ͷ�ļ���
\index{C++!extern ""C""|)}
\index{C++!���ָı�|)}

\begin{figure}
\lstset{escapeinside=`',language=Pascal,%
}
\begin{lstlisting}[language=C++,frame=tlrb]{}
void f( int & x )     // `\&��ʾ��һ�����ò���' { x++; }

int main()
{
  int y = 5;
  f(y);               // `����������y��ע������û��\&!'
  printf("%d\n", y);  // `��ʾ6!'
  return 0;
}
\end{lstlisting}
\caption{���õ�����\label{fig:refex}}
\end{figure}

\subsection{����\index{C++!����|(}}

\emph{����}��C++����һ�������ԡ��������㴫�ݲ�����������������Ҫ��ȷʹ��ָ�롣���磬����ͼ~\ref{fig:refex}�еĴ��롣��ʵ�ϣ����ò����Ƿdz��򵥣�ʵ�������Ǿ���ָ�롣ֻ�DZ������Գ���Ա����������(����Pascal��������{\code
var}��������ָ����ִ��)���������������˺������õĵ�7�д���Ļ�����ʱ������{\code
y}��\emph{��ַ}���ݸ�����������������û��������д��{\code
f}��������ô���Dz�����������ͺ���ԭ�������Ƶģ�\footnote{��Ȼ�����ǿ�����ʹ��C���������������������������ָı࣬����С��~\ref{subsec:mangling}�����۵�}��
\begin{lstlisting}[stepnumber=0]{}
  void f( int * xp);
\end{lstlisting}

�����Ƿdz�����ģ��ر��Ƕ��������������˵�Ƿdz����õġ��������������C++����һ�����ԣ����������ڶԽṹ��������ͽ��в���ʱ������ͨ�������һ�ֹ��ܡ����磬һ���ձ��ʹ���Ǹ���Ӻ�({\code +})������ܽ��ַ����������������Ĺ��ܡ���ˣ����{\code a}��{\code
b}���ַ�������ô{\code a~+~b}���õ�{\code a}��{\code b}���Ӻ���ַ�����ʵ���ϣ�C++���Ե���һ���������������(��ʵ�ϣ�����ı���ʽ�����ú����ı�ʾ������дΪ��{\code operator~+(a,b)})��Ϊ�����Ч�ʣ����˿��ܻ�ϣ�������ַ����ĵ�ַ�����洫�����ǵ�ֵ����û�����ã���ô����Ҫ��������{\code
operator~+(\&a,\&b)}��������Ҫ��������������﷨����дӦΪ��{\code \&a~+~\&b}�����Ƿdz���׾���һ��ҵġ����ǣ�ͨ��ʹ�����ã��������������д��{\code a~+~b}�������Ϳ������dz���Ȼ��
\index{C++!����|)}

\subsection{��������\index{C++!��������|(}}

��ĿǰΪֹ��\emph{��������}����C++����һ������\footnote{
C������ͨ��֧���������ԣ���������ANSI C����չ��}�����������յ���Ӧ�ÿ���ȡ�����׷�����ģ�Я�������ģ�����Ԥ��������ĺꡣ����һ����C�У���дһ��������ƽ���ĺ�����������ģ�
\begin{lstlisting}[stepnumber=0]{}
#define SQR(x) ((x)*(x))
\end{lstlisting}
\noindent ��ΪԤ��������������C�����ü򵥵��滻�������ڴ��������£�Բ������Ҫ��������ȷ���������ֵ�����ǣ���ʹ������汾Ҳ���ܸ���{\code SQR(x++)}����ȷ�𰸡�

\begin{figure}
\begin{lstlisting}[language=C++,frame=tlrb]{}
inline int inline_f( int x )
{ return x*x; }

int f( int x )
{ return x*x; }

int main()
{
  int y, x = 5;
  y = f(x);
  y = inline_f(x);
  return 0;
}
\end{lstlisting}
\caption{�������������� \label{fig:InlineFun}}
\end{figure}


��֮���Ա�ʹ������Ϊ����ȥ�˽���һ���򵥺����ĺ������õĶ���ʱ�俪֧�������ӳ�����һ�������ģ�ִ��һ���������ð����ü���������һ���dz��򵥵ĺ�����˵���������к������õ�ʱ����ܱ�ʵ����ִ�к�����IJ�����ʱ�仹Ҫ�࣡����������һ����Ϊ�Ѻõ�������д����ķ������ô��뿴������һ����׼�ĺ�������������\emph{����}{\code CALL}ָ���ܵ��õ���ͨ����顣�������������ĵ��ñ���ʽ�ĵط�����ִ�к����Ĵ����滻��C++����ͨ���ں�������ǰ����{\code inline}�ؼ�����ʹ������Ϊ���������������������ͼ~\ref{fig:InlineFun}�������ĺ�������10�ж�{\code f}�ĵ��ý�ִ��һ����׼�ĺ�������(�ڻ�������У��ٶ�{\code x}�ĵ�ַΪ{\code ebp-8}��{\code y}��ַΪ{\code ebp-4})��
\begin{AsmCodeListing}
      push   dword [ebp-8]
      call   _f
      pop    ecx
      mov    [ebp-4], eax
\end{AsmCodeListing}
���ǣ���11�ж�{\code inline\_f}�ĵ��ý��õ����½����
\begin{AsmCodeListing}
      mov    eax, [ebp-8]
      imul   eax, eax
      mov    [ebp-4], eax
\end{AsmCodeListing}

��������£�ʹ�����������������ŵ㡣���ȣ������������졣û�в�����Ҫѹ��ջ�У�Ҳ����Ҫ�����ͻٻ���ջ֡��Ҳ����Ҫ���з�֧����Σ�������������ʹ�õĴ����Ƿdz��٣�����һ������������˵����ȷ�ģ����Dz���������������¶�����ȷ�ġ�

������������Ҫ�ŵ����������벻��Ҫ���ӣ����Զ���ʹ������������\emph{����}Դ�ļ���˵�����������Ĵ��붼������Ч��ǰ��Ļ����������չʾ����һ�㡣���ڷ����������ĵ��ã�ֻҪ��֪������������ֵ���ͣ�����Լ���ͺ����ĺ����������е���Щ��Ϣ�����ԴӺ�����ԭ���еõ������ǣ�ʹ�������������ã��ͱ���֪��������������д��롣�����ζ������ı���һ�����������е��κβ��֣���ô\emph{����}ʹ�������������Դ�ļ��������±��롣����һ�¶��ڷ������������������ԭ��û�иı䣬ͨ��ʹ�����������Դ�ļ��Ͳ���Ҫ���±��롣�������е���Щԭ�����������Ĵ���ͨ��������ͷ�ļ��С�������Υ������C�����б�׼���ȶ��Ϳ���׼��ִ�еĴ������\emph{������}������ͷ�ļ��С�
\index{C++!��������|)}

\begin{figure}[t]
\lstset{escapeinside=`',language=Pascal,%
}
\begin{lstlisting}[language=C++,frame=tlrb]{}
class Simple {
public:
  Simple();                // `ȱʡ�Ĺ��캯��'
  ~Simple();               // `��������'
  int get_data() const;    // `������Ա'
  void set_data( int );
private:
  int data;                // `���ݳ�Ա'
};

Simple::Simple()
{ data = 0; }

Simple::~Simple()
{ /* `�ճ�����' */ }

int Simple::get_data() const
{ `����ֵ'; }

void Simple::set_data( int x )
{ data = x; }
\end{lstlisting}
\caption{һ���򵥵�C++��\label{fig:SimpleClass}}
\end{figure}

\subsection{��\index{C++!��|(}}

C++�е���������һ��\emph{����}���͡�һ������������ݳ�Ա(data member)�ͺ�����Ա(function member)\footnote{��C++�У�ͨ����֮Ϊ\emph{��Ա����(member
function)}���߸�Ϊ�ձ�س�֮Ϊ\emph{����(method)}\index{����}��}�����仰˵���ǣ������ɸ�������������ݺͺ�����ɵ�һ��{\code struct}�ṹ�塣������ͼ~\ref{fig:SimpleClass}�ж�����Ǹ��򵥵��ࡣһ��{\code Simple}���͵ı����dz������ڰ���һ��{\code int}��Ա�ı�׼C{\code struct}�ṹ�塣\MarginNote{��ʵ�ϣ�C++ʹ��{\code this}�ؼ��ִӳ�Ա�����ڲ�������ָ��˺����������õĶ����ָ�롣}��Щ������\emph{����}���浽ָ���ṹ����ڴ��С����ǣ���Ա���������������Dz�һ���ġ����Ǵ�����һ��\emph{����}�IJ��������������һ��ָ���Ա�����������õĶ����ָ�롣

\begin{figure}[t]
\begin{lstlisting}[stepnumber=0]{}
void set_data( Simple * object, int x )
{
  object->data = x;
}
\end{lstlisting}
\caption{Simple::set\_data()��C�汾\label{fig:SimpleCVer}}
\end{figure}


\begin{figure}[t]
\begin{AsmCodeListing}
_set_data__6Simplei:           ; �ı�������
      push   ebp
      mov    ebp, esp

      mov    eax, [ebp + 8]   ; eax = ָ������ָ��(this)
      mov    edx, [ebp + 12]  ; edx = �����
      mov    [eax], edx       ; data��ƫ�Ƶ�ַ0��

      leave
      ret
\end{AsmCodeListing}
\caption{����Simple::set\_data( int )����� \label{fig:SimpleAsm}}
\end{figure}


���磬����ͼ~\ref{fig:SimpleClass}�е�{\code Simple}��ij�Ա����{\code set\_data}�������C��������д�˺������������������������ȷ����һ��ָ���Ա�����������õĶ����ָ�룬��ͼ~\ref{fig:SimpleCVer}��ʾ��ʹ��\emph{DJGPP}����������{\code -S}ѡ��(\emph{gcc}��Borland������Ҳ��һ��)�����߱��������һ�������˴�������ĵȼ۵Ļ�����Դ����Դ�ļ�������\emph{DJGPP}��\emph{gcc}���������˻��Դ�ļ�����{\code .s}��չ����β�ģ����Dz��ҵ���ʹ�õ��﷨��AT\&T��������﷨�������﷨��NASM��MASM�﷨����dz���\footnote{\emph{gcc}����ϵͳ������һ�������Լ��ij�Ϊ\emph{gas}\index{gas}�Ļ������
\emph{gas}�����ʹ��AT\&T�﷨����˱�������\emph{gas}�ĸ�ʽ��������롣��ҳ���кü�ҳ��������INTEL��AT\&T�﷨������ͬʱ��һ����Ϊ{\code a2i}����ѳ���
({http://www.multimania.com/placr/a2i.html})���˳���AT\&T��ʽת������NASM��ʽ��}��(Borland��MS����������һ����{\code .asm}��չ����β��Դ�ļ���ʹ�õ���MASM�﷨��)
ͼ~\ref{fig:SimpleAsm}չʾ�˽�\emph{DJGPP}�����ת����NASM�﷨��Ĵ��룬�����˲������Ŀ�ĵ�ע�͡��ڵ�һ���У�ע���Ա����{\code set\_data}�ĺ�������ָ��Ϊһ���ı��ı�ţ��˱����ͨ�������Ա�������������Ͳ�����õ��ġ������������ȥ������Ϊ�������п���Ҳ����Ϊ{\code set\_data}�ij�Ա����������������Ա����\emph{����}ʹ�ò�ͬ�ı�š�����֮���Ա������ȥ��Ϊ������ͨ��Я���������������س�Ա����{\code set\_data}�������׼��C++���������ǣ�����ǰһ������ͬ�ı������ڸı���ʱ������Ϣ�ķ�ʽҲ��ͬ��

����ĵ�2�͵�3�У���������Ϥ�ĺ����Ŀ�ʼ���֡��ڵ�5�У��Ѷ�ջ�еĵ�һ���������浽{\code
EAX}���ˡ��Ⲣ\emph{����}����{\code x}������������Ǹ����صIJ���\footnote{��ƽ��һ����\emph{û�ж���}�������ڻ������У�} ������ָ��˺����������õĶ����ָ�롣��6�н�����{\code x}���浽{\code EDX}���ˣ�����7���ֽ�{\code EDX}���浽��{\code EAX}ָ���˫���С�����{\code Simple}�����е�{\code data}��Ա��Ҳ��������е�Ψһ�����ݣ���������{\code Simple}�ṹ����ƫ�Ƶ�ַΪ0�ĵط���

\begin{figure}[tp]
\lstset{escapeinside=`',language=Pascal,%
}
\begin{lstlisting}[frame=tlrb,language=C++]{}
class Big_int {
public:
   /*
   * Parameters:
   *   size           - `��ʾ�������޷������������δ�С'
   *
   *   initial_value  - `��Big\_int��ֵ��ʼ��Ϊһ���������޷�������'
   */
  explicit Big_int( size_t   size,
                    unsigned initial_value = 0);
  /*
   * Parameters:
   *   size           - `��ʾ�������޷������������δ�С'
   *
   *   initial_value  - `��Big\_int��ֵ��ʼ��Ϊһ������һ����ʮ�����Ʊ�ʾ��ֵ���ַ���'
   *
   */
  Big_int( size_t       size,
           const char * initial_value);

  Big_int( const Big_int & big_int_to_copy);
  ~Big_int();

  // `����Big\_int�Ĵ�С (���޷������ε���ʽ)'
  size_t size() const;

  const Big_int & operator = ( const Big_int & big_int_to_copy);
  friend Big_int operator + ( const Big_int & op1,
                              const Big_int & op2 );
  friend Big_int operator - ( const Big_int & op1,
                              const Big_int & op2);
  friend bool operator == ( const Big_int & op1,
                            const Big_int & op2 );
  friend bool operator < ( const Big_int & op1,
                           const Big_int & op2);
  friend ostream & operator << ( ostream &       os,
                                 const Big_int & op );
private:
  size_t      size_;    // `�޷�������Ĵ�С'
  unsigned *  number_;  // `ָ��ӵ����ֵ���޷��������ָ��'
};
\end{lstlisting}
\caption{Big\_int��Ķ���\label{fig:BigIntClass}}
\end{figure}

\begin{figure}[tp]
\lstset{escapeinside=`',language=Pascal,%
}
\begin{lstlisting}[frame=tlrb,language=C++]{}
// `�������ԭ��'
extern "C" {
  int add_big_ints( Big_int &       res,
                    const Big_int & op1,
                    const Big_int & op2);
  int sub_big_ints( Big_int &       res,
                    const Big_int & op1,
                    const Big_int & op2);
}

inline Big_int operator + ( const Big_int & op1, const Big_int & op2)
{
  Big_int result(op1.size());
  int res = add_big_ints(result, op1, op2);
  if (res == 1)
    throw Big_int::Overflow();
  if (res == 2)
    throw Big_int::Size_mismatch();
  return result;
}

inline Big_int operator - ( const Big_int & op1, const Big_int & op2)
{
  Big_int result(op1.size());
  int res = sub_big_ints(result, op1, op2);
  if (res == 1)
    throw Big_int::Overflow();
  if (res == 2)
    throw Big_int::Size_mismatch();
  return result;
}
\end{lstlisting}
\caption{Big\_int�����������\label{fig:BigIntAdd}}
\end{figure}

\subsubsection{����}
\index{C++!Big\_int����|(}
��һ��ʹ���������е�˼�봴����һ��C++�ࣺ�������������С���޷������Ρ���ΪҪ���������С�����Σ���������Ҫ���浽һ���޷������ε�����(˫�ֵ�)�С�����ʹ�ö�̬������ʵ�������С�����Ρ�˫�������෴�ķ��򴢴��\footnote{Ϊʲô�أ���Ϊ�ӷ����㽫������Ŀ�ʼ����ʼ����ǰ���в�����}  (\emph{Ҳ����˵}��˫�ֵ������Чλ���±�Ϊ0)��ͼ~\ref{fig:BigIntClass}չʾ��{\code Big\_int}��Ķ���\footnote{��������Դ�������õ�������ӵ�ȫ���Ĵ��롣�����н�ֻ���ò��ִ��롣}��{\code Big\_int}�Ĵ�С��ͨ������{\code unsigned}����Ĵ�С�õ��ģ����������������ݡ������е�{\code size\_}���ݳ�Ա��ƫ�Ƶ�ַΪ0����{\code number\_}��Ա��ƫ��Ϊ4��

Ϊ�˼򵥻���Щ���ӣ�ֻ��ӵ�д�С��ͬ������Ķ���ʵ���ſ����໥���мӼ�������

��������������캯��(constructor)����һ�����캯��(��9��)ʹ����һ���������޷�����������ʼ����ʵ�����ڶ������캯��(��18��)ʹ����һ������һ��ʮ������ֵ���ַ�������ʼ����ʵ�������������캯��(��21��)��\emph{�������캯��(copy
constructor)}\index{C++!�������캯��}��

��Ϊ����ʹ�õ��ǻ�����ԣ��������۵Ľ������ڼӷ��ͼ����������ι�����ͼ~\ref{fig:BigIntAdd}չʾ������Щ�������صIJ���ͷ�ļ�������չʾ����δ�������������û�������Ϊ��ͬ�ı�����ʹ����ȫ��ͬ�����ָı�������ı���������������Դ��������������������������C���ӻ��������ʹ���ڲ�ͬ�����������ֲ����������Щ�����Һ�ֱ�ӵ����ٶ�һ���졣�����ͬ����ȥ�˴ӻ�����׳��쳣�ı�Ҫ��

Ϊʲô������ʹ�õ�ȫ���ǻ�������أ�����һ�£���ִ�ж౶��������ʱ����λ�����һ��˫����ȥ����һ����Ч��˫�ֽ��мӷ�������C++(�� C)������������Ա����CPU�Ľ�λ��־λ��ֻ��ͨ����C++���������¼������λ��־λ��������������һ��˫�ֽ��мӷ�����������ִ������ӷ�������ʹ�û����������д��������Ч����Ϊ�����Է��ʽ�λ��־λ������ʹ��{\code ADC}ָ�����Զ�����λ��־λ���ϣ����������е�����

Ϊ�˼򻯣�ֻ��{\code add\_big\_ints}�Ļ������������ۡ��������������Ĵ���(����{\code big\_math.asm})��
\begin{AsmCodeListing}[label=big\_math.asm]
segment .text
        global  add_big_ints, sub_big_ints
%define size_offset 0
%define number_offset 4

%define EXIT_OK 0
%define EXIT_OVERFLOW 1
%define EXIT_SIZE_MISMATCH 2

; �ӷ��ͼ�������IJ���
%define res ebp+8
%define op1 ebp+12
%define op2 ebp+16

add_big_ints:
        push    ebp
        mov     ebp, esp
        push    ebx
        push    esi
        push    edi
        ;
        ; �������ã�esiָ��op1
        ;           ediָ��op2
        ;           ebxָ��res
        mov     esi, [op1]
        mov     edi, [op2]
        mov     ebx, [res]
        ;
        ; Ҫ��֤����3��Big_int��������ͬ���Ĵ�С
        ;
        mov     eax, [esi + size_offset]
        cmp     eax, [edi + size_offset]
        jne     sizes_not_equal                 ; op1.size_ != op2.size_
        cmp     eax, [ebx + size_offset]
        jne     sizes_not_equal                 ; op1.size_ != res.size_

        mov     ecx, eax                        ; ecx = Big_int�Ĵ�С
        ;
        ; ���ڣ��üĴ���ָ�����Ǹ��Ե�����
        ;      esi = op1.number_
        ;      edi = op2.number_
        ;      ebx = res.number_
        ;
        mov     ebx, [ebx + number_offset]
        mov     esi, [esi + number_offset]
        mov     edi, [edi + number_offset]

        clc                                     ; ���λ��־λ
        xor     edx, edx                        ; edx = 0
        ;
        ; �ӷ�ѭ��
add_loop:
        mov     eax, [edi+4*edx]
        adc     eax, [esi+4*edx]
        mov     [ebx + 4*edx], eax
        inc     edx                             ; ��Ҫ�ı��λ��־λ
        loop    add_loop

        jc      overflow
ok_done:
        xor     eax, eax                        ; ����ֵ = EXIT_OK
        jmp     done
overflow:
        mov     eax, EXIT_OVERFLOW
        jmp     done
sizes_not_equal:
        mov     eax, EXIT_SIZE_MISMATCH
done:
        pop     edi
        pop     esi
        pop     ebx
        leave
        ret
\end{AsmCodeListing}

ϣ�������˿�Ϊֹ���������״󲿷�����Ĵ��롣��25�е�27�н�{\code Big\_int}���󴫵ݵ�ָ�봢�浽�Ĵ����С���ס���õĽ�����ָ�롣��31�е�35�м�鱣֤������������Ĵ�С��һ���ġ�(ע�⣬
{\code size\_}��ƫ�Ʊ��ӵ�ָ�����ˣ�Ϊ�˷������ݳ�Ա��)��44�к͵�46�е����Ĵ�����������ָ�򱻸��Զ���ʹ�õ����飬�������ʹ�ö�������(ͬ����{\code number\_}��ƫ�Ʊ��ӵ�����ָ�����ˡ�)

\begin{figure}[tp]
\begin{lstlisting}[language=C++, frame=tlrb]{}
#include "big_int.hpp"
#include <iostream>
using namespace std;

int main()
{
  try {
    Big_int b(5,"8000000000000a00b");
    Big_int a(5,"80000000000010230");
    Big_int c = a + b;
    cout << a << " + " << b << " = " << c << endl;
    for( int i=0; i < 2; i++ ) {
      c = c + a;
      cout << "c = " << c << endl;
    }
    cout << "c-1 = " << c - Big_int(5,1) << endl;
    Big_int d(5, "12345678");
    cout << "d = " << d << endl;
    cout << "c == d " << (c == d) << endl;
    cout << "c > d " << (c > d) << endl;
  }
  catch( const char * str ) {
    cerr << "Caught: " << str << endl;
  }
  catch( Big_int::Overflow ) {
    cerr << "Overflow" << endl;
  }
  catch( Big_int::Size_mismatch ) {
    cerr << "Size mismatch" << endl;
  }
  return 0;
}
\end{lstlisting}
\caption{ {\code Big\_int}�ļ�Ӧ�� \label{fig:BigIntEx}}
\end{figure}

�ڵ�52�е�57�е�ѭ���У��������������������һ����ӣ����ȼӵ��������Ч��˫�֣�Ȼ������һ�����Ч��˫�֣�\emph{�ȵȡ�}�౶�������������������˳�������(��С��~\ref{sec:ExtPrecArith})����59��������������һ���������λ��־λ���������мӷ�����������Чλ��λ����Ϊ�������˫������little endian˳�򴢴�ģ�����ѭ��������Ŀ�ʼ����ʼ��������ǰֱ��������

ͼ~\ref{fig:BigIntEx}չʾ��{\code Big\_int}�ļ�Ӧ�õļ�̵����ӡ�ע�⣬{\code Big\_int}����������ȷ���������16�С���������ԭ�����ȣ�û��ת�����캯������һ���޷�������ת����{\code Big\_int}���͡���Σ�ֻ����ͬ��С��{\code Big\_int}����������������Ӳ����������������ת����������ģ���ΪҪ֪����ת���Ĵ�С�Ƿdz����ѵġ������һ�����߼���ʵ�ֽ����������С����֮�����ӡ����߲�������ΪҪʵ�������С��������Ӷ����������Ū�ù��ȸ��ӡ�(���ǣ�����������ʵ������)
\index{C++!Big\_int����|)}

\begin{figure}[tp]
\begin{lstlisting}[language=C++, frame=tlrb]{}
#include <cstddef>
#include <iostream>
using namespace std;

class A {
public:
  void __cdecl m() { cout << "A::m()" << endl; }
  int ad;
};

class B : public A {
public:
  void __cdecl m() { cout << "B::m()" << endl; }
  int bd;
};

void f( A * p )
{
  p->ad = 5;
  p->m();
}

int main()
{
  A a;
  B b;
  cout << "Size of a: " << sizeof(a)
       << " Offset of ad: " << offsetof(A,ad) << endl;
  cout << "Size of b: " << sizeof(b)
       << " Offset of ad: " << offsetof(B,ad)
       << " Offset of bd: " << offsetof(B,bd) << endl;
  f(&a);
  f(&b);
  return 0;
}
\end{lstlisting}
\caption{�򵥼̳�\label{fig:SimpInh}}
\end{figure}


\subsection{�̳кͶ�̬\index{C++!�̳�|(}}


\begin{figure}[tp]
\begin{AsmCodeListing}
_f__FP1A:                       ; �ı��ĺ�����
      push   ebp
      mov    ebp, esp
      mov    eax, [ebp+8]       ; eaxָ�����
      mov    dword [eax], 5     ; ad��ƫ��Ϊ0
      mov    eax, [ebp+8]       ; ������ĵ�ַ���ݸ�A::m()
      push   eax
      call   _m__1A             ; A::m()�ı��ij�Ա������
      add    esp, 4
      leave
      ret
\end{AsmCodeListing}
\caption{�򵥼̳еĻ����� \label{fig:FAsm1}}
\end{figure}

\emph{�̳�(Inheritance)}����һ����̳���һ��������ݺͳ�Ա���������磬����ͼ~\ref{fig:SimpInh}�еĴ��롣��չʾ�������࣬{\code A}��{\code B}��������{\code B}��ͨ���̳���{\code A}�õ��ġ������������£�
\begin{verbatim}
Size of a: 4 Offset of ad: 0
Size of b: 8 Offset of ad: 0 Offset of bd: 4
A::m()
A::m()
\end{verbatim}
ע�⣬����������ݳ�Ա{\code ad}({\code B}ͨ���̳�{\code A}�õ���)����ͬ��ƫ�ƴ������Ƿdz���Ҫ�ģ���Ϊ{\code f}����������һ��ָ�뵽һ��
{\code A}���������һ����{\code A}����(\emph{Ҳ����}��ͨ���̳еõ�)�Ķ��������С�ͼ~\ref{fig:FAsm1}չʾ�˴˺�����(�༭����)������(\emph{gcc}�õ���)��

\begin{figure}[tp]
\begin{lstlisting}[language=C++, frame=tlrb]{}
class A {
public:
  virtual void __cdecl m() { cout << "A::m()" << endl; }
  int ad;
};

class B : public A {
public:
  virtual void __cdecl m() { cout << "B::m()" << endl; }
  int bd;
};
\end{lstlisting}
\caption{ ��̬�̳� \label{fig:VirtInh}}
\end{figure}

\index{C++!��̬|(}
ע��������У�{\code a}��{\code b}������õĶ���{\code A}�ij�Ա����{\code m}���ӻ������У����ǿ��Կ�����{\code A::m()}�ĵ��ñ�Ӳ���뵽�������ˡ�������������������̣���Ա�����ĵ���ȡ���ڴ��ݸ������Ķ���������ʲô���������ν��\emph{��̬}��ȱʡ����£�C++�ص���������ԡ������ʹ��\emph{virtual} \index{C++!virtual}�ؼ�������������ͼ~\ref{fig:VirtInh}չʾ������޸��������ࡣ�������벻��Ҫ�޸ġ���̬���������෽����ʵ�֡����ҵ��ǣ����������ַ�����д��ʱ��\emph{gcc}��ʵ�ַ��������ڸı��У��������������ʵ�ַ�����ȣ����Ա�ø������ˡ�Ϊ�˼򵥻����۵�Ŀ�ģ�����ֻ�漰����Microsoft��Borland������Windowsʹ�õĶ�̬��ʵ�ַ���������ʵ�ַ����ܶ���û�иı��ˣ����ҿ�����δ������Ҳ����ı䡣

������Щ�ı䣬�����������£�
\begin{verbatim}
Size of a: 8 Offset of ad: 4
Size of b: 12 Offset of ad: 4 Offset of bd: 8
A::m()
B::m()
\end{verbatim}


\begin{figure}[tp]
\begin{AsmCodeListing}[commentchar=!]
?f@@YAXPAVA@@@Z:
      push   ebp
      mov    ebp, esp

      mov    eax, [ebp+8]
      mov    dword [eax+4], 5  ; p->ad = 5;

      mov    ecx, [ebp + 8]    ; ecx = p
      mov    edx, [ecx]        ; edx = ָ��vtable
      mov    eax, [ebp + 8]    ; eax = p
      push   eax               ; ��"this"ָ��ѹ��ջ��
      call   dword [edx]       ; ������vtable��ĵ�һ������
      add    esp, 4            ; ������ջ

      pop    ebp
      ret
\end{AsmCodeListing}
\caption{{\code f()}�����Ļ����� \label{fig:FAsm2}}
\end{figure}

\begin{figure}[tp]
\lstset{escapeinside=`',language=Pascal,%
}
\begin{lstlisting}[language=C++, frame=tlrb]{}
class A {
public:
  virtual void __cdecl m1() { cout << "A::m1()" << endl; }
  virtual void __cdecl m2() { cout << "A::m2()" << endl; }
  int ad;
};

class B : public A {    // `B�̳���A��m2()'
public:
  virtual void __cdecl m1() { cout << "B::m1()" << endl; }
  int bd;
};
/* `��ʾ�����Ķ����vtable' */
void print_vtable( A * pa ) {
  // `p��pa������һ��˫������'
  unsigned * p = reinterpret_cast<unsigned *>(pa);
  // `vt��vtable������һ��ָ������'
  void ** vt = reinterpret_cast<void **>(p[0]);
  cout << hex << "vtable address = " << vt << endl;
  for( int i=0; i < 2; i++ )
    cout << "dword " << i << ": " << vt[i] << endl;

  // `�ü��˵�û��Ȩ�޵ķ����������麯��!'
  void (*m1func_pointer)(A *);   // `����ָ�����'
  m1func_pointer = reinterpret_cast<void (*)(A*)>(vt[0]);
  m1func_pointer(pa);            // `ͨ������ָ����ó�Ա����m1'

  void (*m2func_pointer)(A *);   // `����ָ�����'
  m2func_pointer = reinterpret_cast<void (*)(A*)>(vt[1]);
  m2func_pointer(pa);            // `ͨ������ָ����ó�Ա����m2'
}

int main()
{
  A a;   B b1;  B b2;
  cout << "a: " << endl;   print_vtable(&a);
  cout << "b1: " << endl;  print_vtable(&b);
  cout << "b2: " << endl;  print_vtable(&b2);
  return 0;
}
\end{lstlisting}
\caption{ �����ӵ����� \label{fig:2mEx}}
\end{figure}


\begin{figure}[tp]
\centering
%\epsfig{file=vtable}
\input{vtable.latex}
\caption{{\code b1}���ڲ���ʾ\label{fig:vtable}}
\end{figure}

���ڣ���{\code f}�ĵڶ��ε��õ�����{\code B::m()}�ij�Ա��������Ϊ�������˶���{\code B}�����ǣ��Ⲣ����Ψһ���޸ĵĵط���{\code A}�Ĵ�С����Ϊ8(��{\code B}Ϊ12)��ͬ����{\code
ad}��ƫ��Ϊ4,����0����ƫ��0���ǵ�ʲô�أ��������Ĵ������ʵ�ֶ�̬��ء�

\index{C++!vtable|(} �����������Ա������C++����һ����������ص�������һ��ָ���Ա����ָ�������ָ�����\footnote{����û�����Ա�������࣬C++������ͨ��һ������ͬ�����ݳ�Ա�ı�׼C�ṹ��������������м��ݡ�}�������ͨ����Ϊ\emph{vtable}������
{\code A}��{\code B}�ָ࣬���������ƫ�Ƶ�ַ0����Windows���������ǰѴ�ָ����ŵ��̳�����������Ŀ�ʼ������ӵ�����Ա�����ij���汾(Դ��ͼ~\ref{fig:SimpInh})�е�{\code f}���������Ļ�����(ͼ~\ref{fig:FAsm2})�У�����Կ����Գ�Ա����{\code m}�ĵ��ò���ʹ��һ����š���9�������Ҷ����vtable�ĵ�ַ������ĵ�ַ�ڵ�11���б�ѹ���ջ����12��ͨ����֧��vtable��ĵ�һ����ַ�����������Ա������
\footnote{��Ȼ�����ֵ�Ѿ���{\code ECX}�Ĵ������ˡ������ڵ�8�з��õ��üĴ����ģ����ҿ����Ƴ���10�У��ٰ���һ�иı�Ϊpush {\code ECX}����Щ���벢��ʮ����Ч����Ϊ������û�п����Ż�����ѡ�������²����ġ�}����ε��ò���ʹ��һ����ţ�����֧��{\code EDX}ָ��Ĵ����ַ�����������͵ĵ�����һ��
\emph{����(late binding)}������\index{C++!����}�����󶨽������ĸ���Ա�������ж��ӳٵ���������ʱ�������������Ϊ�������ǡ���ij�Ա��������׼�İ���(ͼ~\ref{fig:FAsm1})Ӳ����ij����Ա�����ĵ��ã�Ҳ��Ϊ\emph{���(early binding)}\index{C++!���}
(��Ϊ�����Ա����������ˣ��ڱ����ʱ��)��

���ĵĶ��߽���������Ϊʲô��ͼ~\ref{fig:VirtInh}�е���ij�Ա����ͨ��ʹ��{\code \_\_cdecl}�ؼ�������ȷ����ʹ�õ���C����Լ����ȱʡ����£�Microsoft����C++���Ա����ʹ�õ��Dz�ͬ�ĵ���Լ���������DZ�׼C����Լ�����˵���Լ����ָ���Ա�����������õĶ����ָ�봫�ݵ�{\code ECX}�Ĵ�����������ʹ�ö�ջ����Ա������������ȷ�IJ�����Ȼʹ�ö�ջ���޸�Ϊ{\code \_\_cdecl}���߱�����ʹ�ñ�׼C����Լ����Borland C++ȱʡ�����ʹ�õ���C����Լ����

\begin{figure}[tp]
\fbox{ \parbox{\textwidth}{\code
a: \\
vtable address = 004120E8\\
dword 0: 00401320\\
dword 1: 00401350\\
A::m1()\\
A::m2()\\
b1:\\
vtable address = 004120F0\\
dword 0: 004013A0\\
dword 1: 00401350\\
B::m1()\\
A::m2()\\
b2:\\
vtable address = 004120F0\\
dword 0: 004013A0\\
dword 1: 00401350\\
B::m1()\\
A::m2()\\
} }
\caption{ͼ~\ref{fig:2mEx}�г������� \label{fig:2mExOut}}
\end{figure}


���������ٿ�һ����΢����һ������ӡ�
(ͼ~\ref{fig:2mEx})������������У���{\code A}��{\code B}����������Ա������{\code m1}��{\code m2}����ס��Ϊ��{\code B}��û�ж����Լ��ij�Ա����{\code m2}�����̳���{\code A}��ij�Ա������ͼ~\ref{fig:vtable}չʾ�˶���{\code b}���ڴ�����δ��档ͼ~\ref{fig:2mExOut}չʾ�˴˳������������ȣ�����ÿ�������vtable�ĵ�ַ������{\code B}�����vtable��ַ��һ���ģ�������ǹ���ͬ����vtable��һ��vtable����������Զ�����һ������(����һ��{\code static}���ݳ�Ա)����Σ�������vtable��ĵ�ַ���ӻ����������У������ȷ����Ա����{\code m1}ָ����ƫ�Ƶ�ַ~0��
(��˫��~0)��{\code m2}��ƫ�Ƶ�ַ~4��(˫��~1)��{\code m2}��Ա����ָ������{\code A}��{\code B}��vtable����һ���ģ���Ϊ��{\code B}����{\code A}�̳��˳�Ա����{\code m2}��

��25�е�32��չʾ�������ͨ���Ӷ����vtable����ַ�ķ���������һ���麯��\footnote{��ס��Щ����ֻ����MS��Borland�����������У�\emph{gcc}���С�}����Ա������ַͨ��һ�������\emph{this}ָ�봢�浽��һ��C���ͺ���ָ�����ˡ���ͼ~\ref{fig:2mExOut}������У�����Կ�����ȷʵ�������С����ǣ���\emph{��Ҫ}������д���룡��ֻ����������˵�����Ա�������ʹ��vtable��

%Looking at the output of Figure~\ref{fig:2mExOut} does demonstrate several
%features of the implementation of polymorphism.  The {\code b1} and {\code b2}
%variables have the same vtable address; however the {\code a} variable
%has a different vtable address. The vtable is a property of the class not
%a variable of the class. All class variables share a common vtable. The two
%{\code dword} values in the table are the pointers to the virtual methods.
%The first one (number 0) is for {\code m1}. Note that it is different for the
%{\code A} and {\code B} classes. This makes sense since the A and B classes
%have different {\code m1} methods. However, the second method pointer is
%the same for both classes, since class {\code B} inherits the {\code m2}
%method from its base class, {\code A}.

���������ǿ���ѧ��һЩʵ���Ľ�ѵ��һ����Ҫ����ʵ�ǵ������д�������һ��������Դ�ļ���ʱ�������dz�С�ġ��㲻���������������н���ʹ��һ�������ƶ���д����Ϊ���ܻ����дԴ�ļ�֮���vtableָ�룡����һ��ָ�����ڳ����ڴ��е�vtable��ָ�룬���Ҳ�ͬ�ij��򽫲�ͬ�� ͬ��������ᷢ����C���ԵĽṹ�У�������C�����У��ṹ��ֻ�е�����Ա��ȷ��ָ��ŵ��ṹ����ʱ���ṹ���ڲ�����ָ�롣��{\code A}����{\code B}�У���û�����Եض����ָ�롣


�ٴΣ���ʶ����ͬ�ı�����ʵ�����Ա�����ķ����Dz�һ�����Ƿdz���Ҫ�ġ���In Windows�У�COM(�������ģ�ͣ�Component Object Model)
\index{COM}�����ʹ��vtable��ʵ��COM�ӿ�\footnote{COM��ͬ��ʹ��{\code \_\_stdcall}
\index{����Լ��!stdcall}����Լ���������DZ�׼C����Լ����}��ֻ����Microsoftһ������ʵ�����Ա�����ı������ſ��Դ���COM�ࡣ��Ҳ��ΪʲôBorland���ú�Microsoftһ����ʵ�ַ�����ԭ��Ҳ��Ϊʲô��������\emph{gcc}������COM���ԭ��֮һ��

���Ա�����Ĵ���ͷdz���ij�Ա�����Ĵ���dz�����ֻ�ǵ������ǵĴ����Dz�ͬ�ġ����������ܾ��Ա�֤�����ĸ����Ա��������ô�����Ժ���vtable��ֱ�ӵ��ó�Ա������(\emph{����}��ʹ�����)��
\index{C++!vtable|)}
\index{C++!��̬|)}
\index{C++!�̳�|)}
\index{C++!��|)}
\index{C++|)}

\subsection{C++����������}

C++�������ԵĹ�����ʽ(\emph{����}�����˴����̳кͶ�̳У���������ʱ����ʶ��)�����ڱ���ķ�Χ���������ϣ���ߵø�ԶһЩ��һ���õ������Ellis
��Stroustrup�\emph{The Annotated C++ Reference
Manual}��Stroustrup�\emph{The Design and Evolution of C++}��

%%-*- latex -*-
\chapter{Dynamic Link Libraries}

\section{Using the Window's API and Dynamic Link Libraries}

UNIX systems provide a simple C based Application Programming
Interface (API).  In contrast, Microsoft Windows\texttrademark
\hspace{0.5em} packages its API in Dynamic Link Libraries that load in
each executables address space before its use. From the C/C++
programmers perspective, it appears as a normal C function call;
however, at the assembly level, it is different.

\subsection{Standard Call Calling Convention}
The Windows API uses the \emph{Standard Call}\index{calling
convention!standard call} calling convention. As stated eariler, this
convention pushes the arguments in reverse order just as the standard
C calling convention. However, there are two important
differences. First, the subroutine is responsible for clearing the
parameters from the stack. Secondly, the label of the function is
generated differently. An underscore is prepended the name as before,
but in addition the \emph{@} character is added to the end of the
function name along a number equal to the number of bytes on the stack
for the parameters of the function (in decimal).

\index{CloseHandle|(}
For example, consider the {\code CloseHandle} Windows API function. It's prototype
looks like:
\begin{lstlisting}[stepnumber=0]{}
BOOL WINAPI CloseHandle( HANDLE hObject );
\end{lstlisting}
Since a {\code HANDLE} is a double word in 32-bit Windows, the label
for this function would be {\code \_CloseHandle@4}. The {\code WINAPI}\index{WINAPI}
in the prototype is a C macro defined to be {\code \_\_stdcall}. Below
is a sample call to the function assuming that the {\code hObject}
value is in {\code EBX}.
\begin{AsmCodeListing}[frame=single]
  push  ebx             ; Push hObject on stack
  call  _CloseHandle@4
  mov   esi, eax        ; Save return value in ESI
\end{AsmCodeListing}
The stack does not have to adjusted after the function call, {\code CloseHandle}
fixes the stack automatically.
\index{CloseHandle|)}

\subsection{Static and Shared Libraries}

\index{static library|(}
A \emph{static library} is a collection of object files that can be
linked to an executable when it is created. The object code is
inserted directly into the executable just like an ordinary object
file. The library file is just a convenience. It allows a single file
to be included in the link step of the build process. All the object
code is probably not stored in the final executable. The linker will
look at which object modules are needed and only include the required
ones. Static libraries are created using the {\code LIB} program under
Windows or the {\code ar} program under UNIX. Windows libraries end in
a {\code .lib} extension and UNIX libraries end in a {\code .a}
extension.
\index{static library|)}

\index{shared library|(}
A shared library (or DLL in Windows) as its name implies shares code
among executables. When the executable is run, the OS finds all the 
shared libraries it requires and loads them into the processes memory
so the executable can use the code in them. Using the virtual memory
mapping capabilities on a protected mode operating system, shared
library code used by two or more concurrently running processes is
only loaded into physical memory once. 

There are advantages and disadvantages to shared libraries. The first
advantage is that executable sizes can be greatly reduced. If a large
library is used by many executables. Only one copy of the code (in the
shared library) is on the system (and in physical memory). If the
library was included statically, each executable would include a copy
of the code.

Another advantage is that if a bug is found in the shared library,
then the library can be replaced with one with the bug fixed. The
executables will automatically use the new fixed code without
recompiling the executable. However, this only works if the interface
of the functions in the shared library remain unchanged.

Shared libraries are also used to allow code written in different
languages to interoperate. For example, C++ can be interfaced to
Visual Basic\index{Visual Basic}, DotNet\index{DotNet} and
Java\index{Java} using shared libraries under Windows.

The disadvantages of shared libraries are that they can be more complicated
to maintain. One of the most common problems is that a new version of the
library breaks older code because it behaves slightly differently. Then
some executables need one version and others need a different one. In Windows,
this condition is known as \emph{DLL hell}\index{DLL hell}.
\index{shared library|)}

\index{DLL|(}
\subsection{Windows DLLs}

A Windows DLL is constructed much like an executable program. Object files are linked
together to create a DLL file. Unlike an executable, a DLL can have many entry points.
An entry point is just a function that can be called externally to the DLL. Only
functions that have been \emph{exported}\index{export} can be called externally.

A function (or global variable) may be exported by either entering its
name in the definition file\index{DLL!definition file} for the DLL or
by using Microsoft specific keywords.

\begin{figure}[t]
\begin{Verbatim}[frame=single,commandchars=\\\{\}]
LIBRARY \textit{library root name}
DESCRIPTION '\textit{short text description}'
EXPORTS
\textit{list of exported functions (one per line)}
\end{Verbatim}
\caption{Windows DLL definition file\label{fig:DefFile}}
\end{figure}

\subsubsection{Definition file}
This is a text file with a {\code .def} extension that lists all the
functions that the DLL exports. This file is used during the link step
of the DLL creation process. Figure~\ref{fig:DefFile} shows the
general format of a definition file.


\index{DLL|)}

\begin{appendix}
%appendix
\chapter{80x86 Instructions}
\section{Non-floating Point Instructions}
This section lists and describes the actions and formats of the 
non-floating point instructions of the Intel 80x86 CPU family.

The formats use the following abbreviations:
\begin{center}
\begin{tabular}{|l|l|}
\hline
R   & general register \\
R8  & 8-bit register \\
R16 & 16-bit register \\
R32 & 32-bit register \\
SR  & segment register \\
M   & memory \\
M8  & byte \\
M16 & word \\
M32 & double word \\
I   & immediate value \\
\hline
\end{tabular}
\end{center}
These can be combined for the multiple operand instructions. For example,
the format \emph{R, R} means that the instruction takes two register operands.
Many of the two operand instructions allow the same operands. The abbreviation
\emph{O2} is used to represent these operands: \emph{R,R R,M R,I M,R M,I}. If
a 8-bit register or memory can be used for an operand, the abbreviation,
\emph{R/M8} is used.

The table also shows how various bits of the FLAGS register are affected by
each instruction. If the column is blank, the corresponding bit is not
affected at all. If the bit is always changed to a particular value, a 1 or
0 is shown in the column. If the bit is changed to a value that depends on
the operands of the instruction, a \emph{C} is placed in the column. Finally,
if the bit is modified in some undefined way a \emph{?} appears in the
column. Because the only instructions that change the direction flag are 
{\code CLD} and {\code STD}, it is not listed under the FLAGS columns.

\begin{longtable}{||l|p{1.5in}|p{0.75in}|c|c|c|c|c|c||}
\hline \hline
\multicolumn{1}{||c}{} & 
   \multicolumn{1}{c}{} &
   \multicolumn{1}{c}{} &
  \multicolumn{6}{c||}{\textbf{Flags}} \\ \cline{4-9}
\multicolumn{1}{||c}{\textbf{Name}} & 
   \multicolumn{1}{c}{\textbf{Description}} &
   \multicolumn{1}{c}{\textbf{Formats}} &
   \multicolumn{1}{c}{\textbf{O}} &
   \multicolumn{1}{c}{\textbf{S}} &
   \multicolumn{1}{c}{\textbf{Z}} &
   \multicolumn{1}{c}{\textbf{A}} &
   \multicolumn{1}{c}{\textbf{P}} &
   \multicolumn{1}{c||}{\textbf{C}} \\ \hline \endhead
\hline \hline \endfoot
%                                              O   S   Z   A   P   C
{\code ADC} & Add with Carry & O2            & C & C & C & C & C & C \\
{\code ADD} & Add Integers   & O2            & C & C & C & C & C & C \\
{\code AND} & Bitwise AND    & O2            & 0 & C & C & ? & C & 0 \\
{\code BSWAP} & Byte Swap    & R32           &   &   &   &   &   &  \\
{\code CALL} & Call Routine  & R M I         &   &   &   &   &   &   \\
{\code CBW} & Convert Byte to Word &         &   &   &   &   &   & \\
{\code CDQ} & Convert Dword to Qword &       &   &   &   &   &   & \\
{\code CLC} & Clear Carry &                  &   &   &   &   &   & 0 \\
{\code CLD} & Clear Direction Flag &         &   &   &   &   &   & \\
{\code CMC} & Complement Carry &             &   &   &   &   &   & C \\
{\code CMP} & Compare Integers & O2          & C & C & C & C & C & C \\
{\code CMPSB} & Compare Bytes &              & C & C & C & C & C & C \\
{\code CMPSW} & Compare Words &              & C & C & C & C & C & C \\
{\code CMPSD} & Compare Dwords &             & C & C & C & C & C & C \\
{\code CWD} & Convert Word to Dword into DX:AX & &   &   &   &   &   & \\
{\code CWDE} & Convert Word to Dword into EAX & &   &   &   &   &   & \\
{\code DEC} & Decrement Integer & R M        & C & C & C & C & C & \\
{\code DIV} & Unsigned Divide & R M          & ? & ? & ? & ? & ? & ? \\
{\code ENTER} & Make stack frame & I,0       &   &   &   &   &   & \\
{\code IDIV} & Signed Divide & R M           & ? & ? & ? & ? & ? & ? \\
{\code IMUL} & Signed Multiply & R M R16,R/M16 R32,R/M32 R16,I R32,I 
                                       {\small R16,R/M16,I R32,R/M32,I}
                                             & C & ? & ? & ? & ? & C \\
{\code INC} & Increment Integer & R M        & C & C & C & C & C & \\
{\code INT} & Generate Interrupt & I         &   &   &   &   &   & \\
{\code JA } & Jump Above & I                 &   &   &   &   &   & \\
{\code JAE } & Jump Above or Equal & I       &   &   &   &   &   & \\
{\code JB } & Jump Below & I                 &   &   &   &   &   & \\
{\code JBE } & Jump Below or Equal  & I      &   &   &   &   &   & \\
{\code JC } & Jump Carry & I                 &   &   &   &   &   & \\
{\code JCXZ } & Jump if CX = 0 & I           &   &   &   &   &   & \\
{\code JE } & Jump Equal & I                 &   &   &   &   &   & \\
{\code JG } & Jump Greater & I               &   &   &   &   &   & \\
{\code JGE } & Jump Greater or Equal & I     &   &   &   &   &   & \\
{\code JL } & Jump Less & I                  &   &   &   &   &   & \\
{\code JLE } & Jump Less or Equal & I        &   &   &   &   &   & \\
{\code JMP } & Unconditional Jump & R M I    &   &   &   &   &   & \\
{\code JNA } & Jump Not Above & I            &   &   &   &   &   & \\
{\code JNAE } & Jump Not Above or Equal& I   &   &   &   &   &   & \\
{\code JNB } & Jump Not Below & I            &   &   &   &   &   & \\
{\code JNBE } & Jump Not Below or Equal & I  &   &   &   &   &   & \\
{\code JNC } & Jump No Carry & I             &   &   &   &   &   & \\
{\code JNE } & Jump Not Equal & I            &   &   &   &   &   & \\
{\code JNG } & Jump Not Greater & I          &   &   &   &   &   & \\
{\code JNGE } & Jump Not Greater or Equal & I&   &   &   &   &   & \\
{\code JNL } & Jump Not Less & I             &   &   &   &   &   & \\
{\code JNLE } & Jump Not Less or Equal & I   &   &   &   &   &   & \\
{\code JNO } & Jump No Overflow & I          &   &   &   &   &   & \\
{\code JNS } & Jump No Sign & I              &   &   &   &   &   & \\
{\code JNZ } & Jump Not Zero & I             &   &   &   &   &   & \\
{\code JO } & Jump Overflow & I              &   &   &   &   &   & \\
{\code JPE } & Jump Parity Even & I          &   &   &   &   &   & \\
{\code JPO } & Jump Parity Odd & I           &   &   &   &   &   & \\
{\code JS } & Jump Sign & I                  &   &   &   &   &   & \\
{\code JZ } & Jump Zero & I                  &   &   &   &   &   & \\
{\code LAHF} & Load FLAGS into AH &          &   &   &   &   &   & \\
{\code LEA} & Load Effective Address & R32,M &   &   &   &   &   & \\
{\code LEAVE} & Leave Stack Frame &          &   &   &   &   &   & \\
{\code LODSB} & Load Byte &                  &   &   &   &   &   & \\
{\code LODSW} & Load Word &                  &   &   &   &   &   & \\
{\code LODSD} & Load Dword &                 &   &   &   &   &   & \\
{\code LOOP}  & Loop       & I               &   &   &   &   &   & \\
{\code LOOPE/LOOPZ} & Loop If Equal & I     &   &   &   &   &   & \\
{\code LOOPNE/LOOPNZ} & Loop If Not Equal & I  &   &   &   &   &   & \\
{\code MOV} & Move Data & O2 \mbox{SR,R/M16} R/M16,SR
                                             &   &   &   &   &   & \\
{\code MOVSB} & Move Byte &                  &   &   &   &   &   & \\
{\code MOVSW} & Move Word &                  &   &   &   &   &   & \\
{\code MOVSD} & Move Dword &                 &   &   &   &   &   & \\
{\code MOVSX} & Move Signed & R16,R/M8 R32,R/M8 R32,R/M16
                                             &   &   &   &   &   & \\
{\code MOVZX} & Move Unsigned & R16,R/M8 R32,R/M8 R32,R/M16
                                             &   &   &   &   &   & \\
{\code MUL} & Unsigned Multiply & R M        & C & ? & ? & ? & ? & C \\
{\code NEG} & Negate & R M                   & C & C & C & C & C & C \\
{\code NOP} & No Operation &                 &   &   &   &   &   & \\
{\code NOT} & 1's Complement & R M           &   &   &   &   &   & \\
{\code OR} & Bitwise OR    & O2              & 0 & C & C & ? & C & 0 \\
{\code POP} & Pop From Stack & R/M16 R/M32   &   &   &   &   &   & \\
{\code POPA} & Pop All &                     &   &   &   &   &   & \\
{\code POPF} & Pop FLAGS &                   & C & C & C & C & C & C \\
{\code PUSH} & Push to Stack & R/M16 R/M32 I &   &   &   &   &   & \\
{\code PUSHA} & Push All &                   &   &   &   &   &   & \\
{\code PUSHF} & Push FLAGS &                 &   &   &   &   &   & \\
{\code RCL} & Rotate Left with Carry & R/M,I R/M,CL
                                             & C &   &   &   &   & C \\
{\code RCR} & Rotate Right with Carry & R/M,I R/M,CL
                                             & C &   &   &   &   & C \\
{\code REP} & Repeat &                       &   &   &   &   &   & \\
{\code REPE/REPZ} & Repeat If Equal&        &   &   &   &   &   & \\
{\code REPNE/REPNZ} & Repeat If Not Equal&  &   &   &   &   &   & \\
{\code RET} & Return &                       &   &   &   &   &   & \\
{\code ROL} & Rotate Left & R/M,I R/M,CL     & C &   &   &   &   & C \\
{\code ROR} & Rotate Right & R/M,I R/M,CL    & C &   &   &   &   & C \\
{\code SAHF} & Copies AH into FLAGS &        &   & C & C & C & C & C \\
{\code SAL} & Shifts to Left & R/M,I R/M, CL &   &   &   &   &   & C \\
{\code SBB}  & Subtract with Borrow & O2     & C & C & C & C & C & C \\
{\code SCASB} & Scan for Byte &              & C & C & C & C & C & C \\
{\code SCASW} & Scan for Word &              & C & C & C & C & C & C \\
{\code SCASD} & Scan for Dword &             & C & C & C & C & C & C \\
{\code SETA } & Set Above & R/M8                 &   &   &   &   &   & \\
{\code SETAE } & Set Above or Equal & R/M8       &   &   &   &   &   & \\
{\code SETB } & Set Below & R/M8                 &   &   &   &   &   & \\
{\code SETBE } & Set Below or Equal  & R/M8      &   &   &   &   &   & \\
{\code SETC } & Set Carry & R/M8                 &   &   &   &   &   & \\
{\code SETE } & Set Equal & R/M8                 &   &   &   &   &   & \\
{\code SETG } & Set Greater & R/M8               &   &   &   &   &   & \\
{\code SETGE } & Set Greater or Equal & R/M8     &   &   &   &   &   & \\
{\code SETL } & Set Less & R/M8                  &   &   &   &   &   & \\
{\code SETLE } & Set Less or Equal & R/M8        &   &   &   &   &   & \\
{\code SETNA } & Set Not Above & R/M8            &   &   &   &   &   & \\
{\code SETNAE } & Set Not Above or Equal& R/M8   &   &   &   &   &   & \\
{\code SETNB } & Set Not Below & R/M8            &   &   &   &   &   & \\
{\code SETNBE } & Set Not Below or Equal & R/M8  &   &   &   &   &   & \\
{\code SETNC } & Set No Carry & R/M8             &   &   &   &   &   & \\
{\code SETNE } & Set Not Equal & R/M8            &   &   &   &   &   & \\
{\code SETNG } & Set Not Greater & R/M8          &   &   &   &   &   & \\
{\code SETNGE } & Set Not Greater or Equal & R/M8&   &   &   &   &   & \\
{\code SETNL } & Set Not Less & R/M8             &   &   &   &   &   & \\
{\code SETNLE } & Set Not LEss or Equal & R/M8   &   &   &   &   &   & \\
{\code SETNO } & Set No Overflow & R/M8          &   &   &   &   &   & \\
{\code SETNS } & Set No Sign & R/M8              &   &   &   &   &   & \\
{\code SETNZ } & Set Not Zero & R/M8             &   &   &   &   &   & \\
{\code SETO } & Set Overflow & R/M8              &   &   &   &   &   & \\
{\code SETPE } & Set Parity Even & R/M8          &   &   &   &   &   & \\
{\code SETPO } & Set Parity Odd & R/M8           &   &   &   &   &   & \\
{\code SETS } & Set Sign & R/M8                  &   &   &   &   &   & \\
{\code SETZ } & Set Zero & R/M8                  &   &   &   &   &   & \\

{\code SAR} & Arithmetic Shift to Right & R/M,I R/M, CL 
                                             &   &   &   &   &   & C \\
{\code SHR} & Logical Shift to Right & R/M,I R/M, CL 
                                             &   &   &   &   &   & C \\
{\code SHL} & Logical Shift to Left & R/M,I R/M, CL 
                                             &   &   &   &   &   & C \\
{\code STC} & Set Carry &                    &   &   &   &   &   & 1 \\
{\code STD} & Set Direction Flag &           &   &   &   &   &   & \\
{\code STOSB} & Store Btye &                 &   &   &   &   &   & \\
{\code STOSW} & Store Word &                 &   &   &   &   &   & \\
{\code STOSD} & Store Dword &                &   &   &   &   &   & \\
{\code SUB} & Subtract & O2                  & C & C & C & C & C & C\\
{\code TEST} & Logical Compare & R/M,R R/M,I & 0 & C & C & ? & C & 0\\
{\code XCHG} & Exchange & R/M,R R,R/M        &   &   &   &   &   & \\
{\code XOR} & Bitwise XOR    & O2            & 0 & C & C & ? & C & 0 \\

\end{longtable}

\newpage
\section{Floating Point Instructions}

\renewcommand{\thefootnote}{\fnsymbol{footnote}} In this section, many
of the 80x86 math coprocessor instructions are described. The
description section briefly describes the operation of the
instruction. To save space, information about whether the instruction
pops the stack is not given in the description. 

The format column shows what type of operands can be used with each
instruction. The following abbreviations are used:
\begin{center}
\begin{tabular}{|l|l|}
\hline
ST\emph{n} & A coprocessor register \\
F          & Single precision number in memory \\
D          & Double precision number in memory \\
E          & Extended precision number in memory \\
I16        & Integer word in memory \\
I32        & Integer double word in memory \\
I64        & Integer quad word in memory \\
\hline
\end{tabular}
\end{center}

Instructions requiring a Pentium Pro or better are marked with an 
asterisk(\footnotemark[1]).

\begin{longtable}{||l|l|l||}
\hline \hline
\multicolumn{1}{||c}{\textbf{Instruction}} & 
  \multicolumn{1}{c}{\textbf{Description}} &
\multicolumn{1}{c||}{\textbf{Format}} \\
\hline
\endhead
\hline \hline \endfoot
{\code FABS} & $\mathtt{ST0} = |\mathtt{ST0}|$ & \\
{\code FADD \emph{src}} & {\code ST0 += \emph{src}} & ST\emph{n} F D \\
{\code FADD \emph{dest}, ST0} & {\code \emph{dest} += STO} & ST\emph{n} \\
{\code FADDP \emph{dest}[,ST0]} & {\code \emph{dest} += ST0} & ST\emph{n} \\
{\code FCHS} & $\mathtt{ST0} = - \mathtt{ST0}$ & \\
{\code FCOM \emph{src}} & Compare {\code ST0} and {\code \emph{src}} &
ST\emph{n} F D \\
{\code FCOMP \emph{src}} & Compare {\code ST0} and {\code \emph{src}} &
ST\emph{n} F D \\
{\code FCOMPP \emph{src}} & Compares {\code ST0} and {\code ST1} & \\
{\code FCOMI\footnotemark[1] \emph{src}} & Compares into FLAGS 
& ST\emph{n} \\
{\code FCOMIP\footnotemark[1] \emph{src}} & Compares into FLAGS 
& ST\emph{n} \\
{\code FDIV \emph{src}} & {\code ST0 /= \emph{src}} & ST\emph{n} F D \\
{\code FDIV \emph{dest}, ST0} & {\code \emph{dest} /= STO} & ST\emph{n} \\
{\code FDIVP \emph{dest}[,ST0]} & {\code \emph{dest} /= ST0} & ST\emph{n} \\
{\code FDIVR \emph{src}} & {\code ST0 = \emph{src}/ST0} & ST\emph{n} F D \\
{\code FDIVR \emph{dest}, ST0} & {\code \emph{dest} = ST0/\emph{dest}} 
& ST\emph{n} \\
{\code FDIVRP \emph{dest}[,ST0]} & {\code \emph{dest} = ST0/\emph{dest}} 
& ST\emph{n} \\
{\code FFREE \emph{dest}} & Marks as empty & ST\emph{n} \\
{\code FIADD \emph{src}} & {\code ST0 += \emph{src}} & I16 I32 \\
{\code FICOM \emph{src}} & Compare {\code ST0} and {\code \emph{src}} &
I16 I32 \\
{\code FICOMP \emph{src}} & Compare {\code ST0} and {\code \emph{src}} &
I16 I32 \\
{\code FIDIV \emph{src}} & {\code STO /= \emph{src}} & I16 I32 \\
{\code FIDIVR \emph{src}} & {\code STO = \emph{src}/ST0} & I16 I32 \\
{\code FILD \emph{src}} & Push \emph{src} on Stack & I16 I32 I64 \\
{\code FIMUL \emph{src}} & {\code ST0 *= \emph{src}} & I16 I32 \\
{\code FINIT} & Initialize Coprocessor & \\
{\code FIST \emph{dest}} & Store {\code ST0} & I16 I32 \\
{\code FISTP \emph{dest}} & Store {\code ST0} & I16 I32 I64\\
{\code FISUB \emph{src}} & {\code ST0 -= \emph{src}} & I16 I32 \\
{\code FISUBR \emph{src}} & {\code ST0 = \emph{src} - ST0} & I16 I32 \\
{\code FLD \emph{src}} & Push \emph{src} on Stack & ST\emph{n} F D E \\
{\code FLD1} & Push 1.0 on Stack & \\
{\code FLDCW \emph{src}} & Load Control Word Register & I16 \\
{\code FLDPI} & Push $\pi$ on Stack & \\
{\code FLDZ} & Push 0.0 on Stack & \\
{\code FMUL \emph{src}} & {\code ST0 *= \emph{src}} & ST\emph{n} F D \\
{\code FMUL \emph{dest}, ST0} & {\code \emph{dest} *= STO} & ST\emph{n} \\
{\code FMULP \emph{dest}[,ST0]} & {\code \emph{dest} *= ST0} & ST\emph{n} \\
{\code FRNDINT} & Round {\code ST0} & \\
{\code FSCALE} & $\mathtt{ST0} = \mathtt{ST0} \times 2^{\lfloor \mathtt{ST1} \rfloor}$ & \\
{\code FSQRT} & $\mathtt{ST0} = \sqrt{\mathtt{STO}}$ & \\
{\code FST \emph{dest}} & Store {\code ST0} & ST\emph{n} F D \\
{\code FSTP \emph{dest}} & Store {\code ST0} & ST\emph{n} F D E \\
{\code FSTCW \emph{dest}} & Store Control Word Register & I16 \\
{\code FSTSW \emph{dest}} & Store Status Word Register & I16 AX \\
{\code FSUB \emph{src}} & {\code ST0 -= \emph{src}} & ST\emph{n} F D \\
{\code FSUB \emph{dest}, ST0} & {\code \emph{dest} -= STO} & ST\emph{n} \\
{\code FSUBP \emph{dest}[,ST0]} & {\code \emph{dest} -= ST0} & ST\emph{n} \\
{\code FSUBR \emph{src}} & {\code ST0 = \emph{src}-ST0} & ST\emph{n} F D \\
{\code FSUBR \emph{dest}, ST0} & {\code \emph{dest} = ST0-\emph{dest}} 
& ST\emph{n} \\
{\code FSUBP \emph{dest}[,ST0]} & {\code \emph{dest} = ST0-\emph{dest}} 
& ST\emph{n} \\
{\code FTST} & Compare {\code ST0} with 0.0 & \\
{\code FXCH \emph{dest}} & Exchange {\code ST0} and {\code \emph{dest}} 
& ST\emph{n} \\
\end{longtable}

\renewcommand{\thefootnote}{\arabic{footnote}}



\end{appendix}
\clearpage
\ifmypdf
\phantomsection % fixes link anchor
\fi
\addcontentsline{toc}{chapter}{Index}
\printindex
\end{document}

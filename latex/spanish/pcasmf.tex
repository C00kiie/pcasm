% -*-LaTex-*-
%front matter of pcasm book

\chapter{Prefacio}

\section*{Prop�sito}

El prop�sito de este libro es dar al lector un mejor entendimiento de
c�mo trabajan realmente los computadores a un nivel m�s bajo que los
lenguajes de alto nivel como Pascal. Teniendo un conocimiento profundo
de c�mo trabajan los computadores, el lector puede ser m�s productivo
desarrollando software en lenguajes de alto nivel tales como C y C++.
Aprender a programar en lenguaje ensamblador es una manera excelente
de lograr este objetivo. Otros libros de lenguaje ensamblador a�n
ense�an a programar el procesador 8086 que us� el PC original en 1981.
El procesador 8086 s�lo soporta el modo \emph{real}. En este modo, 
cualquier programa puede acceder a cualquier direcci�n de memoria
o dispositivo en el computador. Este modo no es apropiado para un sistema
operativo multitarea seguro. Este libro, en su lugar discute c�mo
programar los procesadores 80386 y posteriores en modo \emph{protegido}
(el modo en que corren Windows y Linux). Este modo soporta las
caracter�sticas que los sistemas operativos modernos esperan, como
memoria virtual y protecci�n de memoria.
Hay varias razones para usar el modo protegido
\begin{enumerate}
\item Es m�s f�cil de programar en modo protegido que en el modo real del
8086 que usan los otros libros.
\item Todos los sistemas operativos de PC se ejecutan en modo protegido.
\item Hay disponible software libre que se ejecuta en este modos.
\end{enumerate}
La carencia de libros de texto para la programaci�n en ensamblador de PC
para modo protegido es la principal raz�n por la cual el autor escribi�
este libro.

C�mo lo dicho antes, este libro hace uso de Software Libre: es decir el
ensamblador NASM y el compilador de C/C++ DJGPP. Ambos se pueden
descargar de Internet. El texto tambi�n discute c�mo usar el c�digo del 
ensamblador NASM bajo el sistema operativo Linux y con los compiladores
de C/C++ de Borland y Microsoft bajo Windows. Todos los ejemplos de estas
plataformas se pueden encontrar en mi sitio web:
{\url{http://www.drpaulcarter.com/pcasm}}.
Debe descargar el c�digo de los ejemplos, si desea ensamblar y correr los
muchos ejemplos de este tutorial.

Tenga en cuenta que este libro no intenta cubrir cada aspecto de la
programaci�n en ensamblador. El autor ha intentado cubrir los t�picos m�s
importantes que \emph{todos} los programadores deber�an tener

\section*{Reconocimientos}

El autor quiere agradecer a los muchos programadores alrededor del mundo
que han contribuido al movimiento de Software Libre. Todos los programe y
a�n este libro en s� mismo fueron producidos usando software libre. 
El autor desear�a agradecerle especialmente a John~S.~Fine,
Simon~Tatham, Julian~Hall 
y otros por desarrollar el ensamblador NASM ya que todos los ejemplos de
este libro est�n basados en �l; a DJ Delorie por desarrollar el
compilador usado de C/C++ DJGPP; la numerosa gente que ha contribuido al
compilador GNU gcc en el cual est� basado DJGPP; a Donald Knuth y otros
por desarrollar los lenguajes de composici�n de textos \TeX\ y \LaTeXe\
que fueron usados para producir este libro; a Richar Stallman (fundador
de la Free Software Fundation), Linus Torvalds 
(creador del n�cleo de Linux) y a otros que han desarrollado el software
que el autor ha usado para producir este trabajo.

Gracias a las siguientes personas por correcciones:
\begin{itemize}
\item John S. Fine
\item Marcelo Henrique Pinto de Almeida
\item Sam Hopkins
\item Nick D'Imperio
\item Jeremiah Lawrence
\item Ed Beroset
\item Jerry Gembarowski
\item Ziqiang Peng
\item Eno Compton
\item Josh I Cates
\item Mik Mifflin
\item Luke Wallis
\item Gaku Ueda
\item Brian Heward
\item Chad Gorshing
\item F. Gotti
\item Bob Wilkinson
\item Markus Koegel
\item Louis Taber
\item Dave Kiddell
\item Eduardo Horowitz
\item S\'{e}bastien Le Ray
\item Nehal Mistry
\item Jianyue Wang
\item Jeremias Kleer
\item Marc Janicki
\end{itemize}


\section*{Recursos en Internet}
\begin{center}
\begin{tabular}{|ll|}
\hline
P�gina del autor & \url{http://www.drpaulcarter.com/} \\
%NASM   & {\code http://nasm.2y.net/} \\
P�gina de NASM en SourceForge & \url{http://nasm.sourceforge.net/} \\
DJGPP  & \url{http://www.delorie.com/djgpp} \\
Ensamblador con Linux & \url{http://www.linuxassembly.org/} \\
The Art of Assembly & \url{http://webster.cs.ucr.edu/} \\
USENET & {\code comp.lang.asm.x86} \\
Documentaci�n de Intel & \url{http://www.intel.com/design/Pentium4/documentation.htm} \\
\hline
\end{tabular}
\end{center}


\section*{Comentarios}

El autor agradece cualquier comentario sobre este trabajo.
\begin{center}
\begin{tabular}{ll}
\textbf{E-mail:} & {\code pacman128@gmail.com} \\
\textbf{WWW:}    & \url{http://www.drpaulcarter.com/pcasm} \\
\end{tabular}
\end{center}





\chapter{Basic Assembly Language}

\section{Working with Integers \index{integer|(}}

\subsection{Integer representation \index{integer!representation|(}}

\index{integer!unsigned|(}
Integers come in two flavors: unsigned and signed. Unsigned integers (which
are non-negative) are represented in a very straightforward binary manner.
The number 200 as an one byte unsigned integer would be represented as by
11001000 (or C8 in hex).
\index{integer!unsigned|)}

\index{integer!signed|(}
Signed integers (which may be positive or negative) are represented in a more
complicated ways. For example, consider $-56$. $+56$ as a byte would be
represented by 00111000. On paper, one could represent $-56$ as $-111000$,
but how would this be represented in a byte in the computer's memory.
How would the minus sign be stored?

There are three general techniques that have been used to represent
signed integers in computer memory. All of these methods use the most
significant bit of the integer as a \emph{sign
bit}. \index{integer!sign bit} This bit is 0 if the number is positive
and 1 if negative.

\subsubsection{Signed magnitude \index{integer!representation!signed magnitude}}

The first method is the simplest and is called \emph{signed magnitude}. It
represents the integer as two parts. The first part is the sign bit and
the second is the magnitude of the integer. So 56 would be represented as
the byte $\underline{0}0111000$ (the sign bit is underlined) and $-56$
would be $\underline{1}0111000$. The largest byte value would be
$\underline{0}1111111$ or $+127$ and the smallest byte value would be
$\underline{1}1111111$ or $-127$. To negate a value, the sign bit is reversed.
This method is straightforward, but it does have its drawbacks. First,
there are two possible values of zero, $+0$ ($\underline{0}0000000$) and
$-0$ ($\underline{1}0000000$). Since zero is neither positive nor negative,
both of these representations should act the same. This complicates the logic
of arithmetic for the CPU. Secondly, general arithmetic is also complicated.
If 10 is added to $-56$, this must be recast as 10 subtracted by 56. Again,
this complicates the logic of the CPU.

\subsubsection{One's complement \index{integer!representation!one's complement}}
The second method is known as \emph{one's complement} representation. The
one's complement of a number is found by reversing each bit in the number.
(Another way to look at it is that the new bit value is $1 - \mathrm{old bit value}$.) 
For example, the one's complement of 
$\underline{0}0111000$ ($+56$) is $\underline{1}1000111$. In one's complement
notation, computing the one's complement is equivalent to negation. Thus,
$\underline{1}1000111$ is the representation for $-56$. Note that the sign
bit was automatically changed by one's complement and that as one would
expect taking the one's complement twice yields the original number. As for
the first method, there are two representations of zero: 
$\underline{0}0000000$ ($+0$) and $\underline{1}1111111$ ($-0$). Arithmetic
with one's complement numbers is complicated.

There is a handy trick to finding the one's complement of a number in
hexadecimal without converting it to binary. The trick is to subtract the
hex digit from F (or 15 in decimal). This method assumes that the number of
bits in the number is a multiple of 4. Here is an example: $+56$ is
represented by 38 in hex. To find the one's complement, subtract each
digit from F to get C7 in hex. This agrees with the result above.

\subsubsection{Two's complement \index{integer!representation!two's complement|(}
               \index{two's complement|(}}

The first two methods described were used on early computers. Modern computers
use a third method called \emph{two's complement} representation. The two's
complement of a number is found by the following two steps:
\begin{enumerate}
\item Find the one's complement of the number
\item Add one to the result of step 1
\end{enumerate}
Here's an example using $\underline{0}0111000$ (56). First the one's complement
is computed: $\underline{1}1000111$. Then one is added:
\[
\begin{array}{rr}
 & \underline{1}1000111 \\
+&                    1 \\ \hline
 & \underline{1}1001000
\end{array}
\]

In two complement's notation, computing the two's complement is equivalent
to negating a number. Thus, $\underline{1}1001000$ is the two's complement
representation of $-56$. Two negations should reproduce the original number.
Surprising two's complement does meet this requirement. Take the two's
complement of $\underline{1}1001000$ by adding one to the one's complement.
\[
\begin{array}{rr}
 & \underline{0}0110111 \\
+&                    1 \\ \hline
 & \underline{0}0111000
\end{array}
\]

When performing the addition in the two's complement operation, the addition
of the leftmost bit may produce a carry. This carry is \emph{not} used. 
Remember that all data on the computer is of some fixed size (in terms of
number of bits). Adding two bytes always produces a byte as a result (just
as adding two words produces a word, {\em etc\/}.) This property is important 
for two's complement notation. For example, consider zero as a one byte
two's complement number ($\underline{0}0000000$). Computing its two complement
produces the sum:
\[
\begin{array}{rr}
 & \underline{1}1111111 \\
+&                    1 \\ \hline
c& \underline{0}0000000
\end{array}
\]
where $c$ represents a carry. (Later it will be shown how to detect this carry,
but it is not stored in the result.) Thus, in two's complement notation there
is only one zero. This makes two's complement arithmetic simpler that the
previous methods.

\begin{table}
\centering
\begin{tabular}{||c|c||}
\hline
Number & Hex Representation \\
\hline
0 & 00 \\
1 & 01 \\
127 & 7F \\
-128 & 80 \\
-127 & 81 \\
-2 & FE \\
-1 & FF \\
\hline
\end{tabular}
\caption{Two's Complement Representation \label{tab:twocomp}}
\end{table}

Using two's complement notation, a signed byte can be used to
represent the numbers $-128$ to $+127$. Table~\ref{tab:twocomp} shows some
selected values. If 16 bits are used, the signed numbers $-32,768$ to 
$+32,767$ can be represented. $+32,767$ is represented by 7FFF, 
$-32,768$ by 8000, -128 as FF80 and -1 as FFFF. 32 bit two's complement 
numbers range from $-2$ billion to $+2$ billion approximately. 


The CPU has no idea what a particular byte (or word or double word) is
supposed to represent. Assembly does not have the idea of types that a
high level language has. How data is interpreted depends on what instruction
is used on the data. Whether the hex value FF is considered to represent
a signed $-1$ or a unsigned $+255$ depends on the programmer. The C language
defines signed and unsigned integer types. This allows a C compiler to
determine the correct instructions to use with the data.

\index{two's complement|)}
\index{integer!representation!two's complement|)}
\index{integer!signed|)}

\subsection{Sign extension \index{integer!sign extension|(}}

In assembly, all data has a specified size. It is not uncommon to need to
change the size of data to use it with other data. Decreasing size is the
easiest.

\subsubsection{Decreasing size of data}

To decrease the size of data, simply remove the more significant bits of the
data. Here's a trivial example:
\begin{AsmCodeListing}[numbers=none,frame=none]
      mov    ax, 0034h      ; ax = 52 (stored in 16 bits)
      mov    cl, al         ; cl = lower 8-bits of ax
\end{AsmCodeListing}

Of course, if the number can not be represented correctly in the smaller
size, decreasing the size does not work. For example, if {\code AX}
were 0134h (or 308 in decimal) then the above code would still set
{\code CL} to 34h. This method works with both signed and unsigned
numbers.  Consider signed numbers, if {\code AX} was FFFFh ($-1$ as a
word), then {\code CL} would be FFh ($-1$ as a byte). However, note
that this is not correct if the value in {\code AX} was unsigned!

The rule for unsigned numbers is that all the bits being removed must
be 0 for the conversion to be correct. The rule for signed numbers is
that the bits being removed must be either all 1's or all 0's. In
addition, the first bit not being removed must have the same value as
the removed bits. This bit will be the new sign bit of the smaller value.
It is important that it be same as the original sign bit!

\subsubsection{Increasing size of data}

Increasing the size of data is more complicated than
decreasing. Consider the hex byte FF. If it is extended to a word,
what value should the word have?  It depends on how FF is
interpreted. If FF is a unsigned byte (255 in decimal), then the word
should be 00FF; however, if it is a signed byte ($-1$ in decimal),
then the word should be FFFF.

In general, to extend an unsigned number, one makes all the new bits
of the expanded number 0. Thus, FF becomes 00FF. However, to extend a
signed number, one must \emph{extend} the sign
bit. \index{integer!sign bit} This means that the new bits become
copies of the sign bit. Since the sign bit of FF is 1, the new bits
must also be all ones, to produce FFFF. If the signed number 5A (90 in
decimal) was extended, the result would be 005A.

There are several instructions that the 80386 provides for extension of
numbers. Remember that the computer does not know whether a number is signed
or unsigned. It is up to the programmer to use the correct instruction.

For unsigned numbers, one can simply put zeros in the upper bits using a
{\code MOV} instruction. For example, to extend the byte in AL to an unsigned
word in AX:
\begin{AsmCodeListing}[numbers=none,frame=none]
      mov    ah, 0   ; zero out upper 8-bits
\end{AsmCodeListing}
However, it is not possible to use a {\code MOV} instruction to
convert the unsigned word in AX to an unsigned double word in EAX. Why
not? There is no way to specify the upper 16 bits of EAX in a {\code
MOV}. The 80386 solves this problem by providing a new instruction
{\code MOVZX}. \index{MOVZX} This instruction has two operands. The destination
(first operand) must be a 16 or 32~bit register.  The source (second
operand) may be an 8 or 16~bit register or a byte or word of
memory. The other restriction is that the destination must be larger than
than the source. (Most instructions require the source and destination to be
the same size.) Here are some examples:
\begin{AsmCodeListing}[numbers=none,frame=none]
      movzx  eax, ax      ; extends ax into eax
      movzx  eax, al      ; extends al into eax
      movzx  ax, al       ; extends al into ax
      movzx  ebx, ax      ; extends ax into ebx
\end{AsmCodeListing}

For signed numbers, there is no easy way to use the {\code MOV}
instruction for any case. The 8086 provided several instructions to
extend signed numbers.  The {\code CBW} \index{CBW} (Convert Byte to
Word) instruction sign extends the AL register into AX. The operands
are implicit. The {\code CWD} \index{CWD} (Convert Word to Double
word) instruction sign extends AX into DX:AX. The notation DX:AX means
to think of the DX and AX registers as one 32~bit register with the
upper 16 bits in DX and the lower bits in AX. (Remember that the 8086
did not have any 32~bit registers!) The 80386 added several new
instructions. The {\code CWDE} \index{CWDE} (Convert Word to Double
word Extended) instruction sign extends AX into EAX. The {\code CDQ}
\index{CDQ} (Convert Double word to Quad word) instruction sign
extends EAX into EDX:EAX (64~bits!). Finally, the {\code MOVSX}
\index{MOVSX} instruction works like {\code MOVZX} except it uses the
rules for signed numbers.

\subsubsection{Application to C programming}

Extending \MarginNote{ANSI C does not define whether the {\code char}
type is signed or not, it is up to each individual compiler to decide
this. That is why the type is explicitly defined in
Figure~\ref{fig:charExt}.} of unsigned and signed integers also occurs
in C. Variables in C may be declared as either signed or unsigned
({\code int} is signed). Consider the code in
Figure~\ref{fig:charExt}.  In line~3, the variable {\code a} is
extended using the rules for unsigned values (using {\code MOVZX}), but in 
line~4, the signed rules are used for {\code b} (using {\code MOVSX}).

\begin{figure}[t]
\begin{lstlisting}[frame=tlrb]{}
unsigned char uchar = 0xFF;
signed char   schar = 0xFF;
int a = (int) uchar;     /* a = 255 (0x000000FF) */
int b = (int) schar;     /* b = -1  (0xFFFFFFFF) */
\end{lstlisting}
\caption{}
\label{fig:charExt}
\end{figure}

There is a common C programming bug that directly relates to this subject.
Consider the code in Figure~\ref{fig:IObug}. The prototype of 
{\code fgetc()}{\samepage is:
\begin{CodeQuote}
int fgetc( FILE * );
\end{CodeQuote}
One might question }why does the function return back an {\code int}
since it reads characters? The reason is that it normally does return
back an {\code char} (extended to an {\code int} value using zero
extension). However, there is one value that it may return that is not
a character, {\code EOF}. This is a macro that is usually defined as
$-1$. Thus, {\code fgetc()} either returns back a {\code char}
extended to an {\code int} value (which looks like {\code 000000{\em
xx}} in hex) or {\code EOF} (which looks like {\code FFFFFFFF} in
hex).

\begin{figure}[t]
\begin{lstlisting}[stepnumber=0,frame=tlrb]{}
char ch;
while( (ch = fgetc(fp)) != EOF ) {
  /* do something with ch */
}
\end{lstlisting}
\caption{}
\label{fig:IObug}
\end{figure}

The basic problem with the program in Figure~\ref{fig:IObug} is that
{\code fgetc()} returns an {\code int}, but this value is stored in a
{\code char}. C will truncate the higher order bits to fit the {\code
int} value into the {\code char}. The only problem is that the numbers
(in hex) {\code 000000FF} and {\code FFFFFFFF} both will be truncated
to the byte {\code FF}. Thus, the while loop can not distinguish
between reading the byte {\code FF} from the file and end of file.

Exactly what the code does in this case, depends on whether {\code char} is
signed or unsigned. Why? Because in line~2, {\code ch} is compared with 
{\code EOF}. Since {\code EOF} is an {\code int} value\footnote{It is a
common misconception that files have an EOF character at their end. This is
\emph{not} true!}, {\code ch} will be
extended to an {\code int} so that two values being compared are of the
same size\footnote{The reason for this requirement will be shown later.}.
As Figure~\ref{fig:charExt} showed, where the variable is signed or 
unsigned is very important.

If {\code char} is unsigned, {\code FF} is extended to be {\code
000000FF}. This is compared to {\code EOF} ({\code FFFFFFFF}) and
found to be not equal. Thus, the loop never ends!

If {\code char} is signed, {\code FF} is extended to {\code
FFFFFFFF}. This does compare as equal and the loop ends. However,
since the byte {\code FF} may have been read from the file, the loop
could be ending prematurely.

The solution to this problem is to define the {\code ch} variable as an
{\code int}, not a {\code char}. When this is done, no truncating or
extension is done in line~2. Inside the loop, it is safe to truncate the
value since {\code ch} \emph{must} actually be a simple byte there.

\index{integer!sign extension|)}
\index{integer!representation|)}

\subsection{Two's complement arithmetic \index{two's complement!arithmetic|(}}

As was seen earlier, the {\code add} instruction performs addition and
the {\code sub} instruction performs subtraction. Two of the bits in
the FLAGS register that these instructions set are the \emph{overflow}
and \emph{carry flag}. The overflow flag is set if the true result of
the operation is too big to fit into the destination for signed
arithmetic. The carry flag is set if there is a carry in the msb of an
addition or a borrow in the msb of a subtraction. Thus, it can be used
to detect overflow for unsigned arithmetic. The uses of the carry flag
for signed arithmetic will be seen shortly. One of the great
advantages of 2's~complement is that the rules for addition and
subtraction are exactly the same as for unsigned integers. Thus,
{\code add} and {\code sub} may be used on signed or unsigned
integers.
\[
\begin{array}{rrcrr}
 & 002\mathrm{C} & & & 44\\
+& \mathrm{FFFF} & &+&(-1)\\ \cline{1-2} \cline{4-5}
 & 002\mathrm{B} & & & 43
\end{array}
\]
There is a carry generated, but it is not part of the answer.

\index{integer!multiplication|(}
\index{MUL|(}
\index{IMUL|(}
There are two different multiply and divide instructions. First, to
multiply use either the {\code MUL} or {\code IMUL} instruction. The
{\code MUL} instruction is used to multiply unsigned numbers and
{\code IMUL} is used to multiply signed integers. Why are two
different instructions needed? The rules for multiplication are
different for unsigned and 2's complement signed numbers. How so?
Consider the multiplication of the byte FF with itself yielding a word
result. Using unsigned multiplication this is 255 times 255 or 65025
(or FE01 in hex). Using signed multiplication this is $-1$ times $-1$ or
1 (or 0001 in hex).

There are several forms of the multiplication instructions. The oldest
form looks like:
\begin{AsmCodeListing}[numbers=none,frame=none]
      mul   source
\end{AsmCodeListing}
The \emph{source} is either a register or a memory reference. It can not be an 
immediate value. Exactly what multiplication is performed depends on the
size of the source operand. If the operand is byte sized, it is multiplied by
the byte in the AL register and the result is stored in the 16 bits of AX.
If the source is 16-bit, it is multiplied by the word in AX and the 32-bit
result is stored in DX:AX. If the source is 32-bit, it is multiplied by EAX
and the 64-bit result is stored into EDX:EAX.
\index{MUL|)}

\begin{table}[t]
\centering
\begin{tabular}{|c|c|c|l|}
\hline
{ \bf dest} & { \bf source1 } & {\bf source2} & \multicolumn{1}{c|}{\bf Action} \\ \hline
            & reg/mem8        &               & AX = AL*source1 \\
            & reg/mem16       &               & DX:AX = AX*source1 \\
            & reg/mem32       &               & EDX:EAX = EAX*source1 \\
reg16       & reg/mem16       &               & dest *= source1 \\
reg32       & reg/mem32       &               & dest *= source1 \\
reg16       & immed8          &               & dest *= immed8 \\
reg32       & immed8          &               & dest *= immed8 \\
reg16       & immed16         &               & dest *= immed16 \\
reg32       & immed32         &               & dest *= immed32 \\
reg16       & reg/mem16       & immed8        & dest = source1*source2 \\
reg32       & reg/mem32       & immed8        & dest = source1*source2 \\
reg16       & reg/mem16       & immed16       & dest = source1*source2 \\
reg32       & reg/mem32       & immed32       & dest = source1*source2 \\
\hline
\end{tabular}
\caption{{\code imul} Instructions \label{tab:imul}}
\end{table}

The {\code IMUL} instruction has the same formats as {\code MUL}, but also
adds some other instruction formats. There are two and three operand formats:
\begin{AsmCodeListing}[numbers=none,frame=none]
      imul   dest, source1
      imul   dest, source1, source2
\end{AsmCodeListing}
Table~\ref{tab:imul} shows the possible combinations.
\index{IMUL|)}
\index{integer!multiplication|)}

\index{integer!division|(}
\index{DIV}
The two division operators are {\code DIV} and {\code IDIV}. They perform
unsigned and signed integer division respectively. The general format is:
\begin{AsmCodeListing}[numbers=none,frame=none]
      div   source
\end{AsmCodeListing}
If the source is 8-bit, then AX is divided by the operand. The
quotient is stored in AL and the remainder in AH. If the source is
16-bit, then DX:AX is divided by the operand. The quotient is stored
into AX and remainder into DX. If the source is 32-bit, then EDX:EAX
is divided by the operand and the quotient is stored into EAX and the
remainder into EDX. The {\code IDIV} \index{IDIV} instruction works the same
way. There are no special {\code IDIV} instructions like the special
{\code IMUL} ones. If the quotient is too big to fit into its register
or the divisor is zero, the program is interrupted and terminates. A
very common error is to forget to initialize DX or EDX before division.
\index{integer!division|)}

The {\code NEG} \index{NEG} instruction negates its single operand by computing
its two's complement. Its operand may be any 8-bit, 16-bit, or 32-bit
register or memory location.

\subsection{Example program}
\index{math.asm|(}
\begin{AsmCodeListing}[label=math.asm]
%include "asm_io.inc"
segment .data         ; Output strings
prompt          db    "Enter a number: ", 0
square_msg      db    "Square of input is ", 0
cube_msg        db    "Cube of input is ", 0
cube25_msg      db    "Cube of input times 25 is ", 0
quot_msg        db    "Quotient of cube/100 is ", 0
rem_msg         db    "Remainder of cube/100 is ", 0
neg_msg         db    "The negation of the remainder is ", 0

segment .bss
input   resd 1

segment .text
        global  _asm_main
_asm_main:
        enter   0,0               ; setup routine
	pusha

        mov     eax, prompt
        call    print_string

        call    read_int
        mov     [input], eax

        imul    eax               ; edx:eax = eax * eax
        mov     ebx, eax          ; save answer in ebx
        mov     eax, square_msg
        call    print_string
        mov     eax, ebx
        call    print_int
        call    print_nl

        mov     ebx, eax
        imul    ebx, [input]      ; ebx *= [input]
        mov     eax, cube_msg
        call    print_string
        mov     eax, ebx
        call    print_int
        call    print_nl

        imul    ecx, ebx, 25      ; ecx = ebx*25
        mov     eax, cube25_msg
        call    print_string
        mov     eax, ecx
        call    print_int
        call    print_nl

        mov     eax, ebx
        cdq                       ; initialize edx by sign extension
        mov     ecx, 100          ; can't divide by immediate value
        idiv    ecx               ; edx:eax / ecx
        mov     ecx, eax          ; save quotient into ecx
        mov     eax, quot_msg
        call    print_string
        mov     eax, ecx
        call    print_int
        call    print_nl
        mov     eax, rem_msg
        call    print_string
        mov     eax, edx
        call    print_int
        call    print_nl
        
        neg     edx               ; negate the remainder
        mov     eax, neg_msg
        call    print_string
        mov     eax, edx
        call    print_int
        call    print_nl

        popa
        mov     eax, 0            ; return back to C
        leave                     
        ret
\end{AsmCodeListing}
\index{math.asm|)}

\subsection{Extended precision arithmetic \label{sec:ExtPrecArith} \index{integer!extended precision|(}}}

Assembly language also provides instructions that allow one to perform
addition and subtraction of numbers larger than double words. These
instructions use the carry flag. As stated above, both the {\code ADD}
\index{ADD} and {\code SUB} \index{SUB} instructions modify the carry
flag if a carry or borrow are generated, respectively. This
information stored in the carry flag can be used to add or subtract
large numbers by breaking up the operation into smaller double word
(or smaller) pieces.

The {\code ADC} \index{ADC} and {\code SBB} \index{SBB} instructions
use this information in the carry flag. The {\code ADC} instruction
performs the following operation:
\begin{center}
{\code \emph{operand1} = \emph{operand1} + carry flag + \emph{operand2} }
\end{center}
The {\code SBB} instruction performs:
\begin{center}
{\code \emph{operand1} = \emph{operand1} - carry flag - \emph{operand2} }
\end{center}
How are these used? Consider the sum of 64-bit integers in EDX:EAX and 
EBX:ECX. The following code would store the sum in EDX:EAX:
\begin{AsmCodeListing}[frame=none]
      add    eax, ecx       ; add lower 32-bits
      adc    edx, ebx       ; add upper 32-bits and carry from previous sum
\end{AsmCodeListing}
Subtraction is very similar. The following code subtracts EBX:ECX from 
EDX:EAX:
\begin{AsmCodeListing}[frame=none]
      sub    eax, ecx       ; subtract lower 32-bits
      sbb    edx, ebx       ; subtract upper 32-bits and borrow
\end{AsmCodeListing}

For \emph{really} large numbers, a loop could be used (see 
Section~\ref{sec:control}). For a sum loop, it would be convenient to use
{\code ADC} instruction for every iteration (instead of all but the first
iteration). This can be done by using the {\code CLC} \index{CLC} (CLear Carry)
instruction right before the loop starts to initialize the carry flag to 0.
If the carry flag is 0, there is no difference between the {\code ADD} and
{\code ADC} instructions. The same idea can be used for subtraction, too.
\index{integer!extended precision|)}
\index{two's complement!arithmetic|)}

\section{Control Structures}
\label{sec:control}
High level languages provide high level control structures (\emph{e.g.}, the
\emph{if} and \emph{while} statements) that control the thread of execution.
Assembly language does not provide such complex control structures. It instead
uses the infamous \emph{goto} and used inappropriately can result in
spaghetti code! However, it \emph{is} possible to write structured assembly
language programs. The basic procedure is to design the program logic using
the familiar high level control structures and translate the design into
the appropriate assembly language (much like a compiler would do).

\subsection{Comparisons \index{integer!comparisons|(} \index{CMP|(}}
%TODO: Make a table of all the FLAG bits

\index{register!FLAGS|(}
Control structures decide what to do based on comparisons of data. In
assembly, the result of a comparison is stored in the FLAGS register
to be used later. The 80x86 provides the {\code CMP} instruction to
perform comparisons.  The FLAGS register is set based on the
difference of the two operands of the {\code CMP} instruction. The
operands are subtracted and the FLAGS are set based on the result, but
the result is \emph{not} stored anywhere. If you need the result use
the SUB instead of the {\code CMP} instruction.

\index{integer!unsigned|(}
For unsigned integers, there are two flags (bits in the FLAGS
register) that are important: the zero (ZF) \index{register!FLAGS!ZF} and carry (CF) 
\index{register!FLAGS!CF} flags. The
zero flag is set (1) if the resulting difference would be zero. The
carry flag is used as a borrow flag for subtraction. Consider a
comparison like:
\begin{AsmCodeListing}[frame=none, numbers=none]
      cmp    vleft, vright
\end{AsmCodeListing}
The difference of {\code vleft~-~vright} is computed and the flags are
set accordingly. If the difference of the of {\code CMP} is zero, {\code 
vleft~=~vright}, then ZF is set (\emph{i.e.} 1) and the CF is unset
(\emph{i.e.} 0). If {\code vleft~>~vright}, then ZF is unset and CF
is unset (no borrow). If {\code vleft~<~vright}, then ZF is unset and
CF is set (borrow).
\index{integer!unsigned|)}

\index{integer!signed|(} 
For signed integers, there are three flags
that are important: the zero \index{register!FLAGS!ZF} (ZF) flag, the
overflow \index{register!FLAGS!OF}(OF) flag and the sign
\index{register!FLAGS!SF} (SF) flag. \MarginNote{Why does SF~=~OF if
{\code vleft~>~vright}? If there is no overflow, then the difference
will have the correct value and must be non-negative. Thus,
SF~=~OF~=~0. However, if there is an overflow, the difference will not
have the correct value (and in fact will be negative). Thus,
SF~=~OF~=~1.}The overflow flag is set if the result of an operation
overflows (or underflows). The sign flag is set if the result of an
operation is negative. If {\code vleft~=~vright}, the ZF is set (just
as for unsigned integers). If {\code vleft~>~vright}, ZF is unset and
SF~=~OF.  If {\code vleft~<~vright}, ZF is unset and SF~$\neq$~OF.
\index{integer!signed|)}

Do not forget that other instructions can also change the FLAGS
register, not just {\code CMP}.
\index{CMP|)}
\index{integer!comparisons|)}
\index{register!FLAGS|)}
\index{integer|)}

\subsection{Branch instructions}

Branch instructions can transfer execution to arbitrary points of a program.
In other words, they act like a \emph{goto}. There are two types of branches:
unconditional and conditional. An unconditional branch is just like a goto,
it always makes the branch. A conditional branch may or may not make the
branch depending on the flags in the FLAGS register. If a conditional branch
does not make the branch, control passes to the next instruction.

\index{JMP|(}
The {\code JMP} (short for \emph{jump}) instruction makes
unconditional branches. Its single argument is usually a \emph{code
label} to the instruction to branch to. The assembler or linker will
replace the label with correct address of the instruction. This is
another one of the tedious operations that the assembler does to make
the programmer's life easier. It is important to realize that the
statement immediately after the {\code JMP} instruction will never be
executed unless another instruction branches to it!

There are several variations of the jump instruction:
\begin{description}

\item[SHORT] This jump is very limited in range. It can only move up or
down 128 bytes in memory. The advantage of this type is that it uses less
memory than the others. It uses a single signed byte to store the 
\emph{displacement} of the jump. The displacement is how many bytes to move 
ahead or behind. (The displacement is added to EIP). To specify a
short jump, use the {\code SHORT} keyword immediately before the label
in the {\code JMP} instruction.

\item[NEAR] This jump is the default type for both unconditional and 
conditional branches, it can be used to jump to any location in a
segment. Actually, the 80386 supports two types of near jumps.  One
uses two bytes for the displacement. This allows one to move up or
down roughly 32,000 bytes. The other type uses four bytes for the
displacement, which of course allows one to move to any location in
the code segment. The four byte type is the default in 386 protected
mode. The two byte type can be specified by putting the {\code WORD}
keyword before the label in the {\code JMP} instruction.

\item[FAR] This jump allows control to move to another code segment. This is
a very rare thing to do in 386 protected mode.
\end{description}

Valid code labels follow the same rules as data labels. Code labels
are defined by placing them in the code segment in front of the statement
they label. A colon is placed at the end of the label at its point of
definition. The colon is \emph{not} part of the name.
\index{JMP|)}

\index{conditional branch|(}
\begin{table}[t]
\center
\begin{tabular}{|ll|}
\hline
JZ  & branches only if ZF is set \\
JNZ & branches only if ZF is unset \\
JO  & branches only if OF is set \\
JNO & branches only if OF is unset \\
JS  & branches only if SF is set \\
JNS & branches only if SF is unset \\
JC  & branches only if CF is set \\
JNC & branches only if CF is unset \\
JP  & branches only if PF is set \\
JNP & branches only if PF is unset \\
\hline
\end{tabular}
\caption{Simple Conditional Branches \label{tab:SimpBran} \index{JZ} \index{JNZ}
        \index{JO} \index{JNO} \index{JS} \index{JNS} \index{JC} \index{JNC}
        \index{JP} \index{JNP}}
\end{table}

There are many different conditional branch instructions. They also
take a code label as their single operand. The simplest ones just look
at a single flag in the FLAGS register to determine whether to branch
or not.  See Table~\ref{tab:SimpBran} for a list of these
instructions. (PF is the \emph{parity flag} \index{register!FLAGS!PF}
which indicates the odd or evenness of the number of bits set in the
lower 8-bits of the result.)

The following pseudo-code:
\begin{Verbatim}
if ( EAX == 0 )
  EBX = 1;
else
  EBX = 2;
\end{Verbatim}
could be written in assembly as:
\begin{AsmCodeListing}[frame=none]
      cmp    eax, 0            ; set flags (ZF set if eax - 0 = 0)
      jz     thenblock         ; if ZF is set branch to thenblock
      mov    ebx, 2            ; ELSE part of IF
      jmp    next              ; jump over THEN part of IF
thenblock:
      mov    ebx, 1            ; THEN part of IF
next:
\end{AsmCodeListing}

Other comparisons are not so easy using the conditional branches in 
Table~\ref{tab:SimpBran}. To illustrate, consider the following pseudo-code:
\begin{Verbatim}
if ( EAX >= 5 )
  EBX = 1;
else
  EBX = 2;
\end{Verbatim}
If EAX is greater than or equal to five, the ZF may be set or unset and SF
will equal OF. Here is assembly code that tests for these conditions 
(assuming that EAX is signed):
\begin{AsmCodeListing}[frame=none]
      cmp    eax, 5
      js     signon            ; goto signon if SF = 1
      jo     elseblock         ; goto elseblock if OF = 1 and SF = 0
      jmp    thenblock         ; goto thenblock if SF = 0 and OF = 0
signon:
      jo     thenblock         ; goto thenblock if SF = 1 and OF = 1
elseblock:
      mov    ebx, 2
      jmp    next
thenblock:
      mov    ebx, 1
next:
\end{AsmCodeListing}

\begin{table}
\center
\begin{tabular}{|ll|ll|}
\hline
\multicolumn{2}{|c|}{\textbf{Signed}} & \multicolumn{2}{c|}{\textbf{Unsigned}} \\
\hline
JE & branches if {\code vleft = vright} & JE & branches if {\code vleft = vright} \\
JNE & branches if {\code vleft $\neq$ vright} & JNE & branches if {\code vleft $\neq$ vright} \\
JL, JNGE & branches if {\code vleft < vright} & JB, JNAE & branches if {\code vleft < vright} \\
JLE, JNG & branches if {\code vleft $\leq$ vright} & JBE, JNA & branches if {\code vleft $\leq$ vright} \\
JG, JNLE & branches if {\code vleft > vright} & JA, JNBE & branches if {\code vleft > vright} \\
JGE, JNL & branches if {\code vleft $\geq$ vright} & JAE, JNB & branches if {\code vleft $\geq$ vright} \\
\hline
\end{tabular}
\caption{Signed and Unsigned Comparison Instructions \label{tab:CompBran} \index{JE} \index{JNE}
         \index{JL} \index{JNGE} \index{JLE} \index{JNG} \index{JG} \index{JNLE} \index{JGE}
         \index{JNL}}
\end{table}

The above code is very awkward. Fortunately, the 80x86 provides additional
branch instructions to make these type of tests \emph{much} easier. There
are signed and unsigned versions of each. Table~\ref{tab:CompBran} shows
these instructions. The equal and not equal branches (JE and JNE) are the
same for both signed and unsigned integers. (In fact, JE and JNE are really
identical to JZ and JNZ, respectively.) Each of the other branch 
instructions have two synonyms. For example, look at JL (jump less than) and
JNGE (jump not greater than or equal to). These are the same instruction
because:
\[ x < y \Longrightarrow \mathbf{not}( x \geq y ) \]
The unsigned branches use A for \emph{above} and B for \emph{below} instead of
L and G.

Using these new branch instructions, the pseudo-code above can be translated
to assembly much easier.
\begin{AsmCodeListing}[frame=none]
      cmp    eax, 5
      jge    thenblock
      mov    ebx, 2
      jmp    next
thenblock:
      mov    ebx, 1
next:
\end{AsmCodeListing}
\index{conditional branch|)}

\subsection{The loop instructions}

The 80x86 provides several instructions designed to implement 
\emph{for}-like loops. Each of these instructions takes a code label
as its single operand.
\begin{description}
\item[LOOP] 
\index{LOOP}
Decrements ECX, if ECX $\neq$ 0, branches to label
\item[LOOPE, LOOPZ]
\index{LOOPE} \index{LOOPZ}
Decrements ECX (FLAGS register is not modified), if
                    ECX $\neq$ 0 and ZF = 1, branches
\item[LOOPNE, LOOPNZ]
\index{LOOPNE} \index{LOOPNZ}
Decrements ECX (FLAGS unchanged), if ECX $\neq$ 0
                      and ZF = 0, branches
\end{description}

The last two loop instructions are useful for sequential search loops. The
following pseudo-code:
\begin{lstlisting}[stepnumber=0]{}
sum = 0;
for( i=10; i >0; i-- )
  sum += i;
\end{lstlisting}
\noindent could be translated into assembly as:
\begin{AsmCodeListing}[frame=none]
      mov    eax, 0          ; eax is sum
      mov    ecx, 10         ; ecx is i
loop_start:
      add    eax, ecx
      loop   loop_start
\end{AsmCodeListing}

\section{Translating Standard Control Structures}

This section looks at how the standard control structures of high level
languages can be implemented in assembly language.

\subsection{If statements \index{if statment|(}}
The following pseudo-code:
\begin{lstlisting}[stepnumber=0]{}
if ( condition )
  then_block;
else
  else_block;
\end{lstlisting}
\noindent could be implemented as:
\begin{AsmCodeListing}[frame=none]
      ; code to set FLAGS
      jxx    else_block    ; select xx so that branches if condition false
      ; code for then block
      jmp    endif
else_block:
      ; code for else block
endif:
\end{AsmCodeListing}

If there is no else, then the {\code else\_block} branch can be replaced by
a branch to {\code endif}.
\begin{AsmCodeListing}[frame=none]
      ; code to set FLAGS
      jxx    endif          ; select xx so that branches if condition false
      ; code for then block
endif:
\end{AsmCodeListing}
\index{if statment|)}

\subsection{While loops \index{while loop|(}}
The \emph{while} loop is a top tested loop:
\begin{lstlisting}[stepnumber=0]{}
while( condition ) {
  body of loop;
}
\end{lstlisting}
\noindent This could be translated into:
\begin{AsmCodeListing}[frame=none]
while:
      ; code to set FLAGS based on condition
      jxx    endwhile       ; select xx so that branches if false
      ; body of loop
      jmp    while
endwhile:
\end{AsmCodeListing}
\index{while loop|)}

\subsection{Do while loops \index{do while loop|(}}
The \emph{do while} loop is a bottom tested loop:
\begin{lstlisting}[stepnumber=0]{}
do {
  body of loop;
} while( condition );
\end{lstlisting}
\noindent This could be translated into:
\begin{AsmCodeListing}[frame=none]
do:
      ; body of loop
      ; code to set FLAGS based on condition
      jxx    do          ; select xx so that branches if true
\end{AsmCodeListing}
\index{do while loop|)}

\begin{figure}[t]
\begin{lstlisting}[frame=tlrb]{}
  unsigned guess;   /* current guess for prime      */
  unsigned factor;  /* possible factor of guess     */
  unsigned limit;   /* find primes up to this value */

  printf("Find primes up to: ");
  scanf("%u", &limit);
  printf("2\n");    /* treat first two primes as  */
  printf("3\n");    /* special case               */
  guess = 5;        /* initial guess */
  while ( guess <= limit ) {
    /* look for a factor of guess */
    factor = 3;
    while ( factor*factor < guess &&
            guess % factor != 0 )
     factor += 2;
    if ( guess % factor != 0 )
      printf("%d\n", guess);
    guess += 2;    /* only look at odd numbers */
  }
\end{lstlisting}
\caption{}\label{fig:primec}
\end{figure}

\section{Example: Finding Prime Numbers}
This section looks at a program that finds prime numbers. Recall that
prime numbers are evenly divisible by only 1 and themselves. There is
no formula for doing this. The basic method this program uses is to
find the factors of all odd numbers\footnote{2 is the only even prime
number.} below a given limit. If no factor can be found for an odd
number, it is prime.  Figure~\ref{fig:primec} shows the basic
algorithm written in C.

Here's the assembly version:
\index{prime.asm|(}
\begin{AsmCodeListing}[label=prime.asm]
%include "asm_io.inc"
segment .data
Message         db      "Find primes up to: ", 0

segment .bss
Limit           resd    1               ; find primes up to this limit
Guess           resd    1               ; the current guess for prime

segment .text
        global  _asm_main
_asm_main:
        enter   0,0               ; setup routine
        pusha

        mov     eax, Message
        call    print_string
        call    read_int             ; scanf("%u", & limit );
        mov     [Limit], eax

        mov     eax, 2               ; printf("2\n");
        call    print_int
        call    print_nl
        mov     eax, 3               ; printf("3\n");
        call    print_int
        call    print_nl

        mov     dword [Guess], 5     ; Guess = 5;
while_limit:                         ; while ( Guess <= Limit )
        mov     eax,[Guess]
        cmp     eax, [Limit]
        jnbe    end_while_limit      ; use jnbe since numbers are unsigned

        mov     ebx, 3               ; ebx is factor = 3;
while_factor:
        mov     eax,ebx
        mul     eax                  ; edx:eax = eax*eax
        jo      end_while_factor     ; if answer won't fit in eax alone
        cmp     eax, [Guess]
        jnb     end_while_factor     ; if !(factor*factor < guess)
        mov     eax,[Guess]
        mov     edx,0
        div     ebx                  ; edx = edx:eax % ebx
        cmp     edx, 0
        je      end_while_factor     ; if !(guess % factor != 0)

        add     ebx,2                ; factor += 2;
        jmp     while_factor
end_while_factor:
        je      end_if               ; if !(guess % factor != 0)
        mov     eax,[Guess]          ; printf("%u\n")
        call    print_int
        call    print_nl
end_if:
        mov     eax,[Guess]
        add     eax, 2
        mov     [Guess], eax         ; guess += 2
        jmp     while_limit
end_while_limit:

        popa
        mov     eax, 0            ; return back to C
        leave                     
        ret
\end{AsmCodeListing}
\index{prime.asm|)}

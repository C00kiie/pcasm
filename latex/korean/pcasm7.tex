% -*-latex-*-
\chapter{구조체와 C++}

\section{구조체\index{구조체|(}}

\subsection{서론}

C 에서 사용되는 구조체는 연관된 데이터를 하나의 변수에 보관하는 것이라
말할 수 있다. 이는 몇 가지 장점이 있다. 

\begin{enumerate}
\item 구조체에 정의된 데이터가 연관되어 있음을 보임으로써 코드를 
      명료하게 할 수 있다. 
\item 이를 통해 함수로의 데이터 전달을 단순하게 할 수 있다. 여러 개의 변수를
      독립적으로 전달하는 대신에 이를 이용해 하나의 단위만 전달 하면 된다.
\item 이는 코드의 \emph{지역성(locality)}\index{지역성}\footnote{운영체제에 관련한 책의 가상 메모리 관리 단원을 살펴보라.} 을 향상시킨다.
\end{enumerate}

어셈블리의 관점에서 볼 때 구조체는 원소들의 크기가 \emph{제각각}인 배열로 볼 수 있다. 실제 배열의
원소들의 크기는 언제나 같은 형이자, 같은 크기 여야 한다. 이를 통해 실제 배열에서는 배열의 시작 주소와
원소의 크기, 원소의 번째 수 만 알면 원소의 주소를 계산할 수 있게 된다.

그러나 구조체의 원소들은 동일한 크기를 가질 필요가 없다. (그리고 대부분 동일한 크기가 아니다)
이 때문에 구조체의 각각의 원소들은 뚜렷이 지정되어 있어야 하고 수치적인 위치 대신에
\emph{태그(tag)} (또는 이름)를 가진다. 

어셈블리에서 구조체의 원소에 접근하는 것은 배열의 원소의 접근하는 것과 비슷하다. 원소에 접근하기
위해서는 먼저 구조체의 시작 주소를 알아야 하고, 구조체의 시작 부분으로 부터의 각 원소의 \emph{상대 오프셋(relative offset)}을
알아야 한다. 그러나, 배열의 경우 이 오프셋이 원소의 번째 수 만 알고도 계산될 수 있었지만, 구조체의 원소들은
컴파일러에 의해 오프셋이 지정된다. 

예를 들어 아래의 구조체를 보자. 

\begin{lstlisting}[stepnumber=0, escapeinside=~~]{}
struct S {
  short int x;    /* ~2 바이트 정수~ */
  int       y;    /* ~4 바이트 정수~ */
  double    z;    /* ~8 바이트 float ~*/
};
\end{lstlisting}

\begin{figure}
\centering
\begin{tabular}{r|c|}
\multicolumn{1}{c}{오프셋} & \multicolumn{1}{c}{원소} \\
\cline{2-2}
0 & {\code x} \\
\cline{2-2}
2 & \\
  & {\code y} \\
\cline{2-2}
6 & \\
  & \\
  & {\code z} \\
  & \\
\cline{2-2}
\end{tabular}
\caption{구조체 S \label{fig:structPic1}}
\end{figure}

그림 ~\ref{fig:structPic1} 은 {\code S} 형의 변수가 컴퓨터 메모리 상에 어떻게
나타나는지 보여준다. ANSI C 표준에 따르면 메모리 상에 배열된 구조체의 원소들의 순서는
{\code struct} 정의에서 나타난 순서와 동일해야 한다고 하였다. 또한 첫 번째 원소는 구조체의
최상단에 위치해야 한다고 하였다. (\emph{i.e} 오프셋이 0 ) 이는 또한 {\code stddef.h} 헤더 파일에
{\code offsetof()} 라는 매우 유용한 매크로를 정의하였는데 \index{구조체!offsetof()} 이 매크로는 
구조체의 특정한 원소의 오프셋 값을 계산한 후 리턴한다. 이 매크로는 두 개의 인자를 가지는데 첫 번째는
구조체의 이름이고 두 번째는 오프셋 값을 계산할 원소의 이름이다. 따라서 
그림 ~\ref{fig:structPic1} 에서의 {\code offsetof(S,y)} 의 결과는 2 가 된다. 

%TODO: talk about definition of offsetof() ??

\subsection{메모리 배열하기}

\begin{figure}
\centering
\begin{tabular}{r|c|}
\multicolumn{1}{c}{오프셋} & \multicolumn{1}{c}{ 원소 } \\
\cline{2-2}
0 & {\code x} \\
\cline{2-2}
2 & \emph{사용되지 않음} \\
\cline{2-2}
4 & \\
  & {\code y} \\
\cline{2-2}
8 & \\
  & \\
  & {\code z} \\
  & \\
\cline{2-2}
\end{tabular}
\caption{구조체 S \label{fig:structPic2}}

\end{figure}
\index{구조체!배열하기|(}

\emph{gcc} 컴파일러를 사용시, {\code offsetof} 매크로를 이용하여
{\code y} 의 오프셋을 구한다면 2 가 아닌 4 를 리턴하게 된다. 왜냐하면 
\MarginNote{주소가 4 로 나누어 진다면 주소는 더블워드 경계에 놓여 있다는
점을 상기해라 } \emph{gcc} (그리고 다수의 컴파일러 들이) 는 변수들을
기본적으로 더블워드 경계에 놓기 때문이다. 32 비트 보호 모드에서 CPU 는 메모리 상의 데이타가 
더블워드 경계에 놓여 있을 때 더 빨리 읽어들일 수 있다. 그림 ~\ref{fig:structPic2} 는 {\code S} 구조체가
\emph{gcc} 를 사용했을 때 메모리 상에 어떻게 나타날지 보여준다. 컴파일러는 
구조체에 사용되지 않는 2 바이트를 끼워 넣어서 {\code y} (그리고 {\code z} 도) 를 더블워드
경계에 놓이게 하였다. 이는 {\code offsetof} 를 사용하여 오프셋을 계산하는 것이 
정의된 구조체의 형태 만을 보고 값을 계산하는 것 보다 훨씬 좋은 것인지를 알 수 있다. 

당연하게도 구조체가 오직 어셈블리에서만 사용된다면 프로그래머는 스스로 오프셋을
지정할 수 있다. 그러나 어셈블리와 C 를 함께 사용한다면 어셈블리 코드와 C 코드의 구조체의 원소들의
오프셋이 반드시 일치해야 한다. 한 가지 문제가 되는 점은 C 컴파일러 마다 원소에 대해 다른 오프셋 값을
줄 수 있다는 것이다. 예를 들어 위에서 보았듯이 \emph{gcc} 컴파일러는 {\code S} 구조체를 그림
~\ref{fig:structPic2} 와 같은 모습으로 만들지만, 볼랜드 사의 컴파일러는 구조체를 그림 ~\ref{fig:structPic1} 과
같은 모습으로 만든다. C 컴파일러들은 데이터를 정렬하기 위해 여러가지 방법을 사용한다. 그러나
ANSI C 표준은 그 어떤 방법도 구체화 하지 않았으므로 컴파일러들이 저마다 다른 방법들을 
사용하게 되었다. 

%Borland's compiler has a flag, {\code -a}, that can be
%used to define the alignment used for all data. Compiling with {\code -a 4}
%tells \emph{bcc} to use double word alignment. Microsoft's compiler 
%provides a {\code \#pragma pack} directive that can be used to set
%the alignment (consult Microsoft's documentation for details). Borland's
%compiler also supports Microsoft's pragma 

\emph{gcc} \index{컴파일러!gcc!\_\_attribute\_\_} 컴파일러는 데이터의 정렬 방법을 구체화 하기 위해
유연하고도 복잡한 방법을 지원한다. 이 컴파일러에선 특별한 문법을 사용함으로써 어떠한 형이라도 데이터를
정렬하는 방법을 지정할 수 있게 하였다. 예를 들어 아래의 행을 보면 

\begin{lstlisting}[stepnumber=0]{}
  typedef short int unaligned_int __attribute__((aligned(1)));
\end{lstlisting}
\noindent 이는 {\code unaligned\_int} 라는 새로운 형을 정의하는데
이는 바이트 경계에 맞추어져 있다. ({\code \_\_attribute\_\_} 다음의 괄호들이 꼭 필요하다)
{\code aligned} 인자 1 은 2 의 멱수들로 바뀔 수 있으며 이는 다른 정렬 조건을 나타낸다. 
(2 는 워드 정렬, 4 는 더블워드 정렬 \emph{등등}) 만일 구조체의 {\code y} 원소가 
{\code unaligned\_int} 형이라면 \emph{gcc} 는 {\code y} 를 오프셋 2 에 놓을 것이다. 그러나,
{\code z} 는 오프셋 8 에 놓이게 되는데 왜냐하면 double 형들은 기본값이 더블워드 정렬이기 때문이다.
물론 {\code z} 를 오프셋 6 에 놓이게 할 수 있도록 변경할 수 있다. 

\begin{figure}[t]
\begin{lstlisting}[frame=tlrb,stepnumber=0,escapeinside=~~]{}
struct S {
  short int x;    /* ~2 바이트 정수~ */
  int       y;    /* ~4 바이트 정수~ */
  double    z;    /* ~8 바이트 float ~*/
} __attribute__((packed));
\end{lstlisting}
\caption{압축된 \emph{gcc} 에서의 구조체 사용 \label{fig:packedStruct}}\index{컴파일러!gcc!\_\_attribute\_\_}
\end{figure}

\emph{gcc} 컴파일러는 또한 구조체를 \emph{압축(pack)} 하는 것을 지원한다.
이는 컴파일러로 하여금 구조체 생성시 최소한의 공간만을 사용하게 한다. 그림 ~\ref{fig:packedStruct} 는
{\code S} 가 이 방법으로 어떻게 다시 쓰일 수 있는지 보여준다. 이러한 형태의 {\code S} 는 최소한의 바이트를
사용하여, 이 경우 14 바이트를 사용한다. 

마이크로소프트와 볼랜드 사의 컴파일러는 {\code \#pragma} 지시어를 통해 위와 동일한 작업을 
할 수 있다. \index{컴파일러!마이크로소프트!pragma pack}
\begin{lstlisting}[stepnumber=0]{}
#pragma pack(1)
\end{lstlisting}
이 지시어는 컴파일러가 바이트 경계에 맞추어 구조체의 원소들을 압축하게 한다. (\emph{i.e} 
추가적인 사용 없이) 1 은 2, 4, 8, 16 등으로 변경될 수 있는데 이는 각각 워드, 더블워드, 쿼드워드,
패러그래프 경계에 맞추라는 뜻이다. 이 지시어는 다른 지시어의 의해 변경되기 전 까지 계속 효력을
발휘한다. 이 때문에 이와 같은 지시어가 만이 사용되는 헤더 파일에서는 문제가 발생하기도 한다. 
만일 이 헤더 파일이 다른 구조체를 포함한 헤더파일 보다 먼저 포함(include) 되었다면 이 구조체는
기본값과 다른 방식으로 메모리 상에 배열 될 수 있다. 이와 같은 오류는 찾기가 매우 힘들다. 프로그램의
각기 다른 모듈들이 구조체의 원소들을 각기 \emph{다른} 장소에 배열하게 된다. 

\begin{figure}[t]
\begin{lstlisting}[frame=tlrb,stepnumber=0,escapeinside=~~]{}
#pragma pack(push)    /* ~이전 정렬 상태를 저장한다~*/
#pragma pack(1)       /* ~바이트 기준 정렬로 정의~ */

struct S {
  short int x;    /* ~2 바이트 정수~ */
  int       y;    /* ~4 바이트 정수~ */
  double    z;    /* ~8 바이트 float ~*/
};

#pragma pack(pop)     /* ~이전의 정렬 상태를 복원한다~*/
\end{lstlisting}
\caption{마이크로소프트나 볼랜드 사의 압축된 구조체 \label{fig:msPacked}\index{컴파일러!마이크로소프트!pragma pack}}
\end{figure}

이와 같은 문제를 피하는 방법이 있다. 마이크로소프트와 볼랜드 사는 현재의
정렬 방법을 저장한 후 나중에 복원할 수 있게 한다. 그림 ~\ref{fig:msPacked} 는 이를 
나타낸다. 

\index{구조체!배열하기|)}

\subsection{비트필드\index{구조체!비트필드|(}}

\begin{figure}[t]
\begin{lstlisting}[frame=tlrb,stepnumber=0,escapeinside=~~]{}
struct S {
  unsigned f1 : 3;   /*~3 비트필드~ */
  unsigned f2 : 10;  /*~10 비트필드~*/
  unsigned f3 : 11;  /* ~11 비트필드~ */
  unsigned f4 : 8;   /* ~8 비트필드~  */
};
\end{lstlisting}
\caption{비트필드 예제 \label{fig:bitStruct}}
\end{figure}

비트필드는 구조체의 멤버들로 하여금 특정한 크기의 비트만 사용할 수 있게 한다.
비트의 크기가 꼭 8 의 배수가 되야할 필요는 없다. 비트필드 멤버는 
\lstinline|unsigned int| 나 \lstinline|int| 멤버에 콜론(:)와 비트 크기를 적음으로써 정의된다. 
그림 ~\ref{fig:bitStruct} 는 이 예를 보여준다. 이는 32 비트 변수를 다음과 같은 부분으로
나뉘어 정의한다. 

\begin{center}
\begin{tabular}{|c|c|c|c|}
\multicolumn{1}{c}{8 비트} & \multicolumn{1}{c}{11 비트} 
& \multicolumn{1}{c}{10 비트} & \multicolumn{1}{c}{3 비트} \\ \hline
\hspace{2em} f4 \hspace{2em} & \hspace{3em} f3 \hspace{3em}
& \hspace{3em} f2 \hspace{3em} & f1 \\
\hline
\end{tabular}
\end{center}
첫 번째 비트필드는 더블워드의 최하위 비트들을 가리킨다. 
\footnote{사실 ANSI/ISO C 표준은 컴파일러로 하여금
비트들이 어떻게 나열되는지는 정확히 정하지 않았다. 그러나 대부분의 C 컴파일러들 (\emph{gcc},
\emph{마이크로소프트}, \emph{볼랜드} 사의 컴파일러) 들은 이와 같은 모습으로 배열한다.}

그러나 우리가 이 비트들이 실제로 메모리 상에 어떻게 나타나는지 본다면 이 형식은 그리 간단한 것이
아니다. 이 문제는 비트필드가 바이트 경계에 걸쳐 있을 때 나타난다. 왜냐하면 리틀 엔디안 형식에서
바이트는 메모리에 거꾸로 저장되기 때문이다. 예를 들어 {\code S} 구조체의 비트필드는 메모리에
아래와 같이 나타난다. 

\begin{center}
\begin{tabular}{|c|c||c|c||c||c|}
\multicolumn{1}{c}{5 비트} & \multicolumn{1}{c}{3 비트} 
& \multicolumn{1}{c}{3 비트} & \multicolumn{1}{c}{5 비트} 
& \multicolumn{1}{c}{8 비트} & \multicolumn{1}{c}{8 비트} \\ \hline
f2l & f1 &  f3l  & f2m & \hspace{1em} f3m \hspace{1em} 
& \hspace{1.5em} f4 \hspace{1.5em} \\
\hline
\end{tabular}
\end{center}
\emph{f2l} 라벨은 \emph{f2} 비트필드의 하위 5 비트를 의미한다 (\emph{i.e} LSB) .
\emph{f2m} 라벨은 \emph{f2} 비트필드의 상위 5 비트를 가리킨다. 수직선은 바이트 경계를
표시한다. 만일 모든 바이트들을 거꾸로 배치한다면 \emph{f2} 와 \emph{f3} 의 조각들이
올바른 위치에 합쳐질 것이다. 

\begin{figure}[t]
\centering
\begin{tabular}{|c*{8}{|p{1.3em}}|}
\hline
바이트 $\backslash$ 비트 & 7 & 6 & 5 & 4 & 3 & 2 & 1 & 0 \\ \hline
0 & \multicolumn{8}{c|}{연산 부호 (08h) } \\ \hline
1 & \multicolumn{3}{c|}{논리 단위 \# } & \multicolumn{5}{c|}{논리 블록 주소의 최상위 비트} \\ \hline
2 & \multicolumn{8}{c|}{논리 블록 주소의 중간} \\ \hline
3 & \multicolumn{8}{c|}{논리 블록 주소의 최하위 바이트} \\ \hline
4 & \multicolumn{8}{c|}{전달 크기} \\ \hline
5 & \multicolumn{8}{c|}{제어} \\ \hline
\end{tabular}
\caption{SCSI 읽기 명령 형식\label{fig:scsi-read}}
\end{figure}

\begin{figure}[t]
\begin{lstlisting}[frame=lrtb,escapeinside=~~]{}
#define MS_OR_BORLAND (defined(__BORLANDC__) \
                        || defined(_MSC_VER))

#if MS_OR_BORLAND
#  pragma pack(push)
#  pragma pack(1)
#endif

struct SCSI_read_cmd {
  unsigned opcode : 8;
  unsigned lba_msb : 5;
  unsigned logical_unit : 3;
  unsigned lba_mid : 8;    /* ~중간 비트들~*/
  unsigned lba_lsb : 8;
  unsigned transfer_length : 8;
  unsigned control : 8;
}
#if defined(__GNUC__)
   __attribute__((packed))
#endif
;

#if MS_OR_BORLAND
#  pragma pack(pop)
#endif
\end{lstlisting}
\caption{SCSI 읽기 명령 형식 구조체\label{fig:scsi-read-struct}\index{컴파일러!gcc!\_\_attribute\_\_}
         \index{컴파일러!마이크로소프트!pragma pack}}
\end{figure}

메모리에서의 배열 형태는 데이터가 프로그램 바깥으로 전송되지 않는 한 중요하지 않다.
(사실 이러한 경우는 비트필드를 다룰 때 잘 발생한다) 하드웨어 장치 인터페이스를 만들 때,
홀수의 비트를 가지는 비트필드를 사용하는 경우가 잦은데 이는 비트필드를 사용하면 매우 유용하기 때문이다.

\begin{figure}[t]
\centering
\begin{tabular}{|c||c||c||c||c|c||c|}
\multicolumn{1}{c}{8 비트} & \multicolumn{1}{c}{8 bits} 
& \multicolumn{1}{c}{8 비트} & \multicolumn{1}{c}{8 bits} 
& \multicolumn{1}{c}{3 비트} & \multicolumn{1}{c}{5 bits} 
& \multicolumn{1}{c}{8 비트} \\ \hline
제어 & 전달크기 & lba\_lsb  & lba\_mid &  
논리 단위  & lba\_msb & 연산 부호 \\
\hline
\end{tabular}
\caption{{\code SCSI\_read\_cmd} 필드의 모습\label{fig:scsi-read-map}}
\end{figure}
\index{SCSI|(}
그 예로는 SCSI \footnote{소형 컴퓨터 시스템 인터페이스 (Small Computer Systems Interface) 의 약어로
하드 디스크 \emph{등등}의 장치에 대한 규격이다.} 를 들 수 있다. SCSI 장치에 직접적으로 읽기 명령을 내리는
방법은 그림 ~\ref{fig:scsi-read} 에 나타난 형식 대로 6 바이트의 메세지를 장치에 전달하면 된다. 
이를 비트 필들르 이용하여 표현하는데 어려움은 \emph{논리 블록 주소(logical block address)}에 있는데, 왜냐하면
이것이 3 개의 다른 바이트에 걸쳐 있기 때문이다. 그림 ~\ref{fig:scsi-read} 에는 빅 엔디안 형식으로 저장된 데이터를
볼 수 있다. 그림 ~\ref{fig:scsi-read-struct} 에는 모든 컴파일러에 대해서 사용 가능하게한 정의를 보여준다. 첫 2 행은 매크로를
정의하는데 이는 마이크로소프트나 볼랜드 사의 컴파일러로 컴파일 될 때 참이 된다. 아마 혼동되는 부분은
11 에서 14 행 까지 일 것이다. 먼저 우리는 \lstinline|lba_mid| 와 \lstinline|lba_lsb| 필드가 하나의 16 비트필드로
정의되지 않고 따로 정의됬는지 의문을 가질 것이다. 이는 데이타가 빅 엔디안 순서로 되어 있기 때문이다. 16 비트 
필드는 컴파일러에 의해 리틀 엔디안 순서로 저장될 것이다. 두 번째로 \lstinline|lba_msb| 와 \lstinline|logical_unit| 는 거꾸로
된 것처럼 보인다. 그러나 이 경우엔 적용되지 않는다. 이들은 이 순서대로 집어너져야 한다. 그림 ~\ref{fig:scsi-read-map} 은
48 비트필드가 메모리에 어떻게 표현되는지 보여준다. (바이트 경계는 두 개의 선으로 표현되었다.) 이것이 메모리에 리틀 엔디안 순서로 저장된다면
비트들은 원하는 형식으로 배열될 것이다. (그림~\ref{fig:scsi-read}).

\begin{figure}[t]
\begin{lstlisting}[frame=lrtb,escapeinside=~~]{}
struct SCSI_read_cmd {
  unsigned char opcode;
  unsigned char lba_msb : 5;  
  unsigned char logical_unit : 3;
  unsigned char lba_mid;    /*~중간 비트들~ */
  unsigned char lba_lsb;
  unsigned char transfer_length;
  unsigned char control;
}
#if defined(__GNUC__)
   __attribute__((packed))
#endif
;
\end{lstlisting}
\caption{다른 형태의 SCSI 읽기 명령 형식 구조체 \label{fig:scsi-read-struct2}
         \index{컴파일러!gcc!\_\_attribute\_\_}\index{컴파일러!마이크로소프트!pragma pack}}
\end{figure}

일을 더 복잡하게 하는 것은, \lstinline|SCSI_read_cmd| 은 마이크로소프트 C 에서 완벽하게 작동하지 않는다는 점이다. 
\lstinline|sizeof(SCSI_read_cmd)| 라는 식을 계산하였을 때, 마이크로소프트 C 에서는
6 이 아닌 8 을 리턴한다. 왜냐하면 마이크로소프트 컴파일러는 비트를 메모리 나타낼 때
비트필드의 형을 기준으로 나타내기 때문이다. 모든 비트필드들이 \lstinline|unsigned| 로
정의되어 있으므로 컴파일러는 구조체 맨 마지막에 2 바이트를 덧붙여서 더블워드의 배수로
만들어 준다. 이와 같은 현상은 모든 필드를 \lstinline|unsigned short| 로 정의해 줌으로써
해결할 수 있다. 그렇게 한다면 마이크로소프트는 6 바이트가 2 바이트 워드의 정수배 이므로
부수적인 바이트를 추가할 필요가 없어지게 된다. 
\footnote{여러가지 다른 형의 비트필드를 사용하는 것은 매우 혼란스러운 .} 
다른 컴파일러는 이와 같이 바꾸어도 정확하게 동작한다. 그림 ~\ref{fig:scsi-read-struct2} 는 
3 개 컴파일러 모두에서 잘 작동하는 정의를 보여준다. \lstinline|unsigned char| 를 이용하여
오직 두 개만 비트필드를 이용했다. 

\index{SCSI|)}

이전에 우리가 이야기 하였던 것이 혼란스러워도 걱정할 필요가 없다. 왜냐하면 정말로
혼란스러운 내용이니까 말이다. 저자는 덜 혼란스럽기 위해서 비트필드를 사용하지 않고
비트를 각각 다루기 위해서 비트 연산을 사용한다. 

\index{구조체!비트필드|)}

%TODO:discuss alignment issues and struct size issues

\subsection{어셈블리에서 구조체 사용하기}

위에서 이야기 하였듯이 어셈블리에서 구조체를 사용하는 것은 배열을 사용하는 것과
비슷한다. 간단한 예로 구조체 {\code S} 의 원소 {\code y} 를 0 으로 만드는 루틴을
어떻게 만들지 생각해 보자. 이 루틴의 원형은 아래과 같다. 

\begin{lstlisting}[stepnumber=0]{}
void zero_y( S * s_p );
\end{lstlisting}
\noindent 어셈블리 루틴은 
\begin{AsmCodeListing}
%define      y_offset  4
_zero_y:
      enter  0,0
      mov    eax, [ebp + 8]      ; 스택에서 s_p(구조체 포인터) 를 불러온다
      mov    dword [eax + y_offset], 0
      leave
      ret
\end{AsmCodeListing}

C 는 함수에게 구조체를 값으로 전달하는 것을 허용한다. 그러나 이는 좋지 않은 생각이다. 
값으로 전달을 할 경우, 구조체의 모든 데이터들이 스택에 복사된 후 루틴에게 전달될 것이다.
단순히 구조체를 가리키는 포인터를 전달하는 것이 더 효과적이라 할 수 있다. 

C 는 또한 구조체 타입이 함수의 리턴값으로도 사용할 수 있게 한다. 하지만 명백히도, 구조체는
{\code EAX} 레지스터에 저장되어 리턴될 수 없다. 이 상황을 여러 컴파일러들이 각기 다른 방법으로 해결 한다.
가장 많이 쓰이는 방법은 컴파일러가 내부적으로 함수를 구조체를 가리키는 포인터를 인자로 갖게
변경하는 것이다. 이 포인터는 루틴이 호출된 곳 밖에서 정의된 구조체에 반환값을 대입하는데 사용된다. 

대부분의 어셈블러 (NASM 포함) 은 여러분의 어셈블리 코드에서 구조체를 정의하는 것을 지원하는
자체적인 방법이 있다. 자세한 내용은 이에 관련한 문서들을 참조하여라. 
% add section on structure return values for functions

\index{구조체|)}

\section{어셈블리와 C++\index{C++|(}}

C++ 프로그래밍 언어는 C 언어의 확장이라 볼 수 있다. 어셈블리와 C 가 서로
작업하는 방법 대부분이 C++ 에서도 동일하게 적용된다. 그러나, 몇 가지 규칙들은
수정되야 한다. 또한, C++ 의 몇 가지 확장된 부분들은 어셈블리로 이해하기가 편하다. 
이 단원은 독자 여러분이 C++ 의 기본적인 이해가 되어있다고 생각한다. 

\subsection{오버로딩과 이름 맹글링(Name Mangling)\index{C++!name mangling|(}}
\label{subsec:mangling}
\begin{figure}
\centering
\begin{lstlisting}[frame=tlrb]{}
#include <stdio.h>

void f( int x )
{
  printf("%d\n", x);
}

void f( double x )
{
  printf("%g\n", x);
}
\end{lstlisting}
\caption{두 개의 {\code f()} 함수 \label{fig:twof}}
\end{figure}

C++ 은 같은 이름으로 서로 다른 함수 (그리고 클래스 멤버 함수)를 정의하는 것을 허용한다.
2 개 이상의 함수가 동일한 이름을 사용한다면, 이 함수들을 \emph{오버로드(overloaded)} 되었다
라고 한다. 만일 이 두 함수들이 C 에서 같은 이름으로 정의되었다면 C 링커는 오류를 출력할 것이다. 왜냐하면
링킹시 오브젝트 파일에서 동일한 심볼에 두 개의 정의를 찾을 것이기 때문이다. 예를 들어 그림 ~\ref{fig:twof} 의
코드를 고려해 보자. 어셈블리에서 두 개의 {\code \_f} 라 이름 붙여진 라벨은 명백히 오류를 출력한다. 

C++ 은 C 와 동일한 링킹 과정을 거치지만, 함수에 이름 붙여진 라벨을 \emph{이름 맹글링(name mangling)}이라는
과정을 통해 오류를 피해갈 수 있다. 사실 C 도 이름 맹글링을 사용한다. C 에서 함수에 대한 라벨을 정의하기 위해서
함수에 이름에 \_ 를 붙인다. 그러나 C 는 그림 ~\ref{fig:twof} 의 두 함수의 이름을 동일한 라벨로 변경해
오류를 출력할 것이다. C++ 는 이름 맹글링시 조금 더 복잡한 방법을 사용하여 이 두 함수에 대한 서로 다른 라벨을
만들어 준다. 예를 들어 그림 ~\ref{fig:twof} 의 첫 번째 함수는 DJGPP  에 의해 {\code \_f\_\_Fi} 라는 라벨을 가지고
두 번째 함수는 {\code \_f\_\_Fd} 라는 라벨을 부여받게 된다. 이를 통해 링크 오류를 피해나갈 수 있다. 

% check to make sure that DJGPP does still but an _ at beginning for C++

불행히도 C++ 에서 라벨을 어떻게 만드는지에 대한 기준이 없기 때문에 각기 다른 컴파일러들이 자신들 만의 방법으로
이름 맹글링을 한다. 예를 들어 볼랜드 C++ 은 그림 ~\ref{fig:twof} 의 두 함수에 대해 {\code @f\$qi} 와 {\code @f\$qd}
라는 라벨을 부여하게 된다. 그러나 이러한 규칙들은 완전히 무작위적이지는 않다. 맹글링 된 이름은 함수의
\emph{특징}을 나타낸다. 여기서 이 특징은 함수의 인자들의 배열 순서와 형에 의해 정해진다. 예를 들어 {\code int} 인자만을
가지는 함수는 맹글링 된 이름 끝 부분에 \emph{i} 가 붙게 된다. (이는 볼랜드와 DJGPP 모두 동일) 또한 {\code double} 인자를
가지는 함수는 끝에 \emph{d} 가 붙게 된다. 만일 아래와 같은 원형을 가지는 함수 {\code f} 가 있다면, 

\begin{lstlisting}[stepnumber=0]{}
  void f( int x, int y, double z);
\end{lstlisting}
\noindent DJGPP 는 이 함수의 이름을 {\code \_f\_\_Fiid} 로 맹글링 하고 볼랜드는 {\code @f\$qiid} 로 한다.

함수의 리턴 형식은 함수의 특징에 포함되지 \emph{않으며}, 맹글링 된 이름에 나타나지도 않는다. 
이를 통해 우리는 C++ 의 오버로딩 규칙에 대해 알 수 있다. 오직 고유의 특징을 가지는 함수많이 오버로드 될 수 있다.
만일 C++ 에서 동일한 이름과 특징을 가지는 함수가 정의된다면, 이는 동일한 맹글링된 이름을 가지게 되므로
링커 애러를 발생하게 된다. 기본적으로 C++ 함수들은 오버로드 되지 않아도 이름이 맹글링 된다. 왜냐하면
컴파일러는 컴파일 시에 이 함수가 오버로드 됐는지 안됐는지 알 수 있는 방법이 없기 때문이다.  사실 함수의 이름을
맹글링하는 방법과 비슷한 방법으로 전역변수의 이름을 맹글링한다. 따라서, 만일 어떤 파일에 특정한 형을 가지는
전역 변수를 정의하였는데 다른 파일에서 다른 형으로 사용한다면 링커 오류가 발생하게 된다. 이러한 C++ 의 특징은 
\emph{형이 안전한 링킹(typesafe linking)} 이라 불린다. \index{C++!형이 안전한 링킹} 이를 통해 일정하지 않은 원형에
대한 오류까지도 잡아낼 수 있다. 이는 특정한 모듈에서의 함수의 정의가 다른 모듈에서의 원형과 일치하지 않을 때 발생한다. 
C 에서는 이를 디버그 하기 매우 어렵다. 왜냐하면 C 는 이 오류를 잡아내지 못하기 때문이다. 프로그램은 오류없이 컴파일되고
링크되지만 함수를 호출시 함수가 예상했던 값들과 다른 값들을 스택에 넣어 프로그램이 동작을 하지 못한다. 하지만 C++ 에서는
이는 링커 오류를 발생해 미연에 이러한 일을 방지할 수 있다. 

C++ 컴파일러가 함수 호출을 분석(parsing)할 때, 함수에 전달되는 인자들의 형을 살펴 보아서 일치하는 함수가 존재하는지
살펴본다. \footnote{이 형은 정확히 일치할 필요가 없다. 컴파일러는 인자들을 형변환 것들도 고려해 보기 때문이다. 이 
과정에 대한 규칙은 이 책의 범위를 훨씬 뛰어 넘는다. 자세한 내용은 C++ 책들을 참조해라.} 만일 일치하는 함수를 찾으면 
컴파일러는 이름 맹글링 규칙을 사용하여 올바른 함수에 대한 {\code CALL} 명령을 수행한다. 

각기 다른 컴파일러가 각기 다른 맹글링 규칙을 사용하므로 서로 다른 컴파일러에서 컴파일된 C++ 코드는 링크가
되지 않을 수 있다. 이 사실은 우리가 미리 컴파일된 C++ 라이브러리를 사용할 때 중요하게 작용할 수 있다. 만일
우리가 C++ 코드와 함께 사용될 수 있는 함수를 어셈블리에서 작성한다고 한다면 C++ 코드를 컴파일하는데 사용되는
컴파일러를 보고 이름 맹글링 규칙을 적용시켜야 한다. (또는 아래에 나온 기술을 이용한다) 

아마 눈치가 빠른 학생들은 그림 ~\ref{fig:twof} 의 코드가 예상했던 대로 작동하는 것인지 물어볼 것이다. C++ 는
모든 함수의 이름을 맹글링하기에 {\code printf} 함수도 맹글링 되므로 컴파일러는 라벨 {\code \_printf} 에 올바른
{\code CALL} 을 할 수 없다는 것이다. 이는 올바른 생각이다. {\code printf} 의 원형이 단순히 파일 상단에 놓여 있다면
이와 같은 일이 발생할 것이다. 그 원형은:

\begin{lstlisting}[stepnumber=0]{}
  int printf( const char *, ...);
\end{lstlisting}
\noindent DJGPP 는 이를 {\code \_printf\_\_FPCce} 로 맹글링할 것이다.
({\code F} 는 \emph{함수(function)}를, {\code P} 는 \emph{포인터(pointer)},
{\code C} 는 \emph{상수(const)}, {\code c} 는 \emph{char}, {\code e} 는 \emph{생략(ellipsis)}
을 각각 뜻한다) 이는 보통의 C 라이브러리의 {\code printf} 함수를 호출하지 않을 것이다!
당연하게도 C++ 코드에서 C 함수를 호출하는 방법이 있다. 이는 매우 유용한데 왜냐하면 
\emph{엄청난 수의} 유용한 C 코드들이 있기 때문이다. 게다가 C 코드를 호출하는 것을 할 수 있다면
C++ 에서 또한 보통의 C 맹글링 규약을 이용한 어셈블리 코드도 호출할 수 있게 된다. 

\index{C++!extern ""C""|(}
C++ 에선 {\code extern} 키워드를 이용하여 특정한 함수와 전역변수에 C 맹글링 규칙을 적용할 수 있다. 
C++ 용어로 이렇게 선언된 함수와 변수는 \emph{C 연계(C linkage)}를 사용한다고 말한다. 예를 들어, 
{\code printf} 를 C 연계시키려면 아래와 같이 원형을 쓰면 된다.

\begin{lstlisting}[language=C++,stepnumber=0]{}
extern "C" int printf( const char *, ... );
\end{lstlisting}
\noindent 이 명령은 컴파일러로 하여금 이 함수에 대해 C++ 이름 맹글링을 
적용하지 않으며, 그 대신에 C 규칙에 따라 맹글링을 하게 한다. 그러나, 
위와 같이 정의된 {\code printf} 는 오버로드 되지 않는다. 이와 같은 방법을 통해 
C++ 와 어셈블리가 손쉽게 소통할 수 있다. 함수를 C 연계를 사용하게 정의한 뒤, C 호출
규약을 사용하는 것이다.

편의를 위해, C++ 은 또한 여러개의 함수들과 전역 변수를 한꺼번에 연계시킬 수 있게 해준다.
이는 중괄호를 사용하면 된다. 

\begin{lstlisting}[stepnumber=0,language=C++,escapeinside=~~]{}
extern "C" {
  /* ~C 연계 전역 변수들과 함수 원형들~*/
}
\end{lstlisting}

만일, 여러분이 C/C++ 컴파일러에서 모두 작동되는 ANSI C 헤더를 보게 된다면,
헤더 파일 상단에 다음과 같은 코드가 있음을 알 수 있다.

\begin{lstlisting}[stepnumber=0,language=C++]{}
#ifdef __cplusplus
extern "C" {
#endif
\end{lstlisting}
\noindent 그리고 하단에 중괄호로 닫는 부분이 있음을 알 수 있다. 
C++ 컴파일러들은 {\code \_\_cplusplus} 를 매크로 (\_ 가 \emph{두}
개) 를 정의한다. 위 코드는 헤더파일이 C++ 에서 컴파일 될 경우 헤더파일
전체를 {\code extern~"C"} 로 감싸고, C 에서 컴파일 된다면 어떠한 짓도 하지
않는다. (이는 C 컴파일러를 {\code extern~"C"} 를 문법상의 오류로 
처리하기 때문에 필요하다). 위와 같은 방법으로 C 와 C++ 모두에서
돌아가는 어셈블리 루틴을 위한 헤더파일도 작성할 수 있다. 

\index{C++!extern ""C""|)}
\index{C++!이름 맹글링|)}

\begin{figure}
\begin{lstlisting}[language=C++,frame=tlrb,escapeinside=~~]{}
void f( int & x )     // ~ \&는 레퍼런스 인자를 의미한다~
{ x++; }

int main()
{
  int y = 5;
  f(y);    // ~y 에 대한 레퍼런스가 전달되었다. \& 가 없음을 유의!~
  printf("%d\n", y);  // ~ 6 을 출력 ~
  return 0;
}
\end{lstlisting}
\caption{레퍼런스(Reference) 예제\label{fig:refex}}
\end{figure}

\subsection{레퍼런스\index{C++!레퍼런스|(}}

\emph{레퍼런스(Reference)} 는 C++ 의 또다른 새 기능이다. 이는 함수의 인자들을
포인터를 직접적으로 사용하지 않고도 전달될 수 있게 한다. 예를 들어 그림 ~\ref{fig:refex} 의
코드를 고려해 보자. 사실, 레퍼런스 인자들은 정말로 단순하다. 이들은 정말 포인터 같다. 
컴파일러는 단순히 이를 프로그래머에게 숨기는 것이다. (이는 파스칼 컴파일러가 {\code var} 인자들을
포인터로 실행하는 것처럼) 컴파일러가 함수를 호출하는 ~7 행에서의 어셈블리 코드를 생성할 때,
이는 {\code y} 의 \emph{주소}를 전달한다. 만일 {\code f} 를 어셈블리로 작성한다면 그것이
함수의 원형이였던 것처럼 \footnote{물론, ~\ref{subsec:mangling} 에서 이야기 했던 것처럼 이름 맹글링을 피하기 위해
C 연계로 작성해야 한다. }

\begin{lstlisting}[stepnumber=0]{}
  void f( int * xp);
\end{lstlisting}

레퍼런스는 단순히 편의를 위해서 만들어 졌는데, 특히 오버로딩 연산자에
편리하다. 이는 C++ 의 또다른 기능으로 구조체나 클래스 형에 대한 평범한
연산자에 의미를 부여하는 것이다. 예를 들어 더하기 ({\code +}) 기호는
보통 문자열 객체를 합칠 때 사용한다. 따라서, 만일 {\code a} 와 {\code b} 가
문자열이라면 {\code a~+~b} 는 문자열 {\code a} 와 {\code b} 가 합친
결과가 리턴될 것이다. C++ 은 사실 이 작업을 하기 위해 함수를 호출한다. 
(사실, 이러한 식들은 {\code operator~+(a,b)} 와 같은 함수를 이용해 나타낼 수 있다) 
편의를 위해, 문자열 객체의 값을 전달하기 보단 주소를 전달하는 것이 좋을 것이다. 
레퍼런스를 사용하지 않는다면 이는 {\code operator~+(\&a, \&b)} 로 가능하지만, 
연산자 사용시 {\code \&a~+~\&b} 와 같이 나타내야 하는 불편함이 있다. 그러나
레퍼런스를 이용하면 이를 단순히 {\code a~+~b} 만으로 가능하고, 자연스러워 보인다. 

\index{C++!references|)}

\subsection{인라인 함수 \index{C++!인라인 함수|(}}

\emph{인라인 함수(Inline functions)} 는 C++ 의 또다른 기능이다.
\footnote{C 컴파일러들은 종종 이 기능을 ANSI C 의 확장 형태로 지원한다} 
인라인 함수들은 오류가 잦은, 인자를 갖는 전처리기 기반 매크로들을 대체하기
위하여 만들어 졌다. C 에서 숫자를 제곱하는 메크로는 아래와 같이 생겼다. 

\begin{lstlisting}[stepnumber=0]{}
#define SQR(x) ((x)*(x))
\end{lstlisting}
\noindent 
전처리기는 C 를 이해하지 못하고, 단순한 치환만을 하기 때문에 대부분의 경우에
올바른 값을 얻기 위해서 소괄호로 묶어야 한다. 그러나, 이와 같이 함에도 불구하고
{\code SQR(x++)} 와 같은 식에서는 올바른 값을 얻을 수 없다. 

\begin{figure}
\begin{lstlisting}[language=C++,frame=tlrb]{}
inline int inline_f( int x ) 
{ return x*x; }

int f( int x ) 
{ return x*x; }

int main()
{
  int y, x = 5;
  y = f(x);
  y = inline_f(x);
  return 0;
}
\end{lstlisting}
\caption{인라인 예제 \label{fig:InlineFun}}
\end{figure}

매크로들은 단순한 함수의 호출에서 복잡한 함수 호출과정을 만드는 것을
생략하기 위해 만들어진다. 서브프로그램에 관한 장에서 다루었듯이, 함수를
호출하는 과정은 몇 개의 단계를 거친다. 매우 단순한 함수를 호출 할 때에는
함수를 호출하는 과정이 함수를 실행하는 과정보다 더 오래 걸릴 수도 있다!
인라인 함수는 보통의 함수를 작성하는 것과 매우 비슷하지만, 코드에서 이
함수를 {\code CALL} 하는 명령은 이루어 지지 \emph{않는다}. 그 대신에, 
인라인 함수의 호출에 관한 부분은 함수가 수행하는 코드로 대체된다. C++ 에선 
함수의 정의 앞에 {\code inline} 키워드를 붙임으로써 이 함수를 인라인으로
선언할 수 있다. 예를 들어 그림 ~\ref{fig:InlineFun} 에서 선언된 함수를
살펴보자. ~10 행에서의 {\code f} 함수의 호출은 보통의 함수 호출을 수행한다.
(어셈블리에서 {\code x} 가 {\code ebp-8} 에, {\code y} 가 {\code ebp-4} 에 있다고
생각하자) 

\begin{AsmCodeListing}
      push   dword [ebp-8]
      call   _f
      pop    ecx
      mov    [ebp-4], eax
\end{AsmCodeListing}
그러나 ~11 행에서 나타난 {\code inline\_f} 함수의 호출은 아래와 같다. 
\begin{AsmCodeListing}
      mov    eax, [ebp-8]
      imul   eax, eax
      mov    [ebp-4], eax
\end{AsmCodeListing}

위 경우 인라인시 두 가지 장점이 있다. 먼저, 인라인 함수는 더 빠르다. 어떠한 인자도
스택에 푸시되지 않으며, 어떠한 스택 프레임도 생성되거나 파괴되지 않고, 어떠한 분기도
이루어 지지 않는다. 두 번째로 인라인 함수는 코드를 적게 사용한다. 물론, 위 경우 이것이
사실이였지만 모든 경우에서 사실인 것은 아니다. 

인라인 함수의 가장 큰 단점은 인라인 코드는 링크가 되지 않기 때문에 인라인 함수의 코드는 
이를 사용하는 \emph{모든} 파일에서 있어야 한다. 이전의 예제의 어셈블리 코드는 이를 보여준다.
인라인이 아닌 함수를 호출하는데 필요한 정보는 오직 인자, 리턴 값의 형, 호출 규약과 함수에 대한
라벨의 이름이다. 이 모든 정보는 함수의 원형에서 얻을 수 있다. 그러나, 인라인 함수는 함수의
모든 코드에 대한 정보를 필요로 하다. 이 말은 인라인 함수의 \emph{어떠한} 부분이라도 바뀐다면
이 함수를 사용하는 \emph{모든} 소스 파일들은 반드시 다시 컴파일 되야만 한다. 인라인이 아닌 함수들의
경우 함수의 원형이 바뀌지 않는 한, 이 함수를 사용하는 파일들은 굳이 다시 컴파일이 될 필요가 없다. 
이러한 이유 때문에 인라인 함수에 대한 코드는 보통 헤더파일에 위치한다. 이는 C  에서 실행 가능한 코드는
\emph{절대로} 헤더파일에 위치하면 안된다라는 규칙과 상충된다. 

\index{C++!인라인 함수|)}

\begin{figure}[t]
\begin{lstlisting}[language=C++,frame=tlrb,escapeinside=@@]{}
class Simple {
public:
  Simple();                // @디폴트 생성자@ 
  ~Simple();               // @소멸자@
  int get_data() const;    // @멤버 함수들@
  void set_data( int );
private:
  int data;                // @멤버 데이타@
};

Simple::Simple()
{ data = 0; }

Simple::~Simple()
{ /*@아무 것도 안함@*/ }

int Simple::get_data() const
{ return data; }

void Simple::set_data( int x )
{ data = x; }
\end{lstlisting}
\caption{단순한 C++ 클래스 \label{fig:SimpleClass}}
\end{figure}

\subsection{클래스\index{C++!클래스|(}}

C++ 클래스는 \emph{객체(object)}의 타입을 설명한다. C++ 에서 객체는 데이타 멤버와
함수 맴버\footnote{보통 \emph{맴버 함수(member function)} 이라 부른다.}, 혹은 일반적으로 
\emph{메서드(method)} \index{메서드} 를 포함할 수 있다. 다시말해, 이는 데이타와 함수가
연관이 되어 있는 \emph{구조체} 이다. 그림 ~\ref{fig:SimpleClass}에 정의된 단순한 클래스를
살펴보자. {\code Simple} 타입의 변수는 단순히 C 에서의 하나의 {\code int} 를 멤버로 가지는 
\emph{구조체} 와 같다. 
\MarginNote{사실 C++ 은 {\code this} 키워드를 이용하여 맴버 함수 내부에서
객체를 가리키는 포인터에 접근한다.} 이 함수들은 구조체에 해당하는 메모리에 저장되지 \emph{않는다}.
그러나 멤버 함수들은 다른 함수들과 다르다. 이 들은 \emph{숨겨진} 인자를 주고 받는다. 그 인자는
멤버 함수가 작업하는 객체를 가리키는 포인터 이다. 

\begin{figure}[t]
\begin{lstlisting}[stepnumber=0]{}
void set_data( Simple * object, int x )
{
  object->data = x;
}
\end{lstlisting}
\caption{C 버전의 Simple::set\_data()\label{fig:SimpleCVer}}
\end{figure}


\begin{figure}[t]
\begin{AsmCodeListing}
_set_data__6Simplei:           ; 맹글링 된 이름 
      push   ebp
      mov    ebp, esp

      mov    eax, [ebp + 8]   ; eax = 객체를 가리키는 포인터 (this)
      mov    edx, [ebp + 12]  ; edx = 정수 인자 
      mov    [eax], edx       ; 데이타는 오프셋 0 에 위치 

      leave
      ret
\end{AsmCodeListing}
\caption{Simple::set\_data( int ) 가 컴파일 된 모습 \label{fig:SimpleAsm}}
\end{figure}

예를 들어 그림 ~\ref{fig:SimpleClass} 의 {\code Simple} 클래스의 {\code set\_data} 메서드를
보자. 이것이 C 에서 작성된다면 그림 ~\ref{fig:SimpleCVer} 과 같이 함수의 코드에서 작업하는
객체에 대한 포인터를 인자로 받아들이는 함수가 될 것이다. \emph{DJGPP} 컴파일러 (그리고 
\emph{gcc} 와 볼랜드 컴파일러들도 동일)에 {\code -S} 스위치를 사용하면 컴파일러가 코드와
동일한 어셈블리 코드를 만들어 낸다. \emph{DJGPP} 와 \emph{gcc} 가 생성한 어셈블리 파일에는
{\code .s} 확장자가 붙는데, 불행이도 NASM 과 MASM 문법과 다른 AT\&T 어셈블리를 사용한다. 
\footnote{\emph{gcc} 컴파일러 시스템은 자체적인 어셈블러인 \emph{gas} \index{gas} 를 가지고 있다.
\emph{gas} 어셈블러는 AT\&T 문법을 사용하므로 컴파일러에 의해 생성된 어셈블리 코드는 \emph{gas} 형식
으로 되어 있다. 인터넷 상에서 INTEL 과 AT\&T 문법의 차이점을 서술한 페이지가 몇 개 있다. 
또한 {\code a2i} 라는 이름의 무료 프로그램 ({http://www.multimania.com/placr/a2i.html}) 을 통해
AT\&T 형식을 NASM 형식으로 변환할 수 있다.} (Borland 와 MS 컴파일러들은 {\code .asm} 확장자를 가지는
MASM 문법의 어셈블리 파일을 만든다) 그림 ~\ref{fig:SimpleAsm} 은 \emph{DJGPP} 의 어셈블리 코드를
NASM 문법으로 변환한 것이다. 문장의 의미를 파악하기 쉽게 하기 위해 일부 주석을 달았다. 가장
첫 번째 행에서 {\code set\_data} 메서드에 클래스의 이름과 인자에 대한 정보를 맹글링한 것으로 이루어진
라벨이 붙었음을 알 수 있다. 라벨에 클래스에 이름에 관한 정보도 포함된 이유는 다른 클래스가
{\code set\_data} 라는 이름의 메서드를 가질 수 있기 때문이다. 또한 인자들에 관한 정보도 평범한 C++ 함수들
처럼 오버로드 할 수 있도록 포함되었다. 그러나 이전과 같이 다른 컴파일러들은 이 정보를 다른 식으로
맹글링 할 것이다. 

그리고 ~2 와 3 행에서 우리에게 친숙한 부분이 나타난다. ~5 행에서 스택에 저장된 인자는 {\code EAX}로 저장된다.
이는 {\code x} 인자가 \emph{아니다}. 대신에 이는 숨겨진 인자 \footnote{어셈블리에서는 숨겨지는 것이 \emph{없다}!}로
작업하는 객체를 가리키는 포인터 이다. ~6 행에서, {\code x} 인자를 {\code EDX} 에 저장하고 ~7 행에서는 {\code EDX} 를 
{\code EAX} 가 가리키는 더블워드에 저장한다. 이는 우리가 현재 작업하는 {\code Simple} 의 객체의 {\code data} 멤버 이다. 
이 멤버는 클래스의 유일한 데이타로 {\code Simple} 의 오프셋 0 에 위치해 있다. 

\begin{figure}[tp]
\begin{lstlisting}[frame=tlrb,language=C++,escapeinside=@@]{}
class Big_int {
public:
   /* 
   * @인자@:
   *   size           - @정수의 크기로, unsigned int 가 @
   *                     @사용된 수로 나타낸다@
   *   initial_value  - @Big\_int 가 보통의 unsigned int 였을 때 초기값@
   */
  explicit Big_int( size_t   size,
                    unsigned initial_value = 0);
  /*
   * @인자@:
   *   size           - @정수의 크기로, unsigned int 가@
   *                     @사용된 수로 나타낸다@ 
   *   initial_value  - @Big\_int 의 초기값의 16 진수의 문자열 형태@
   */
  Big_int( size_t       size,
           const char * initial_value);

  Big_int( const Big_int & big_int_to_copy);
  ~Big_int();

  // @Big\_int 의 크기를 반환(unsigned int 의 수로)@
  size_t size() const;  

  const Big_int & operator = ( const Big_int & big_int_to_copy);
  friend Big_int operator + ( const Big_int & op1,
                              const Big_int & op2 );
  friend Big_int operator - ( const Big_int & op1,
                              const Big_int & op2);
  friend bool operator == ( const Big_int & op1,
                            const Big_int & op2 );
  friend bool operator < ( const Big_int & op1,
                           const Big_int & op2);
  friend ostream & operator << ( ostream &       os,
                                 const Big_int & op );
private:
  size_t      size_;    // @부호 없는 배열의 크기@
  unsigned *  number_;  //@값을 가지는 부호 없는 배열을 가리키는 포인터@
};
\end{lstlisting}
\caption{Big\_int 클래스의 정의\label{fig:BigIntClass}}
\end{figure}

\begin{figure}[tp]
\begin{lstlisting}[frame=tlrb,language=C++,escapeinside=~~]{}
//~어셈블리 루틴의 원형~
extern "C" {
  int add_big_ints( Big_int &       res, 
                    const Big_int & op1, 
                    const Big_int & op2);
  int sub_big_ints( Big_int &       res, 
                    const Big_int & op1, 
                    const Big_int & op2);
}

inline Big_int operator + ( const Big_int & op1, const Big_int & op2)
{
  Big_int result(op1.size());
  int res = add_big_ints(result, op1, op2);
  if (res == 1)
    throw Big_int::Overflow();
  if (res == 2)
    throw Big_int::Size_mismatch();
  return result;
}

inline Big_int operator - ( const Big_int & op1, const Big_int & op2)
{
  Big_int result(op1.size());
  int res = sub_big_ints(result, op1, op2);
  if (res == 1)
    throw Big_int::Overflow();
  if (res == 2)
    throw Big_int::Size_mismatch();
  return result;
}
\end{lstlisting}
\caption{Big\_int 클래스 산술 연산 코드\label{fig:BigIntAdd}}
\end{figure}

\subsubsection{예제}
\index{C++!Big\_int 예제|(}
이 단원은 이 장의 내용을 바탕으로 임의의 크기의 부호가 없는 정수를
나타낸는 C++ 클래스를 만든다. 정수의 크기가 임의로 정해지기 때문에 이는
부호가 없는 정수(더블워드)들의 배열에 저장될 것이다. 이 배열은 동적 할당을
이용하면 아무 크기로나 만들 수 있다. 더블워드는 역순으로 저장된다.
\footnote{왜냐하면 덧셈 연산이 배열의 처음 부분 부터 진행되기 
때문이다.} (\emph{i.e} 최하위 더블워드는 0 의 위치에 있다) 그림 ~\ref{fig:BigIntClass} 는
{\code Big\_int} 클래스의 정의를 보여준다. 
\footnote{이 예제의 완전한 코드에 대한 예제 소스 코드를 보라. 본문은 오직
코드의 일부분만 이야기 할 것이다. } {\code Big\_int} 의 크기는 데이타를 저장하는데 사용되는 
부호 없는 배열의 크기로 계산된다. 클래스의 {\code size\_} 데이타 멤버는 오프셋이 0 이고,
{\code number\_} 멤버는 오프셋이 4 이다. 

이 예제들을 단순화 하기 위해 같은 크기의 배열을 가지는 객체 인스턴스(instance) 만이 
더해지거나 빼질 수 있다. 

이 클래스는 3 개의 생성자를 가진다. 하나는 (~9 행) 보통의 부호 없는 정수를 이용하여
클래스 인스턴스를 초기화 하는 것이고 다른 하나는 (~18 행) 16 진수의 값을 보관하는 문자열을 이용하여
인스턴스를 초기화 한다. 마지막으로는 (~21 행) \emph{복사 생성자(copy constructor)} \index{C++!복사 생성자}
이다. 

여기에서는 덧셈과 뺄셈 연산자들이 어떻게 작동하는지에 대해 초점을 맞출 것인데,
왜냐하면 여기서 어셈블리가 사용되기 때문이다. 그림 ~\ref{fig:BigIntAdd} 은 
이 연산자들과 관계있는 헤더파일의 일부분을 보여주고 있다. 이는 어셈블리 루틴을
호출하기 위해 연산자들이 어떻게 설정되어 있는지 보여준다. 서로 다른 컴파일러가 
연산자 함수를 위해 다른 종류의 맹글링 규칙을 사용하기 때문에 인라인 연산자 함수들은
C 연계 어셈블리 루틴을 호출하게 설정되어 있다. 이는 보통의 직접적인 호출과 속도가
비슷하며, 다른 컴파일러에 의해 컴파일 되어도 잘 작동하게 할 수 있다. 이 기술은 또한
어셈블리에서 예외를 던지는(throw) 상황을 막을 수 있다. 

왜 여기에 어셈블리가 사용될까? 다중 정밀도 연산을 사용하기 위해, 캐리가 하나의
더블워드로 부터 다음 더블워드로 전달되야 한다는 사실을 상기해라. C++ (그리고 C)는
프로그래머로 하여금 CPU 의 캐리 플래스에 접근하는 것을 막고 있다. 이 덧셈을 수행하는
유일한 방법은 C++ 이 독립적으로 캐리 플래그를 다시 계산한 뒤에 그 다음 더블워드에
더하는 수 밖에 없다. 하지만 어셈블리에선 캐리 플래그에 마음대로 접근할 수 있으며,
{\code ADC} 명령을 통해 자동적으로 캐리 플래그를 더할 수 있으므로 효과적이다. 

너무 글이 길어지는 것을 막기 위해 오직 {\code add\_big\_ints} 어셈블리 루틴만
여기에서 다룰 것이다. 아래는 이 루틴을 위한 코드 이다. ({\code big\_math.asm} 에서 가져옴)

\begin{AsmCodeListing}[label=big\_math.asm]
segment .text
        global  add_big_ints, sub_big_ints
%define size_offset 0
%define number_offset 4

%define EXIT_OK 0
%define EXIT_OVERFLOW 1
%define EXIT_SIZE_MISMATCH 2

; 덧셈과 뺄셈 서브 루틴들을 위한 인자들
%define res ebp+8
%define op1 ebp+12
%define op2 ebp+16

add_big_ints:
        push    ebp
        mov     ebp, esp
        push    ebx
        push    esi
        push    edi
        ;
        ; op1 을 가리키기 위해 esi 를 설정 
        ; op2 를 가리키기 위한 edi 
        ; res 를 가리키기 위한 ebx
        mov     esi, [op1]
        mov     edi, [op2]
        mov     ebx, [res]
        ;
        ; 모든 Big_int 들이 같은 크기여야 함을 명심해라 
        ;
        mov     eax, [esi + size_offset]
        cmp     eax, [edi + size_offset]
        jne     sizes_not_equal                 ; op1.size_ != op2.size_
        cmp     eax, [ebx + size_offset]
        jne     sizes_not_equal                 ; op1.size_ != res.size_

        mov     ecx, eax                        ; ecx = Big_int 의 크기
        ;
        ; 레지스터를 이에 대응하는 배열을 가리키게 맞춘다. 
        ;      esi = op1.number_
        ;      edi = op2.number_
        ;      ebx = res.number_
        ;
        mov     ebx, [ebx + number_offset]
        mov     esi, [esi + number_offset]
        mov     edi, [edi + number_offset]
        
        clc                                     ; 캐리 플래그 초기화
        xor     edx, edx                        ; edx = 0
        ;
        ; addition loop
add_loop:
        mov     eax, [edi+4*edx]
        adc     eax, [esi+4*edx]
        mov     [ebx + 4*edx], eax
        inc     edx                             ; 캐리 플래그를 변경하지 말라 
        loop    add_loop

        jc      overflow
ok_done:
        xor     eax, eax                        ; 리턴값 = EXIT_OK
        jmp     done
overflow:
        mov     eax, EXIT_OVERFLOW
        jmp     done
sizes_not_equal:
        mov     eax, EXIT_SIZE_MISMATCH
done:
        pop     edi
        pop     esi
        pop     ebx
        leave
        ret
\end{AsmCodeListing}

위의 대부분의 코드는 독자 여러분이 이해하기에 쉽다. ~25 에서 27 행에서
함수를 통해 인자로 전달된 {\code Big\_int} 객체를 가리키는 포인터가 저장된 레지스터를  
저장한다. 레퍼런스는 단지 포인터에 불과하다는 것을 상기해라. ~31 에서 35 행은 세 객체
배열들의 크기가 같은지 확인한다. (데이타 멤버에 접근하기 위해 {\code size\_} 의 오프셋이 
포인터에 더해졌다.) ~44 에서 46 행에선 레지스터 값들을 조정하여, 
객체 자기 자신이 아닌 각 객체의 배열을 가리키게 하였다.
(여기서도 객체 포인터에 {\code number\_} 멤버의 오프셋이 더해졌다) 

\begin{figure}[tp]
\begin{lstlisting}[language=C++, frame=tlrb]{}
#include "big_int.hpp"
#include <iostream>
using namespace std;

int main()
{
  try {
    Big_int b(5,"8000000000000a00b");
    Big_int a(5,"80000000000010230");
    Big_int c = a + b;
    cout << a << " + " << b << " = " << c << endl;
    for( int i=0; i < 2; i++ ) {
      c = c + a;
      cout << "c = " << c << endl;
    }
    cout << "c-1 = " << c - Big_int(5,1) << endl;
    Big_int d(5, "12345678");
    cout << "d = " << d << endl;
    cout << "c == d " << (c == d) << endl;
    cout << "c > d " << (c > d) << endl;
  }
  catch( const char * str ) {
    cerr << "Caught: " << str << endl;
  }
  catch( Big_int::Overflow ) {
    cerr << "Overflow" << endl;
  }
  catch( Big_int::Size_mismatch ) {
    cerr << "Size mismatch" << endl;
  }
  return 0;
}
\end{lstlisting}
\caption{{\code Big\_int} 의 단순한 사용 \label{fig:BigIntEx}}
\end{figure}

~52 에서 57 행의 루프는 최하위 더블워드 순으로 배열에 저장된 정수들을 더한다.
이런 순서로 더하는 이유는 확장 정밀도 산술 연산 (~\ref{sec:ExtPrecArith} 참조) 때문이다.
~59 행은 오버플로우가 발생하였는지 확인한다. 오버플로우 시, 최상위 더블워드의
마지막 덧셈에 의해 캐리 플래그가 세트된다. 배열에 저장된 더블워드들이 리틀 엔디안 형식으로
되어 있기 때문에 루프는 배열의 처음 부분 부터 시작되어 배열의 끝에서 끝난다. 

그림 ~\ref{fig:BigIntEx} 는 {\code Big\_int} 를 사용하는 짧은 예제 클래스를 보여준다.
{\code Big\_int} 상수는 ~16 행과 같이 반드시 명확하게 정의되어야 한다. 이는 두 가지 연유에서 중요하다. 
먼저, 부호가 없는 정수를 {\code Big\_int} 로 변환하는 변환 생성자가 없다. 두 번째로는 
오직 같은 크기의 {\code Big\_int} 들만이 더해질 수 있다. 이는 변환을 복잡하게 하는데 왜냐하면
어떠한 크기로 변환해야 할 지 모르기 때문이다. 클래스를 좀 더 복잡하게 만든다면 크기가 서로 다른
것들도 더해질 수 있게 할 수 있다. 그러나 이를 여기에서 보임으로써 이 예제를 복잡하게
만들고 싶은 생각은 없다. (그러나, 독자 여러분이 이를 한 번 해보기를 권한다) 

\index{C++!Big\_int 예제|)}

\begin{figure}[tp]
\begin{lstlisting}[language=C++, frame=tlrb]{}
#include <cstddef>
#include <iostream>
using namespace std;

class A {
public:
  void __cdecl m() { cout << "A::m()" << endl; }
  int ad;
};

class B : public A {
public:
  void __cdecl m() { cout << "B::m()" << endl; }
  int bd;
};

void f( A * p )
{
  p->ad = 5;
  p->m();
}

int main()
{
  A a;
  B b;
  cout << "Size of a: " << sizeof(a)
       << " Offset of ad: " << offsetof(A,ad) << endl;
  cout << "Size of b: " << sizeof(b)
       << " Offset of ad: " << offsetof(B,ad)
       << " Offset of bd: " << offsetof(B,bd) << endl;
  f(&a);
  f(&b);
  return 0;
}
\end{lstlisting}
\caption{상속 예제 \label{fig:SimpInh}}
\end{figure}


\subsection{상속과 다형성\index{C++!상속|(}}


\begin{figure}[tp]
\begin{AsmCodeListing}
_f__FP1A:                       ; 맹글링 된 함수 이름 
      push   ebp
      mov    ebp, esp
      mov    eax, [ebp+8]       ; eax 는 객체를 가리킨다. 
      mov    dword [eax], 5     ; ad 를 위해 오프셋 0 을 사용 
      mov    eax, [ebp+8]       ; 객체의 주소를 A::m() 에 전달
      push   eax
      call   _m__1A             ; A::m() 의 맹글링 된 메서드 이름 
      add    esp, 4
      leave
      ret
\end{AsmCodeListing}
\caption{간단한 상속 예제를 위한 어셈블리 코드 \label{fig:FAsm1}}
\end{figure}

\emph{상속(Inheritance)} 을 통해 하나의 클래스를 다른 클래스에 데이터와 메서드를
상속 할 수 있다. 예를 들어 그림 ~\ref{fig:SimpInh} 의 코드를 보자. 이는 두 개의 클래스
{\code A} 와 {\code B} 가 나오는데, {\code B} 클래스가 {\code A} 로 부터 상속받는다.
프로그램의 출력 결과는 

\begin{verbatim}
Size of a: 4 Offset of ad: 0
Size of b: 8 Offset of ad: 0 Offset of bd: 4
A::m()
A::m()
\end{verbatim}
두 개의 클래스({\code A} 와 상속받은 {\code B})의 {\code ad} 데이타 멤버가 
동일한 오프셋을 가짐을 주목해라. 이는 {\code f} 함수에 {\code A} 객체 혹은
{\code A} 를 물려받은(\emph{i.e.} 상속받은) 클래스의 객체를 가리키는 포인터를 전달했을 수 도 있기 때문이다. (\emph{i.e} 상속된)
그림 ~\ref{fig:FAsm1} 은 함수에 대한 어셈블리 코드를 보여준다. (\emph{gcc} 에 의해 생성되었고, NASM 형식으로 변환됨)

\begin{figure}[tp]
\begin{lstlisting}[language=C++, frame=tlrb]{}
class A {
public:
  virtual void __cdecl m() { cout << "A::m()" << endl; }
  int ad;
};

class B : public A {
public:
  virtual void __cdecl m() { cout << "B::m()" << endl; }
  int bd;
};
\end{lstlisting}
\caption{다형적 상속(Polymorphic Inheritance)\label{fig:VirtInh}}
\end{figure}

\index{C++!다형성|(}
출력 결과에서 {\code A} 의 {\code m} 메서드가 {\code a} 와 {\code b} 객체로 부터 호출 되었을 때 모두
호출되었다는 것을 주목하자. 어셈블리에서 {\code A::m()} 으로 호출하는 것은 
어셈블리에서 코딩하기가 어렵다. 완전한 객체 지향 프로그래밍에서 호출되는 메서드는 
어떠한 객체가 함수에 전달되느냐에 따라 달려있다. 이를 \emph{다형성(polymorphism)} 이라
한다. C++ 은 기본적으로 이 기능을 사용하지 않는다. 우리는 \emph{virtual} 키워드 \index{C++!virtual} 를
통해 이 기능을 사용할 수 있다. 그림 ~\ref{fig:VirtInh} 는 이 두 클래스가 어떻게 바뀌는지 보여준다. 다른
코드들은 바뀔 필요가 없다. 다형성은 여러가지 형태로 구현 될 수 있다. 불행이도 현재 \emph{gcc} 에서의 구현은
이전의 구현 되었던 것보다 훨씬 복잡하게 되었다. 여기에 대해선 깊이 들어가지는 않고 윈도우즈 기반의
마이크로소프트와 볼랜드 컴파일러들이 사용하는 형태에 대해서만 이야기 하고자 한다. 이 컴파일러들에서의 구현은 몇 년 동안 바뀌지
않았을 뿐더러 앞으로도 바뀌지 않을 것이다. 

\emph{virtual} 키워드를 붙임으로써 프로그램의 출력은 아래와 같이 바뀐다:

\begin{verbatim}
Size of a: 8 Offset of ad: 4
Size of b: 12 Offset of ad: 4 Offset of bd: 8
A::m()
B::m()
\end{verbatim}


\begin{figure}[tp]
\begin{AsmCodeListing}[commentchar=!]
?f@@YAXPAVA@@@Z:
      push   ebp
      mov    ebp, esp

      mov    eax, [ebp+8]
      mov    dword [eax+4], 5  ; p->ad = 5;

      mov    ecx, [ebp + 8]    ; ecx = p
      mov    edx, [ecx]        ; edx = vtable 을 가리키는 포인터 
      mov    eax, [ebp + 8]    ; eax = p
      push   eax               ; "this" 포인터를 푸시
      call   dword [edx]       ; vtable 의 첫 번째 함수 호출
      add    esp, 4            ; 스택을 정리한다 

      pop    ebp
      ret
\end{AsmCodeListing}
\caption{{\code f()} 함수를 위한 어셈블리 코드 \label{fig:FAsm2}}
\end{figure}

\begin{figure}[tp]
\begin{lstlisting}[language=C++, frame=tlrb,escapeinside=~~]{}
class A {
public:
  virtual void __cdecl m1() { cout << "A::m1()" << endl; }
  virtual void __cdecl m2() { cout << "A::m2()" << endl; }
  int ad;
};

class B : public A {    // B inherits A's m2()
public:
  virtual void __cdecl m1() { cout << "B::m1()" << endl; }
  int bd;
};
/*~주어진 객체의 vtable 을 출력~ */
void print_vtable( A * pa )
{
  // ~p 는 pa 를 더블워드의 배열로 생각한다~
  unsigned * p = reinterpret_cast<unsigned *>(pa);
  // ~vt 는가상테이블을 포인터들의 배열로 생각한다~
  void ** vt = reinterpret_cast<void **>(p[0]);
  cout << hex << "vtable address = " << vt << endl;
  for( int i=0; i < 2; i++ )
    cout << "dword " << i << ": " << vt[i] << endl;

  // ~가상 함수를 이식성이 매우 떨어지는 방법으로 호출한다!~
  void (*m1func_pointer)(A *);   // ~함수 포인터 변수~
  m1func_pointer = reinterpret_cast<void (*)(A*)>(vt[0]);
  m1func_pointer(pa);       //~함수 포인터를 통해 메서드 m1 을 호출~

  void (*m2func_pointer)(A *);   // ~함수 포인터 변수~
  m2func_pointer = reinterpret_cast<void (*)(A*)>(vt[1]);
  m2func_pointer(pa);      // ~함수 포인터를 통해 메서드 m2 을 호출~
}

int main()
{
  A a;   B b1;  B b2;
  cout << "a: " << endl;   print_vtable(&a);
  cout << "b1: " << endl;  print_vtable(&b);
  cout << "b2: " << endl;  print_vtable(&b2);
  return 0;
}
\end{lstlisting}
\caption{ 좀 더 복잡한 예제 \label{fig:2mEx}}
\end{figure}


\begin{figure}[tp]
\centering
%\epsfig{file=vtable}
\input{vtable.latex}
\caption{{\code b1} 의 내부적 표현\label{fig:vtable}}
\end{figure}

이제, {\code f} 의 두번째 호출에선 {\code B} 객체를 전달했기 때문에 {\code B::m()} 메서드를
호출하게 된다. 하지만 달라진 것은 이것이 다가 아니다. {\code A} 의 크기는 8 이 된다. (그리고
{\code B} 는 12) 또한 {\code ad} 의 오프셋은 0 이 아닌 4 가 된다. 그렇다면 오프셋 0 에는 무엇이
있을까? 이것에 대한 대답은 어떻게 다형성이 구현되었는지에 따라 달라진다. 

\index{C++!vtable|)} 가상 메서드를 한 개 라도 가지는 C++ 클래스들은 메서드 포인터들의 배열을
가리키는 포인터의 추가적인 숨겨진 영역이 있다. \footnote{가상 메서드가 없는 클래스들의 경우
C++ 컴파일러들은 언제나 동일한 데이타 멤버를 가지는 보통의 C 구조체와 호환 가능한 클래스를
만든다} 이 테이블은 종종 \emph{가상테이블(vtable)} 이라 불린다. {\code A} 와 {\code B} 클래스의 가상테이블을 가리키는 포인터는
오프셋 0 에 저장되어 있다. 윈도우즈 컴파일러는 언제나 이 포인터들을 클래스 시작 부분인 상속 트리의 최상단에 위치해
놓는다. 함수 {\code f} (그림 ~\ref{fig:SimpInh})의 가상 메서드 버전의 어셈블리 코드 (그림 ~\ref{fig:FAsm2})를
보면 {\code m} 메서드 호출은 라벨이 아님을 알 수 있다. ~9 행은 객체로 부터 가상테이블의 주소값을 찾는다.
~11 행에서 객체의 주소값은 스택에 푸시된다. ~12 행에서 가상테이블의 첫 번째 주소\footnote{당연하게도 이 값은 이미 
{\code ECX} 레지스터에 들어 있다. 그 값은 ~8 행에서 들어갔다. 또한 ~10 행을 제거하는 대신에 그 다음 행에
{\code ECX} 를 푸시하게 바꾸면 된다. 이 코드는 별로 효율적이지 않는데 왜냐하면 컴파일러 최적화 옵션이
켜지지 않은 채 생성된 코드 이기 대문이다} 로 분기함으로써 가상 메서드를 호출한다.  이 호출은 라벨을 사용하지 않고,
{\code EDX} 가 가리키는 코드 주소로 분기한다. 이러한 형식의 호출은 \emph{나중 바인딩(late binding)} 라 한다. 
\index{C++!나중 바인딩}. 나중 바인딩은 코드가 실행될 때 까지 어떠한 메서드를 호출할지 결정할 수 없게 한다. 
이를 통해 객체의 올바른 메서드를 호출하게 한다. 보통의 경우 (그림 ~\ref{fig:FAsm1})는
호출되는 형태는 \emph{빠른 바인딩(early binding)} 이라 한다. \index{C++!빠른 바인딩} 
(컴파일 타임(compile time)에 무슨 메서드로 분기해야 할 지 알 수 있다.)

세심한 독자들의 경우 왜 그림 ~\ref{fig:VirtInh} 의 클래스 메서드들을 {\code \_\_cdecl} 키워드를 사용함으로써  
명백하게 C 호출 규약으로 정의하였는지에 대해 궁금해 할 것이다. 기본적으로 마이크로소프트는 표준 C 규약과 다른
호출 규약을 C++ 클래스 메서드에 사용한다. 이는 대신에 메서드에서 작업하는 객체를 가리키는 포인터를 스택을 사용하는 대신에 {\code ECX} 레지스터에
넣어 전달한다. 스택은 메서드의 다른 인자들을 전달하는데 사용된다. {\code \_\_cdecl} 변경자는 표준 C 
호출 규약을 사용하게끔 한다. 볼랜드~C+++ 은 C 호출 규약을 기본으로 사용한다. 

\begin{figure}[tp]
\fbox{ \parbox{\textwidth}{\code
a: \\
vtable address = 004120E8\\
dword 0: 00401320\\
dword 1: 00401350\\
A::m1()\\
A::m2()\\
b1:\\
vtable address = 004120F0\\
dword 0: 004013A0\\
dword 1: 00401350\\
B::m1()\\
A::m2()\\
b2:\\
vtable address = 004120F0\\
dword 0: 004013A0\\
dword 1: 00401350\\
B::m1()\\
A::m2()\\
} }
\caption{그림~\ref{fig:2mEx} 의 프로그램의 출력값 \label{fig:2mExOut}}
\end{figure}

이번에는 약간 더 복잡한 예제를 살펴보자 (그림 ~\ref{fig:2mEx}). 여기서,
클래스 {\code A} 와 {\code B} 는 각각 2 개의 메서드를 가지고 있다:
{\code m1} 과 {\code m2}. {\code B} 클래스가 자신의 {\code m2} 메서드를
정의하는 것이 아니라 {\code A} 클래스의 메서드를 상속 받는 다는 것을 기억해라.
그림 ~\ref{fig:vtable} 은 {\code b} 객체가 메모리 상에 어떻게 나타나는지 보여준다.
그림 ~\ref{fig:2mExOut} 은 프로그램의 출력 결과를 보여준다. 먼저, 각 객체의 가상테이블 주소를
보자. 두 개의 {\code B} 객체들의 가상 테이블 주소들은 같으므로 이들은 같은 가상테이블을 공유한다. 가상테이블은
클래스의 특징이지 객체가 아니다 (정적 데이타 멤버 처럼). 다음으로 가상테이블에 있는 주소값들을 보라.
출력 결과를 보면 우리는 {\code m1} 메서드 포인터가 오프셋 ~0 (혹은 더블워드 ~0) 에 있고 
{\code m2} 는 오프셋~4 (더블워드~1) 에 있음을 알 수 있다. {\code m2} 메서드 포인터는 {\code A} 와
{\code B} 의 가상테이블들에서 동일한데, 왜냐하면 {\code B} 는 {\code m2} 메서드를 {\code A}
클래스로 부터 상속 받았기 때문이다. 

~25 에서 32 행은 객체에 대한 객체의 가상테이블의 주소값을 읽어서 가상 함수를 호출할 수 있는지를
보여주고 있다. \footnote{이 코드는 오직 MS 와 볼랜드 컴파일러에서만 작동된다. \emph{gcc} 에서는
작동하지 않는다}. 메서드 주소는 C 유형의 함수 포인터와 \emph{this} 포인터가 저장된다. 
그림 ~\ref{fig:2mExOut} 에서의 출력값에서 우리는 이것이 실제로 작동한다는 것을 알 수 있다.
그러나, 코드를 이렇게 쓰지는 \emph{말라}. 이는 오직 가상 메서드들이 가상테이블을 어떻게 사용하는지
보여주기 위해서 쓴 것이다. 

%Looking at the output of Figure~\ref{fig:2mExOut} does demonstrate several
%features of the implementation of polymorphism.  The {\code b1} and {\code b2}
%variables have the same vtable address; however the {\code a} variable
%has a different vtable address. The vtable is a property of the class not
%a variable of the class. All class variables share a common vtable. The two
%{\code dword} values in the table are the pointers to the virtual methods.
%The first one (number 0) is for {\code m1}. Note that it is different for the
%{\code A} and {\code B} classes. This makes sense since the A and B classes
%have different {\code m1} methods. However, the second method pointer is 
%the same for both classes, since class {\code B} inherits the {\code m2}
%method from its base class, {\code A}.

여기서 몇 가지 교훈을 얻을 수 있다. 하나는 우리가 이진 파일에
클래스 변수들을 읽고 쓸 때 매우 주의해야 한다는 점이다. 우리는 단순히 전체 객체를 읽거나
쓸 수 없는데, 왜냐하면 이는 파일에 가상테이블 포인터를 읽거나 쓸 것이기 때문이다.
이는 프로그램의 메모리 상에 가상테이블이 있는 곳을 가리키는 포인터 이고, 아는 프로그램마다
다를 것이기 때문이다. 이같은 문제가 C 의 구조체에서도 발생 할 수 있지만, C 구조체의 경우
프로그래머가 특별히 지정해야만 포인터를 가지게 된다. {\code A} 나 {\code B} 클래스에선
명백히 정의된 포인터들이 없다. 

재차 강조하지만 컴파일러들은 서로 다른 방식으로 가상 메서드를 구현했다는 점을
잊지 말아야 한다. 윈도우즈 에서는 COM (Component Object Model) \index{COM} 
클래스 객체들이 COM 인터페이스를 구현하기 위해 가상테이블을 사용한다.\footnote{
COM 클래스는 또한 {\code \_\_stdcall} \index{호출 규약!stdcall} 호출 규약을 사용한다.
표준 C 가 아니다} 마이크로소프트와 같은 방법으로 가상 메서드를 가상테이블을 이용하여 구현한 컴파일러만이 
COM 클래스를 생성할 수 있다. 이는 볼랜드가 마이크로소프트와 같은 구현을 사용하였고,
왜 \emph{gcc} 에서는 COM 클래스를 만들 수 없는지 알 수 있다. 

가상 메서드를 위한 코드는 비-가상 메서드 것과 정확이 같다. 오직 이를 호출하는
코드만이 다를 뿐이다. 만일 컴파일러가 어떤 가상 메서드가 호출되는지 정확히 알 수 있다면,
이는 가상테이블을 무시하고 직접적으로 메서드를 호출 할 수 있다. (\emph{e.g} 빠른 바인딩)

\index{C++!vtable|)}
\index{C++!다형성|)}
\index{C++!상속|)}
\index{C++!클래스|)}
\index{C++|)}

\subsection{다른 C++ 기능들}

다른 C++ 기능(\emph{e.g} 런타임 형식의 정보, 예외 처리 및 다중 상속)에 대해 이야기 하는 것은
이 책의 범위를 훌쩍 뛰어 넘는다. 만일 독자 여러분이 더 자세히 알고 싶다면,
Ellis 와 Stroustrup 이 쓴 \emph{The Annotated C++ Reference Manual} 과 
Stroustrup 이 쓴 \emph{The Design and Evolution of C++} 을 읽어 보기를 권한다. 

% -*-latex-*-
\chapter{부동 소수점}\index{부동 소수점|(}

\section{부동 소수점 표현}\index{부동 소수점!표현|(}

\subsection{정수가 아닌 이진수}

우리가 첫 번째 장에서 기수법에 대해 이야기 하였을 때, 오직 정수들에 대해서만 이야기 하였다.
그러나, 명백히도 십진법과 같이 다른 진법 체계에서도 정수가 아닌 수를 표현할 수 있다. 십진법에선
소수점 다음으로 나타나는 숫자들은 10 에 음수 승을 취하게 된다. 

\[ 0.123 = 1 \times 10^{-1} + 2 \times 10^{-2} + 3 \times 10^{-3} \]

물론, 이진수들에 대해서도 동일하게 적용된다. 
\[ 0.101_2 = 1 \times 2^{-1} + 0 \times 2^{-2} + 1 \times 2^{-3} = 0.625 \]

~1 장에서 정수 이진수를 십진수로 바꾸는데 사용했던 방법을 동일하게 적용하여 아래와 같이
정수가 아닌 이진수를 십진수로 바꿀 수 있다. 

\[ 110.011_2 = 4 + 2 + 0.25 + 0.125 = 6.375 \]

십진수를 이진수로 바꾸는 것도 크게 어렵진 않다. 보통, 십진수를 두 개의 부분으로 나눈다:
정수 부분과 소수 부분. 정수 부분은 ~1 장에서 다루었던 방법으로 이진수로 바꾼다. 소수 부분은
아래의 설명한 방법을 통해 이진수로 변경될 수 있다. 

\begin{figure}[t]
\centering
\fbox{
\begin{tabular}{p{2in}p{2in}}
\begin{eqnarray*}
0.5625 \times 2 & = & 1.125 \\
0.125 \times 2 & = & 0.25 \\
0.25 \times 2 & = & 0.5 \\
0.5 \times 2 & = & 1.0 \\
\end{eqnarray*}
&
\begin{eqnarray*}
\mbox{first bit} & = & 1 \\
\mbox{second bit} & = & 0 \\
\mbox{third bit} & = & 0 \\
\mbox{fourth bit} & = & 1 \\
\end{eqnarray*}
\end{tabular}
}
\caption{0.5625 를 이진수로 바꾸기\label{fig:binConvert1}}
\end{figure}

예를 들어서 각각의 비트를 $a,b,c, \ldots$ 로 나타낸 이진수를 생각하자. 
이는 아래와 같이 나타난다:

\[ 0.abcdef\ldots \]
위 수에 2 를 곱한 수는 아래와 같이 이진법으로 나타난다. 

\[ a.bcdef\ldots \]
첫 번째 비트가 이제 1 의 자리에 있음을 유의해라. $a$ 를 $0$ 으로 바꾼다면
\[ 0.bcdef\ldots \]
그리고 2 를 다시 곱한다면 
\[ b.cdef\ldots \]
이제, 두 번째 비트 ($b$) 가 1 의 자리에 오게 된다. 이 방법을 여러번 반복하여
필요한 비트 수 만큼까지 값을 찾을 수 있다. 그림 ~\ref{fig:binConvert1} 는
0.5625 를 이진수로 변환하는 예를 보여주고 있다. 이 방법은 소수 부분의 값이
0 이 될 때 중단을 한다. 

\begin{figure}[t]
\centering
\fbox{\parbox{2in}{
\begin{eqnarray*}
0.85 \times 2 & = & 1.7 \\
0.7 \times 2 & = &  1.4 \\
0.4 \times 2 & = &  0.8 \\
0.8 \times 2 & = &  1.6 \\
0.6 \times 2 & = &  1.2 \\
0.2 \times 2 & = &  0.4 \\
0.4 \times 2 & = &  0.8 \\
0.8 \times 2 & = &  1.6 \\
\end{eqnarray*}
}}
\caption{0.85 를 이진수로 바꾸기\label{fig:binConvert2}}
\end{figure}

또다른 예제에선 23.85 를 이진수로 바꾸었다. 정수 부분은
이진수로 바꾸기가 매우 쉽다 ($23 = 10111_2$). 하지만 소수 부분 ($0.85$)은
어떨까? 그림 ~\ref{fig:binConvert2} 는 소수 부분을 이진수로 바꾸는
처음 몇 부분의 과정을 보여주고 있다. 사실, 이 수를 완전히 이진수로 바꾸기 위해선
무한번의 연산이 필요하다. 이 말은 0.85 는 순환 이진 소수라는 것이다.
\footnote{이는 전혀 놀라운 일이 아닌데, 어떤 수는 특정한 진법을 통해 
표현한다면 무한 소수가 될 수 있지만 다른 진법에서는 유한 소수가 될 수 있다.
예를 들어서 $\frac{1}{3}$ 을 생각해 보면 십진법으로 표현시 3 이 무한히 반복 되지만
3 진법으로 표현한다면 $0.1_3$ 과 같이 깔끔하게 나타날 수 있다.}
위 계산에서 나타나는 숫자들은 특정한 규칙들이 있다. 이를 잘 살펴본다면 
$0.85 = 0.11\overline{0110}_2$ 임을 알 수 있다. 따라서, $23.85 = 10111.11\overline{0110}_2$
이다. 

따라서, 위 계산 결과를 통해 알 수 있는 점은 23.85 는 유한 개의 이진 비트를 사용하여
수를 \emph{정확히} 나타낼 수 없다라는 점이다. (이는 $\frac{1}{3}$ 이 십진법으로
유한개의 자리수를 가지는 소수로 정확하게 나타낼 수 없는 점과 같다) 이 장에서는
C 에서의 {\code float} 과 {\code double} 변수들이 이진수로 어떻게 저장되는지 살펴 볼 것이다.
따라서 23.85 와 같은 값들은 이 변수에 정확하게 저장될 수 없다. 오직 23.85 의 근사값 만이
변수에 저장될 것이다. 

하드웨어를 단순화 하기 위해 부동 소수점 수들은 일정한 형식으로 저장된다. 이 형식은
과학적 기수법 표현 (다만 이진수에서는 10 의 멱수를 사용하지 않고 2 의 멱수를 사용한다)
을 사용한다. 예를 들어 23.85, 혹은 $10111.11011001100110\ldots_2$ 은 다음과 같이
저장된다. 

\[ 1.011111011001100110\ldots \times 2^{100} \]
(지수 (100) 도 이진수로 나타난다.). \emph{정규화} 된 부동 소수점 수는
아래의 같은 꼴을 가진다. 

\[ 1.ssssssssssssssss \times 2^{eeeeeee} \]
이 때 $1.sssssssssssss$ 은\emph{가수(significand)} 라 하고 $eeeeeeee$ 은 
\emph{지수(exponent)} 라 한다. 

\subsection{IEEE 부동 소수점 표현}\index{부동 소수점!표현!IEEE|(}

IEEE (전기 전자 기술자 협회, Institute of Electrical and Electronic Engineers) 는 
부동 소수점 수들을 저장하기 위한 구체적인 이진 형식을 만든 국제적인 조직이다.
이 형식은 오늘날 만들어진 대부분 (모든 컴퓨터는 아니다!) 의 컴퓨터에서 사용된다. 
많은 경우 이 형식은 컴퓨터 하드웨어 자체에서 지원된다. 예를 들어 인텔의 수치
보조처리기(numeric coprocessor) 이다. (이는 Pentium 부터 모든 인텔 CPU 들에게
추가되었다) IEEE 는 정밀도가 다른 두 개의 형식을 지원한다. 이는 단일 정밀도(single precision)
와 2배 정밀도(double precision) 이다. 단일 정밀도는 C 언어의 {\code float} 변수들이
사용하며 2배 정밀도는 {\code double} 변수들이 사용한다.

인털의 수치 보조처리기는 또한 더 높은 정밀도를 가지는 \emph{확장 정밀도(extended precision)}를
사용한다. 사실, 보조처리기에 있는 모든 데이터들은 이 정밀도로 표현된다. 보조처리기는
이를 메모리에 저장할 때에 단일 또는 2배 정밀도를 가지는 수들로 자동으로 변환한다.
\footnote{ 일부 컴파일러 (예를 들어 Borland) 의 {\code long double} 형은 이 확장 정밀도를
이용한다. 그러나 대부분의 경우 {\code double} 과 {\code long double} 형 모두 
2배 정밀도를 사용한다. (이는 ANSI C 에서 허용된다)} 확장 정밀도는 IEEE 의 float 이나 double에 비해
다른 형식을 사용하며 여기서는 이야기 하지 않도록 하겠다. 

\subsubsection{IEEE 단일 정밀도}\index{부동 소수점!표현!단일 정밀도|(}

\begin{figure}[t]
\fbox{
\centering
\parbox{5in}{
\begin{tabular}{|c|c|c|}
\multicolumn{1}{p{0.3cm}}{31} &
\multicolumn{1}{p{2.5cm}}{30 \hfill 23} &
\multicolumn{1}{p{6cm}}{22 \hfill 0} \\
\hline
s & e & f \\
\hline
\end{tabular}
\\[0.4cm]
\begin{tabular}{cp{4.5in}}
s & 부호 비트 - 0 = 양수, 1 = 음수 \\
e & 편향 지수(8 비트) = 실제 지수값 + 7F (십진수로 127). 
    지수의 값이 00 과 FF 일 때는 특별한 의미를 가진다.  \\
f & 가수 - 1. 다음의 처음 23 비트 
\end{tabular}
}}
\caption{IEEE 단일 정밀도\label{fig:IEEEsingle}}
\end{figure}

단일 정밀도 부동 소수점 형식은 수를 32 비트로 표현한다. 이는 대체로 십진수로 7 자리 정도
정확하다. 부동 소수점 수들은 정수들에 비해 훨씬 복접한 형식으로 저장된다.
그림 ~\ref{fig:IEEEsingle} 은 IEEE 단일 정밀도 수의 기본적인 형식을 보여준다. 
이 형식에선 약간의 이상한 점이 있다. 부동 소수점 수들은 음수를 위해 2 의 보수 표현법을 사용하지
않는다. 이들은 부호 있는 크기 표현(signed magnitude) 을 사용한다. 31 째 비트는 위에 나와있듯이
수의 부호를 정한다. 

이진 지수는 그 값 그대로 저장되지 않는다. 그 대신 7F 를 더해서 23 ~ 30 번째 비트에 더해진다.
따라서 이 \emph{편향 된 지수(biased exponent)} 는 언제나 음수가 아니다. 

소수 부분은 언제나 정규화 된 가수 부분이라 생각한다 ($1.sssssssss$ 형식으로). 언제나 첫 번째 비트가
1 이므로, 1 은 \emph{저장되지 않는다}. 이를 통해 비트 하나를 추가적으로 저장할 수 있게 되어 정확성을
향상 시킬 수 있다. 이 아이디어는 \emph{숨겨진 1 표현법(hidden one representation)} 이라 한다.\index{부동 소수점!표현!숨겨진 1}

그렇다면 23.85 는 어떻게 저장될까? 
 \MarginNote{언제나 명심해 둘 것은 프로그램이 41 BE CC CD 를 어떻게 해석하느냐에 따라
 그 값이 달라질 수 있다는 사실이다! 예를 들어 단일 정밀도 부동 소수점 수라고 생각한다면 
 이 값은 23.850000381 을 나타나게 된다. 하지만 더블워드 정수라고 생각한다면 이는 
1,103,023,309 를 의미한다. CPU 는 어는 것이 이 수에 대한 정확한 표현인지 알 수 없다. }
먼저 이 수는 양수이므로 부호 비트는 0 이다. 지수 값은 4 이므로, 편향된 지수 값은
$7\mathrm{F} + 4 = 83_{16}$ 가 된다. 마지막으로 가수 부분은 01111101100110011001100 이다.
(맨 처음의 1 은 숨겨져 있다는 사실을 명심해라) 이를 모두 합치면 (참고로 이해를 돕기 위해서
부호 비트와 가수 부분은 밑줄이 처져 있고, 비트 들은 4 비트 니블들로 나뉘어 나타냈다)
% 오타 faction -> fraction
\[ \underline{0}\,100\;0001\;1
   \,\underline{011\;1110\;1100\;1100\;1100\;1100}_2 = 41 \mathrm{BE} 
\mathrm{CC} \mathrm{CC}_{16} \]
이는 23.85 를 정확하게 표현한 것은 아니다 (왜냐하면 이진법에서 순환소수로 나타나기 때문).
만일 위 수를 십진수로 바꾼다면 그 값은 대략 23.849998474 가 될 것이다. 이 값은
23.85 와 매우 가깝지만 정확히 같지는 않다. 사실, C 에선 23.85 를 위와 같이 나타내지 않는다.
C 에선, 정확한 이진 표현에서 단일 정밀도 형식으로 변환 시, 마지막으로 잘라낸 비트가 1 이므로 마지막 비트는 1 로 반올림 된다. 
따라서 23.85 는 41 BE CC CD 로 나타나게 된다. 이는 십진수로 바꾸면 23.850000381 이므로 좀더
23.85 를 가깝게 표현할 수 있다.  

-23.85 는 어떻게 저장될까? 단순히 부호 비트만 바꾸면 된다. 따라서 C1 BE CC CD 가 된다. 
2의 보수 표현법을 취하지 \emph{않는다}!

\begin{table}[t]
\fbox{
\begin{tabular}{lp{3.1in}}
$e=0 \quad\mathrm{,}\quad f=0$ & 이면 0 을 나타낸다. (이는 정규화 
                         될 수 없다)  +0 와 -0 가 있음을 유의해라. \\
$e=0 \quad\mathrm{,}\quad f \neq 0$ & 는 \emph{비정규화 된 수} 임을
                              나타낸다. 이것은 다음 단원에서 다룬다. \\
$e=\mathrm{FF} \quad\mathrm{,}\quad f=0$ 
& 는 ($\infty$) 을 나타낸다. 음의 무한과 양의 무한이 있을 수 
있다. \\
$e=\mathrm{FF} \quad\mathrm{,}\quad f\neq 0$ 
& \emph{NaN}(Not A Number) 라 부르는 정의되지 않은 값을 나타낸다.
\end{tabular}
}
\caption{\emph{f} 와 \emph{e} 의 특별한 값들\label{tab:floatSpecials}}
\end{table}

\emph{e} 와 \emph{f} 의 특정한 조합은 IEEE float 에서 특별한 의미를 가진다.
표 ~\ref{tab:floatSpecials} 는 이러한 특별한 값들을 보여준다. 무한대는 오버 플로우(over flow)나
0 으로의 나눗셈에 의해 발생된다. 정의되지 않은 값(NaN) 은 올바르지 못한 연산, 예를 들면
음수의 제곱근을 구한다던지, 두 개의 무한대를 더할 때 \emph{등등} 발생된다.

정규화된 단일 정밀도 수들은 $1.0 \times 2^{-126}$ ($\approx 1.1755 \times 10^{-35}$) 에서 
$1.11111\ldots \times 2^{127}$ ($\approx 3.4028 \times 10^{35}$) 까지의 값을 가질 수 있다.

\subsubsection{비정규화 된 수\index{부동 소수점!표현!비정규화|(}}

비정규화 된 수(Denormalized number)들은 너무 값이 작아서 정규화 할 수 없는 수들을 나타내기 위해 사용된다.
(\emph{i.e.} $1.0 \times 2^{-126}$ 미만) 예를 들어 $1.001_2 \times 2^{-129}$ ($\approx 1.6530 \times 10^{-39}$) 라는
수를 생각하자. 이 수를 정규화 하기에는 지수가 너무 작다. 그러나, 이 수는 비정규화 된 꼴:
$0.01001_2 \times 2^{-127}$ 로 나타낼 수 있다. 이 수를 저장하기 위해 편향된 지수는
0 으로 맞춰 진다. (표 ~\ref{tab:floatSpecials} 참조) 그리고 $2^{-127}$ 와 곱해진
소수 부분은 맨 앞 자리의 비트를 포함한 완전한 수로 나타나게 된다. 따라서, $1.001 \times 2^{-129}$는:
\[ \underline{0}\,000\;0000\;0
   \,\underline{001\;0010\;0000\;0000\;0000\;0000} \]
\index{부동 소수점!표현!비정규화|)}
\index{부동 소수점!표현!단일 정밀도|)}


\subsubsection{IEEE 2배 정밀도\index{부동 소수점!표현!2배 정밀도|(}}

\begin{figure}[t]
\centering
\begin{tabular}{|c|c|c|}
\multicolumn{1}{p{0.3cm}}{63} &
\multicolumn{1}{p{3cm}}{62 \hfill 52} &
\multicolumn{1}{p{7cm}}{51 \hfill 0} \\
\hline
s & e & f \\
\hline
\end{tabular}
\caption{IEEE 2배 정밀도\label{fig:IEEEdouble}}
\end{figure}

IEEE 2배 정밀도 표현은 64 비트를 사용하며, 십진법으로 대략 15 자리 정도 정확하게
나타낼 수 있다. 그림 ~\ref{fig:IEEEdouble} 에서 나타나듯이 기본적인 형식은
단일 정밀도와 매우 유사하다. 다만 편향된 지수와 가수 부분이 각각 11 비트와 
52 비트로 늘어났다.  

편향된 지수 부분의 크기가 더 커졌으므로 두 가지 변화를 야기하게 된다. 먼저
편향된 지수 값을 구할 때 진짜 지수에 3FF(1023) 을 더한다 (단일 정밀도의 7F 가 아니다).
두 번째로 진짜 지수가 가질 수 있는 값의 범위가 넓어 졌다.
2배 정밀도가 표현할 수 있는 수의 범위는 대략  $10^{-308}$ to $10^{308}$ 정도 된다. 

또한 가수 부분이 더 커졌으므로 유효 숫자의 자리수가 더 늘어나게 되었다. 

예를 들어 23.85 를 다시 생각해 보자. 편향된 지수는 16 진수로 $4 + \mathrm{3FF} = 403$가
될 것이다. 따라서 2배 정밀도 표현으로는 :
\[ \underline{0}\,100\;0000\;0011\;\underline{0111\;1101\;1001\;1001\;1001\;
   1001\;1001\;1001\;1001\;1001\;1001\;1001\;1010} \]
혹은 40 37 D9 99 99 99 99 9A 로 나타나게 된다. 이 수를 십진수로 변환한다면 
그 값은 23.8500000000000014 가 되어 (무려 12 개의 0 이 있다!) 23.85 를 훨씬 정밀하게
표현할 수 있게 된다.

2배 정밀도도 단일 정밀도 표현과 같이 특별한 값들이 있다. 
\footnote{유일한 차이점은 무한대와 NaN 의 경우 인데, 편향된 지수 값이 7FF 가 아니라
FF 가 된다. }
비정규화 된 수도 또한 매우 비슷하다. 유일한 차이점은 비정규화 된 수는$2^{-127}$ 가 아닌 $2^{-1023}$ 
가 된다. 

\index{부동 소수점!표현!2배 정밀도|)}
\index{부동 소수점!표현!IEEE|)}
\index{부동 소수점!표현|)}

\section{부동 소수점 수들의 산술 연산\index{부동 소수점!산술 연산|(}}

컴퓨터에서의 부동 소수점 수들의 산술 연산은 수들이 연속된 수학에서와는 다르다. 
수학에서는 모든 수들이 정확한 값을 가졌다고 생각된다. 위에서 다루었듯이 
컴퓨터가 다루는 많은 수들은 유한개의 비트를 통해 정확히 나타낼 수 없다. 모든 연산들은
제한된 정밀도를 가진다. 이 단원에서의 예제들은 단순화를 위해 8 비트의 유효 숫자를 가진다고
생각한다. 

\subsection{덧셈}
두 개의 부동 소수점 수들을 더하기 위해선 지수가 반드시 같아야 한다. 만일, 이 지수들이 같지 않다면
수의 가수를 쉬프트 하여 큰 지수에 맞추어야 한다. 예를 들어 $10.375 + 6.34375 = 16.71875$
를 이진법에서 수행한다고 하자. 

\[
\begin{array}{rr}
 & 1.0100110 \times 2^3 \\
+& 1.1001011 \times 2^2 \\ \hline
\end{array}
\]
이 두 수들은 지수값이 같지 않으므로 가수 부분을 쉬프트 하여 두 수의 지수가
같게 한 뒤 더한다.

\[
\begin{array}{rr@{.}l}
 &  1&0100110 \times 2^3 \\
+&  0&1100110 \times 2^3 \\ \hline
 & 10&0001100 \times 2^3
\end{array}
\]
이 때 $1.1001011 \times 2^2$ 에 쉬프트 연산을 취하면서 마지막 비트를 없애고
반올림을 하여 $0.1100110 \times 2^3$ 가 되었음을 유의하자. 덧셈의 결과는 
$10.0001100 \times 2^3$ (혹은 $1.00001100 \times 2^4$) 로 $10000.110_2$
혹은 16.75 와 같이 같다. 이는 정확한 계산 결과인 16.71875 와 같지 \emph{않다}! 이는 단지
덧셈 과정에서 반올림에 의해 오차가 나타난 근사값에 불과하다. 

컴퓨터 (혹은 계산기)에서의 부동 소수점 연산은 언제나 근사임을 명심해야 한다.
수학에서의 법칙들은 컴퓨터에서의 부동 소수점들에 적용될 수 없다. 수학은 언제나
컴퓨터가 절대로 만들어 낼 수 없는 무한대의 정밀도를 가정한다. 예를 들어서
수학에선 $(a + b) - b = a$ 라 말하지만 컴퓨터에서는 이 식이 성립하지 않을 수 있다!

\subsection{뺄셈}
뺄셈은 덧셈과 매우 유사하며, 같은 문제를 가지고 있다. 예를 들어서 $16.75 - 15.9375 = 0.8125$
를 생각해보자. 

\[
\begin{array}{rr}
 & 1.0000110 \times 2^4 \\
-& 1.1111111 \times 2^3 \\ \hline
\end{array}
\]
$1.1111111 \times 2^3$ 를 쉬프트 하면 $1.0000000 \times 2^4$ (반올림 됨)
\[
\begin{array}{rr}
 & 1.0000110 \times 2^4 \\
-& 1.0000000 \times 2^4 \\ \hline
 & 0.0000110 \times 2^4
\end{array}
\]
$0.0000110 \times 2^4 = 0.11_2 = 0.75$ 이므로 원 답과 일치하지 않는다. 

\subsection{곱셈과 나눗셈}

곱셈의 경우 유효 숫자 부분은 곱해지고, 지수 부분은 더해 진다. 예를 들어 
$10.375 \times 2.5 = 25.9375$ 을 생각해 보면 

\[
\begin{array}{rr@{}l}
 &  1.0&100110 \times 2^3 \\
\times &  1.0&100000 \times 2^1 \\ \hline
 &     &10100110 \\
+&   10&100110   \\ \hline
 &   1.1&0011111000000 \times 2^4
\end{array}
\]
당연하게도 실제 결과는 8 비트로 반올림 되어 
\[1.1010000 \times 2^4 = 11010.000_2 = 26 \]

나눗셈은 좀더 복잡하지만 반올림으로 인하여 같은 문제를 가진다. 

\subsection{프로그래밍을 위한 조언}

이 단원의 핵심적인 내용은 부동 소수점 연산이 정확하지 않다는 것이다. 
프로그래머는 언제나 이를 고려해야 한다. 프로그래머들이 가장 많이
범하는 실수로는 연산이 정확하다고 가정 하에 부동 소수점 수들을 비교하는 것이다.
예를 들어 \lstinline|f(x)| 라는 이름의 함수를 고려하자. 이 함수는 복잡한 연산을 수행한다.
우리의 프로그램의 목적은 이 함수의 근을 찾는 것이다. 
\footnote{함수의 근이라 하면 $f(x) = 0$ 을 만족하는 값 $x$ 를 말한다.} 여러분은 아마
\lstinline|x| 가 근인지 확인하기 위해 아래의 식을 이용할 것이다. 

\begin{lstlisting}[stepnumber=0]{}
  if ( f(x) == 0.0 )
\end{lstlisting}
그러나 만일 \lstinline|f(x)| 가 $1 \times 10^{-30}$ 을 리턴하면? 이는
진짜 근의 훌륭한 근사값이다. 그러나, 위 식은 성립하지 않는다. 
아마도 \lstinline|f(x)| 에서의 반올림으로 인한 오차로 인해 어떠한 IEEE 부동 소수점 형식으로 표현된
\lstinline|x|값도 정확히 0 을 리턴하지 않을 것이다. 

위 보다 훨씬 낳은 방법은 아래와 같다. 

\begin{lstlisting}[stepnumber=0]{}
  if ( fabs(f(x)) < EPS )
\end{lstlisting}
\lstinline|EPS| 는 매우 작은 양수 값인 매크로로 정의되어 있다. (예를 들면 $1 \times 10^{-10}$)
이는 \lstinline|f(x)| 가 0 에 매우 가까울 때 성립이 된다. 통상적으로 두 개의 부동 소수점 값
(\lstinline|x|) 을 다른 (\lstinline|y|) 값에 비교 할 때 아래와 같이 사용한다: 

\begin{lstlisting}[stepnumber=0]{}
  if ( fabs(x - y)/fabs(y) < EPS )
\end{lstlisting}
\index{부동 소수점!산술 연산|)}

\section{수치 부프로세서}
\index{부동 소수점 부프로세서|(}
\subsection{하드웨어}
\index{부동 소수점 부프로세서!하드웨어|(}
가장 초기의 인털 프로세서들은 부동 소수점 연산을 지원하는 하드웨어가 없었다.
이 말은 부동 소수점 연산을 할 수 없었다는 것이 아니다. 이는 단지 부동 소수점 수들의
연산을 위해 많은 수의 비-부동 소수점 명령들을 수행해야 했다는 뜻이다. 이러한
초기의 시스템에선 인텔은 부수적인\emph{수치 부프로세서(math coprocessor)}를 제공하였다.
이 수치 부프로세서는 부동 소수점 수들의 연산을 위한 소프트웨어 상의 프로시저의 비해
훨씬 빠른 속도로 명령을 수행하였다 (초기의 프로세서의 경우 대략 10 배 이상!). 8086/8088을
위한 부프로세서는 8087 이라 불리었다. 80286 에선 80287이, 80386 에선 80387 이 부프로세서 였다. 
80486DX 프로세서에선 80486에 수치 부프로세서가 통합되었다. 
\footnote{그러나 80486SX 에는 통합된 부프로세서가 \emph{없었다}. 이를 위해 독립적인 80487SX
칩이 있었다.} 펜티엄 이후 80x86 세대의 모든 프로세서들에게는 수치 부프로세서가 통합되었다. 그러나, 
독립적으로 떨어져 있었을 때 처럼 프로그램 된다. 심지어 부프로세서가 없던 초기의 시스템들은 
수학 부프로세서를 에뮬레이트 할 수 있는 소프트웨어가 있었다. 이 에뮬레이터는 프로그램이
부프로세서 명령을 실행하면 자동적으로 활성화 되어 소프트웨어 프로시저를 통해 부프로세서가 
내놓았을 같은 값을 내놓았다. (비록 매우 느렸지만)

수치 부프로세서는 8 개의 부동 소수점 레지스터가 있다. 각각의 레지스터는 80 비트의 데이터를 보관한다.
부동 소수점 수들은 이 레지스터들에 \emph{언제나} 80 비트 확장 정밀도로 저장된다. 이 레지스터들은 각각
{\code ST0}, {\code ST1}, {\code ST2}, $\ldots$ {\code ST7} 로 이름 붙여졌다. 부동 소수점 레지스터들은
주  CPU 에서의 정수 레지스터들과는 달리 다른 용도로 사용된다. 부동 소수점 레지스터들은
\emph{스택(stack)}으로 구성되어 있다. 스택이 \emph{후입 선출} (LIFO) 리스트임을
상기하자. {\code ST0} 는 언제나 스택의 최상단의 값을 가리킨다. 모든 새로운 수들은 스택의 최상단에 들어간다.
기존의 수들은 새로운 수를 위한 공간을 만들기 위해 스택 한 칸 내려간다. 

수치 부프로세서에는 상태 레지스터(status register) 가 있다. 이는 몇 가지의 플래그를 가진다.
비교 연산에서는 오직 4 개의 플래그들이 사용된다:C$_0$, C$_1$, C$_2$, C$_3$. 이것들의
사용은 나중에 다룰 것이다. 

\index{부동 소수점 부프로세서!하드웨어|)}

\subsection{명령}

보통의 CPU 명령들과 부프로세서의 명령들을 쉽게 구별하기 위해 모든 부프로세서의 연상기호들은
{\code F} 로 시작한다. 

\subsubsection{데이터를 불러오고 저장하기}\index{부동 소수점 부프로세서!불러오고 저장하기|(}
부프로세서의 레지스터 스택의 최상위 데이터에 값을 저장하는 몇 가지 연산들이 있다.\\
\begin{tabular}{lp{4in}}
{\code FLD \emph{source}} \index{FLD} & 
는 메모리에서 부동 소수점 값을 불러와 스택 최상단에 저장한다. \emph{source} 는
단일 혹은 2배 정밀도 값, 아니면 부프로세서의 레지스터 여야 한다. \\ 
{\code FILD \emph{source}} \index{FILD} &
는 \emph{정수}를 메모리로 부터 읽어들어, 부동 소수점 형식으로 변환한 뒤
그 결과를 스택 최상단에 저장한다. \emph{source} 는 워드, 더블워드, 쿼드워드
중 하나여야 한다. \\
{\code FLD1} \index{FLD1} &
1 을 스택 최상단에 저장한다.  \\
{\code FLDZ} \index{FLDZ} &
0 을 스택 최상단에 저장한다. \\
\end{tabular}


스택으로 부터의 데이터를 메모리에 저장하는 명령들도 몇 가지 있다. 이러한 명령들의 일부는
스택에서 수를 \emph{팝(pop)} (\emph{i.e.} 제거하다) 한 후에 그 값을 저장한다. 


\begin{tabular}{lp{4in}}
{\code FST \emph{dest}} \index{FST} &
스택의 최상단({\code ST0}) 을 메모리에 저장한다. \emph{dest} 은 
단일 혹은 2배 정밀도 수 이거나 부프로세서 레지스터가 될 수 있다. \\
{\code FSTP \emph{dest}} \index{FSTP} &
은 스택의 최상단 값을 {\code FST} 와 같이 메모리에 저장한다. 그러나
수가 저장된 다음에는 그 값은 스택에서 팝 된다. \emph{dest} 은
단일 혹은 2배 정밀도 수 이거나 부프로세서 레지스터가 될 수 있다. \\
{\code FIST \emph{dest}} \index{FIST} &
는 스택의 최상단의 값을 정수로 변환한 뒤 메모리에 저장한다. \emph{dest}는
워드 혹은 더블워드가 되어야만 한다. 스택 그 자체로는 바뀌지 않는다. 부동 소수점
수가 정수로 어떻게 변환되느냐넨 부프로세서의 \emph{제어 워드(control word)}에 
몇 개의 비트값에 따라 달라진다. 이는 특별한 (비-부동 소수점) 워드 레지스터로 부프로세서가
어떻게 작동할지 제어한다. 기본적으로 제어 워드는 초기화 되어 있는데 이 때에는
수에 가장 가까운 정수로 변환되어 진다. 그러나, {\code FSTCW} (저장 제어 워드
Store Control Word) 와 {\code FLDCW} (불러오기 제어 워드, Load Control Word) 명령들을 통해
어떻게 변환 할지를 바꿀 수 있다. \index{FSTCW} \index{FLDCW} \\

{\code FISTP \emph{dest}} \index{FIST} &
{\code FIST} 와 두 가지 빼고 동일하다. 하나는, 스택의 최상단이 팝 된다는 것이고,
다른 하나는 \emph{dest} 가 쿼드워드도 될 수 있다는 것이다. 
\end{tabular}

스택 자체에서 데이터를 이동, 혹은 제거할 수 있는 명령이 두 개가 있다. \\

\begin{tabular}{lp{4in}}
{\code FXCH ST\emph{n}} \index{FXCH}  &
스택에서의 {\code ST0} 와 {\code ST\emph{n}} 의 값을 바꾼다. 
(이 때, \emph{n} 은 1 부터 7 까지의 값을 가질 수 있다.) \\
{\code FFREE ST\emph{n}} \index{FFREE} &
레지스터가 비었거나 미사용중임을 표시하여 스택에서의 레지스터의 값을 비운다. 
\end{tabular}
\index{부동 소수점 부프로세서!불러오고 저장하기|)}

\subsubsection{덧셈과 뺄셈}\index{부동 소수점 부프로세서!덧셈과 뺄셈|(}

각각의 덧셈 명령들은 {\code ST0} 과 다른 피연산자의 합을 계산한다. 그 결과는 언제나
부프로세서 레지스터에 저장된다. \\
\begin{tabular}{p{1.5in}p{3.5in}}
{\code FADD \emph{src}} \index{FADD} &
{\code ST0 += \emph{src}}. \emph{src} 는 어떤 부프로세서 레지스터이거나 
메모리 상의 단일 혹은 2배 정밀도 수여도 상관 없다. \\
{\code FADD \emph{dest}, ST0} &
{\code \emph{dest} += ST0}. \emph{dest}는 임의의 부프로세서 레지스터 이다. \\
{\code FADDP \emph{dest}} 또는 \newline {\code FADDP \emph{dest}, STO} \index{FADDP} &
{\code \emph{dest} += ST0} 하고 팝 한다. \emph{dest} 는 임의의 
부프로세서 레지스터 이다. \\
{\code FIADD \emph{src}} \index{FIADD} &
{\code ST0 += (float) \emph{src}}. 정수를 {\code ST0} 에 더한다. 
\emph{src} 는 메모리 상의 워드 혹은 더블워드 값이다. 
\end{tabular}

\begin{figure}[t]
\begin{AsmCodeListing}[frame=single]
segment .bss
array        resq SIZE
sum          resq 1

segment .text
      mov    ecx, SIZE
      mov    esi, array
      fldz                  ; ST0 = 0
lp:
      fadd   qword [esi]    ; ST0 += *(esi)
      add    esi, 8         ; 다음 double 로 이동
      loop   lp
      fstp   qword sum      ; sum 에 결과를 저장
\end{AsmCodeListing}
\caption{배열의 원소 합 구하는 예제\label{fig:addEx}}
\end{figure}

두 개의 피연산자에 대해 덧셈 명령은 한 가지 이지만, 뺄셈 명령은 2 가지의 서로 다른 뺄셈 명령이
존재할 수 있다. 왜냐하면 피연산자의 순서가 뺄셈에서는 중요하기 때문이다. (\emph{i.e.} $a + b = b + a$, 이지만
$a - b \neq b - a$ 은 아니다!) 각 명령에 대해 뺄셈을 역으로 수행하는 명령이 있다. 이 역명령들은 끝에 언제나
{\code R} 이나 {\code RP} 가 붙는다. 그림 ~\ref{fig:addEx} 는 배열의 double 형 원소들을 모두 더하는 코드를 보여주고
있다. ~10 에서 13 행 까지 우리는 메모리 피연산의 크기를 확실히 지정해야 한다. 그렇지 않는 다면 어셈블러는 메모리
피연산자가 float(더블워드) 인지 double(쿼드워드) 인자 알 수 없기 때문이다. 

\begin{tabular}{p{1.5in}p{3.5in}}
{\code FSUB \emph{src}} \index{FSUB} &
{\code ST0 -= \emph{src}}. \emph{src} 는 임의의 부프로세서 레지스터 거나
메모리 상의 단일 혹은 2배 정밀도 수 이면 된다. \\
{\code FSUBR \emph{src}} \index{FSUBR} &
{\code ST0 = \emph{src} - ST0}. \emph{src} 는 임의의 부프로세서 레지스터 거나
메모리 상의 단일 혹은 2배 정밀도 수 이면 된다. \\
{\code FSUB \emph{dest}, ST0} &
{\code \emph{dest} -= ST0}. \emph{dest} 는 임의의 부프로세서 레지스터 이다. \\
{\code FSUBR \emph{dest}, ST0} &
{\code \emph{dest} = ST0 - \emph{dest}}.\emph{dest} 는 임의의 부프로세서 
레지스터 이다. \\
{\code FSUBP \emph{dest}} 혹은 \newline {\code FSUBP \emph{dest}, STO} \index{FSUBP} &
{\code \emph{dest} -= ST0} 한 후 팝 한다. \emph{dest} 는 임의의 부프로세서
레지스터 이다. \\
{\code FSUBRP \emph{dest}} or \newline {\code FSUBRP \emph{dest}, STO} \index{FSUBRP} &
{\code \emph{dest} = ST0 - \emph{dest}} 한 후 팝 한다. \emph{dest} 는 임의의 
부프로세서 레지스터 이다. \\
{\code FISUB \emph{src}} \index{FISUB} &
{\code ST0 -= (float) \emph{src}}.{\code ST0} 에서 정수를 뺀다. 
\emph{src} 는 반드시 메모리 상의 워드 혹은 더블워드 여야 한다. \\
{\code FISUBR \emph{src}} \index{FISUBR} &
{\code ST0 = (float) \emph{src} - ST0}.{\code ST0} 에서 정수를 뺀다.
\emph{src} 는 반드시 메모리 상의 워드 혹은 더블워드 여야 한다.
\end{tabular}

\index{부동 소수점 부프로세서!덧셈과 뺄셈|)}

\subsubsection{곱셈과 나눗셈}\index{부동 소수점 부프로세서!곱셈과 나눗셈|(}

곰셉 명령은 덧셈 명령과 완벽히 동일하다. 

\begin{tabular}{p{1.5in}p{3.5in}}
{\code FMUL \emph{src}} \index{FMUL} &
{\code ST0 *= \emph{src}}. \emph{src} 는 임의의 부프로세서 레지스터 거나
메모리 상의 단일 혹은 2배 정밀도 수 이면 된다. \\
{\code FMUL \emph{dest}, ST0} &
{\code \emph{dest} *= ST0}.\emph{dest} 는 임의의 부프로세서 레지스터 이다. \\
{\code FMULP \emph{dest}} 혹은 \newline {\code FMULP \emph{dest}, STO} \index{FMULP} &
{\code \emph{dest} *= ST0} 한 후 팝 한다. \emph{dest} 는 임의의 부프로세서
레지스터 이다. \\
{\code FIMUL \emph{src}} \index{FMUL} &
{\code ST0 *= (float) \emph{src}}. {\code ST0} 에 정수를 곱한다.
\emph{src} 는 메모리 상의 워드 혹은 더블워드 여야 한다. 
\end{tabular}

놀랍지 않게도 나눗셈 명령은 뺄셈 명령과 거의 동일하다. 다만 0 으로 나눈다면
무한대가 된다.\\

\begin{tabular}{p{1.5in}p{3.5in}}
{\code FDIV \emph{src}} \index{FDIV} &
{\code ST0 /= \emph{src}}. \emph{src} 는 임의의 부프로세서 레지스터 거나
메모리 상의 단일 혹은 2배 정밀도 수 이면 된다. \\
{\code FDIVR \emph{src}} \index{FDIVR} &
{\code ST0 = \emph{src} / ST0}. \emph{src} 는 임의의 부프로세서 레지스터 거나
메모리 상의 단일 혹은 2배 정밀도 수 이면 된다. \\
{\code FDIV \emph{dest}, ST0} &
{\code \emph{dest} /= ST0}. \emph{dest} 는 임의의 부프로세서 레지스터 이다. \\
{\code FDIVR \emph{dest}, ST0} &
{\code \emph{dest} = ST0 / \emph{dest}}. \emph{dest} 는 임의의 부프로세서
레지스터 이다. \\
{\code FDIVP \emph{dest}} 혹은 \newline {\code FDIVP \emph{dest}, STO} \index{FDIVP} &
{\code \emph{dest} /= ST0} 한 후 팝 한다. \emph{dest} 는 임의의 부프로세서
레지스터 이다. \\
{\code FDIVRP \emph{dest}} 혹은 \newline {\code FDIVRP \emph{dest}, STO} \index{FDIVRP} &
{\code \emph{dest} = ST0 / \emph{dest}} 한 후 팝한다. \emph{dest} 는 임의의
부프로세서 레지스터 이다. \\
{\code FIDIV \emph{src}} \index{FIDIV} &
{\code ST0 /= (float) \emph{src}}. {\code ST0}를 정수로 나눈다. 
\emph{src} 는 메모리 상의 워드 혹은 더블워드 여야 한다. \\
{\code FIDIVR \emph{src}} \index{FIDIVR} &
{\code ST0 = (float) \emph{src} / ST0}. 정수를 {\code ST0} 로 나눈다.
\emph{src} 는 메모리 상의 워드 혹은 더블워드 여야 한다. 
\end{tabular}
\index{부동 소수점 부프로세서!곱셈과 나눗셈|)}
\subsubsection{비교\index{부동 소수점 부프로세서!비교|(}}

부프로세서는 부동 소수점 수들에 대한 비교 명령도 수행한다. {\code FCOM} 계열의
명령들이 이 작업을 한다. \\

\begin{tabular}{lp{4in}}
{\code FCOM \emph{src}} \index{FCOM} & 
{\code ST0} 와 {\code \emph{src}} 를 비교. \emph{src} 는 임의의 
부프로세서 레지스터 거나 메모리 상의 단일 혹은 2배 정밀도 수 이면 된다. \\
{\code FCOMP \emph{src}} \index{FCOMP} & 
{\code ST0} 와 {\code \emph{src}} 를 비교한 후 팝 한다.\emph{src} 는
임의의 부프로세서 레지스터 거나 메모리 상의 단일 혹은 2배 정밀도 수 이면 된다.\\
{\code FCOMPP} \index{FCOMPP} & 
{\code ST0} 와 {\code ST1} 를 비교 한 후 팝을 두 번 한다. \\
{\code FICOM \emph{src}} \index{FICOM} & 
{\code ST0} 와 {\code (float) \emph{src}} 를 비교한다. \emph{src} 는 메모리 상의  
워드 혹은 더블워드 정수 이다. \\
{\code FICOMP \emph{src}} \index{FICOMP} & 
{\code ST0} 와 {\code (float)\emph{src}} 를 비교 한 후 팝 한다. 
\emph{src} 는 메모리 상의 워드 혹은 더블워드 정수 이다.  \\
{\code FTST } \index{FTST} &
{\code ST0} 와 0 을 비교한다.
\end{tabular}

\begin{figure}[t]
\begin{AsmCodeListing}[frame=single]
;     if ( x > y )
;
      fld    qword [x]       ; ST0 = x
      fcomp  qword [y]       ; STO 와 y 를 비교
      fstsw  ax              ; 플래그 레지스터에 C 비트를 이동 
      sahf
      jna    else_part       ;  x>y 가 아니면 else_part 로 분기
then_part:
      ; then 을 위한 부분
      jmp    end_if
else_part:
      ; else 를 위한 부분
end_if:
\end{AsmCodeListing}
\caption{비교 예제\label{fig:compEx}}
\end{figure}

이 명령들은 부프로세서 상태 레지스터의 C$_0$, C$_1$, C$_2$, C$_3$ 비트들을
바꾼다. 불행이도 CPU 가 직접 이러한 비트에 접근하는 것은 불가능 하다. 조건 분기 명령은
부프로세서의 상태 레지스터가 아닌 플래그 레지스터를 사용한다.그러나 상태 워드의 비트들을
이에 대응되는 플래그 레지스터의 비트에 옮기는 것은 아래의 몇 가지 새로운 명령을
이용한다면 쉽다
\\

\begin{tabular}{lp{4in}}
{\code FSTSW \emph{dest}} \index{FSTSW} & 
부프로세서의 상태 워드를 메모리의 워드나 AX 레지스터에 저장한다.\\
{\code SAHF} \index{SAHF} & 
AH 레지스터를 플래그 레지스터에 저장한다. \\
{\code LAHF} \index{LAHF} & 
플래그 레지스터의 비트들과 함게 AH 를 불러온다. \\
\end{tabular}

\begin{figure}
\begin{AsmCodeListing}[frame=single]
global _dmax

segment .text
; function _dmax
; 두 개의 double 인자 들 중 큰 것을 리턴한다.
; C 원형
; double dmax( double d1, double d2 )
; 인자들:
;   d1   - 첫 번째 double
;   d2   - 두 번째 double
; 리턴값:
;  d1 과 d2 중 큰 것 (ST0 에 대입하여)
%define d1   ebp+8
%define d2   ebp+16
_dmax:
        enter   0, 0

        fld     qword [d2]
        fld     qword [d1]          ; ST0 = d1, ST1 = d2
        fcomip  st1                 ; ST0 = d2
        jna     short d2_bigger
        fcomp   st0                 ; d2 를 팝 한다.
        fld     qword [d1]          ; ST0 = d1
        jmp     short exit
d2_bigger:                          ; d2 가 더 크면 아무것도 안함
exit:
        leave
        ret
\end{AsmCodeListing}
\caption{{\code FCOMIP} 예제 \label{fig:fcomipEx}}
\index{FCOMIP}
\end{figure}

그림 ~\ref{fig:compEx} 는 코드 예제를 보여준다. ~5 행과 6 행은 부프로세서의
상태 워드의 C$_0$, C$_1$, C$_2$, C$_3$ 비트들을 플래그 레지스터에 대입한다. 
전달한 비트들은 두 개의 \emph{부호가 없는} 정수의 비교 결과와 동일한다. 이는
우리가 ~7 행에서 {\code JNA} 를 사용한 이유이다. 

펜티엄 프로(그리고 그 이후의 프로세서들 (펜티엄 II, III)) 는 두 개의 새로운 비교
명령을 지원하는데 이는 CPU 의 플래그 레지스터의 값을 직접적으로 변경할 수 있다. 

\begin{tabular}{lp{4in}}
{\code FCOMI \emph{src}} \index{FCOMI} & 
{\code ST0} 와 {\code \emph{src}} 를 비교한다. \emph{src} 는 반드시
부프로세서 레지스터 여야 한다. \\
{\code FCOMIP \emph{src}} \index{FCOMIP} & 
{\code ST0} 와 {\code \emph{src}} 를 비교한 후 팝 한다.\emph{src} 는
반드시 부프로세서 레지스터 여야 한다. \\
\end{tabular}
그림 ~\ref{fig:fcomipEx} 은 두 개의 double 값 중 큰 것을 찾는 {\code FCOMIP} 를 사용한
예제 서브루틴을 보여주고 있다. 이 명령들을 정수 비교 함수들과 혼동하지 말라 ({\code FICOM} 과
{\code FICOMP}).

\index{부동 소수점 부프로세서!비교|)}

\subsubsection{잡다한 명령들}
%FINIT?

\begin{figure}
\begin{AsmCodeListing}[frame=single]
segment .data
x            dq  2.75          ; double 형식으로 변환
five         dw  5

segment .text
      fild   dword [five]      ; ST0 = 5
      fld    qword [x]         ; ST0 = 2.75, ST1 = 5
      fscale                   ; ST0 = 2.75 * 32, ST1 = 5
\end{AsmCodeListing}
\caption{{\code FSCALE} 예제\label{fig:fscaleEx}}
\index{FSCALE}
\end{figure}

이 단원은 부프로세서가 제공하는 여러 잡다한 명령들에 대해 다루어 보도록 
한다. 

\begin{tabular}{lp{4in}}
{\code FCHS} \index{FCHS} & 
{\code ST0 = - ST0}, {\code ST0} 의 부호를 바꾼다. \\
{\code FABS} \index{FABS} & 
$\mathtt{ST0} = |\mathtt{ST0}|$ ,{\code ST0} 의 절대값을 취한다.\\
{\code FSQRT} \index{FSQRT} &
$\mathtt{ST0} = \sqrt{\mathtt{STO}}$, {\code ST0} 의 제곱근을 구한다. \\
{\code FSCALE} \index{FSCALE} &
$\mathtt{ST0} = \mathtt{ST0} \times 2^{\lfloor \mathtt{ST1} \rfloor}$
,{\code ST0} 에 2 의 멱수를 빠르게 곱한다. {\code ST1} 는 스택에저 제거되지 않는다.
그림 ~\ref{fig:fscaleEx} 는 이 명령을 어떻게 사용하는지에 대한 예제를 보여준다. 

\end{tabular}

\subsection{예제}

\subsection{2차 방정식의 근의 공식\index{quad.asm|(}}
우리의 첫 번째 예제는 2차 방정식의 근의 공식을 어셈블리로 나타낸 것이다.

\[ a x^2 + b x + c = 0 \]
근의 공식을 통해 두 개의 근을 구할 수 있다. $x$: $x_1$ 와 $x_2$.
\[ x_1, x_2 = \frac{-b \pm \sqrt{b^2 - 4 a c}}{2 a} \]
제곱근 안의 식 ($b^2 - 4 a c$) 은 \emph{판별식(discriminant)} 라 부른다.
이 값은 근이 어떠한 성질을 띄는지 판별하는데 유용하게 쓰인다. 

\begin{enumerate}
\item 실수인 중근을 가진다. $b^2 - 4 a c = 0$
\item 두 개의 실수근을 가진다. $b^2 - 4 a c > 0$
\item 두 개의 복소근을 가진다. $b^2 - 4 a c < 0$
\end{enumerate}

아래 어셈블리 서브루틴을 사용하는 작은 C 프로그램을 나타냈다. 

\LabelLine{quadt.c}
\begin{lstlisting}{}
#include <stdio.h>

int quadratic( double, double, double, double *, double *);

int main()
{
  double a,b,c, root1, root2;

  printf("Enter a, b, c: ");
  scanf("%lf %lf %lf", &a, &b, &c);
  if (quadratic( a, b, c, &root1, &root2) )
    printf("roots: %.10g %.10g\n", root1, root2);
  else
    printf("No real roots\n");
  return 0;
}
\end{lstlisting}
\LabelLine{quadt.c}

여기 어셈블리 서브루틴이 있다:
\begin{AsmCodeListing}[label=quad.asm,commentchar=$]
; 함수 quadratic
; 2차 방정식의 근을 찾는다.  
;       a*x^2 + b*x + c = 0
; C 원형:
;   int quadratic( double a, double b, double c,
;                  double * root1, double *root2 )
; 인자:
;   a, b, c - 2차 방정식의 각 계수들 (위를 참조)
;   root1   - 첫 번째 근을 저장할 double 을 가리키는 포인터 
;   root2   - 두 번째 근을 저장할 double 을 가리키는 포인터 
; 리턴값:
;  실근을 찾으면 1 을 리턴, 아니면 0 리턴

%define a               qword [ebp+8]
%define b               qword [ebp+16]
%define c               qword [ebp+24]
%define root1           dword [ebp+32]
%define root2           dword [ebp+36]
%define disc            qword [ebp-8]
%define one_over_2a     qword [ebp-16]

segment .data
MinusFour       dw      -4

segment .text
        global  _quadratic
_quadratic:
        push    ebp
        mov     ebp, esp
        sub     esp, 16         ; 2 개의 더블을 할당한다 (disc 와 one_over_2a)
        push    ebx             ; 원래의 ebx 값을 저장

        fild    word [MinusFour]; 스택 -4
        fld     a               ; 스택: a, -4
        fld     c               ; 스택: c, a, -4
        fmulp   st1             ; 스택: a*c, -4
        fmulp   st1             ; 스택: -4*a*c
        fld     b
        fld     b               ; 스택: b, b, -4*a*c
        fmulp   st1             ; 스택: b*b, -4*a*c
        faddp   st1             ; 스택: b*b - 4*a*c
        ftst                    ; 0 과 비교
        fstsw   ax
        sahf
        jb      no_real_solutions ; 만일 disc < 0, 이면 해가 없다.
        fsqrt                   ; 스택: sqrt(b*b - 4*a*c)
        fstp    disc            ; 저장 후 스택을 팝 한다.
        fld1                    ; 스택: 1.0
        fld     a               ; 스택: a, 1.0
        fscale                  ; 스택: a * 2^(1.0) = 2*a, 1
        fdivp   st1             ; 스택: 1/(2*a)
        fst     one_over_2a     ; 스택: 1/(2*a)
        fld     b               ; 스택: b, 1/(2*a)
        fld     disc            ; 스택: disc, b, 1/(2*a)
        fsubrp  st1             ; 스택: disc - b, 1/(2*a)
        fmulp   st1             ; 스택: (-b + disc)/(2*a)
        mov     ebx, root1
        fstp    qword [ebx]     ;  *root1 에 저장
        fld     b               ; 스택: b
        fld     disc            ; 스택: disc, b
        fchs                    ; 스택: -disc, b
        fsubrp  st1             ; 스택: -disc - b
        fmul    one_over_2a     ; 스택: (-b - disc)/(2*a)
        mov     ebx, root2
        fstp    qword [ebx]     ; *root2 에 저장
        mov     eax, 1          ; 리턴값은 1
        jmp     short quit

no_real_solutions:
        mov     eax, 0          ; 리턴값은 0

quit:
        pop     ebx
        mov     esp, ebp
        pop     ebp
        ret
\end{AsmCodeListing}
\index{quad.asm|)}

\subsection{파일로 부터 배열을 읽기\index{read.asm|(}}
이번 예제에서는 어셈블리가 파일로 부터 double 을 읽어들인다. 아래는
짧은 C 테스트 프로그램이다. 

\LabelLine{readt.c}
\begin{lstlisting}[escapeinside=~~]{}
/*
 * ~이 프로그램은 32 비트 read\_doubles() 어셈블리 프로시저를 테스트 한다.~
 * ~이는 stdin 으로 부터 double 을 읽어 들인다. (파일로 부터 읽기 위해 ~
 * ~ 리다이렉션(redirection)을 사용한다.)~
 */
#include <stdio.h>
extern int read_doubles( FILE *, double *, int );
#define MAX 100

int main()
{
  int i,n;
  double a[MAX];

  n = read_doubles(stdin, a, MAX);

  for( i=0; i < n; i++ )
    printf("%3d %g\n", i, a[i]);
  return 0;
}
\end{lstlisting}
\LabelLine{readt.c}

아래는 어셈블리 루틴이다. 
\begin{AsmCodeListing}[label=read.asm]
segment .data
format  db      "%lf", 0        ; format for fscanf()

segment .text
        global  _read_doubles
        extern  _fscanf

%define SIZEOF_DOUBLE   8
%define FP              dword [ebp + 8]
%define ARRAYP          dword [ebp + 12]
%define ARRAY_SIZE      dword [ebp + 16]
%define TEMP_DOUBLE     [ebp - 8]

;
; _read_doubles 함수
; C 원형:
;   int read_doubles( FILE * fp, double * arrayp, int array_size );
; 이 함수는 텍스트 파일로 부터 double 을 읽어 들여서 EOF 가 나올 때 까지나 
; 배열이 꽉 찰 때 까지 배열에 저장한다.
; 인자:
;   fp         - 읽어 들일 FILE 포인터 (읽기 가능해야 한다)
;   arrayp     - 읽어 들인 값을 저장할 double 배열을 가리키는 포인터
;   array_size - 배열의 원소의 개수 
; 리턴:
;  배열에 저장된 double 의 개수 (EAX 에 저장됨)

_read_doubles:
        push    ebp
        mov     ebp,esp
        sub     esp, SIZEOF_DOUBLE      ; 스택에 하나의 double 을 정의 

        push    esi                     ; esi 저장
        mov     esi, ARRAYP             ; esi = ARRAYP
        xor     edx, edx                ; edx = 배열 원소 위치 (처음이 0)

while_loop:
        cmp     edx, ARRAY_SIZE         ; edx < ARRAY_SIZE 인가?
        jnl     short quit              ; 아니면 루프 루프 종료 
;
; TEMP_DOUBLE 로 double 값을 읽어 들이기 위해 fscanf() 를 호출
; fscanf() 는 edx 값을 바꿀 수 있으므로 저장해 놓는다. 
;
        push    edx                     ; edx 저장
        lea     eax, TEMP_DOUBLE
        push    eax                     ; &TEMP_DOUBLE 푸시
        push    dword format            ; &format 푸시
        push    FP                      ; 파일 포인터 푸시 
        call    _fscanf
        add     esp, 12
        pop     edx                     ; edx 복원
        cmp     eax, 1                  ; fscanf 가 1 을 리턴했는가?
        jne     short quit              ; 아니면 루프 종료
;
; TEMP_DOUBLE 를 ARRAYP[edx] 에 대입
; (8 바이트의 double 은 두 개의 4 바이트 레지스터를 통해 대입된다.)
;
        mov     eax, [ebp - 8]
        mov     [esi + 8*edx], eax      ; 하위 4 바이트를 복사 
        mov     eax, [ebp - 4]
        mov     [esi + 8*edx + 4], eax  ; 상위 4 바이트를 복사 

        inc     edx
        jmp     while_loop

quit:
        pop     esi                     ; esi 를 복원한다. 

        mov     eax, edx                ; 리턴값을 eax 에 저장 

        mov     esp, ebp
        pop     ebp
        ret 
\end{AsmCodeListing}
\index{read.asm|)}

\subsection{소수 찾기\index{prime2.asm|(}}

마지막 예제는 소수를 찾는 프로그램이다. 이미 앞에서 한 번 다루어 보았지만 
이번 것은 좀더 속도가 향상되었다. 이번 프로그램은 앞에서 찾은 소수를 배열에 저장해
놓고 특정한 수가 소수 인지를 판별하기 위해 특정한 수 이전의 모든 짝수로 나누어 보는
것이 아니라 배열에 저장해 놓은 소수들만으로 나누어 본다. 

또다른 차이점은 특정한 수가 소수 인지를 판별하기 위해 그 수의 제곱근 이하의 소수들만으로
나누어 본다는 것이다. 이 때문에 부프로세서의 제어 워드를 변경하여 제곱근을 정수로 저장할 때
반올림 하기 보단 내림을 해야한다. 이는 제어 워드의 10 과 11 번째 비트를 변경하면 된다. 
이 비트들은 RC (반올림 제어, Rounding Control) 비트로 불린다. 만일 두 비트 모두 0 이라면 (기본값)
부프로세서는 정수로 변환시에 반올림을 한다. 두 비트가 모두 1 이라면 부프로세서는 정수로
변환시 내림을 한다. 유의해야 할 점은 기존의 제어 워드를 저장하고 리턴하기 전에 복원해야 한다는
것이다. 

아래는 C 드라이버 프로그램 이다
\LabelLine{fprime.c}
\begin{lstlisting}[escapeinside=~~]{}
#include <stdio.h>
#include <stdlib.h>
/*
 * ~ find\_primes 함수 ~
 * ~ 지정된 개수의 소수들을 찾는다~
 * ~인자들:~
 *   a -~ 배열을 보관할 배열~
 *   n - ~찾을 소수의 개수~
 */
extern void find_primes( int * a, unsigned n );

int main()
{
  int status;
  unsigned i;
  unsigned max;
  int * a;

  printf("How many primes do you wish to find? ");
  scanf("%u", &max);

  a = calloc( sizeof(int), max);

  if ( a ) {

    find_primes(a,max);

    /*~찾은 마지막 20 개의 소수를 출력 ~*/
    for(i= ( max > 20 ) ? max - 20 : 0; i < max; i++ )
      printf("%3d %d\n", i+1, a[i]);

    free(a);
    status = 0;
  }
  else {
    fprintf(stderr, "Can not create array of %u ints\n", max);
    status = 1;
  }

  return status;
}
\end{lstlisting}
\LabelLine{fprime.c}

아래는 어셈블리 루틴이다.

\begin{AsmCodeListing}[label=prime2.asm]
segment .text
        global  _find_primes
;
; find_primes 함수
; 지정한 수 만큼의 소수를 찾는다 
; 인자:
;   array  - 소수를 보관할 배열
;   n_find - 찾을 소수의 개수
; C 원형:
;extern void find_primes( int * array, unsigned n_find )
;
%define array         ebp + 8
%define n_find        ebp + 12
%define n             ebp - 4           ; 현재까지 찾은 소수의 개수 
%define isqrt         ebp - 8           ; guess 의 제곱근의 내림한 값 
%define orig_cntl_wd  ebp - 10          ; 원래 제어 워드 값
%define new_cntl_wd   ebp - 12          ; 새로운 제워 워드 값

_find_primes:
        enter   12,0                    ; 지역 변수를 위한 공간을 할당 

        push    ebx                     ; 가능한 레지스터 변수들을 저장 
        push    esi

        fstcw   word [orig_cntl_wd]     ; 현재의 제어 워드를 저장 
        mov     ax, [orig_cntl_wd]
        or      ax, 0C00h               ; 반올림 비트를 11 로 한다. (내림)
        mov     [new_cntl_wd], ax
        fldcw   word [new_cntl_wd]

        mov     esi, [array]            ; esi 는 배열을 가리킨다. 
        mov     dword [esi], 2          ; array[0] = 2
        mov     dword [esi + 4], 3      ; array[1] = 3
        mov     ebx, 5                  ; ebx = guess = 5
        mov     dword [n], 2            ; n = 2
;
; 이 바깥 루프는 각 루프 마다 새로운 소수를 하나씩 찾는다. 이 새로운 소수는
; 배열의 끝부분에 추가된다. 기존의 소수 찾는 프로그램들과는 달리 이 함수는 
; 소수 판별을 위해 그 수 미만의 모든 홀수들로 나누어 보지 않는다.
; 오직 이미 찾은 소수들 만으로 나누어 본다. (이 때문에 이 소수들을 배열에
; 저장한 것이다.)
;
while_limit:
        mov     eax, [n]
        cmp     eax, [n_find]           ; while ( n < n_find )
        jnb     short quit_limit

        mov     ecx, 1                  ; ecx 는 원소의 위치를 가리키는데 사용 
        push    ebx                     ; guess 를 스택에 저장 
        fild    dword [esp]             ;guess 를 부프로세서 스택에 불러온다
        pop     ebx                     ; 스택에서 guess 를 뺀다.
        fsqrt                           ; sqrt(guess) 값을 구한다. 
        fistp   dword [isqrt]           ; isqrt = floor(sqrt(quess))
;
; 이 내부의 루프는 guess (ebx)를 이전에 찾은 소수들로 나누어서
; guess 의 소수 인수를 찾을 때 까지 혹은  
; 소수 인수가 floor(sqrt(guess)) 보다 클 때 까지 나눈다. 
;(floor 은 내림 함수)
while_factor:
        mov     eax, dword [esi + 4*ecx]        ; eax = array[ecx]
        cmp     eax, [isqrt]                    ; while ( isqrt < array[ecx] 
        jnbe    short quit_factor_prime
        mov     eax, ebx
        xor     edx, edx
        div     dword [esi + 4*ecx]     
        or      edx, edx                        ; && guess % array[ecx] != 0 )
        jz      short quit_factor_not_prime
        inc     ecx                             ; 다음 소수로 시도한다 
        jmp     short while_factor

;
; 새 소수를 찾음!
;
quit_factor_prime:
        mov     eax, [n]
        mov     dword [esi + 4*eax], ebx        ; guess 를 배열의 끝 부분에 추가 
        inc     eax
        mov     [n], eax                        ; inc n

quit_factor_not_prime:
        add     ebx, 2                          ; 다음 홀수를 시도해 본다 
        jmp     short while_limit

quit_limit:

        fldcw   word [orig_cntl_wd]             ; 제어 워드를 복원 
        pop     esi                             ; 레지스터 변수들을 복원 
        pop     ebx

        leave
        ret 
\end{AsmCodeListing}
\index{prime2.asm|)}
\index{부동 소수점 부프로세서|)}
\index{부동 소수점|)}
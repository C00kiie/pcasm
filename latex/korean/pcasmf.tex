% -*-LaTex-*-
%front matter of pcasm book

\chapter{머리말}

\section*{이 책을 쓴 이유}

이 책을 쓰게 된 동기는 바로 독자들에게 파스칼과 같은 언어들 보다
낮은 수준에서 컴퓨터가 실제로 어떻게 작동하는지에 대한 자세한 
이해를 위해서 이다. 컴퓨터가 실제로 어떻게 작동하는 지에 대해 자세히
알게 된다면 독자 여러분은 C 나 C++ 과 같은 고급 언어를 통해 
프로그램을 개발하는것을 훨씬 효율적으로 할 수 있을 것이다. 
이 목적을 이루기 위해선 어셈블리 언어를 배우는 것이 가장 효과적이다. 
다른 PC 어셈블리 언어 책들은 아직도 1981년 PC 에 사용되었던
8086 프로세서에서 어떻게 프로그램을 만드는지에 대해 다루고 있다. 이
8086 프로세서는 오직 \emph{실제} 모드 만을 지원한다. 이 모드에선
어떠한 프로그램이라도 컴퓨터의 메모리나 다른 장치들에게 자유롭게 접근
이 가능하다. 이 모드는 안정적인 멀티 태스킹 운영체제에는 적합하지 않다.
이 책은 그 대신 \emph{보호} 모드에서 작동하는 (현재의 윈도우즈나 리눅스가 실행되는)
80386 이나 그 이후에 나온 프로세서들에서 어떻게 프로그램을 만드는지에 대해
다루고 있다. 이 모드는 현대의 운영체제들이 바라는 기능들을 가지고 있다.
예를 들자면 가상 메모리나 메모리 보호 등이다. 
보호 모드를 이용하는 데에는 여러 이유들이 있다. 

\begin{enumerate}
\item 다른 책이 다루는 8086 실제 모드에서 보다 프로그래밍 하기가
      편하다
\item 현대 모든 PC 운영체제는 보호 모드에서 작동된다. 
\item 보호 모드에서 작동되는 많은 수의 무료 프로그램들이 있다. 
\end{enumerate}
따라서 제가 이 책을 쓰게된 가장 큰 동기는 바로 보호 모드 PC 어셈블리 
프로그래밍에 대한 책의 부족 때문이다. 

앞에서 살짝 말했듯이 이 책은 무료/공개 소스 소프트웨어만을 사용한다. 
그 예로, NASM 어셈블러와 DJGPP C/C++ 컴파일러가 있다. 이 둘은 모두 
인터넷을 통해서 다운 로드가 가능하다. 또한 이 책은 앞으로 리눅스에서
NASM 어셈블리 코드를 어떻게 사용할 것인지와 Windows 에서 볼랜드사
와 마이크로소프트사의 C/C++ 컴파일러를 어떻게 사용할 것인지에 대해 
다룰 것 이다.
앞서 말한 모든 프로그램들은 나의 웹사이트 http://www.drpaulcarter.com/pcasm 
에서 다운로드가 가능하다. 이 책에서 다룰 예제들을 어셈블하고 실행시킬 것이라면
나의 사이트에서 프로그램들을 \emph{꼭} 다운로드 해야 한다. 

이 책이 어셈블리 프로그래밍의 모든 부분을 다루지 않았으리라 걱정 할
필요는 없다. 나는 \emph{모든} 프로그래머가 보아야 할 어셈블리 언어의
중요한 주제들을 모두 다루려고 노력했다. 


\section*{감사의 말}

저는 무료/공개 소스 운동에 기여해 주신 많은 프로그래머들에게 감사의 말을 전합니다. 
많은 프로그램, 심지어 이 책 자체도 무료 소프트웨어를 통해서 만들어졌습니다. 특히 저는 
John S. Fine, Simon Tatham, Julian Hall, 그리고 NASM 어셈블러를 개발하신 다른 모든 분들께 
감사의 말을 전합니다. 또한 DJGPP C/C++ 컴파일러를 개발한 DJ Delorie 와 DJGPP 가 기반으로 
한 GNU gcc 컴파일러를 개발한 많은 프로그래머들에게도, \TeX 와 \LaTeXe 를 개발한 Donald Knuth,
마지막으로 자유 소프트웨어 재단을 창설한 Richard Stallman 과 리눅스 커널을 개발한 Linus Torvalds, 
등 에게도 감사의 말을 전합니다. 

아래는 이 책의 오류를 정정해 주신 분들의 목록 입니다. 
\begin{itemize}\setlength{\parskip}{-0.5em}
\item John S. Fine
\item Marcelo Henrique Pinto de Almeida
\item Sam Hopkins
\item Nick D'Imperio
\item Jeremiah Lawrence
\item Ed Beroset
\item Jerry Gembarowski
\item Ziqiang Peng
\item Eno Compton
\item Josh I Cates
\item Mik Mifflin
\item Luke Wallis
\item Gaku Ueda
\item Brian Heward
\item Chad Gorshing
\item F. Gotti
\item Bob Wilkinson
\item Markus Koegel
\item Louis Taber
\item Dave Kiddell
\item Eduardo Horowitz
\item S\'{e}bastien Le Ray
\item Nehal Mistry
\item Jianyue Wang
\item Jeremias Kleer
\item Marc Janicki
\item Trevor Hansen
\item Giacomo Bruschi
\item Leonardo Rodr\'{i}guez M\'{u}jica
\item Ulrich Bicheler
\item Wu Xing
\end{itemize}

\section*{인터넷 레퍼런스}
\begin{minipage}[c]{\textwidth}
\begin{tabular}{|ll|}
\hline
저자의 페이지 & {\code http://www.drpaulcarter.com/} \\
%NASM   & {\code http://nasm.2y.net/} \\
NASM SourceForge 페이지 & {\code http://sourceforge.net/projects/nasm/} \\
DJGPP  & {\code http://www.delorie.com/djgpp} \\
리눅스 어셈블리 & {\code http://www.linuxassembly.org/} \\
어셈블리 언어의 미학& {\code http://webster.cs.ucr.edu/} \\
유즈넷 & {\code comp.lang.asm.x86 } \\
인텔 문서 & {\code http://developer.intel.com/design/}\\ & {\code Pentium4/documentation.htm} \\
\hline
\end{tabular}
\end{minipage}

\section*{피드백}

저는 이 작업에 대한 피드백을 환영합니다. 
\begin{center}
\begin{tabular}{ll}
\textbf{E-mail:} & \texttt{\href{mailto:pacman128@gmail.com}{pacman128@gmail.com}} \\
\textbf{WWW:}    & \url{http://www.drpaulcarter.com/pcasm} \\
\end{tabular}
\end{center}
